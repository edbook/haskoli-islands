\lecture[1]{Vika 1:
Hvað er töluleg greining? \\ Skekkjur og mat á þeim
}{lecture-text}
\date{7.~og 9.~janúar 2015}

\begin{document}

\begin{frame}
	\maketitle
\end{frame}

\section{Lagt af stað}

% \frame{
%  \begin{center}
% \textcolor{blue}{\bf \Large Töluleg greining}\\
% \textcolor{blue}{\bf \large Kafli  1. Lagt af stað}  \ (Ch1:1-53)
% \end{center}
\begin{frame}{Yfirlit}
\begin{block}{Vika 1: Lagt af stað}
\begin{center}
\begin{tabular}{|l|l|l|l|}\hline
Nr. &Heiti á viðfangsefni &Bls.&Glærur\\
\hline
1.1 & Hvað er töluleg greining?&1-16&1-5\\
1.2 & Samleitni runa, setning Taylors og $O$-ritháttur& 20-27&6-14\\
1.3 & Skekkjur, skekkjumat og fleira &30-39 &15-36 \\
1.4 & Fleytitalnakerfið& 42-50&37-42\\
% 1.X & Ástandsgildi falla & &24-31\\

\hline
\end{tabular}
\end{center}
\end{block}
\end{frame}

\subsection{Hvað er töluleg greining?}

\begin{frame}{1.1 Hvað er töluleg greining?} \pause
\smallskip
\begin{block}{Tilraun að svari} 
\begin{itemize} 
\item Fagið {\it töluleg greining} snýst um að búa til, greina og forrita
aðferðir til þess að nálga á lausnum á stærðfræðilegum
verkefnum.\pause
\item  Aðferðirnar eru settar fram með reikniritum sem síðan eru
forrituð og það  þarf góðan skilning á eiginleikum lausnanna sem verið
er að nálga til þess að geta greint hvernig forritin munu virka.\pause
\item Greining á reikniritum er aðallega fólgin í skekkjumati og
mati á þeim aðgerðafjölda sem þarf til þess að ná að nálga lausn með
fyrirfram gefinni nákvæmni, \pause þ.e.~hagkvæmni og nákvæmni reikniritsins.
\end{itemize} 
\end{block}
\end{frame}

\begin{frame}{1.1 Dæmi}
 \begin{block}{}
Gerum ráð fyrir að við höfum eftirfarandi eldflaug undir höndum: \pause
\begin{itemize}
 \item Eldsneytið dugir í 18 sek., $t\in [0,18]$.
\item Loftmótstaðan er $d=0,1v^2$, þar sem $v(t)$ er hraðinn.
\item Krafturinn sem knýr flaugina er $T=5000$ 
\item Massi eldsneytisins er $m=180-10t$.
\item Massi flaugarinnar er $M = 120 + m = 300 - 10t$.
\end{itemize}
\pause

Spurningin er: Í hvaða hæð er eldflaugin þegar eldsneytið klárast?

\pause

Úr öðru lögmáli Newtons fæst að $F = (Mv)'$. \pause
Kraftarnir sem verka á eldflaugina er $T$ upp á við og 
loftmótstaðan og þyngdarkrafturinn niður á við. Þannig fæst
$$
  (Mv)' = F = T - Mg - d
$$\pause
það er
$$
M'v + Mv' = T - Mg -d.
$$
\end{block}
\end{frame}

\begin{frame}{1.1 Dæmi}

\begin{block}{}
Þetta jafngildir því að 
\begin{equation}
v' = \frac{T-Mg-d-M'v}{M} = \frac{5000-(300-10t)g-0,1v^2+10v}{300-10t},
\label{eldflaug}
\end{equation}\pause
og upphafsskilyrðin eru $v(0) =0$.

\pause\smallskip
Þar sem $h' = v$, \pause þá er hæðin á tíma $t$ gefin með 
$h(t) =\int_0^t v(s)\, ds$. \pause 
Þegar eldsneytið klárast þá er hæðin $h(18) = \int_0^{18} v(s)\, ds$.

\pause\smallskip
Verkefnið er því að finna $v$, og reikna svo heildið.

\pause\smallskip
Diffurjafnan (\ref{eldflaug}) er ólínuleg þannig að við getum ekki vænst þess 
finna lausn með þeim aðferðum sem við höfum þegar lært. 
Eins er ekki víst að við getum auðveldlega fundið stofnfall $h$ fyrir
$v$ til þess að reikna heildið, jafnvel þótt við hefðum $v$.

\pause\smallskip
Hins vegar getum við leyst diffurjöfnuna tölulega með Runge-Kutta aðferð (Kafli 7)
og heildið reiknum við svo tölulega (Kafli 6).
 \end{block}

\end{frame}


% \begin{frame}{1.1 Dæmi: Finna á nálgun á núllstöð  $f(x)=0$:} 
% 
% \begin{block}{Upprifjun}
% Munum að talan  $p\in I$ sögð vera {\it núllstöð} fallsins $f:I\to \R$ ef
% \begin{equation*}
% 	f(p)=0.
% \end{equation*}\pause
% Milligildissetningin úr stærðfræðigreiningu segir: 
% \begin{quote}
% 	Ef $f$ er samfellt á $[a,b]$ og $y$ er einhver tala á milli $f(a)$ og $f(b)$, þá er til $c$ þannig að $a < c < b$ og $f(c) = y$.
% \end{quote}
% \end{block} 
% \pause
% 
% \begin{block}{Afleiðing}
% Svo ef við höfum $a$ og $b$ þannig að $a < b$ og þannig að
% $f(a)$ og $f(b)$ hafi ólík formerki, þá hefur $f$ núllstöð $p$ á bilinu
% $[a,b]$.
% \end{block}
% \end{frame}
% 
% %


%

\subsection{Samleitni runa, setning Taylors og $O$-ritháttur}
\begin{frame}{1.2 Nokkur atriði um samleitni runa} 

Mörg reiknirit til nálgunar á einhverri rauntölu eru hönnuð þannig að 
reiknuð er runa $x_0,x_1,x_2,\dots$ sem á að nálgast lausnina
okkar.\pause

\begin{block}{Skilgreining}
{\it Rauntalnaruna} $(x_n)$ er sögð vera {\it samleitin \emph{(e.~convergent)}} að {\it
  markgildinu} $r$ ef um sérhvert $\varepsilon>0$ gildir að til er
$N>0$ þannig að  
\begin{equation*}
	|x_n-r|<\varepsilon, \qquad \text{ ef } \quad n\geq N.
\end{equation*}\pause
Þetta er táknað annað hvort með
\begin{equation*}
	\lim_{n\to \infty}x_n=r \qquad \pause\text{ eða } \qquad  x_n\to r
	\text{ ef } n\to \infty.
\end{equation*} \pause
Ef runan $(x_n)$ er samleitin að markgildinu $r$ þá segjum við einnig
að hún {\it stefni á} $r$. 
\end{block}

\pause
Hugsum okkur nú að $(x_n)$ sé gefin runa sem stefnir á $r$ og táknum
skekkjuna með $e_n=r-x_n$.
\end{frame}
%

%
\begin{frame}{1.2 Nokkur atriði um samleitni runa} 
Runan er sögð vera {\it  línulega samleitin} (e.~linear convergence) ef til er $\lambda\in ]0,1[$ þannig að
\begin{equation*}
	\lim_{n\to \infty}\dfrac{|e_{n+1}|}{|e_n|}=\lambda,
\end{equation*}\pause
{\it ofurlínulega samleitin} (e.~superlinear convergence), ef 
\begin{equation*}
	\lim_{n\to \infty}\dfrac{|e_{n+1}|}{|e_n|}=0,
\end{equation*}\pause
{\it ferningssamleitin} (e.~quadratic convergence) ef til er $\lambda>0$ þannig að
\begin{equation*}
	\lim_{n\to \infty}\dfrac{|e_{n+1}|}{|e_n|^2}=\lambda,
\end{equation*}\pause
og {\it samleitin af stigi $\alpha$} (e.~convergence of order $\alpha$), 
þar sem $\alpha> 1$, 
ef til er $\lambda>0$ þannig að
\begin{equation*}
	\lim_{n\to \infty}\dfrac{|e_{n+1}|}{|e_n|^\alpha}=\lambda.
\end{equation*}
\pause
Ath: Runa er ofurlínulega saml.~ef hún er samleitin af
stigi $\alpha>1$.
\end{frame}
%
%
\begin{frame}{1.2 Nokkur atriði um samleitni runa} 
Oft eru notuð veikari hugtök til þess að lýsa samleitni runa
(t.d.~ef við getum ekki fundið $\lambda$ og $\alpha$ nákvæmlega). \pause

\smallskip

Þannig segjum við að runan $(x_n)$ sé  
{\it að minnsta kosti línulega samleitin} ef til er 
$\lambda\in ]0,1[$ og $N >0$ þannig að 
\begin{equation*}
	|e_{n+1}|\leq \lambda |e_n|, \qquad n\geq N,
\end{equation*}\pause
{\it að minnsta kosti ferningssamleitin} ef til er $\lambda>0$ og $N>0$ þannig að
\begin{equation*}
	|e_{n+1}|\leq \lambda |e_n|^2, \qquad n\geq N,
\end{equation*}\pause
og  {\it að minnsta kosti samleitin af stigi $\alpha$}, þar sem $\alpha> 1$, ef til eru $\lambda>0$ og $N>0$ þannig að
\begin{equation*}
	|e_{n+1}|\leq \lambda |e_n|^\alpha, \qquad n\geq N.
\end{equation*}
\end{frame}

\begin{frame}{1.2 Ritháttur fyrir deildanleg föll}
Látum nú $f : I \to \C$ vera fall á bili $I$ sem tekur gildi í
tvinntölunum. Ef $f$ er deildanlegt í sérhverjum punkti í $I$, þá
táknum við afleiðuna með $f'$. Ef $f'$ er deildanlegt í sérhverjum
punkti í $I$, þá táknum við \em aðra afleiðu \em $f$ með $f''$, og svo
framvegis. 

\pause
\smallskip
Við skilgreinum með þrepun $f^{(k)}$ fyrir $k = 0,1,2,
\ldots$ þannig að $f^{(0)} = f$ og ef $f^{(k-1)}$ er deildanlegt í
sérhverjum punkti í $I$, þá er $f^{(k)} = (f^{(k-1)})'$. 

\pause
\smallskip
Við látum
$C^{k}(I)$ tákna línulega rúmið sem samanstendur af öllum föllum $f :
I \to \C$ þannig að $f', \ldots, f^{(k)}$ eru til í sérhverjum punkti
í $I$ og $f^{(k)}$ er samfellt fall á $I$. 
\end{frame}
%
\begin{frame}{1.2 Nálgun með Taylor-margliðu} 
Ef $a \in I$, $m$ er jákvæð heiltala og $f \in C^{m}(I)$, 
þá nefnist margliðan 
\begin{equation*}
  p(x) = %T_mf(x;a) = 
f(a) + f'(a)(x-a) + \ldots   + \frac{f^{(m)}(a)}{m!}(x-a)^m
\end{equation*}
Taylor-margliða fallsins $f$ í punktinum $a$ af stigi $m$, og er 
stundum táknuð með $T_m f(x;a)$.

\pause
Athugið að stig margliðunnar $p$ er minna eða jafnt og $m$. 

\end{frame}

%
\begin{frame}{1.2 Skekkja í nálgun með Taylor-margliðu} 

\begin{block}{Setning Taylors} Látum $I \subseteq \R$ vera bil, $f : I \to
\C$ vera fall, $m \geq 0$ vera heiltölu og gerum ráð fyrir að $f \in
C^m(I)$ og að $f^{(m+1)}(x)$ sé til í sérhverjum innri punkti bilsins
$I$.  Þá er til punktur $\xi$ á milli $a$ og $x$
þannig að 
\begin{equation*}
  f(x) - T_mf(x;a)= \frac{f^{(m+1)}(\xi)}{(m+1)!}(x-a)^{m+1}.
\end{equation*}
Hægri hliðin er oft táknuð $R_m(x)$.
\end{block}

\pause

\begin{block}{Viðbót}
Ef $f^{(m+1)}$ er samfellt á lokaða bilinu með endapunkta $a$ og $x$,
þá er
\begin{align*}
  f(x) - T_mf(x;a)&= \int\limits_a^x 
  \frac{(x-t)^m}{m!}f^{(m+1)}(t) dt \notag \\
  &= (x-a)^{m+1} \int\limits_0^1 
  \frac{(1-s)^m}{m!} f^{(m+1)}(a + s(x-a)) ds\\
&= (x-a)^{m+1}g_m(x)
\end{align*}
\end{block}

\end{frame}
%
\begin{frame}{1.2 Sýnidæmi: Nálgun á fallgildum $x-\sin x$} 
Vitum að $x \approx \sin x$ ef $x$ er lítið. \pause 
Tökum $x=0.1$ og hugsum okkur að
við séum að reikna á vél með 8 stafa nákvæmni.     \pause Hún gefur 
\begin{equation*}
    \sin 0.1 = 0.099833417
\end{equation*}\pause
Af því leiðir
\begin{equation*}
    0.1 - \sin 0.1 = 1.66583\cdot 10^{-4}
\end{equation*}
Við höfum tapað tveimur markverðum stöfum í nákvæmni. 

Ef við notum Taylor-nálgunina fyrir $\sin(x)$,
\begin{equation*}
    \sin x = x - \frac{x^3}{3!} + \frac{x^5}{5!} 
    - \frac{x^7}{7!} \cdots
\end{equation*}
og tökum fyrstu þrjá liðina, þ.e.~skoðum 6.~stigs Taylor-margliðu fallsins. \pause


$x-\sin(x)$ er þá u.þ.b.
$$
x - \left(x - \frac{x^3}{3!} + \frac{x^5}{5!}\right) = \frac{x^3}{3!} - \frac{x^5}{5!}.
$$
\end{frame}

\begin{frame}{1.2 Sýnidæmi: Nálgun á fallgildum $x-\sin x$, framh.}
Fallgildið er þá 
$$
\frac {0.1^3}{3!} - \frac{0.1^5}{5!} = 1.6658334 \cdot 10^{-4}.
$$
\pause
Skekkjan er gefin með
$$
    |R_6(0.1)| = \left|\frac{\sin^{(7)}(\xi)}{7!}0.1^7\right|
    = \left|\frac{-\cos(\xi)}{7!}0.1^7\right| 
    \leq \frac{1}{7!}0.1^7 < 0.2\cdot 10^{-10}.
$$
Sem þýðir að við höfum enn 8 markverða stafi.

\begin{block}{Ritháttur}
 $\sin^{(7)}$ hér að ofan táknar 7.~afleiðu $\sin$, sem er $-\cos$.
\end{block}
\end{frame}
%
\begin{frame}{1.2 Sýnidæmi: Nálgun á $x-\sin x$} 

Ef við tökum $x = 0.01$ er þetta enn greinilegra. 
Reiknivélin gefur
$$
    \sin(0.01) = 0.0099998333
$$
Þannig að 
$$
    0.01 - \sin 0.01 = 0.1667\cdot 10^{-7}
$$
og við erum bara með 4 markverða stafi.
\pause

Hér dugir að taka aðeins þriðja stigs liðinn í Taylor-formúlunni
\begin{equation*}
    0.01 - \sin (0.01) \approx \frac{0.01^3}{3!} 
    = 0.16666667 \cdot 10^{-7},
\end{equation*}
því skekkjan er
$$
R_4(0.01) \leq \frac{0.01^5}{5!} < 10^{-12}
$$
\end{frame}

\subsection{Skekkjur}
\begin{frame}{1.3 Skekkjur} 

Við allar úrlausnir á verkefnum í tölulegri greininingu 
þarf  að fást við skekkjur. \pause 
Þær eru af ýmsum toga: \pause


\begin{itemize} 
\item Gögn eru oft niðurstöður
mælinga og þá fylgja þeim {\it mæliskekkjur}. Eins getum við þurft 
að notast við nálganir á föstum sem koma fyrir (t.d. $\pi$, Avogadrosar
talan, \ldots).\pause
\item Við nálganir á lausnum á
stærðfræðilegum verkefnum verða til {\it aðferðarskekkjur}.
Þær verða til þegar reikniritin eru hönnuð og
greining á reikniritum snýst fyrst og fremst um mat á 
aðferðarskekkjum.\pause
\item  {\it Reikningsskekkjur} verða til í tölvum á öllum
stigum, jafnvel þegar tölur eru lesnar inn í tugakerfi og þeim snúið
yfir í tvíundarkerfi. Þær verða líka til vegna þess að tölvur geta
einungis unnið með endanlegt mengi af tölum og allar útkomur þarf að
nálga innan þess mengis.  Þessar skekkjur nefnast oft {\it
  afrúningsskekkjur}.  \pause
\item {\it Mannlegar villur} eru óumflýjanlegar. Það sem við getum gert
er temja okkur vinnubrögð sem lágmarka líkur á þeim og auðvelda okkur að
finna villur sem við gerum.
\end{itemize} 
\end{frame}
%
%
\begin{frame}{1.3 Skekkja í nálgun á rauntölu $r$} 

Við getum stillt upp jöfnunum svona
\begin{equation*}
	r \text{ (rétt gildi) } = x\text{ (nálgunargildi)} + 
	e \text{ (skekkja)}
\end{equation*} \pause
þar sem talan $x$ er nálgun á tölunni $r$, og þá nefnist 
\begin{equation*}
	e=r-x 
\end{equation*}
{\it skekkjan \emph{(e.~error)} í nálgun á $r$ með $x$} eða bara {\it skekkja}.
\pause

{\it Algildi skekkju  \emph{(e.~absolute error)}} er tölugildið 
%\begin{equation*}
	$|e|=|r-x|$
%\end{equation*}
\pause

Ef vitað er að $r\neq 0$, þá nefnist 
\begin{equation*}
	\dfrac{|e|}{|r|}=\dfrac{|r-x|}{|r|}
\end{equation*}
{\it hlutfallsleg skekkja \emph{(e.~relative error)}} í nálgun á $r$ með $x$.
\pause

\smallskip
%\begin{block}{Athugasemdir}
 \textbf{Ath:} Auðvitað er talan $r$ sem við leitum að óþekkt 
 (annars þyrftum við ekki að framkvæma alla þessa reikninga), sem þýðir 
 að við getum hvergi notað hana í reikninum.
%\end{block}
\end{frame}


\begin{frame}{1.3 Vítt og breitt um skekkjumat } 
\begin{block}{Fyrirframmat á skekkju} 
Metið er áður en reikningar hefjast hversu umfangsmikla reikninga þarf 
að framkvæma til þess að nálgunin náist innan fyrirfram gefinna
skekkjumarka.

\medskip
Ef lausnin er fundin með ítrekunaraðferð er yfirleitt metið hversu
margar ítrekarnir þarf til þess að nálgun verði innan skekkjumarka.
\end{block}

\pause

\begin{block}{Eftirámat á skekkju}   Um leið og reikningar eru
framkvæmdir er lagt mat á skekkju og reikningum er hætt þegar matið
segir að nálgun sé innan skekkjumarka. Það gerist yfirleitt þegar 
gildið sem við reiknum út breytist orðið lítið í hverju skrefi.
\end{block}
\end{frame}
%
%
\begin{frame}{1.3 Eftirámat á skekkju samleitinnar runu (ofurlínuleg samleitni)} 

Hugsum okkur að við séum að nálga töluna $r$ með gildum rununnar
$x_n$, að við höfum reiknað út $x_0,\dots,x_n$ 
og viljum fá mat á skekkjunni $e_n=r-x_n$ í $n$-ta skrefi. 

\smallskip
Við reiknum næst út $x_{n+1}$ og skrifum $e_{n+1}=\lambda_ne_n$.  Þá er 
\begin{equation*}
    x_{n+1}-x_n = (r-x_n)-(r-x_{n+1})
    = e_n-e_{n+1} = (1-\lambda_n)e_n
\end{equation*}
og við fáum 
\begin{equation*}
    e_n = \dfrac{x_{n+1}-x_n}{1-\lambda_n}.
\end{equation*}

Ef við vitum að runan er {\it ofurlínulega samleitin}, þá stefnir 
$\lambda_n$ á $0$ og þar með er 
\begin{equation*}
    e_n\approx x_{n+1}-x_n. 
\end{equation*}
Við hættum því útreikningi þegar $|x_{n+1}-x_n|<\varepsilon$
þar sem $\varepsilon$ er fyrirfram gefin tala, sem lýsir þeirri
nákvæmni sem við viljum ná.  
\end{frame}
%

%
\begin{frame}{1.3 Eftirámat á skekkju samleitinnar runu (amk.~línuleg samleitni)} 


Ef við vitum ekki meira en að runan $x_n$ sé {\it að minnsta kosti
  línulega samleitin} ; segjum $|e_{n+1}|\leq c|e_n|$, $n\geq N$, 
þar sem $c\in(0,1)$, þá á $\lambda_n$ að stefna á fasta $\lambda$ 
og $|\lambda|\leq c$. Við höfum
\begin{equation*}
    \lambda_n = \dfrac{e_{n+1}}{e_n} = 
    \dfrac{1-\lambda_n}{1-\lambda_{n+1}}
    \cdot\dfrac{x_{n+2}-x_{n+1}}{x_{n+1}-x_n}\approx 
    \dfrac{x_{n+2}-x_{n+1}}{x_{n+1}-x_n}
\end{equation*}

\pause
Nú þurfum við að átta okkur á því hvernig þetta er nýtt í
útreikningum. 

\pause
\smallskip
Hugsum okkur að við höfum reiknað út
$x_0,\dots,x_n$ og viljum fá mat á $e_n$. \pause Við reiknum þá  út $x_{n+1}$
og  $x_{n+2}$ og síðan hlutfallið $\kappa_n=(x_{n+2} - x_{n+1})/(x_{n+1} -
x_n)$ sem við notum sem mat á $\lambda_n$.  \pause Eftirámatið á skekkjunni í
ítrekunarskrefi númer $n$ verður síðan 
\begin{equation*}
    e_n\approx \dfrac{x_{n+1}-x_n}{1-\kappa_n}.
\end{equation*}
\pause
\smallskip
Ef stærðin í hægri hliðinni er komin niður fyrir fyrirfram 
gefin skekkjumörk $\varepsilon$, þá stöðvum við útreikningana.
\end{frame}
%
%
\begin{frame}{1.3 Sýnidæmi} 

Okkur er gefin runa af nálgunum á lausn jöfnunnar
\begin{equation*}
    f(x) = e^x\sin x-x^2 = 0
\end{equation*}
og eigum að staðfesta hvort nálgunaraðferðin er ferningssamleitin:

\pause
\begin{center}
\begin{tabular}{l|l|l|l|}
$n$  & $x_n$ &$|x_{n+1}-x_n|$ &  $\frac{|x_{n+1}-x_n|}{|x_n-x_{n-1}|^2}$\\ \hline 
   0&   3.00000000000000 & & \\
   1&   2.73251570951922 &  0.10052257507862 &  1.404\\
   2&   2.63199313444060 &  0.01373904283351 &  1.359\\
   3&  2.61825409160709  & 0.00024006192208  & 1.273\\
   4&   2.61801402968501 &  0.00000007236005 &  1.256\\
   5&   2.61801395732496 &  0.00000000000001 &  1.272\\
\end{tabular}   
\end{center}

\pause
\smallskip
Við metum $e_n\approx |x_{n+1}-x_n|$ og þar af leiðandi 
$|e_n|/|e_{n-1}|^2\approx |x_{n+1}-x_n|/|x_n-x_{n-1}|^2$.

\pause
Við sjáum að hlutfallið $|x_{n+1}-x_n|/|x_n-x_{n-1}|^2$ helst
stöðugt og því ályktum við að aðferðin sé ferningssamleitin.
\end{frame}


\begin{frame}{1.3 Útreikningur á samleitnistigi} 

Skoðum lítið dæmi um útreikninga á samleitnistigi. 

\begin{block}{Dæmi}
Eftirfarandi runa stefnir á  $\sqrt 3$. 
\begin{center}
	\begin{tabular}{c|c|}
		$n$ & $x_n$ \\ \hline
		0&  2.000000000000000 \\
		1&  1.666666666666667 \\
		2&  1.727272727272727 \\
		3&  1.732142857142857 \\
		4&  1.732050680431722 \\
		5&  1.732050807565499 \\
	\end{tabular}
\end{center}
\pause

Er samleitnistigið $1.618$?

Ef ekki, hvert er þá samleitnistigið?
\end{block}
\end{frame}
%

%
\begin{frame}{1.3 Útreikningur á samleitnistigi} 
{\it Lausn:} Ef miðað er við að  runan $(x_n)$ sé  ofurlínulega
samleitin, þá er eðlilegt að taka $e_n\approx x_{n+1}-x_n$ sem mat á
skekkjunni $e_n=\sqrt 3-x_n$ í $n$-ta  
ítrekunarskrefinu. 

\pause

\smallskip
Við byrjum á því að kanna hvernig tilgátan
um að samleitnistigið kemur út á þessum tölum með $e_n=x_{n+1}-x_n$:
\begin{center}
\begin{tabular}{c|c|c|c|}
$n$ & $x_n$ & $|e_n|$ & $|e_n|/|e_{n-1}|^{1.618}$  \\ \hline
0&  2.000000000000000 & 3.3333$\cdot 10^{-1}$ & \\
1&  1.666666666666667 & 6.0606$\cdot 10^{-2}$ & 3.5851$\cdot 10^{-1}$\\
2&  1.727272727272727 & 4.8701$\cdot 10^{-3}$ & 4.5439$\cdot 10^{-1}$\\
3&  1.732142857142857 & 9.2177$\cdot 10^{-5}$ & 5.0837$\cdot 10^{-1}$\\
4&  1.732050680431722 & 1.2713$\cdot 10^{-7}$ & 4.3004$\cdot 10^{-1}$\\
5&  1.732050807565499 & & \\
\end{tabular}
\end{center}\pause
Tveimur síðustu tölunum í aftasta dálki ber ekki nógu vel saman, svo
það er vafasamt hvort talan $1.618$ er rétta
samleitnistigið. 
\end{frame}
%

%
\begin{frame}{1.3 Útreikningur á samleitnistigi} 
Ef $(x_n)$ er samleitin af stigi $\alpha$, þá gildir
$\lim_{n\to \infty}|e_{n+1}|/|e_n|^\alpha=\lambda$, þar sem
$\lambda>0$. Þar með höfum við nálgunarjöfnu ef $n$ er nógu stórt,
\begin{equation*}
    \dfrac{|e_{n+1}|}{|e_n|^\alpha} \approx
    \dfrac{|e_{n+2}|}{|e_{n+1}|^\alpha}
    \qquad \text{ þá og því aðeins að } \qquad 
    \dfrac{|e_{n+1}|}{|e_{n+2}|} \approx
    \bigg|\dfrac{e_{n}}{e_{n+1}} \bigg|^\alpha.
\end{equation*}\pause
Ef við lítum á þetta sem jöfnu og leysum út $\alpha$, þá fáum við
\begin{equation*}
    \alpha_n = 
    \dfrac{\ln(|e_{n+1}|/|e_{n+2}|)}{\ln(|e_{n}|/|e_{n+1}|)}.
\end{equation*}
\end{frame}
%

%
\begin{frame}{1.3 Útreikningur á samleitnistigi} 

Við getum reiknað út þrjú gildi á $\alpha$  úr þeim gögnum sem við
höfum, $\alpha_0= 1.479$, $\alpha_1 = 1.573$ og $\alpha_2=1.660$. \pause 

Ef við endurtökum útreikninga okkar hér að framan með $1.660$ í stað
$1.618$, þá fæst  

\begin{center}
\begin{tabular}{c|c|c|c|}
$n$ & $p_n$ & $|e_n|$ & $|e_n|/|e_{n-1}|^{1.660}$ \\  \hline 
0&  2.000000000000000 & 3.3333$\cdot 10^{-1}$ &\\
1&  1.666666666666667 & 6.0606$\cdot 10^{-2}$ & 3.7551$\cdot 10^{-1}$\\
2&  1.727272727272727 & 4.8701$\cdot 10^{-3}$ & 5.1143$\cdot 10^{-1}$\\
3&  1.732142857142857 & 9.2177$\cdot 10^{-5}$ & 6.3639$\cdot 10^{-1}$\\
4&  1.732050680431722 & 1.2713$\cdot 10^{-7}$ & 6.3639$\cdot 10^{-1}$\\
5&  1.732050807565499 & &\\
\end{tabular}
\end{center}
Tölunum neðst í aftasta dálki ber saman með fimm réttum stöfum og því
ályktum við að $1.660$ sé nær því að vera rétta samleitnistigið.  
\end{frame}

\begin{frame}{1.3 Markverðir stafir}
 \begin{block}{Skilgreining}
  Gerum ráð fyrir að $r\neq 0$, þá segjum við að $x$ sé 
  \emph{nálgun á $r$ með $t$ markverðum stöfum 
  \emph{(e.~significant digits)}} ef 
  $$
    \frac{|r-x|}{|r|} \leq 10^{-t}.
  $$
  \pause
  
  Getum útfært þetta aðeins ítarlegra. Ef
  $$
  10^{-(t+1)} < \frac{|r-x|}{|r|} \leq 10^{-t}.
  $$
  þá segjum við að nálgunin á $r$ með $x$ sé rétt með að minnsta kosti
  $t$ markverðum stöfum og að hámarki með $t+1$ markverðum stöfum.
  
  \pause
  
   Athugið að ef $e$ er minnsta heila talan þannig að $|r|<10^e$, 
þá gefur seinni ójafnan matið 
\begin{equation*}
    |r-x| = 0.0\dots 0 a_t a_{t+1}\ldots \ \cdot\  10^e,
\end{equation*}
þar sem núllin aftan við punkt eru $t$ talsins.  
  
  Einnig er hægt að útfæra þetta fyrir aðrar grunntölur en 10, sjá 
  bók bls.~36.
 \end{block}

\end{frame}


\begin{frame}{1.3 Úrlausn annars stigs jöfnu} 

Þegar núllstöðvar annars stigs jöfnunnar $ax^2+bx+c=0$  eru reiknaðar
út úr formúlunni  
\begin{equation*}
    x = \dfrac{-b\pm\sqrt{b^2-4ac}}{2a},
\end{equation*}
verður til styttingarskekkja ef $b^2$ er miklu stærra heldur en $4ac$
vegna $|b|\approx\sqrt{b^2-4ac}$. \pause Við komumst hjá þessum vandræðum með
því að líta á margliðuna fullþáttaða $a(x-x_1)(x-x_2)$ og notfæra
okkur að núllstöðvarnar  $x_1$ og $x_2$ uppfylla
$x_1x_2=c/a$. 

\pause
\smallskip
Ef $b>0$, þá reiknum við $x_1$ fyrst út úr formúlunni
\begin{equation*}
    x_1 = \dfrac{-b-\sqrt{b^2-4ac}}{2a}
    \quad \text{ og  síðan } \quad
    x_2 = \dfrac{c/a}{x_1}.
\end{equation*}\pause
Ef aftur á móti $b<0$, þá reiknum við fyrst $x_1$ út úr  formúlunni
\begin{equation*}
    x_1 = \dfrac{-b+\sqrt{b^2-4ac}}{2a} 
    \qquad \text{ og síðan } \qquad 
    x_2 = \dfrac{c/a}{x_1}.
\end{equation*}\pause
Ef $b^2\approx 4ac$ þá lendum við í styttingarskekkjum, en við
neyðumst til þess að lifa með þeim. 
\end{frame}

\begin{frame}{1.3 Áhrif gagnaskekkju} 

\begin{block}{}
Hugsum okkur að við séum að finna nálgun á núllstöð falls $x\mapsto f(x,\alpha)$.
 Við viljum finna nálgun $x$ á lausninni $r=r(\alpha)$ sem uppfyllir 
\begin{equation*}
    f(r,\alpha) = 0
\end{equation*}
og við lítum á $\alpha$ sem stika (t.d.~náttúrulegur fasti). 
\end{block}

\pause
\begin{block}{}
Gerum ráð fyrir að $\alpha_0$ sé
nálgun á $\alpha$ og að við þekkjum nálgun á $r(\alpha_0)$ sem er
lausn á jöfnunni $f(x,\alpha_0)=0$. 
\end{block}

\pause
\begin{block}{}
Við viljum athuga hversu mikil áhrif nálgun á $\alpha$ með 
$\alpha_0$ hefur
á lausnina okkar, þ.e. við þurfum að meta skekkjuna
$r(\alpha)-r(\alpha_0)$. 
\end{block}
\end{frame}

\begin{frame}{1.3 Áhrif gagnaskekkju} 
Ef við gefum okkur að $f$ sé samfellt deildanlegt í grennd um punktinn
$(x_0,\alpha_0)$, þar sem $x_0=r(\alpha_0)$ og
${\partial}_xf(x_0,\alpha_0)\neq 0$, þá segir setningin um fólgin föll
að til sé grennd $I$ um punktinn $\alpha_0$ í $\R$ og samfellt
deildanlegt fall $r:I\to \R$, þannig að $r(\alpha_0)=x_0$ og
$f(r(\alpha),\alpha)=0$ fyrir öll $\alpha\in I$.

\pause
\smallskip
 Með öðrum orðum má
segja að við getum alltaf leyst jöfnuna $f(x,\alpha)=0$ með tilliti
til $x$ þannig að út komi lausn $x=r(\alpha)$ sem er samfellt
diffranlegt fall af $\alpha$. 
\end{frame}

\begin{frame}{1.3 Áhrif gagnaskekkju} 


Keðjureglan gefur okkur nú gildi
afleiðunnar, því af jöfnunni $f(r(\alpha),\alpha)=0$ 
leiðir að fallið $I \ni \alpha \mapsto f(r(\alpha),\alpha)$ er fast, 
þannig að 
\begin{equation*}
    0 =\frac {\partial}{\partial \alpha}f(r(\alpha),\alpha) = f_x'(r(\alpha), \alpha)\cdot r'(\alpha) 
    + f_{\alpha}'(r(\alpha),
    \alpha).
\end{equation*}\pause
Þetta gefur
\begin{equation*}
    r'(\alpha) = \frac{-f_{\alpha}'(r(\alpha),\alpha)}
        {f_x'(r(\alpha),\alpha)}.
\end{equation*}\pause
Nú látum við $e$ tákna skekkjuna í nálguninni á $\alpha$ með
$\alpha_0$, $e=\alpha-\alpha_0$. Þá fáum við skekkjumatið
\begin{equation*}
    r(\alpha) - r(\alpha_0) \approx r'(\alpha_0)\cdot e 
    = \frac{-f_{\alpha}'(r(\alpha_0),\alpha_0)}
        {f_x'(r(\alpha_0),\alpha_0)}\cdot e
\end{equation*}
\pause
og jafnframt mat á hlutfallslegri skekkju
\begin{equation*}
    \dfrac{|r(\alpha) - r(\alpha_0)|}
    {|r(\alpha)|} \approx \frac{|f_{\alpha}'(r(\alpha_0),\alpha_0)|}
    {|r(\alpha_0)f_x'(r(\alpha_0),\alpha_0)|}\cdot
    |e|. 
\end{equation*}
\end{frame}
%

%
\begin{frame}{1.3 Sýnidæmi} 

Við skulum nú líta á það verkefni að finna nálgun á minnstu jákvæðu 
lausn jöfnunnar $\sin(\pi x)=1-e^{-x}$, þar sem við gerum ráð fyrir 
því að þurfa að nálga $\pi$ með $3.14$.

\pause

Okkur eru gefnar niðurstöður úr nálguninni með einhverri aðferð. 
Við setjum $f(x,\alpha)=1-e^{-x}-\sin(\alpha x)$ og fáum
\begin{center}
\begin{tabular}{l|l|l|l|}
$n$  & $x_n$ &$|x_{n+1}-x_n|$ &  $\frac{|x_{n+1}-x_n|}{|x_n-x_{n-1}|^2}$\\ \hline
   0&  & & 0.8\\
   1&  0.81276894538752 &  0.00014017936338 &  0.8597\\
   2&  0.81262876602414 &  0.00000001621651 &  0.8253\\
   3&   0.81262874980763 &  0.00000000000000 &  0.8444\\
\end{tabular}
\end{center}
\end{frame}
%
%
\begin{frame}{1.3 Sýnidæmi} 

Hér er $\alpha=\pi$ og $\alpha_0=3.14$ og þar með $|e|<0.0016$. 

\pause

Hlutafleiðurnar eru $f'_x(x,\alpha)=e^{-x}-\alpha\cos(\alpha x)$ og
$f'_\alpha(x,\alpha)=-x\cos(\alpha x)$. 

\pause

Við stingum tölunum okkar inn í matið og notum punktinn $(x_3,\alpha_0)=(0.8126,3.14)$. Það gefur 
\begin{align*}
    r(\pi)-r(3.14)&\approx r'(3.14) \cdot e\\
    &\approx
    \dfrac{|0.8126\cdot \cos(0.8126\cdot 3.14)|}{|e^{-0.8126}-3.14
    \cdot \cos(0.8126 \cdot 3.14)|}\ 
    0.0016 \\
    &\approx 0.4\cdot 10^{-3}
\end{align*}\pause
Þetta mat segir okkur að við eigum að gera ráð fyrir að áhrif
gagnaskekkjunnar séu þau að við fáum lausn með þremur réttum stöfum,
$r(\pi) \approx 0.813$. Nálgun okkar á minnstu jákvæðu lausn jöfnunnar $\sin(\pi
x)=1-e^{-x}$ er því $0.813$. 
%Við getum verið örlítið nákvæmari með því að
%segja að matið okkar sé að lausnin sé á bilinu  $[0.8122,0.8130]$. 

\end{frame}
%


%
% \begin{frame}{1.10 Annað sýnidæmi í sama dúr} 
% 
% Við viljum ákvarða nálgun á lausn $r$ á jöfnunni $\tan(\frac 15\pi
% x)=x$ sem er nálægt $x=1.6$. Við nálgum $\pi$ með $3.14$, og fáum
% gefin nálgunargildin $x_0=1.5$ og fáum út $x_1=1.653315$,
% $x_2=1.622950$ og $x_3=1.621155$ á jöfnunni $\tan(\frac 15\cdot 3.14
% x)-x=0$. Metið $|r-s|$, skerið úr um hvort ástæða sé til þess að bæta
% við fleiri ítrekunum og skrifið nálgunina með réttum fjölda aukastafa
% ef svo er ekki. 
% 
% 
% \end{frame}
% %
% 
% \begin{frame}{1.10 Annað sýnidæmi í sama dúr} 
% 
% {\it Lausn:} Við skilgreinum fallið $f(x,\alpha)=\tan(\frac 1 5\pi
% x)-x$ og nálgum $\alpha=\pi$ með $\alpha_0=3.14$. Skekkjan er
% $e=\pi-3.14\leq 0.0016$. Til þess að fá mat á skekkjunni í nálgun á
% núllstöðinni $r(\pi)$ með $r(3.14)$ notum við formúluna 
% \begin{equation*}
%     r(\pi)-r(3.14)\approx 
%     \dfrac{-f'_\alpha(r(\alpha_0),\alpha_0)}
%     {f'_x(r(\alpha_0),\alpha_0)}\cdot e
% \end{equation*}
% Við höfum að
% \begin{equation*}
%     f'_\alpha(x,\alpha) = 
%     \sec^2(\frac 15 \alpha x)\cdot \frac 15 x
%     \qquad \text{ og } \qquad
%     f'_x(x,\alpha) = 
%     \sec^2(\frac 15 \alpha x)\cdot \frac 15 \alpha-1
% \end{equation*}
% 
% \end{frame}
% 
% 
% \begin{frame}{1.10 Annað sýnidæmi í sama dúr} 
% 
% Við setjum nú inn besta gildið $x_3$ sem við höfum fyrir
% $r(\alpha_0)\approx x_3$ úr töflunni hér að framan inn í formúluna
% fyrir skekkjumatið og fáum $r(\pi)-r(3.14)\approx 1.5\cdot
% 10^{-3}$. Áætluð skekkja í nálguninni $r(3.14)\approx x_3$ er miklu
% minni en talan $1.5\cdot 10^{-3}$ og því þurfum við ekki að taka
% fleiri ítrekanir í nálgunaraðferðinni og ályktum að það eigi að gefa
% upp þrjá markverða stafi í $r(\pi)=1.62$.  
% \end{frame} 



%
\begin{frame}{1.3 $O$-ritháttur}

 
Látum $f$ og $g$ vera tvö föll sem skilgreind eru á bili $I \subset
\mathbb{R}$  og látum $c$ vera tölu á $I$ eða annan hvorn endapunkt $I$. 
\pause

Við segjum að $f(t)$ {\it sé stórt O
  af} $g(t)$ og skrifum 
\begin{equation*}
    f(t) = O(g(t)), \qquad t \rightarrow c,
\end{equation*}
ef til er fasti $C>0$ þannig að ójafnan
\begin{equation*}
    |f(t)| \leq C|g(t)|
\end{equation*}
gildi fyrir öll $t$ í einhverri grennd um $c$. 

\pause
\smallskip
Athugið
að grennd um $c=+\infty$ er bil af gerðinni $]\alpha,+\infty[$ og
grennd um $c=-\infty$ er bil af gerðinni $]-\infty,\alpha[$. 

% \pause
% \smallskip
% Við skrifum 
% \begin{equation*}
%     f(t) = o(g(t)), \quad t \rightarrow c
% \end{equation*}
% og segjum að $f(t)$ sé {\it óvera} af $g(t)$ þegar $t$ stefnir á $c$
% ef  til er fall $h$ á $I$ þ.a. $h(t) \to 0$ ef $t\to c$ og ójafnan  
% \begin{equation*}
%     |f(t)| \leq |h(t)||g(t)|,
% \end{equation*}
% gildi fyrir öll $t$ í einhverri grennd um $c$.

\end{frame}
%

%
\begin{frame}{1.3 $O$-ritháttur og skekkja í Taylor-nálgnum} 


Oft er $O$-ritháttur notaður þegar fjallað er um skekkjur í
Taylor-nálgunum,
\begin{align*}
    f(x) - T_n f(x;c) &= f(x) - f(c) - f'(x-c) - \cdots 
    - \frac{f^{(n)}(c)}{n!}(x-c)^n \\
    &= \frac{f^{(n+1)}(\xi)}{(n+1)!}(x-c)^{n+1} =
    O\big((x-c)^{n+1}\big),  \quad x \to c
\end{align*}
\end{frame}
%

%
\begin{frame}{1.3 Sýnidæmi} 

Það eru til haugar af dæmum, sem við þekkjum vel. \pause

% \begin{block}{Dæmi}
% Við vitum að 
% $t^{\alpha}e^{-t} \to 0$ ef $t\to\infty$. Þannig að 
% $$
% h(t):= e^{-t}t^\alpha = \frac{e^{-t}}{t^{-\alpha}},
% $$
% stefnir á $0$ þegar $t$ stefnir á $\infty$. \pause
% Það er 
% \begin{equation*}
%     e^{-x} = o(x^{-\alpha}), 
%     \qquad x\to \infty \quad \text{  fyrir öll }
%     \alpha > 0.
% \end{equation*}
% \end{block}
% 
% \pause
\begin{block}{Dæmi}
Setning Taylors gefur okkur:
\begin{gather*}
    x - \sin x = O(x^3), \quad x \to 0\\
    x - \frac{x^3}{3!} - \sin x = O(x^5), \quad x \to 0
\end{gather*}
\end{block}
\end{frame}
%

%
\begin{frame}{1.3 $O$-ritháttur fyrir runur} 


Látum nú $(a_n)$ og $(b_n)$ vera tvær talnarunur. 
Við segjum að $a_n$ {\it sé stórt O af} $b_n$ og skrifum 
\begin{equation*}
    a_n = O(b_n), 
\end{equation*}
ef til er fasti $C>0$ þannig að ójafnan
\begin{equation*}
    |a_n| \leq C|b_n|
\end{equation*}
gildi fyrir öll $n=0,1,2,3,\dots$. 
% Við segjum að $a_n$ sé {\it óvera} af $b_n$ og skrifum
% \begin{equation*}
%     a_n = o(b_n), 
% \end{equation*}
% ef til er runa $\varepsilon_n\searrow 0$ þannig að ójafnan 
% \begin{equation*}
%     |a_n|\leq \varepsilon_n|b_n|
% \end{equation*}
% gildi fyrir öll $n=0,1,2,3\dots$.

\end{frame}

%
\begin{frame}{1.3 Tvö sýnidæmi} 
\begin{itemize}
 \item Út frá Taylor-röðinni fyrir $\cos x$ fáum við að
\begin{equation*}
    \cos(1/n)-1+1/(2n^2) = O(1/n^4)
\end{equation*} \pause
\item Út frá 
\begin{equation*}
    \sqrt{n+1}-\sqrt n = \dfrac{1}{\sqrt{n+1}+\sqrt n} \leq \frac{1}{2\sqrt n}
\end{equation*}\pause
sjáum við að
\begin{equation*}
    \sqrt{n+1}-\sqrt n = O\big(\dfrac 1{\sqrt n}\big)
\end{equation*}
\end{itemize}
\end{frame}


\subsection{Fleytitalnakerfið}
\begin{frame}{1.4 Fleytitalnakerfið -- Framsetning á tölum} 
Ef $r$ er rauntala frábrugðin $0$ og $\beta$ er náttúrleg tala, $2$ eða stærri, þá er til einhlýtt ákvörðuð framsetning á $r$ af gerðinni 
\begin{equation*}
    r = 
    \pm (0.d_1d_2\dots d_kd_{k+1}\dots)_\beta\times \beta^e
\end{equation*}
þar sem $e$ er heiltala og $d_j$ eru heiltölur
\begin{itemize}
 \item  $1\leq d_1<\beta$, 
 \item $0\leq d_j<\beta$, $j=2,3,4,\dots$.
\end{itemize}
\pause

\smallskip

 Tölvur reikna ýmist í {\it tvíundarkerfi} með $\beta=2$ 
eða í {\it sextánundarkerfi} með $\beta=16$, en við mannfólkið með okkar 
tíu fingur reiknum í {\it tugakerfi} með $\beta=10$.
\end{frame}
%
%
\begin{frame}{1.4 Mantissa}

Formerkið og runan
\begin{equation*}
    \pm(0.d_1d_2\dots d_kd_{k+1}\dots)_\beta =
    \pm\sum_{j=1}^\infty \dfrac{d_j}{\beta^j}
\end{equation*}
nefnist {\it mantissa} tölunnar $r$. \pause

\smallskip

Við skrifum
\begin{equation*}
    (0.d_1d_2\dots d_k)_\beta = 
    \sum_{j=1}^k \dfrac{d_j}{\beta^j}
\end{equation*}
ef $d_{k+1} = d_{k+2} = \cdots = 0$ og segjum þá að talan $r$ hafi
$k$-stafa mantissu. 
\end{frame}
%
%
\begin{frame}{1.4 Markverðir $\beta$-stafir} 

Ef rauntalan $x$ er nálgun á $r$, þá segjum við að $x$ sé nálgun á 
$r$ með {\it að minnsta kosti $t$ markverðum $\beta$-stöfum} ef
\begin{equation*}
    \dfrac{|r-x|}{|r|}\leq \beta^{-t}.
\end{equation*}
\pause
Ef við höfum að auki að
\begin{equation*}
\beta^{-t-1}<\dfrac{|r-x|}{|r|}\leq \beta^{-t}.
\end{equation*}
þá segjum við að $x$ sé nálgun á $r$ með  {\it $t$ markverðum
$\beta$-stöfum}.

\pause
 Athugið að ef $e$ er minnsta heila talan þannig að $|r|<\beta^e$, 
þá gefur seinni ójafnan matið 
\begin{equation*}
    |r-x| = (0.0\dots 0a_ta_{t+1}\dots)_\beta \times \beta^e,
\end{equation*}
þar sem núllin aftan við punkt eru $t$ talsins.  

\end{frame}
%
%
\begin{frame}{1.4 Afrúningur talna} 

Ef $r$ er sett fram á stöðluðu $\beta$-fleytitöluformi, þá nefnist talan
\begin{equation*}
    x = (\pm 0.d_1d_2\dots d_k)_\beta\times \beta^e
\end{equation*}
{\it afskurður tölunnar $r$ við $k$-ta aukastaf $r$},  \pause en talan
\begin{equation*}
    x = \begin{cases} 
    \pm (0.d_1d_2\dots d_k)_\beta\times \beta^e, & 
    d_{k+1}<\beta/2,\\
    \pm ((0.d_1d_2\dots d_k)_\beta+\beta^{-k})\times \beta^e,
    &d_{k+1}\geq \beta/2.
    \end{cases}
\end{equation*}
nefnist {\it afrúningur tölunnar $r$ við $k$-ta aukastaf}.

\pause
Við  köllum þessar aðgerðir {\it afskurð} (e.~chopping) og {\it afrúning}
(e.~rounding).  
\end{frame}
%
%
\begin{frame}{1.4 Fleytitölukerfi}
 
{\it Fleytitölukerfi} er endanlegt hlutmengi í $\R$, sem samanstendur
af öllum tölum  
\begin{equation*}
    \pm (0.d_1d_2\dots d_k)_\beta\times \beta^e
\end{equation*}
þar sem $d_j$ eru heiltölur eins og áður var lýst, $k$ er föst tala
og við höfum mörk á veldisvísinum $m\leq e\leq M$. \pause


Allar tölvur vinna með eitthvert fleytitölukerfi, oftast
með grunntölu $\beta=2$ eða $\beta=16$ eins og áður sagði.

\pause
Eftir hverja aðgerð í tölvunni þarf að nálga útkomuna með 
{\it afskurði} eða {\it afrúningu}. 

\pause
Ef við förum ekki varlega þá getur þetta magnað upp skekkju.

\pause
\begin{block}{IEEE staðlar}
\begin{itemize}
 \item Single: $\beta = 2, k=24, m=-125$ og $M = 128$,
 \item Double: $\beta = 2, k=53, m=-1021$ og $M=1024$.
\end{itemize}
Sjá nánar bls.~37 í kennslubók.
\end{block}\end{frame}

\begin{frame}{1.4 Útreikningur í tugakerfi:} 

Þegar reiknað er í tugakerfi er tölurnar afrúnaðar við $k$-ta aukastaf
ef skekkjan í nálgun á þeim er minni en $\frac 12\times 10^{-k}$. Ef 
\begin{equation*}
    \dfrac{|r-x|}{|r|}<10^{-k-1}
\end{equation*}
þá treystum við öllum $k$ stöfum mantissunnar, en ef
\begin{equation*}
    \dfrac{|r-x|}{|r|}>10^{-k+q},
\end{equation*}
þá eru síðustu $q$ stafir mantissunnar marklausir auk þess
sem vænta má nokkurs fráviks í $d_{k-q}$.
\end{frame}




% \subsection{Ástandsgildi}
% \begin{frame}{1.6 Ástandsgildi falla} 
% 
% Hugsum okkur nú að $f: I\to \R$ sé fall sem skilgreint er á hlutmengi 
% $I\subset \R$ og tekur gildi í $\R$ og að  $r\in I$. \pause
% 
% Gerum ráð fyrir að $x$ sé nálgun á $r$ 
% og að við viljum nota $f(x)$ sem nálgun á $f(r)$. \pause
% 
% Skekkjan í nálguninni  á $f(r)$ með $f(x)$ er þá
% \begin{equation*}
%     f(r)-f(x) = f(x+e)-f(x)\approx f'(x)e.
% \end{equation*}
% 
% \pause
% Ef við vitum að $f(r)\neq 0$, þá er hlutfallsleg skekkja í nálgun á $f(r)$ með $f(x)$
% \begin{equation*}
%     \dfrac{|f(r)-f(x)|}{|f(r)|}\approx 
%     \dfrac{|f'(x)e|}{|f(r)|}
%     \approx\dfrac{|xf'(x)|}{|f(x)|}\cdot \dfrac{|e|}{|r|}.
% \end{equation*}
% 
% \end{frame}
% %
% 
% %
% \begin{frame}{1.6 Ástandsgildi} 
% 
% 
% Við skilgreinum {\it ástandsgildi fallsins $f$ í punktinum $x$} sem
% \begin{equation*}
%     \ast_x(f) = \dfrac{|xf'(x)|}{|f(x)|}.
% \end{equation*}\pause
% Ef $\ast_x(f)\leq 1$ og $x$ er nálgun á $r$ með $m$ markverðum stöfum,
% þá segir nálgunarjafnan  
% \begin{equation*}
%     \dfrac{|f(r)-f(x)|}{|f(r)|} \approx 
%     \ast_x(f)\cdot \dfrac{|e|}{|r|}
% \end{equation*}
% okkur að við getum búist við því að $f(x)$ sé nálgun á $f(r)$ með jafn
% mörgum markverðum stöfum. 
% 
% \pause
% \begin{block}{Athugasemd}
%  Niðurstöðuna að ofan má orða svona, hlutfallsleg skekkja á fallgildinu
% $f(x)$ er ástandsgildið sinnum hlutfallsleg skekkja $x$.
% \end{block}
% 
% \end{frame}
% %
% 
% %
% \begin{frame}{1.6 Ástand er gott eða slæmt}
% 
%  \begin{block}{Skilgreining}
%  Ef $\ast_x(f)\leq 10$, þá segjum við að ástand verkefnisins að reikna
%  út $f(x)$ sé {\it gott} eða að verkefnið sé {\it reikningslega
%    stöðugt} eða einfaldlega að {\it ástand fallsins $f$ sé gott í
%    punktinum $x$}. 
% \end{block}
% \pause
% \begin{block}{Skilgreining}
% Ef hins vegar $\ast_x(f)>10$, þá segjum við að
%  ástand verkefnisins að reikna út $f(x)$ sé {\it slæmt} eða að það sé
%  {\it reikningslega óstöðugt}. 
% \end{block}
% \pause
% \begin{block}{Athugasemd}
% Ef ástandið er gott þá getum við búist
%  við því að það tapist í mesta lagi einn markverður stafur í
%  hlutfallslegri skekkju við útreikning á $f(x)$. Ef hins vegar
%  $\ast_x(f)\approx 10^q$ 
% og $x$ er nálgun á $r$ með $m$ markverðum stöfum, þá er ekki ástæða til
% þess að ætla að $f(x)$ sé betri nálgun á $f(r)$ en að við höfum $m-q$
% markverða stafi. 
% \end{block}
% \end{frame}
% 
% \begin{frame}{1.6 Ástandsgildi falla af mörgum breytistærðum} 
% Ef $f: D\to \R$  er deildanlegt fall af mörgum breytistærðum sem
% skilgreint er á hlutmengi $D\subset \R^n$ og tekur gildi í $\R$, þá
% skilgreinum við ástandsgildi $f$ í punktinum $x$ með tilliti til allra
% breytistærðanna $x_i$ með formúlunni 
% \begin{equation*}
%     \ast_{x_i}(f(\mathbf{x})) = 
%     \dfrac{|x_i f'_{x_i}(\mathbf{x})|}{|f(\mathbf{x})|},
%     \qquad i=1,2,\dots,n,
% \end{equation*}
% þar sem ritháttur okkar fyrir hlutafleiður er 
% \begin{equation*}
%     f'_{x_i} = 
%     \dfrac{{\partial}f}{{\partial}x_i}={\partial}_{x_i}f.
% \end{equation*}
% Ef $\ast_{x_i}(f(\mathbf{x}))\leq 10$ fyrir öll $i$, þá segjum við að
% ástand þess verkefnis að ákvarða fallgildið $f(\mathbf{x})$ sé gott,
% en að það sé slæmt ef $\ast_{x_i}(f(\mathbf{x})) > 10$ fyrir eitthvert
% $i$. 
% \end{frame}
% %
% 
% %
% \begin{frame}{1.6 Ástandsgildi kvaðratrótar} 
% 
% Lítum á kvaðratrótarfallið á jákvæða
% raunásnum
% \begin{equation*}
%     \ast_x(\sqrt x) = 
%     \dfrac{x\cdot \frac 12 x^{-\frac 12}}{x^{\frac 12}} =
%     \dfrac 12, \qquad x>0,
% \end{equation*}
% svo ástand verkefnisins að ákvarða kvaðratrót er gott á öllum ásnum.
% 
% \end{frame}
% %
% %
% \begin{frame}{1.6 Ástandsgildi lograns} 
% 
% Lítum næst á náttúrlega logrann á sama mengi
% \begin{equation*}
%     \ast_x(\ln x) = 
%     \dfrac{x\cdot 1/x}{|\ln x|}=\dfrac 1{|\ln x|}, 
%     \qquad x>0.
% \end{equation*}
% Ástandsgildið er til fyrir öll $x\neq 1$. Ástand verkefnisins að reikna út $\ln x$ er gott ef $1/|\ln x|\leq 10$, sem jafngildir því að $0<x\leq e^{-\frac 1{10}}$ eða 
% $x \geq e^{\frac 1{10}}$, og það er slæmt ef $e^{-\frac
% 1{10}}<x<e^{\frac 1{10}}$ og fer versnandi þegar við nálgumst $1$.
% %
% \end{frame}
% 
% %
% \begin{frame}{1.6 Ástandsgildi grunnaðgerðanna fjögurra} 
% 
% 
% Við getum litið á grunnaðgerðirnar fjórar, samlagningu $(x,y)\mapsto x+y$, frádrátt $(x,y)\mapsto x-y$, margföldun $(x,y)\mapsto xy$ og deilingu $(x,y)\mapsto x/y$ sem föll af tveimur breytistærðum og reiknað út ástandsgildi þeirra,
% \begin{align*}
%     \ast_x(x\pm y) &= 
%     \dfrac {|x\cdot 1|}{|x\pm y|} = 
%     \dfrac{1}{|1\pm y/x|},\\
%     \ast_y(x\pm y) &= 
%     \dfrac {|y\cdot 1|}{|x\pm y|} = 
%     \dfrac{1}{|1\pm x/y|},
% \end{align*}
% sem sýnir okkur að ástand samlagningar er gott nema þegar 
% $x\approx -y$ og ástand frádráttar er gott nema þegar $x\approx y$.
% 
% \end{frame}
% %
% 
% %
% \begin{frame}{1.6 Ástandsgildi grunnaðgerðanna fjögurra} 
% 
% 
%  Eins höfum við 
% \begin{align*}
%     \ast_x(xy) &= \dfrac{|x\cdot y|}{|xy|} = 1 \\
%     \ast_x(xy) &= \dfrac{|y\cdot x|}{|xy|} = 1,
% \end{align*}
% sem sýnir okkur að ástand margföldunar er alltaf gott, \pause
% og
% \begin{align*}
%     \ast_x(x/y) &= \dfrac{|x\cdot 1/y|}{|x/y|} = 1 \\
%     \ast_y(x/y) &= \dfrac{|y\cdot (-x/y^2)|}{|x/y|} = 1,
% \end{align*}
% sem sýnir okkur að ástand deilingar er einnig alltaf gott.
% 
% \pause 
% \smallskip
% 
% 
% Hlutfallsleg skekkja sem verður til þegar frádráttur er
% framkvæmdur á tveimur álíka stórum stærðum er oft nefnd
% {\it styttingarskekkja}. Þegar við hönnum reiknirit og  
% forritum þau verðum við að forðast styttingarskekkjur.
% \end{frame}

%


\begin{frame}{Kafli 1: Fræðilegar spurningar}
  \begin{enumerate}
  \item  Hverjar eru helstu tegundir af skekkjum sem þarf að taka
    tillit til í tölulegum útreikningum?
  \item  Hvernig eru {\it skekkja} og {\it hlutfallsleg skekkja} í nálgun á
    rauntölu skilgreindar?
  \item  Hver er skilgreiningin á því að {\it rauntalnaruna} $(x_n)$
    er sögð vera {\it samleitin} að {\it  markgildinu} $r$?
  \item  Ef $(x_n)$ er gefin runa sem stefnir á $r$ og skekkjan
er  $e_n=r-x_n$, hvað þýðir  þá að runan sé  
{\it  að minnsta kosti línulega samleitin}, 
{\it að minnsta kosti  ferningssamleitin} og  
{\it að minnsta kosti samleitin af stigi $\alpha$} 
  \item  Eftir hvaða reglum eru tölur afrúnaðar í tugakerfi?
  \item  Hvernig er fyrirframmat á skekkju framkvæmt?
  \item  Útskýrið hvernig er eftirámat á skekkju framkvæmt við nálgun á
    rauntölu $r$ er aðferðin er ofurlínulega samleitin?
  \item  Útskýrið hvernig er eftirámat á skekkju framkvæmt við nálgun á
    rauntölu $r$ ef aðferðin er að minnsta kosti línulega samleitin.
   \end{enumerate}
\end{frame}

\begin{frame}{Kafli 1: Fræðilegar spurningar}
  \begin{enumerate}
  \item[9.] Útskýrið hvernig samleitnistig runu er metið.
%   \item[10.] Hvernig eru ástandsgildi falla af einni og mörgum
%     breytistærðum skilgreint og hvernig eru þau notuð?
  \item[10.] Útskýrið hvernig forðast á styttingarskekkjur þegar
    núllstöðvar annars stigs margliðu $ax^2+bx+c$ eru reiknaðar
  \item[11.]  Hvernig er setning Taylors og hvernig er skekkjan í
    Taylor-nálgun?
  \item[12.]   Útskýrið hvernig hægt er að meta hlutfallslega skekkju
    í núllstöð $r(\alpha)$ fallsins $x\mapsto f(x,\alpha)$ ef gefið er
    að það er skekkja í gildinu sem notað er fyrir $\alpha$. 
  \item[13.]  Hvað þýðir að $f(t) = O(g(t))$ ef $t \rightarrow c$ þar
    sem $f$ og $g$ eru föll sem skilgreind eru á bili sem inniheldur
    $c$ eða á hálfás $x>a$ í tilfellinu þegar $c=+\infty$?
%   \item[15.]   Hvað þýðir að $f(t) = o(g(t))$ ef $t \rightarrow c$ þar
%     sem $f$ og $g$ eru föll sem skilgreind eru á bili sem inniheldur
%     $c$ eða á hálfás $x>a$ í tilfellinu þegar $c=+\infty$?
%    \end{enumerate}
% \end{frame}
% 
% \begin{frame}{Kafli 1: Fræðilegar spurningar}
%   \begin{enumerate}
  \item[14.]  Hvað þýðir að $ a_n = O(b_n)$ ef $n\to \infty$ þegar
 $(a_n)$ og $(b_n)$ eru tvær talnarunur?
%    \item[17.]  Hvað þýðir að $ a_n = o(b_n)$ ef $n\to \infty$ þegar
%  $(a_n)$ og $(b_n)$ eru tvær talnarunur?
    \end{enumerate}
\end{frame}
%
\end{document}
