\lecture[8]{Kafli 8: Jaðargildisverkefni fyrir venjulegar afleiðujöfnur}{lecture-text}
\date{18.~og 20.~mars, 2015}

\begin{document}

\begin{frame}
	\maketitle
\end{frame}

\begin{frame}{Yfirlit}
\begin{block}{Kafli 8: Jaðargildisverkefni fyrir 
  venjulegar afleiðujöfnur}
\begin{center}
\begin{tabular}{|l|l|l|l|}\hline
Kafli &Heiti á viðfangsefni &Bls. & Glærur\\
\hline
8.0 &Almenn atriði um jaðargildisverkefni & 656-660 & 3-4\\
8.1 & Línulegar jöfnur -- Dirichlet-jaðarskilyrði & 660-670 & 5-11\\
8.2 & Línulegar jöfnur -- Blönduð jaðarskilyrði & 673-683 & 12-18 \\
%6.3 & Ólínuleg jaðargildisverkefni & 687-696 \\ 
%6.4 & Úrlausn á ólínulegum jöfnum & 688-694 \\ 
\hline
\end{tabular}
\end{center}
\end{block}
\end{frame}

\section*{8.0 Almenn atriði um jaðargildisverkefni}
\begin{frame}{8.0  Jaðargildisverkefni fyrir venjulegar afleiðujöfnur} 
Við ætlum að finna nálgunarlausnir á verkefnum af gerðinni
  \begin{gather*}
    y''=f(x,y,y'), \qquad a\leq x\leq b,\\
\alpha_1y(a)+\alpha_2 y'(a)=\alpha_3,\\
\beta_1 y(b)+\beta_2y'(b)=\beta_3.  
  \end{gather*}

\pause
Afleiðujafnan er sögð vera línuleg ef hún er á forminu
$$
y''=p(x)y'+q(x)y+r(x), \qquad x\in [a,b].
$$
\end{frame}

\begin{frame}{8.0 Jaðarskilyrðin nefnast}
\begin{tabular}{lll}
(i)  &Dirichlet-jaðarskilyrði: &  $y(a)=\alpha$, \ \  $y(b)=\beta$\\
&(Fallsjaðarskilyrði:) \\ \pause
(ii)&Neumann-jaðarskilyrði: 
& $y'(a)=\alpha$, \ \  $y'(b)=\beta$\\
&(Afleiðujaðarskilyrði:)\\
&(Flæðisjaðarskilyrði:)\\\pause
(iii)&Robin-jaðarskilyrði: 
&$\alpha_1y(a)+\alpha_2 y'(a)=\alpha_3$ \\ 
&(Blandað jaðarskilyrði:)&$\beta_1 y(b)+\beta_2y'(b)=\beta_3$\\
&&$(\alpha_1,\alpha_2)\neq (0,0)$
\end{tabular}
\pause

\bigskip
Athugið að blönduð jaðarskilyrði með $\alpha_2=0$ (eða $\beta_2=0$)\\
er Dirichlet skilyrði með 
$\alpha=\alpha_3/\alpha_1$ (eða $\beta=\beta_3/\beta_1$).
\pause

\smallskip
Athugið að blandað jaðarskilyrði með $\alpha_1=0$ (eða $\beta_1=0$)\\
er  Neumann skilyrði með 
$\alpha=\alpha_3/\alpha_2$ (eða $\beta=\beta_3/\beta_2$).
\end{frame}


\section*{8.1 Línulegar jöfnur -- Dirichlet-jaðarskilyrði}
\begin{frame}{8.1 Skiptipunktar / Hnútpunktar} 
Gefum okkur jafna skiptingu á bilinu $[a,b]$, $x_j=a+hj$, 
$h=(b-a)/N$,
$$
a=x_0<x_1<x_2<\cdots<x_{N-1}<x_N=b.
$$

\pause
Við nefnum $x_j$  {\it skiptipunkta} eða {\it hnútpunkta} skiptingarinnar.  

\pause
\smallskip
Punktarnir $a=x_0$ og $b=x_N$ nefnast {\it endapunktar}
skiptingarinnar og $x_j$, með $j=1,\dots,N-1$, 
nefnast {\it innri punktar} skiptingarinnar.

\pause
\smallskip
Í fyrstu atrennu ætlum við aðeins að nálga lausnina fyrir
línulegar jöfnur,
$$
y''=p(x)y'+q(x)y+r(x), \qquad x\in [a,b].
$$
Við reiknum út nálgun á réttu lausninni $y(x)$ í
hnútpunktunum. 

\pause
\smallskip
Rétta gildið í punktinum $x_j$ táknum við með $y_j$ og nálgunargildið
með $w_j$, 
$$
y_j=y(x_j)\approx w_j.
$$\pause 
Eins skrifum við 
$$
p_j=p(x_j), \qquad q_j=q(x_j), \qquad  r_j=r(x_j).
$$
\end{frame}

\begin{frame}{8.1 Línulegar afleiðujöfnur:} 
Nú leiðum við út nálgunarjöfnur, eina fyrir hvern innri skiptipunkt.
Við byrjum á því að stinga punkti $x_j$ inn í afleiðujöfnuna
$$
\big\{ y''(x)= p(x)y'(x)+q(x)y(x) + r(x)\big\}_{x=x_j}.
$$
Næst skiptum við á afleiðum og mismunakvótum i þessari jöfnu,
$$
\dfrac{y_{j+1}-2y_j+y_{j-1}}{h^2} +O(h^2)
=p_j\dfrac{y_{j+1}-y_{j-1}}{2h}+q_jy_j+r_j+ O(h^2).
$$

\pause
Síðan  stillum við upp nálgunargildunum  í stað réttu gildanna:
\end{frame}

\begin{frame}{8.1 Skipt á afleiðum og mismunakvótum} 
Endurtökum réttu jöfnuna
$$
\dfrac{y_{j+1}-2y_j+y_{j-1}}{h^2} +O(h^2)
=p_j\dfrac{y_{j+1}-y_{j-1}}{2h}+q_jy_j+r_j+ O(h^2).
$$

\pause
Nú fellum við niður leifaliðina og 
setjum nálgunargildin í stað réttu gildanna:
$$
\dfrac{w_{j+1}-2w_j+w_{j-1}}{h^2}
=p_j\dfrac{w_{j+1}-w_{j-1}}{2h}+q_jw_j+r_j
$$
Hér fáum við eina jöfnu fyrir sérhvern innri skiptipunkt
$j=1,\dots,N-1$.
\end{frame}


\begin{frame}{8.1 Dirichlet-jaðarskilyrði} 
Við erum komin með $N-1$ 
nálgunarjöfnu til þess að finna $N+1$
nálgunargildi $w_0,\dots,w_N$ fyrir $y_0,\dots,y_N$. 

\smallskip
Ef við erum að leysa línulegt jaðargildisverkefni með 
Dirichlet-jaðarskilyrðum,
  \begin{gather*}
    y''=p(x)y'+q(x)y+r(x), \qquad a\leq x\leq b,\\
y(a)=\alpha \quad \text{ og } \quad y(b)=\beta,
  \end{gather*}

\pause
þá fæst nálgunin með því að leysa línulega jöfnuhneppið
\begin{align*}
w_0&=\alpha,\\
\dfrac{w_{j+1}-2w_j+w_{j-1}}{h^2}
&=p_j\dfrac{w_{j+1}-w_{j-1}}{2h}+q_jw_j+r_j, \qquad j=1,\dots,N-1,\\
w_N&=\beta.  
\end{align*}
\end{frame}


\begin{frame}{8.1 Jafngild framsetning á hneppinu} 
Við lítum aftur á línulegu nálgunarjöfnurnar
$$
\dfrac{w_{j+1}-2w_j+w_{j-1}}{h^2}
=p_j\dfrac{w_{j+1}-w_{j-1}}{2h}+q_jw_j+r_j.
$$

\pause
Margföldum alla liði með $-h^2$ og röðum síðan óþekktu stærðunum
vinstra mengin jafnaðarmerkisins. Þá fæst línulega jöfnuhneppið
$$
\big(-1-\tfrac 12 h p_j\big)w_{j-1}
+\big(2+h^2q_j\big) w_j
+\big(-1+\tfrac 12 h p_j\big)w_{j+1}
=-h^2\, r_j
$$
fyrir $j=1,2,3,\dots,N-1$.
\end{frame}


\begin{frame}{8.1 Línulega jöfnuhneppið á fylkjaformi} 
$$
  A\wv=\bv
$$
$$
  A=\left[\begin{matrix}
  1&0\\
  l_1&d_1&u_1\\
  &l_2&d_2&u_2\\
  &&\cdot&\cdot&\cdot \\
  &&&\cdot&\cdot&\cdot \\
  &&&&\cdot&\cdot&\cdot \\
  &&&&&l_{N-2}&d_{N-2}&u_{N-2} \\
  &&&&&&l_{N-1}&d_{N-1}&u_{N-1} \\
  &&&&&&&0&1
  \end{matrix}\right]
$$
Þar sem stuðlarnir $l_j$, $d_j$ og $u_j$ eru gefnir með 
\begin{align*}
  l_j&=-1-\tfrac 12 hp_j\\
d_j&=2+h^2q_j\\
u_j&=-1+\tfrac 12 hp_j
\end{align*}
\end{frame}


\begin{frame}{8.1 Óþekktar stærðir og hægri hlið} 
$$
  \wv=\left[
  \begin{matrix}
w_0\\ w_1\\ w_2\\ \cdot\\ \cdot\\ \cdot\\ 
w_{N-2}\\ w_{N-1}\\ w_N  
\end{matrix}\right]
\qquad \text{ og } \qquad 
\bv=\left[
\begin{matrix}
\alpha \\ -h^2r_1\\ -h^2r_2\\ \cdot \\ \cdot\\ \cdot\\
-h^2r_{N-2}\\ -h^2r_{N-1}\\ \beta
\end{matrix}\right]
$$
Þetta jöfnuhneppi er leyst og þar með eru nálgunargildin fundin.
\end{frame}


\section*{8.2 Línulegar jöfnur -- Blönduð jaðarskilyrði}
\begin{frame}{8.2 Línulega jafna -- Blönduð jaðarskilyrði} 
Við skulum gera ráð fyrir að rétta lausnin $y(x)$ uppfylli blandað
jaðarskilyrði í $x=a$, 
$$
\alpha_1y(a)+\alpha_2 y'(a)=\alpha_3.
$$

\pause
Til þess að líkja eftir afleiðujöfnunni í punktinum $x=a$ þá hugsum
við okkur að við bætum einum punkti $x_{-1}=a-h$ við og látum 
$w_f$ tákna ímyndað gildi lausnarinnar í $x_{-1}$.  

\pause
\smallskip
Svona punktur $x_{-1}$ utan við skiptinguna er kallaður 
{\it felupunktur} við skiptinguna og ímyndað gildi $w_f$ í 
felupunkti er kallað  {\it felugildi}.   \pause

\smallskip
Takið eftir því að lausnin er ekki til í felupunktinum, en við reiknum
eins og $w_f$ sé gildi hennar þar. 

\pause
\smallskip
Mismunajafnan sem líkir eftir afleiðujöfnunni í punktinum $x_0$  er 
$$
\big(-1-\tfrac 12 hp_0\big)w_f+\big(2+h^2 q_0\big)w_0
+\big(-1+\tfrac 12 hp_0\big)w_1=-h^2r_0
$$

\pause
Mismunajafnan sem líkir eftir jaðarskilyrðinu er
$$
\alpha_1w_0+\alpha_2 \dfrac{w_1-w_f}{2h}=\alpha_3.
$$
\end{frame}


\begin{frame}{8.2 Felugildið leyst út} 
Jafnan sem líkir eftir jaðarskilyrðinu er:
$$
\alpha_1w_0+\alpha_2 \dfrac{w_1-w_f}{2h}=\alpha_3.
$$
Út úr henni leysum við
$$
w_f=w_1-\dfrac{2h}{\alpha_2}\big(\alpha_3-\alpha_1w_0\big)
$$

\pause
Við stingum síðan þessu gildi inn í jöfnuna sem líkir eftir
afleiðujöfnunni 
$$
\big(-1-\tfrac 12 hp_0\big)w_f+\big(2+h^2 q_0\big)w_0
+\big(-1+\tfrac 12 hp_0\big)w_1=-h^2r_0
$$

\pause
Útkoman verður:
\end{frame}


\begin{frame}{8.2 Jöfnur fyrir gildin í endapunktum} 
Fyrsta jafna hneppisins:
$$
\bigg(2+h^2q_0-\big(2+hp_0\big)h\dfrac{\alpha_1}{\alpha_2}\bigg)w_0
-2w_1=-h^2r_0-\big(2+hp_0\big)h\dfrac{\alpha_3}{\alpha_2}.
$$

\pause
\smallskip
Með því að innleiða felupunkt  $x_{N+1}=b+h$ hægra megin við 
skiptinguna, tilsvarandi
felugildi $w_f$ og leysa saman tvær jöfnur, þá fáum við síðustu jöfnu
hneppisins :
$$
-2w_{N-1}
+\bigg(2+h^2q_N+\big(2-hp_N\big)h\dfrac{\beta_1}{\beta_2}\bigg)w_N
=-h^2r_N-\big(2-hp_N\big)h\dfrac{\beta_3}{\beta_2}
$$

\pause
\smallskip
Við erum því aftur komin með $(N+1)\times (N+1)$-jöfnuhneppi
\end{frame}


\begin{frame}{8.2 Hneppið á fylkjaformi} 
$$
A\wv=\bv
$$
$$
A=\left[\begin{matrix}
a_{11}&a_{12}\\
l_1&d_1&u_1\\
&l_2&d_2&u_2\\
&&\cdot&\cdot&\cdot \\
&&&\cdot&\cdot&\cdot \\
&&&&\cdot&\cdot&\cdot \\
&&&&&l_{N-2}&d_{N-2}&u_{N-2} \\
&&&&&&l_{N-1}&d_{N-1}&u_{N-1} \\
&&&&&&&a_{N+1,N}&a_{N+1,N+1}
\end{matrix}\right]
$$
Þar sem stuðlarnir $l_j$, $d_j$ og $u_j$ fyrir 
$j=1,2,3\dots,N-1$ eru þeir sömu og áður.
\begin{align*}
  l_j&=-1-\tfrac 12 hp_j\\
d_j&=2+h^2q_j\\
u_j&=-1+\tfrac 12 hp_j
\end{align*}
\end{frame}


\begin{frame}{8.2 Fyrsta og síðasta lína hneppisins} 
  \begin{align*}
a_{11}&=
\begin{cases}
  1,&\text{Dirichlet í } x=a: \alpha_1\neq 0, \alpha_2=0,\\
d_0&\text{Neumann í } x=a:  \alpha_1=0, \alpha_2\neq 0,\\
d_0+2hl_0\alpha_1/\alpha_2&\text{Robin í } x=a:  \alpha_2\neq 0.
\end{cases} \\
a_{12}&=
\begin{cases}
  0,&\text{Dirichlet í } x=a: \alpha_1\neq 0, \alpha_2=0,\\
-2,&\text{annars}.
\end{cases}
\\ 
a_{N+1,N+1}&=
\begin{cases}
  1,&\text{Dirichlet í } x=b: \beta_1\neq 0, \beta_2=0,\\
d_N&\text{Neumann í } x=b:  \beta_1=0, \beta_2\neq 0,\\
d_N-2hu_N\beta_1/\beta_2&\text{Robin í } x=a:  \beta_2\neq 0.
\end{cases} 
\\
a_{N+1,N}&=
\begin{cases}
  0,&\text{Dirichlet í } x=b: \beta_1\neq 0, \beta_2=0,\\
-2&\text{annars}.
\end{cases} 
  \end{align*}
\end{frame}


\begin{frame}{8.2 Hægri hlið hneppisins} 
$$
\bv=\left[
\begin{matrix}
b_1 \\ -h^2r_1\\ -h^2r_2\\ \cdot \\ \cdot\\ \cdot\\
-h^2r_{N-2}\\ -h^2r_{N-1}\\ b_{N+1}
\end{matrix}\right]
$$

$$
b_{1}=
\begin{cases}
  \alpha=\alpha_3/\alpha_1,
&\text{Dirichlet í } x=a: \alpha_1\neq 0, \alpha_2=0,\\
-h^2r_0+2hl_0\alpha_3/\alpha_2
&\text{Neumann í } x=a:  \alpha_1=0, \alpha_2\neq 0,\\
-h^2r_0+2hl_0\alpha_3/\alpha_2&\text{Robin í } x=a:  \alpha_2\neq 0.
\end{cases} 
$$
$$
b_{N+1}=
\begin{cases}
  \beta=\beta_3/\beta_1,
&\text{Dirichlet í } x=a: \beta_1\neq 0, \beta_2=0,\\
-h^2r_N-2hu_N\beta_3/\beta_2&\text{Neumann í } x=a:  \beta_1=0, \beta_2\neq 0,\\
-h^2r_N-2hu_N\beta_3/\beta_2&\text{Robin í } x=a:  \beta_2\neq 0.
\end{cases} 
$$
\end{frame}


\begin{frame}{8.2 Samantekt} 
Gildi lausnarinnar $y(x)$ á  línulega jaðargildisverkefninu
  \begin{gather*}
    y''=p(x)y'+q(x)y+r(x), \qquad a\leq x\leq b,\\
\alpha_1y(a)+\alpha_2 y'(a)=\alpha_3,\\
\beta_1 y(b)+\beta_2y'(b)=\beta_3  
  \end{gather*}
í punktunum $x_j=a+jh$, þar sem $h=(b-a)/N$ og $j=0,\dots,N$,
eru nálguð með
$$
w_j\approx y(x_j)=y_j
$$ 

\pause
\smallskip
Dálkvigurinn
$$
\wv=[w_0,w_1,\dots,w_N]^T
$$
er lausn á línulegu jöfnuhneppi
$A\wv=\bv$.

\smallskip
Stuðlum $(N+1)\times(N+1)$ fylkisins $A$ og
$(N+1)$-dálkvigursins $\bv$ hefur verið lýst hér að framan.
\end{frame}


%
\frame{\frametitle{Kafli 8: Fræðilegar spurningar}
  \begin{enumerate}
  \item Hvað er átt við með því að lausn afleiðujöfnu á bili $[a,b]$
    uppfylli {\it Dirichlet-jaðarskilyrði}? (Samheiti er {\it fallsjaðarskilyrði}.)
  \item Hvað er átt við með því að lausn afleiðujöfnu á bili $[a,b]$
    uppfylli {\it Neumann-jaðarskilyrði}? (Samheiti eru {\it
      afleiðujaðarskilyrði}
og {\it flæðisjaðarskilyrði}.)
  \item Hvað er átt við með því að lausn afleiðujöfnu á bili $[a,b]$
    uppfylli {\it Robin-jaðarskilyrði}? (Samheiti er {\it blandað jaðarskilyrði}.)
  \item Hvernig er  nálgunarjafna fyrir 
línulegu afleiðujöfnuna $y''=p(x)y'+q(x)y+r(x)$ í innri skiptipunkti
á bilinu $[a,b]$ leidd út?
  \item Hvernig eru {\it felupunktur}  og {\it felugildi} notuð til
    þess að meðhöndla blandað jaðarskilyrði $\alpha_1y(a)+\alpha_2
    y'(a)=\alpha_3$ í vinstri endapunkti bilsins $[a,b]$? 
  \item Hvernig er {\it felupunktur}  og {\it felugildi} notuð til
    þess að meðhöndla blandað jaðarskilyrði $\alpha_1y(a)+\alpha_2
    y'(a)=\alpha_3$ í vinstri endapunkti bilsins $[a,b]$ og hvernig
    verður nálgunarjafnan í punktinum $x=0$ þegar þetta er gert? 
      \end{enumerate}
}

\end{document}
%
\frame{\frametitle{6.3 Ólínulegar jöfnur -- Fallsjaðargildi:}


Nú erum við búin að leysa línulega jaðargildisverkefnið og
snúum okkur því að almenna verkefninu:
  \begin{gather*}
    y''=f(x,y,y'), \qquad a\leq x\leq b,\\
\alpha_1y(a)+\alpha_2 y'(a)=\alpha_3,\\
\beta_1 y(b)+\beta_2y'(b)=\beta_3.  
  \end{gather*}

\pause
\smallskip
Þar sem 
$f:[a,b]\times R\times \R\to \R$
er fall af þremur breytistærðum.  


\pause
\smallskip
Forsendur okkar eru:
  \begin{enumerate}
  \item[(i)]  föllin $f$, $\partial f/\partial y$ og $\partial
    f/\partial y'$ eru samfelld.
  \item[(ii)]  $\partial f/\partial y>0$
  \item[(iii)] $\big|\partial f/\partial y'\big|\leq L$
  \end{enumerate}
}


%
\frame{\frametitle{6.3 Skipt á afleiðum og mismunakvótum:} 

Athugið að við notum táknin $y$ og $y'$ í tvennum tilgangi.\\  
Annars vegar eru þetta tákn fyrir breytur og \\
hins vegar fyrir réttu lausn verkefnisins $y=y(x)$ og \\
afleiðu hennar $y'=y'(x)$.


\pause
\smallskip
Að örðu leyti notum við nákvæmlega sama rithátt og við skilgreindum hér að
framan. 


\pause
\smallskip
Við lítum á afleiðujöfnuna í punktinum $x_j$, $j=1,\dots,N-1$.
$$
\big\{ y''(x)= f(x,y(x),y'(x))\big\}_{x=x_j}
$$
Við skiptum út afleiðum fyrir  mismunakvóta og höldum í skekkjuliðinn.  
Þá er þessi sama jafna 
$$
\dfrac{y_{j+1}-2y_j+y_{j-1}}{h^2} +O(h^2)
=f\big(x_j, y_j, \dfrac{y_{j+1}-y_{j-1}}{2h}+ O(h^2)\big).
$$
}


%
\frame{\frametitle{6.3 Nálgunarjafnan leidd út:} 

Endurtökum réttu jöfnuna
$$
\dfrac{y_{j+1}-2y_j+y_{j-1}}{h^2} +O(h^2)
=f\big(x_j,y_j,\dfrac{y_{j+1}-y_{j-1}}{2h}+ O(h^2)\big).
$$


\pause
\medskip
Til þess að stytta ritháttinn augnablik skulum við 
við setja $a=(y_{j+1}-y_{j-1})/(2h)$ og tákna leifarliðinn
í nálgun okkar á $y'(x_j)$ með $k$. 

\smallskip
Sem sagt
$y'(x_j)=a+k$.  Þá gefur meðalgildissetningin að til er tala
$\xi$ á milli $a$ og $a+k$, þannig að
$$
f(x_j,y_j,a+k)=f(x_j,y_j,a)+f'_{y'}(x_j,y_j,\xi)k
$$


\pause
Nú er $k=O(h^2)$ og hlutafleiðan í hægri hliðinni er 
takmörkuð samkvæmt forsendum okkar. Af því  leiðir að
rétta lausnin uppfyllir
$$
\dfrac{y_{j+1}-2y_j+y_{j-1}}{h^2} 
=f\big(x_j,y_j,\dfrac{y_{j+1}-y_{j-1}}{2h}\big)+ O(h^2).
$$

}

%
\frame{\frametitle{6.3 Nálgunarjafnan leidd út:} 

Nú innfærum við nálgunargildin með því að skoða jöfnuna sem 
rétta lausnin uppfyllir
$$
\dfrac{y_{j+1}-2y_j+y_{j-1}}{h^2} 
=f\big(x_j,y_j,\dfrac{y_{j+1}-y_{j-1}}{2h}\big)+ O(h^2).
$$
fellum niður leifarliðinn og 
setjum nálgunargildin $w_0,\dots,w_N$ í stað réttu 
gildanna $y_0,\dots,y_N$:
\pause
$$
\dfrac{w_{j+1}-2w_j+w_{j-1}}{h^2}
=f\big(x_j,w_j,\dfrac{w_{j+1}-w_{j-1}}{2h}\big)
$$

\pause
Hér fáum við eina jöfnu fyrir sérhvern innri skiptipunkt
$x_j$ með $j=1,\dots,N-1$.
}


%
\frame{\frametitle{6.3 Staðalframsetning nálgunarjöfnu:} 


Nálgunarjafnan er 
$$
\dfrac{w_{j+1}-2w_j+w_{j-1}}{h^2}
=f\big(x_j,w_j,\dfrac{w_{j+1}-w_{j-1}}{2h}\big)
$$
\pause
Ef við margföldum með $-h^2$ og færum alla liði vinstra megin
jafnaðarmerkisins, þá fáum við staðalform nálgunarjöfnunnar
$$
-w_{j-1}+2w_j-w_{j+1}+h^2
f\big(x_j,w_j,\dfrac{w_{j+1}-w_{j-1}}{2h}\big)=0
$$
\pause
Táknum nú vinstri hliðina í jöfnunni með
$g_j(\wv)$
þar sem $\wv$ er dálkvigurinn
$\wv=[w_0,\dots,w_N]^T$.



\pause
\smallskip
Við erum komin með $N-1$ nálgunarjöfnu á stöðluðu formi
$$
g_j(\wv)=0, \qquad j=1,2,\dots,N-1.
$$
til þess að finna $N+1$
nálgunargildi $w_0,\dots,w_N$ fyrir $y_0,\dots,y_N$. 


\pause
\smallskip
Jaðarskilyrðin gefa þær tvær jöfnur sem upp á vantar:
}

%
\frame{\frametitle{6.3 Jaðarskilyrði:} 



Almenna jaðarskilyrðið í vinstri endapunkti $x=a$ er
$$
\alpha_1y(a)+\alpha_2 y'(a)=\alpha_3.
$$
Ef $\alpha_2=0$ þá höfum við Dirichlet-jaðarskilyrði
$$
y_0=y(a)=\alpha_3/\alpha_1.
$$
Við skilgreinum í þessu tilfelli
$$
g_0(\wv)=w_0-\alpha_3/\alpha_1.
$$
Jaðarskilyrðið er uppfyllt þá og því aðeins að
$g_0(\wv)=0$.


\pause
\smallskip
Almenna jaðarskilyrðið í hægri endapunkti $x=b$ er
$$
\beta_1 y(b)+\beta_2y'(b)=\beta_3.  
$$
Ef það er Dirichlet-skilyrði, $\beta_2=0$, þá fáum við jöfnu númer 
$N+1$, 
$$
g_N(\wv)=w_N-\beta_3/\beta_1=0.
$$ 
}


%
\frame{\frametitle{6.3 Ólínulegt jöfnuhneppi:} 

Við höfum nú séð að nálgunargildin $\wv=[w_0,\dots,w_N]^T$ fyrir 
jaðargildisverkefnið með Dirichlet-jaðarskilyrði í báðum endapunktum 
jöfnuhneppi uppfylla ólínulega hneppið
$$
\Gv(\wv)=\ov
$$
þar sem $\Gv(\wv)=[g_0(\wv),\dots,g_N(\wv)]^T$.

\pause
\smallskip
Almennu jaðarskilyrðin eru meðhöndluð með nákvæmlega sama hætti 
og  hér að framan.  Nú förum við í gegnum það:

}

%
\frame{\frametitle{6.3 Meðhöndlun á blönduðu jaðarskilyrði í $x=a$:} 


Við skulum gera ráð fyrir að rétta lausnin $y(x)$  á almennu
ólínulegu jöfnuni 
$$
y''=f(x,y,y')
$$
uppfylli blandað jaðarskilyrði í $x=a$, 
$$
\alpha_1y(a)+\alpha_2 y'(a)=\alpha_3.
$$
með  $\alpha_2\neq 0$.

\pause
\smallskip
Til þess að líkja eftir afleiðujöfnunni í punktinum $x=a$ þá 
setjum við inn  {\it felupunktinn} $x_{-1}=a-h$ og látum 
$w_f$ vera tilsvarandi {\it felugildi í honum}.  

\pause
\smallskip
Mismunajafnan sem líkir eftir afleiðujöfnunni í punktinum $x_0$  er 
$$
-w_f+2w_0-w_1+h^2
f\big(x_0,w_0,\dfrac{w_1-w_f}{2h}\big)=0
$$
\pause
Mismunajafnan sem líkir eftir jaðarskilyrðinu er
$$
\alpha_1w_0+\alpha_2 \dfrac{w_1-w_f}{2h}=\alpha_3.
$$
}


%
\frame{\frametitle{6.3 Felugildið leyst út:} 


Jafnan sem líkir eftir jaðarskilyrðinu er:
$$
\alpha_1w_0+\alpha_2 \dfrac{w_1-w_f}{2h}=\alpha_3.
$$
\pause
Út úr henni leysum við
$$
w_f=w_1-2h\dfrac{\alpha_3}{\alpha_2}+
2h\dfrac{\alpha_1}{\alpha_2}w_0
\quad \text{ og } \quad
\dfrac{w_1-w_f}{2h}=
\dfrac{\alpha_3-\alpha_1w_0}{\alpha_2} 
$$


\pause
Við stingum síðan þessu gildi inn í jöfnuna sem líkir eftir
afleiðujöfnunni 
$$
-w_f+2w_0-w_1+h^2
f\big(x_0,w_0,\dfrac{w_1-w_f}{2h}\big)=0
$$
og útkoman verður:
$$
g_0(\wv)=2\big(1-h\dfrac{\alpha_1}{\alpha_2}\big)w_0-2w_1
+h^2
f\big(x_0,w_0,\dfrac{\alpha_3-\alpha_1w_0}{\alpha_2}\big)
+2h\dfrac{\alpha_3}{\alpha_2} =0
$$
}


%
\frame{\frametitle{6.3 Meðhöndlun á blönduðu jaðarskilyrði í $x=b$:} 


Eins er farið að ef rétta lausnin $y(x)$  á almennu
ólínulegu jöfnuni 
$$
y''=f(x,y,y')
$$
uppfyllir blandað jaðarskilyrði í $x=b$, 
$$
\beta_1y(a)+\beta_2 y'(a)=\beta_3.
$$
með  $\beta_2\neq 0$.

\smallskip
Við innleiðum þá felupunkt hægra megin við bilið í 
$x_{N+1}=b+h$ og tilsvarandi felugildi.  Með nákvæmlega sams konar
reikningi og við vorum að fara í gegnum fáum við  síðustu jöfnuna í
hneppið 
$$
g_N(\wv)=-2w_{N-1}+2\big(1+h\dfrac {\beta_1}{\beta_2}\big)w_N
+h^2f\big(x_N,w_N,\dfrac{\beta_3-\beta_1w_N}{\beta_2}\big)
-2h\dfrac{\beta_3}{\beta_2}
$$
}
%

%
\frame{\frametitle{Samantekt:} 

Gildi lausnarinnar $y(x)$ á  almenna  jaðargildisverkefninu
  \begin{gather*}
    y''=f(x,y,y'), \qquad a\leq x\leq b,\\
\alpha_1y(a)+\alpha_2 y'(a)=\alpha_3,\\
\beta_1 y(b)+\beta_2y'(b)=\beta_3  
  \end{gather*}
í punktunum $x_j=a+jh$, þar sem $h=(b-a)/N$ og $j=0,\dots,N$,
eru nálguð með
$$
w_j\approx y(x_j)=y_j
$$ 


\pause
\smallskip
Dálkvigurinn
$$
\wv=[w_0,w_1,\dots,w_N]^T
$$
er lausn á ólínulegu jöfnuhneppi
$$\Gv(\wv)=[g_0(\wv),\dots,g_N(\wv)]^T=\ov$$
þar sem 
$$
g_j(\wv)=-w_{j-1}+2w_j-w_{j+1}+h^2
f\big(x_j,w_j,\dfrac{w_{j+1}-w_{j-1}}{2h}\big)
$$
fyrir $j=1,\dots,N-1$.  
}

%
\frame{\frametitle{Jaðarskilyrðin í vinstri endapunkti $x=a$ 
ákvarða $g_0$:} 

Ef við höfum Dirichlet-jaðarskilyrði, 
$\alpha_2=0$ og setjum $\alpha=\alpha_3/\alpha_1$, þá fáum við
$$
g_0(\wv)=w_0-\alpha_3/\alpha_1=w_0-\alpha.
$$

\pause
Ef við höfum Robin-jaðarskilyrði, 
$\alpha_2\neq 0$, þá fáum við
$$
g_0(\wv)=2\big(1-h\dfrac{\alpha_1}{\alpha_2}\big)w_0-2w_1
+h^2
f\big(x_0,w_0,\dfrac{\alpha_3-\alpha_1w_0}{\alpha_2}\big)
+2h\dfrac{\alpha_3}{\alpha_2} =0
$$

\pause
Ef við höfum Neumann-jaðarskilyrði, sem er  sértilfellið 
$\alpha_1=0$ og setjum $\alpha=\alpha_3/\alpha_2$, þá verður fallið
$$
g_0(\wv)=2w_0-2w_1
+h^2
f\big(x_0,w_0,\alpha\big)+2h\alpha
$$
}
%
\frame{\frametitle{Jaðarskilyrðin í hægri endapunkti $x=b$ 
ákvarða $g_N$:} 

Ef við höfum Dirichlet-jaðarskilyrði, 
$\beta_2=0$ og setjum $\beta=\beta_3/\beta_1$, þá fáum við
$$
g_N(\wv)=w_N-\beta_3/\beta_1=w_N-\beta.
$$ 
\pause
Ef við höfum Robin-jaðarskilyrði, 
$\beta_2\neq 0$, þá fáum við
$$
g_N(\wv)=-2w_{N-1}+2\big(1+h\dfrac {\beta_1}{\beta_2}\big)w_N
+h^2f\big(x_N,w_N,\dfrac{\beta_3-\beta_1w_N}{\beta_2}\big)
-2h\dfrac{\beta_3}{\beta_2}.
$$

\pause
Ef við höfum Neumann-jaðarskilyrði, sem er  sértilfellið 
$\beta_1=0$ og setjum $\beta=\beta_3/\beta_2$, þá verður fallið
$$
g_N(\wv)=-2w_{N-1}+2w_N+h^2f\big(x_N,w_N,\beta\big)-2h\beta.
$$
}

%
\frame{\frametitle{6.4 Úrlausn á ólínulegum jöfnum:}

Við höfum séð að ólínulega jaðargildisverkefnið   
\begin{gather*}
    y''=f(x,y,y'), \qquad a\leq x\leq b,\\
\alpha_1y(a)+\alpha_2 y'(a)=\alpha_3,\\
\beta_1 y(b)+\beta_2y'(b)=\beta_3.  
  \end{gather*}
gefur af sér ólínulegt jöfnuhneppi  $\Gv(\wv)=\ov$ þar sem óþekktu 
stærðirnar $\wv=[w_0,\dots,w_N]^T$ eru nálgunargildi lausnarinnar $y(x)$
í punktunum $x_0,\dots,x_N$.  
Lausn ólínulega jöfnuhneppisins er nú nálguð með aðferð Newtons.



\smallskip
Byrjað er að giska á lausn $\wv^{(0)}$ og síðan er nálgunarrunan
$\wv^{(n)}$ reiknuð út með ítrekun á venjulegan máta
$$
\wv^{(n+1)}=\wv^{(n)} +\Delta \wv^{(n)} 
$$
þar sem fylgt er reikniritinu fyrir aðferð Newtons og 
$\Delta \wv^{(n)}$ er leyst úr úr jöfnuheppinu
$$
J_{\Gv}(\wv^{(n)}) \Delta \wv^{(n)}= -\Gv(\wv^{(n)}).
$$  
}

%
\frame{\frametitle{6.4 Jacobi-fylki vörpunarinnar $\Gv$:}

Jacobi-fylkið  $J_{\Gv}(\wv)$ er af stærðinni $(N+1)\times (N+1)$,
$$
J_{\Gv}(\wv)=\big[ J_{m,k}(\wv)\big]_{m,k=1}^{N+1}
$$ 
Línurnar $m=2,\dots,N$ eru óháðar jaðarskilyrðunum, því
í þeim er $m=j+1$ og tilsvarandi hnitafall er 
$$
g_j(\wv)=-w_{j-1}+2w_j-w_{j+1}+h^2f \big(x_j,w_j,\dfrac{w_{j+1}-w_{j-1}}{2h}\big)
$$
Þetta segir okkur að það eru aðeins þrjú stök í línu $m$ sem eru
frábrugðin $0$.

}

%
\frame{\frametitle{6.3 Útreikningur á Jacobi-fylki:}

Við sáum að hnitafall númer $m=j+1$ í $\Gv$ er
$$
g_j(\wv)=-w_{j-1}+2w_j-w_{j+1}+h^2f \big(x_j,w_j,\dfrac{w_{j+1}-w_{j-1}}{2h}\big)
$$
Þetta segir að öll stökin í línu $m$ í $J_{\Gv}(\wv)$ eru $0$ nema
þrjú 
\begin{align*}
  J_{m,m-1}(\wv)&=\dfrac{\partial g_j}{\partial w_{j-1}}(\wv)
=-1-\tfrac 12 h\dfrac{\partial f}{\partial
  y'}\big(x_j,w_j,\dfrac{w_{j+1}-w_{j-1}}{2h}\big),\\ 
  J_{m,m}(\wv)&=\dfrac{\partial g_j}{w_{j}}(\wv)
=2+ h^2\dfrac{\partial f}{\partial
  y}\big(x_j,w_j,\dfrac{w_{j+1}-w_{j-1}}{2h}\big),\\ 
  J_{m,m+1}(\wv)&=\dfrac{\partial g_j}{\partial w_{j+1}}(\wv)
=-1+\tfrac 12 h\dfrac{\partial f}{\partial
  y'}\big(x_j,w_j,\dfrac{w_{j+1}-w_{j-1}}{2h}\big).
\end{align*}
}

%
\frame{\frametitle{Fyrsta línan í Jacobi-fylkinu  --  Dirichlet-jaðarskilyrði:}

Ef við höfum Dirichlet-jaðarskilyrði í $x=a$, þá er tilsvarandi
hnitafall 
$$
g_0(\wv)= w_0-\alpha.
$$
Þetta segir okkur að öll stökin í fyrstu línu $J_{\Gv}$ eru $0$ nema
eitt
$$
J_{1,1}(\wv)=\dfrac{\partial g_0}{\partial w_0}=1.
$$

}

%
\frame{\frametitle{Fyrsta línan í Jacobi-fylkinu  --  Robin-jaðarskilyrði:}

Ef við höfum Robin-jaðarskilyrði í $x=a$, 
$$
\alpha_1y(a)+\alpha_2 y'(a)=\alpha_3,
$$
með $\alpha_2\neq 0$, þá er tilsvarandi
hnitafall 
$$
g_0(\wv)=2\big(1-h\dfrac{\alpha_1}{\alpha_2}\big)w_0-2w_1
+h^2
f\big(x_0,w_0,\dfrac{\alpha_3-\alpha_1w_0}{\alpha_2}\big)
+2h\dfrac{\alpha_3}{\alpha_2}
$$

Þetta segir okkur að öll stökin í fyrstu línu $J_{\Gv}$ eru $0$ nema
tvö
\begin{align*}
J_{1,1}(\wv)&=\dfrac{\partial g_0}{\partial w_0}
=2\bigg(1-h\dfrac{\alpha_1}{\alpha_2}\bigg)
+h^2\dfrac{\partial f}{\partial y} \big(x_0,w_0,\dfrac{\alpha_3-\alpha_1
  w_0}{\alpha_2} \big)\\
& -h^2\dfrac {\alpha_1}{\alpha_2} \dfrac{\partial
  f}{\partial y'} \big( x_0,w_0,\dfrac{\alpha_3-\alpha_1
  w_0}{\alpha_2} \big)
\\
J_{1,2}(\wv)&=\dfrac{\partial g_0}{\partial w_1}=-2  
\end{align*}

}

%
\frame{\frametitle{6.4 Síðasta línan -- Dirichlet-jaðarskilyrði:}


Ef við höfum Dirichlet-jaðarskilyrði í $x=b$, þá er tilsvarandi
hnitafall 
$$
g_N(\wv)= w_N-\beta.
$$
Þetta segir okkur að öll stökin í $N+1$ línu fylkisins $J_{\Gv}$ eru $0$ nema
eitt
$$
J_{N+1,N+1}(\wv)=\dfrac{\partial g_N}{\partial w_N}=1.
$$
}

%
\frame{\frametitle{6.4 Síðasta línan -- Robin-jaðarskilyrði:}

Ef við höfum Robin-jaðarskilyrði, 
$\beta_2\neq 0$, þá fáum við að $N+1$ hnitafallið í $\Gv$ er
$$
g_N(\wv)=-2w_{N-1}+2\big(1+h\dfrac {\beta_1}{\beta_2}\big)w_N
+h^2f\big(x_N,w_N,\dfrac{\beta_3-\beta_1w_N}{\beta_2}\big)
-2h\dfrac{\beta_3}{\beta_2}.
$$
Af því leiðir að öll stökin í $N+1$ línu fylkisins $J_{\Gv}$ eru $0$ nema
tvö
\begin{align*}
J_{N+1,N}(\wv)&=\dfrac{\partial g_{N}}{\partial w_{N-1}}=-2 \\ 
J_{N+1,N+1}(\wv)&=\dfrac{\partial g_N}{\partial w_N}
=2\bigg(1+h\dfrac{\beta_1}{\beta_2}\bigg)
+h^2\dfrac{\partial f}{\partial y} \big(x_N,w_N,\dfrac{\beta_3-\beta_1
  w_N}{\beta_2} \big)\\
& -h^2\dfrac {\beta_1}{\beta_2} \dfrac{\partial
  f}{\partial y'} \big( x_N,w_N,\dfrac{\beta_3-\beta_1
  w_N}{\beta_2} \big)
\end{align*}
}

%
\frame{\frametitle{6.4  $J_{\Gv}$ er þríhornalínufylki:}

Út úr þessum reikningum sést að fylkið $J_{\Gv}$ er af sömu gerð og
fylkið sem fæst úr úr línulega verkefninu
$$
J_{\Gv}(\wv)=\left[\begin{matrix}
J_{11}&J_{12}\\
l_1&d_1&u_1\\
&l_2&d_2&u_2\\
&&\cdot&\cdot&\cdot \\
&&&\cdot&\cdot&\cdot \\
&&&&\cdot&\cdot&\cdot \\
&&&&&l_{N-2}&d_{N-2}&u_{N-2} \\
&&&&&&l_{N-1}&d_{N-1}&u_{N-1} \\
&&&&&&& J_{N+1,N}&J_{N+1,N+1}
\end{matrix}\right]
$$
Við höfum  hér að framan skrifað upp  formúlur fyrir öllum stökum þess.
Eins og áður hefur komið fram nefnast fylgi af þessari gerð 
{\it þríhornalínufylki}.  Það er tiltölulega auðvelt að
leysa jöfnuhneppið 
$$
J_{\Gv}(\wv^{(n)})\Delta\wv^{(n)}=-\Gv(\wv^{(n)})
$$
með  sérsniðnu reikniriti fyrir Gauss-eyðingu.
}



%
\frame{\frametitle{Kafli 6: Fræðilegar spurningar:}

  \begin{enumerate}
  \item Hvað er átt við með því að lausn afleiðujöfnu á bili $[a,b]$
    uppfylli {\it Dirichlet-jaðarskilyrði}? (Samheiti er {\it fallsjaðarskilyrði}.)
  \item Hvað er átt við með því að lausn afleiðujöfnu á bili $[a,b]$
    uppfylli {\it Neumann-jaðarskilyrði}? (Samheiti eru {\it
      afleiðujaðarskilyrði}
og {\it flæðisjaðarskilyrði}.)
  \item Hvað er átt við með því að lausn afleiðujöfnu á bili $[a,b]$
    uppfylli {\it Robin-jaðarskilyrði}? (Samheiti er {\it blandað jaðarskilyrði}.)
  \item Hvernig er  nálgunarjafna fyrir 
línulegu afleiðujöfnuna $y''=p(x)y'+q(x)y+r(x)$ í innri skiptipunkti
á bilinu $[a,b]$ leidd út?
  \item Hvernig eru {\it felupunktur}  og {\it felugildi} notuð til
    þess að meðhöndla blandað jaðarskilyrði $\alpha_1y(a)+\alpha_2
    y'(a)=\alpha_3$ í vinstri endapunkti bilsins $[a,b]$? 
  \item Hvernig er  nálgunarjafna fyrir 
ólínulegu afleiðujöfnuna $y''=f(x,y,y')$ í innri skiptipunkti
á bilinu $[a,b]$ leidd út?
  \end{enumerate}
}

%
\frame{\frametitle{Kafli 6: Fræðilegar spurningar:}


  \begin{enumerate}

  \item[7.] Hvernig er {\it felupunktur}  og {\it felugildi} notuð til
    þess að meðhöndla blandað jaðarskilyrði $\alpha_1y(a)+\alpha_2
    y'(a)=\alpha_3$ í vinstri endapunkti bilsins $[a,b]$ og hvernig
    verður nálgunarjafnan í punktinum $x=0$ þegar þetta er gert? 
  \item [8.]  Ólínulega jaðargildisverkefnið á bilinu $[a,b]$ með
    ólínulegri afleiðujöfnu  $y''=f(x,y,y')$ gefur af sér 
 ólínulega jöfnu fyrir nálgunargildin $\Gv(\wv)=\ov$.  Lýsið því
 hvernig lausn þessarar jöfnu er nálguð.
    \end{enumerate}
}
\end{document}