\def \kaflanr {15}
\lecture[\kaflanr]{\kaflanr. Þreföld heildi}{lecture-text}
\date{23.~febrúar 2015}
\newcounter{mycount}
\refstepcounter{mycount}

\begin{document}

\subsection{}
	\maketitle





\subsection{} 

\subsubsection{Umræða \kaflanr.\arabic{mycount}}\stepcounter{mycount}
Heildi falls $f(x,y,z)$ yfir kassa $K=[a,b]\times[c,d]\times[u,v]$ í $\R^3$ er skilgreint á sambærilegan hátt og  tvöfalt heildi er skilgreint.  

\medskip
Á sama hátt og fyrir tvöföld heildi má svo skilgreina heildi fyrir almennari rúmskika í $\R^3$. 

\medskip
Heildi falls $f(x,y,z)$ yfir rúmskika $R$ er táknað með 
$$\thrint_R f(x,y,z)\,dV.$$
($dV$ stendur fyrir að heildað er með tilliti til rúmmáls.)




\subsection{} 

\subsubsection{Setning \kaflanr.\arabic{mycount}}\stepcounter{mycount}
 Látum $f(x,y,z)$ vera fall sem er heildanlegt yfir kassa $K=[a,b]\times[c,d]\times[u,v]$ í $\R^3$.  Þá er
$$\thrint_K f(x,y,z)\,dV=
\int_a^b\!\int_c^d\!\int_u^v f(x,y,z)\,dz\,dy\,dx.$$ 
Breyta má röð heilda að vild, t.d.\ er 
$$\thrint_K f(x,y,z)\,dV=
\int_u^v\!\int_c^d\!\int_a^b f(x,y,z)\,dx\,dy\,dz.$$ 




\subsection{} 

\subsubsection{Setning \kaflanr.\arabic{mycount}}\stepcounter{mycount}
 Látum $f(x,y,z)$ vera fall sem er heildanlegt yfir rúmskika $R$ og gerum ráð fyrir að $R$ hafi lýsingu á forminu
$$R=\{(x,y,z)\mid a\leq x\leq b,\ c(x)\leq y\leq d(x),\ u(x,y)\leq z\leq v(x,y)\}.$$
Þá er
$$\thrint_R f(x,y,z)\,dV=
\int_a^b\!\int_{c(x)}^{d(x)}\!\int_{u(x,y)}^{v(x,y)} f(x,y,z)\,dz\,dy\,dx.$$ 
Breyturnar $x, y, z$ geta svo skipt um hlutverk.




\subsection{} 

\subsubsection{Setning \kaflanr.\arabic{mycount} (Almenn breytuskiptaformúla fyrir þreföld heildi.) }\stepcounter{mycount}
 Látum 
$$(u,v,w)\mapsto (x(u,v,w), y(u,v,w), z(u,v,w))$$
vera gagntæka vörpun milli rúmskika $R$ í $xyz$-rúmi og rúmskika $S$ í $uvw$-rúmi.  Gerum ráð fyrir að föllin $x(u,v,w), y(u,v,w), z(u,v,w)$ hafi öll samfelldar fyrsta stigs hlutafleiður.  Ef $f(x,y,z)$ er fall sem er heildanlegt yfir $R$ þá er
\begin {align*}
\thrint_R& f(x,y,z)\,dV \\&=\thrint_S f(x(u,v,w), y(u,v,w), z(u,v,w))
\bigg|\frac{\partial(x,y,z)}{\partial(u,v,w)}\bigg|\,du\,dv\,dw.
\end {align*}




\subsection{} 

\subsubsection{Skilgreining \kaflanr.\arabic{mycount}}\stepcounter{mycount}
 Látum $(x,y,z)$ vera punkt í $\R^3$.  {\em Sívalningshnit} $(x,y,z)$ eru þrennd talna $r, \theta, z$ þannig að 
$$x=r\cos\theta\qquad\qquad y=r\sin\theta\qquad\qquad z=z.$$
Athugið að $[r,\theta]$ eru pólhnit punktsins $(x,y)$. 




\subsection{} 

\subsubsection{Setning \kaflanr.\arabic{mycount} (Breytuskipti yfir í sívalningshnit.)}\stepcounter{mycount}

Látum $R$ vera rúmskika í $\R^3$ og látum $f(x,y,z)$ vera heildanlegt fall yfir $R$.  Gerum ráð fyrir að $R$ megi lýsa með eftirfarandi skorðum á sívalningshnit punktanna sem eru í $R$
$$\alpha\leq \theta\leq \beta,\ a(\theta)\leq r\leq  b(\theta), u(r,\theta)\leq z\leq v(r,\theta),$$ 
þar sem $0\leq \beta-\alpha\leq 2\pi$.  Þá er
$$\thrint_R f(x,y,z)\,dV= 
\int_\alpha^\beta
\!\int_{a(\theta)}^{b(\theta)}\int_{u(r,\theta)}^{v(r,\theta)}      
f(r\cos\theta,r\sin\theta,z)r\,dz\,dr\,d\theta.$$
 




\end{document}