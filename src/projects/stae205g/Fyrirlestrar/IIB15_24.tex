\def \kaflanr {24}
\lecture[\kaflanr]{\kaflanr. Sundurleitnisetningin I}{lecture-text}
\date{25.~mars 2015}
\newcounter{mycount}
\refstepcounter{mycount}

\begin{document}

\begin{frame}
	\maketitle
\end{frame}




\begin{frame}{} 

\begin {block}{Setning \rtask{} (Sundurleitnisetning I)}\stepcounter{mycount}
 Látum $\Fv$ vera samfellt
diffranlegt vigursvið skilgreint á opnu mengi $D$ í $\R^3$.    Látum
$P$ vera punkt á skilgreiningarsvæði $\Fv$ og ${\cal S}_\epsilon$
kúluskel með miðju í $P$ og geisla $\epsilon$.  Látum svo $\Nv$ vera
einingarþvervigrasvið á ${\cal S}_\epsilon$ þannig að $\Nv$ vísar út á
við.  Þá er 
$$\dive \Fv(P)=\lim_{\epsilon\rightarrow 0^+}
\frac{1}{V_\epsilon}\tvint_{{\cal S}_\epsilon}\Fv\cdot\Nv\,dS.$$
þar sem $V_\epsilon = 4\pi\epsilon^3/3$ er rúmmálið innan í ${\cal S}_\epsilon$.
\end{block}

\end{frame}




\begin{frame}{} 

\begin {block}{Setning  \rtask{} (Setning Stokes I)}\stepcounter{mycount}
  Látum $\Fv$ vera samfellt
diffranlegt vigursvið skilgreint á opnu mengi $D$ í $\R^3$.    Látum
$P$ vera punkt á skilgreiningarsvæði $\Fv$ og $C_\epsilon$ vera
hring með miðju í $P$ og geisla $\epsilon$.  Látum $\Nv$ vera
einingarþvervigur á planið sem hringurinn liggur í.  Áttum hringinn
jákvætt.
Þá er
$$\Nv\cdot\curl \Fv(P)=\lim_{\epsilon\rightarrow 0^+}
\frac{1}{A_\epsilon}\oint_{C_\epsilon}\Fv\cdot d\rv.$$
 þar sem $A_\epsilon = \pi\epsilon^2$ er flatarmálið sem afmarkast af  ${\cal C}_\epsilon$.

\end{block}

\end{frame}



\begin{frame}{} 

\begin {block}{Túlkun \rtask{}}\stepcounter{mycount}
 Hugsum $\Fv$ sem lýsingu á vökvastreymi í $\R^3$.

$\dive \Fv(P)$ lýsir því hvort vökvinn er að þenjast út eða dragast
saman í punktinum $P$.  Sundurleitnisetningin (næsti fyrirlestur)
segir að samanlögð útþensla á rúmskika $R$ er jöfn streymi út um jaðar svæðisins $\mathcal{S}$,
eða 
$$\thrint_R\dive\Fv\,dV=\tvint_{\mathcal{S}} \Fv\cdot\Nv\,dS.$$

$\curl \Fv(P)$ lýsir hringstreymi í kringum punktinn $P$.  Setning
Stokes (þar næsti fyrirlestur) segir að samanlagt hringstreymi á fleti $\mathcal{S}$
er jafnt hringstreymi á jaðri flatarins, sem við táknum með $\mathcal{C}$, eða
$$\tvint_{\cal S} \curl\Fv\cdot\Nv\,dS=\oint_\mathcal{C} \Fv\cdot d\rv.$$
\end{block}

\end{frame}



\begin{frame}{} 

\begin {block}{Skilgreining \rtask{}}\stepcounter{mycount}
 Látum $R$ vera svæði í $\R^2$ og $\cal C$
jaðar $R$.  Gerum ráð fyrir að $\cal C$ samanstandi af endanlega
mörgum ferlum ${\cal C}_1, \ldots, {\cal C}_n$.  Jákvæð áttun á
ferlunum felst í því að velja fyrir hvert $i$ stikun $\rv_i$ á ${\cal
  C}_i$ þannig að ef labbað eftir ${\cal C}_i$ í stefnu stikunar þá er
$R$ á vinstri hönd.
\end{block}

\end{frame}



\begin{frame}{} 

\begin {block}{Setning Green \rtask{}}\stepcounter{mycount}
  Látum $R$ vera svæði í planinu þannig að
jaðar $R$, táknaður með $\cal C$,  
samanstendur af endanlega mörgum samfellt diffranlegum
ferlum.  Áttum $\cal C$ jákvætt.  Látum
$\Fv(x,y)=F_1(x,y)\,\iv+F_2(x,y)\,\jv$ vera samfellt diffranlegt
vigursvið skilgreint á $R$.  Þá er 
$$\oint_{\cal C}F_1(x,y)\,dx+F_2(x,y)\,dy=\tvint_R
\frac{\partial  F_2}{\partial x}- 
\frac{\partial  F_1}{\partial y}\,dA.$$
\end{block}

\end{frame}



\begin{frame}{} 

\begin {block}{Fylgisetning \rtask{}}\stepcounter{mycount}
 Látum $R$ vera svæði í planinu þannig að
jaðar $R$ táknaður með $\cal C$, 
samanstendur af endanlega mörgum samfellt diffranlegum
ferlum.  Áttum $\cal C$ jákvætt. 
Þá er 
$$\mbox{Flatarmál } R=\oint_{\cal C}x\,dy= 
-\oint_{\cal C}y\,dx=\frac{1}{2}\oint_{\cal C}x\,dy-y\,dx.$$
\end{block}

\end{frame}



\begin{frame}{} 

\begin {block}{Sundurleitnisetningin í tveimur víddum \rtask{}}\stepcounter{mycount}

Látum $R$ vera svæði í planinu þannig að
jaðar $R$, táknaður með $\cal C$,  
samanstendur af endanlega mörgum samfellt diffranlegum
ferlum.  Látum $\Nv$ tákna einingarþvervigrasvið á $\cal C$ þannig að
$\Nv$ vísar út úr $R$.  Látum
$\Fv(x,y)=F_1(x,y)\,\iv+F_2(x,y)\,\jv$ vera samfellt diffranlegt
vigursvið skilgreint á $R$.  Þá er 
$$\tvint_R\dive \Fv\,dA=\oint_{\cal C} \Fv\cdot\Nv\,ds.$$
\end{block}

\end{frame}
\end{document}