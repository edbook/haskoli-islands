\lecture[1]{1. Ferlar}{lecture-text}
\date{5.~janúar 2015}
\newcounter{mycount}
\stepcounter{mycount}

\begin{document}

\begin{frame}{}
	\maketitle
\end{frame}

\begin{frame}{Inngangur}
\begin {itemize}
 \item Viðfangsefni námskeiðsins er varpanir sem skilgreindar eru á
hlutmengi í $\Rn$ og taka gildi í $\Rm$. 
\item Fáumst við stærðfræðigreiningu í mörgum breytistærðum.
\item Sambærileg verkefni og í stærðfræðigreiningu í einni breytistærð: Samfelldni, diffrun, heildun. Rúmfræðileg túlkun skiptir nú miklu máli.
\item 
Gerir okkur kleift að fást við mörg raunveruleg verkefni þar sem margar breytistærðir koma við sögu.
\end {itemize}

\end{frame}


\begin{frame}{Stikaferlar} 

\begin {block}{Skilgreining 1.\arabic{mycount}}\stepcounter{mycount}
Vörpun $\rv:  [a,b]\rightarrow \Rn$ þannig að $\rv(t)=(r_1(t),\ldots,r_n(t))$ kallast {\em vigurgild vörpun}.  Slík vörpun er sögð samfelld ef föllin $r_1, \ldots, r_n$ eru öll samfelld.  Samfelld vörpun $\rv:  [a,b]\rightarrow \Rn$ er oft kölluð {\em stikaferill}.
\end{block}

\end{frame}

\begin{frame}{Stikaferlar}
\begin{block}{Ritháttur 1.\arabic{mycount}}\stepcounter{mycount}
 Þegar fjallað er um stikaferil $\rv:  [a,b]\rightarrow \R^2$ þá er oft ritað
$$\rv=\rv(t)=(x(t),y(t))=x(t)\iv+y(t)\jv,$$
og þegar fjallað er um stikaferil $\rv:  [a,b]\rightarrow \R^3$ þá er oft ritað
$$\rv=\rv(t)=(x(t),y(t),z(t))=x(t)\iv+y(t)\jv+z(t)\kv.$$
\end{block}
\end{frame}

\begin {frame}{Ferlar og stikanir á ferlum}
\begin {block}{Skilgreining 1.\arabic{mycount}}\stepcounter{mycount}
{\em Ferill í plani} er mengi punkta $(x,y)$ í planinu þannig að skrifa má $x=f(t)$ og $y=g(t)$ fyrir $t$ á bili $I$ þar sem $f$ og $g$ eru samfelld föll á $I$. Bilið $I$ ásamt föllunum $(f,g)$ kallast {\emph stikun} á ferlinum. Ferill í rúmi og stikun á ferli í rúmi eru skilgreind á sambærilegan hátt.
\end{block}
\end {frame}

\begin {frame} {Ferlar og stikanir á ferlum}
Ferill í plani/rúmi er \textbf{ekki} það sama og stikaferill. Fyrir gefinn feril eru til (óendanlega) margar ólíkar stikanir.
\pause
\begin{block}{Dæmi 1.\arabic{mycount} - Eðlisfræðileg túlkun}\stepcounter{mycount}
Líta má á veginn milli Reykjavíkur og Akureyrar sem feril.

Líta má á ferðalag eftir veginum frá Reykjavík til Akureyrar þar sem staðsetning er þekkt á hverjum tíma sem stikaferil þar sem tíminn er stikinn.
\end {block}
\pause
\begin{block}{Dæmi 1.\arabic{mycount}}\stepcounter{mycount}
Jafnan
$$x^2+y^2 = 1$$
lýsir ferli í planinu sem er hringur með miðju í (0,0) og geisla 1. Dæmi um ólíkar stikanir:
\begin{align*}
\rv_1(t) &= (\cos(t),\sin(t)), \quad \text{fyrir $t$ á bilinu $[0,2\pi].$} \\
\rv_2(t) &= \left\{\begin{array}{ll}
(t,\sqrt{1-t^2}) & \text{fyrir $t$ á bilinu $[-1,1[,$} \\
(2-t,-\sqrt{1-(2-t)^2}) & \text{fyrir $t$ á bilinu $[1,3].$} 
\end{array}\right.
\end{align*}

\end {block}

\end {frame}


\begin {frame}{Diffrun stikaferla}
 \begin {block}{Skilgreining 1.\arabic{mycount}}\stepcounter{mycount}
Stikaferill $\rv:  [a,b]\rightarrow \Rn$ er
{\em diffranlegur í punkti} $t$ ef markgildið 
$$\rv'(t)=\lim_{\Delta t\rightarrow 0}\frac{\rv(t+\Delta t)-\rv(t)}{\Delta t}$$
er til.  Stikaferillinn $\rv$ er sagður {\em diffranlegur} ef hann er
diffranlegur í öllum punktum á bilinu $[a,b]$.  (Í endapunktum bilsins
$[a,b]$ er þess krafist að einhliða afleiður séu skilgreindar.)
\end {block}
\end {frame}

\begin{frame}{Diffrun stikaferla}
\begin{block}{\nopagebreak Setning 1.\arabic{mycount}}\stepcounter{mycount}  
   Stikaferill $\rv:  [a,b]\rightarrow \Rn$ er
{\em diffranlegur í punkti} $t$ ef og aðeins ef föllin $r_1,\ldots,r_n$ eru
öll diffranleg í $t$.  Þá gildir að  
$$\rv'(t)=(r'_1(t),\ldots,r'_n(t)).$$ 
 \end{block}
 \pause
\begin{block}{\nopagebreak Ritháttur 1.\arabic{mycount}}\stepcounter{mycount}   Látum $\rv:  [a,b]\rightarrow \Rn$ vera diffranlegan stikaferil.  Venja er að rita $\vv(t)=\rv'(t)$ og tala um
$\vv(t)$ sem {\em hraða} eða {\em hraðavigur}.   Talan $|\vv(t)|$ er
kölluð {\em ferð}.   Einnig er ritað $\av(t)=\vv'(t)=\rv''(t)$ og talað
um $\av(t)$ sem {\em hröðun} eða {\em hröðunarvigur}.  
\end{block}

\end{frame}

\begin {frame}{Diffrun stikaferla}
 \begin{block}{\nopagebreak Dæmi 1.\arabic{mycount}}\stepcounter{mycount}
Lítum á eftirfarand stikaferla sem stika hring með miðju í (0,0) og geisla 1.
 \begin{align*}
\rv_1(t) &= (\cos(t),\sin(t)), \quad \text{fyrir $t$ á bilinu $[0,2\pi].$} \\
\rv_2(t) &= (\cos(t^2),\sin(t^2)), \quad \text{fyrir $t$ á bilinu $[0,\sqrt{2\pi}].$} 
\end {align*}
Þá er tilsvarandi hraði
 \begin{align*}
\vv_1(t) = \rv_1'(t) &= (-\sin(t),\cos(t)), \quad \text{fyrir $t$ á bilinu $[0,2\pi].$} \\
\vv_2(t) = \rv_2'(t) &= (-2t\sin(t^2),2t\cos(t^2)),  \quad \text{fyrir $t$ á bilinu $[0,\sqrt{2\pi}].$}
\end {align*}
og ferðin $|\vv_1(t)| = 1$ og $|\vv_2(t)| = 2t$.
 \end {block}
\end {frame}


\begin{frame}{Diffrun stikaferla}
\begin{block}{\nopagebreak Setning 1.\arabic{mycount}}\stepcounter{mycount}   Látum $\uv,\vv:[a,b]\rightarrow \Rn$ vera
diffranlega stikaferla og $\lambda$ diffranlegt fall.  Þá eru stikaferlarnir
$\uv(t)+\vv(t), \lambda(t)\uv(t)$ og $\uv(\lambda(t))$ diffranlegir,
og ef $n=3$ þá er stikaferillinn $\uv(t)\times \vv(t)$ líka diffranlegur.
Fallið $\uv(t)\cdot\vv(t)$ er líka diffranlegt.  Eftirfarandi listi sýnir
formúlur fyrir afleiðunum: 

{\bf (a)} $\frac{d}{dt}(\uv(t)+\vv(t))=\uv'(t)+\vv'(t)$,

{\bf (b)} $\frac{d}{dt}(\lambda(t)\uv(t))=\lambda'(t)\uv(t)+\lambda(t)\uv'(t)$,

{\bf (c)}  $\frac{d}{dt}(\uv(t)\cdot\vv(t))=\uv'(t)\cdot\vv(t)+\uv(t)\cdot\vv'(t)$,

{\bf (d)}  $\frac{d}{dt}(\uv(t)\times\vv(t))=\uv'(t)\times\vv(t)+\uv(t)\times\vv'(t)$,

{\bf (e)}  $\frac{d}{dt}(\uv(\lambda(t)))=\uv'(\lambda(t))\lambda'(t)$.

\noindent
Ef $\uv(t)\neq\ov$ þá er 

{\bf (f)}  $\frac{d}{dt}|\uv(t)|=\frac{\uv(t)\cdot\uv'(t)}{|\uv(t)|}$.
\end {block}
\medskip

 
\end{frame}

\begin {frame}{Diffrun stikaferla}

\begin {block}{Skilgreining 1.\arabic{mycount}}\stepcounter{mycount}
Látum $\rv:  [a,b]\rightarrow \Rn; \rv(t)=(r_1(t),\ldots,r_n(t))$ vera stikaferil.  

Stikaferillinn er sagður {\em samfellt diffranlegur} ef föllin
$r_1(t),\ldots,r_n(t)$ eru öll diffranleg og afleiður þeirra eru
samfelldar.  Samfellt diffranlegur stikaferill er sagður {\em þjáll}
(e.~smooth) ef $\rv'(t)\neq\ov$ fyrir öll $t$. 

\medskip
Stikaferillinn er sagður {\em samfellt diffranlegur á köflum} ef til eru
tölur $b_0,\ldots,b_k$ þannig að  $a=b_0<b_1<\cdots<b_k=b$ og
stikaferillinn er samfellt diffranlegur á hverju bili $[b_{i-1}, b_i]$.
Það að stikaferill sé {\em þjáll á köflum}  
(e.~piecewise smooth curve) er
skilgreint á sambærilegan hátt. 
\end{block}
\end {frame}

\begin {frame}{Diffrun stikaferla}
 \begin {block}{Setning 1.\arabic{mycount}}\stepcounter{mycount}
Látum $\rv=f(t)\iv+g(t)\jv$ vera samfellt diffranlegan stikaferil fyrir $t$ á bili $I$. Ef $f'(t) \neq 0$ á $I$ þá hefur ferilinn snertilínu fyrir hvert gildi á $t$ og hallatala hennar er 
\begin {equation*}
 \frac{dy}{dx} = \frac{g'(t)}{f'(t)}.
\end {equation*}
Ef $g'(t) \neq 0$ á $I$ þá hefur ferilinn snertilínu fyrir hvert gildi á $t$ og hallatala hennar er
\begin {equation*}
 -\frac{dx}{dy} = -\frac{f'(t)}{g'(t)}.
\end {equation*}
\end {block}
% \begin {block}{Regla 1.\arabic{mycount}}\stepcounter{mycount}
% Ef $f'$ og $g'$ eru ekki bæði núll í $t_0$ þá má stika snertilínuna með
% \begin {align*}
%  x &= f(t_0)+f'(t_0)(t-t_0) \\
%  y &= g(t_0)+g'(t_0)(t-t_0) 
% \end {align*}
% fyrir $t\in \R$.
% \end {block}
 
\end {frame}


\begin {frame}{Lengd stikaferils}
\begin {block}{Regla 1.\arabic{mycount}}\stepcounter{mycount}
Látum $\rv:  [a,b]\rightarrow \Rn$ vera samfellt diffranlegan stikaferil.  {\em Lengd} eða 
{\em bogalengd} stikaferilsins er skilgreind með formúlunni 
$$s=\int_a^b |\vv(t)|\,dt.$$
\end {block}
\end {frame}

\begin {frame}

\begin {block}{Skilgreining og umræða 1.\arabic{mycount}}\stepcounter{mycount}
Látum $\rv: [a,b]\rightarrow \Rn$ vera samfellt diffranlegan stikaferil.   Sagt er að
stikaferillinn sé {\em stikaður með  bogalengd} ef fyrir allar tölur $t_1,
t_2$ þannig að $a\leq t_1<t_2\leq b$ þá gildir 
$$t_2-t_1= \int_{t_1}^{t_2} |\vv(t)|\,dt.$$
(Skilyrðið segir að lengd stikaferilsins á milli punkta $\rv(t_1)$ og
$\rv(t_2)$ sé jöfn muninum á $t_2$ og $t_1$.)
Stikun með bogalengd má líka þekkja á þeim eiginleika að $|\vv(t)|=1$ fyrir öll gildi á $t$.
\end{block}
\end {frame}

%Ath2015: Muna að taka mörg dæmi. Fyrirlestur í styttri kantinum en nægir ef umræða um skipulag námskeiðs. Útskýring á krossfeldi.
%Bæta við: Kúpni, teikning á stikaferli.

\end{document}
