\lecture[2]{2. Ferlar í plani og pólhnit}{lecture-text}
\date{7.~janúar 2015}
\newcounter{mycount}
\refstepcounter{mycount}

\begin{document}

\subsection{}
	\maketitle




\subsection{Inngangur}
 \begin {itemize}
  \item Þegar við fáumst við verkefni í mörgum víddum höfum við frelsi til að velja hnitakerfi.
  \item Heppilegt val á hnitakerfi getur skipt sköpum við lausn verkefnis.
 \end {itemize}




\subsection{Pólhnit} 

\subsubsection{Skilgreining 2.\arabic{mycount}}\stepcounter{mycount}
Látum $P=(x,y)\neq \ov$ vera punkt í plani.  {\em Pólhnit} $P$ er talnapar $[r,\theta]$ þannig að $r$ er fjarlægð $P$ frá 
$O=(0,0)$ og $\theta$ er hornið á milli striksins $\overline{OP}$ og $x$-ássins.  (Hornið er mælt þannig að rangsælis stefna telst jákvæð,  og leggja má við $\theta$ heil margfeldi af $2\pi$.)




\subsection{Pólhnit}
\subsubsection{Regla 2.\arabic{mycount}}\stepcounter{mycount}
  Ef pólhnit punkts í plani eru $[r, \theta]$ þá má reikna $xy$-hnit hans (e.~{\em rectangular coordinates} eða {\em Cartesian coordinates}) með formúlunum
$$x=r\cos\theta \qquad\mbox{og}\qquad y=r\sin\theta.$$

Ef við þekkjum $xy$-hnit punkts þá má finna pólhnitin út frá jöfnunum
$$r=\sqrt{x^2+y^2}\qquad\mbox{og}
\qquad \tan\theta=\frac{y}{x}.$$

(Ef $x=0$ þá má taka $\theta=\frac{\pi}{2}$ ef $y>0$ en 
$\theta=-\frac{\pi}{2}$ ef $y<0$.  Þegar jafnan $\tan\theta=\frac{y}{x}$ er notuð til að ákvarða $\theta$ þá er tekin lausn á milli $-\frac{\pi}{2}$  og $\frac{\pi}{2}$ ef $x>0$ en á milli $\frac{\pi}{2}$ og $\frac{3\pi}{2}$ ef $x<0$.)



\subsection{Pólhnitagraf}
\subsubsection {Skilgreining og umræða 2.\arabic{mycount} }\refstepcounter{mycount}Látum $f$ vera fall skilgreint fyrir 
$\theta$ þannig að 
$\alpha\leq\theta\leq\beta$.  Jafnan $r=f(\theta)$ lýsir mengi allra punkta í planinu sem hafa pólhnit á forminu $[f(\theta),\theta]$ þar sem $\alpha\leq\theta\leq\beta$.  Þetta mengi kallast {\em pólhnitagraf} fallsins $f$.

Pólhnitagraf er ferill í planinu sem má stika með stikaferlinum 
$$\rv:[\alpha,\beta]\rightarrow\R^2$$
með formúlu
$$\rv(\theta)=[f(\theta),\theta]=
(f(\theta)\cos\theta, f(\theta)\sin\theta).$$
 


\subsection{Snertill við pólhnitagraf}
 \subsubsection{Setning 2.\arabic{mycount}}\refstepcounter{mycount}
  Látum $r=f(\theta)$ vera pólhnitagraf fallsins $f$ og gerum ráð fyrir að fallið $f$ sé samfellt diffranlegt.  Látum $\rv(\theta)$ tákna stikunina á pólhnitagrafinu sem innleidd er í 2.3.  Ef vigurinn 
$\rv'(\theta)\neq \ov$ þá gefur þessi vigur stefnu snertils við pólhnitagrafið og út frá $\rv'(\theta)$ má reikna hallatölu snertils við pólhnitagrafið.
 



\subsection{Flatarmál}
 \subsubsection{Setning 2.\arabic{mycount}}\refstepcounter{mycount}
  Flatarmál svæðisins sem afmarkast af geislunum 
$\theta=\alpha$ og $\theta=\beta$ (með $\alpha\leq \beta$ og $\beta-\alpha\leq 2\pi$) og pólhnitagrafi $r=f(\theta)$ ($f$ samfellt) er
$$A=\frac{1}{2}\int_\alpha^\beta r^2\,d\theta
=\frac{1}{2}\int_\alpha^\beta f(\theta)^2\,d\theta.$$
 



\subsection{Bogalengd}
 \subsubsection{Setning 2.\arabic{mycount}}\refstepcounter{mycount}
   Gerum ráð fyrir að fallið $f(\theta)$ sé diffranlegt. Bogalengd pólhnitagrafsins $r=f(\theta)$, þegar $\alpha\leq\theta\leq\beta$, er gefin með formúlunni 
$$s=\int_\alpha^\beta \sqrt{f'(\theta)^2+f(\theta)^2}\,d\theta.$$
 






\end{document}