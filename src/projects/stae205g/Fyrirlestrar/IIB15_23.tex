\def \kaflanr {23}
\lecture[\kaflanr]{\kaflanr. grad, div og curl}{lecture-text}
\date{23.~mars 2015}
\newcounter{mycount}
\refstepcounter{mycount}

\begin{document}

\begin{frame}
	\maketitle
\end{frame}




\begin{frame}{} 

\begin {block}{Skilgreining \rtask{}}\stepcounter{mycount}
 Skilgreinum {\em nabla}-virkjann sem diffurvirkja
$$\nabla=\iv\,\frac{\partial}{\partial x}+\jv\,\frac{\partial}{\partial y}+\kv\,\frac{\partial}{\partial z}.$$

\end{block}

\end{frame}



\begin{frame}{} 

\begin {block}{Skilgreining \rtask{}}\stepcounter{mycount}
 Látum
$\Fv(x,y,z)=F_1(x,y,z)\,\iv+F_2(x,y,z)\,\jv+F_3(x,y,z)\,\kv$ vera
vigursvið og $\phi(x,y,z)$ vera fall. 

Skilgreinum  {\em stigul} $\phi$ sem vigursviðið 
$$\grad \phi=\nabla\phi=\frac{\partial \phi}{\partial x}\,\iv+
\frac{\partial \phi}{\partial y}\,\jv+\frac{\partial \phi}{\partial z}\,\kv.$$

Skilgreinum {\em sundurleitni} (e.~divergens) vigursviðsins $\Fv$ sem 
 $$\dive\Fv=\nabla\cdot\Fv=\frac{\partial F_1}{\partial x}+\frac{\partial F_2}{\partial y}+\frac{\partial F_3}{\partial z}.$$

Skilgreinum {\em rót}  vigursviðsins $\Fv$ sem 
 \begin{align*}
 \curl\Fv&=\nabla\times\Fv =\begin{vmatrix} \iv&\jv&\kv\\
 \frac{\partial} {\partial x}&\frac{\partial}{\partial y}&\frac{\partial}{\partial z}\\F_1&F_2&F_3\end{vmatrix} \\
 \noalign{\smallskip}
 &=\bigg(\frac{\partial F_3}{\partial y}-\frac{\partial F_2}{\partial
 z}\bigg)\,\iv+\bigg(\frac{\partial F_1}{\partial z}-\frac{\partial
 F_3}{\partial x}\bigg)\,\jv+\bigg(\frac{\partial F_2}{\partial
 x}-\frac{\partial F_1}{\partial y}\bigg)\,\kv. 
 \end{align*}
\end{block}

\end{frame}




\begin{frame}{} 

\begin {block}{Varúð \rtask{}}\stepcounter{mycount}
  Ef $\phi(x,y,z)$ er fall þá er $\nabla \phi(x,y,z)$ 
stigullinn af $\phi(x,y,z)$ en $\phi(x,y,z)\nabla$ er diffurvirki.
\end{block}

\end{frame}




\begin{frame}{} 

\begin {block}{Varúð \rtask{}}\stepcounter{mycount}
 Sundurleitnin $\dive\Fv$ er fall $\R^3\rightarrow\R$ en rótið $\curl\Fv$ er vigursvið $\R^3\rightarrow\R^3$.
\end{block}

\end{frame}




\begin{frame}{} 

\begin {block}{Skilgreining \rtask{}}\stepcounter{mycount}
 Látum
$\Fv(x,y)=F_1(x,y)\,\iv+F_2(x,y)\,\jv$ vera vigursvið.  Skilgreinum
{\em sundurleitni} $\Fv$ sem  
$$\dive\Fv=\nabla\cdot\Fv=\frac{\partial F_1}{\partial
  x}+\frac{\partial F_2}{\partial y}.$$ 
og {\em rót} $\Fv$ skilgreinum við sem 
$$\curl\Fv=\bigg(\frac{\partial F_2}{\partial x}-\frac{\partial
  F_1}{\partial y}\bigg)\,\kv.$$ 
\end{block}

\end{frame}




\begin{frame}{} 

\begin {block}{Reiknireglur \rtask{}}\stepcounter{mycount}
  Gerum ráð fyrir að $\Fv$ og $\Gv$ séu
vigursvið og $\phi$ og $\psi$ föll.  Gerum ráð fyrir að þær
hlutafleiður sem við þurfum að nota séu skilgreindar og samfelldar. 

(a) $\nabla(\phi\psi)=\phi\nabla\psi+\psi\nabla\phi$.

(b)  $\nabla\cdot(\phi\Fv)=(\nabla\phi)\cdot\Fv+\phi(\nabla\cdot\Fv)$.

(c) $\nabla\times(\phi\Fv)=(\nabla\phi)\times\Fv+\phi(\nabla\times\Fv)$. 

(d)  $\nabla\cdot(\Fv\times\Gv)=(\nabla\times\Fv)\cdot\Gv
-\Fv\cdot(\nabla\times\Gv)$.

(e) $\nabla\times(\Fv\times\Gv)=(\nabla\cdot\Gv)\Fv
+(\Gv\cdot\nabla)\Fv-(\nabla\cdot\Fv)\Gv-(\Fv\cdot\nabla)\Gv$.

(f) $\nabla(\Fv\cdot\Gv)=\Fv\times(\nabla\times \Gv)+\Gv\times(\nabla\times \Fv)+(\Fv\cdot\nabla)\Gv+(\Gv\cdot\nabla)\Fv$.

(g) $\nabla\cdot(\nabla\times \Fv)=0$\qquad\qquad$\dive\curl=0$

(h) $\nabla\times(\nabla\phi)=\ov$\qquad\qquad$\curl\grad=\ov$

(i)  $\nabla\times(\nabla\times \Fv)=\nabla(\nabla\cdot\Fv)-\nabla^2\Fv$.

\end{block}

\end{frame}




\begin{frame}{} 

\begin {block}{Skilgreining \rtask{}}\stepcounter{mycount}
 Látum $\Fv$ vera vigursvið skilgreint á svæði $D$.  

(a) Vigursviðið $\Fv$ er sagt vera {\em sundurleitnilaust}
(e.~solenoidal) ef $\dive\Fv=0$ i öllum punktum  $D$.

(b) Vigursviðið $\Fv$ er sagt vera {\em rótlaust} (e.~irrotational) ef $\curl\Fv=\ov$ á öllu $D$.
    
\end{block}

\end{frame}




\begin{frame}{} 

\begin {block}{Athugasemd \rtask{}}\stepcounter{mycount}
 Vigursvið   $\Fv(x,y,z)=F_1(x,y,z)\,\iv+F_2(x,y,z)\,\jv+F_3(x,y,z)\,\kv$ er rótlaust ef og aðeins ef 
$$\frac{\partial F_1}{\partial y}=
\frac{\partial F_2}{\partial x},\quad
\frac{\partial F_1}{\partial z}=
\frac{\partial F_3}{\partial x},\quad
\frac{\partial F_2}{\partial z}=
\frac{\partial F_3}{\partial y}.$$

\end{block}

\end{frame}




\begin{frame}{} 

\begin {block}{Setning \rtask{}}\stepcounter{mycount}
 (a) Rót vigursviðs er sundurleitnilaus.

(b) Stigulsvið er rótlaust.

   \medskip
\end{block}

\end{frame}




\begin{frame}{} 

\begin {block}{Skilgreining \rtask{}}\stepcounter{mycount}
  Svæði $D$ í rúmi eða plani kallast {\em
  stjörnusvæði} ef til er punktur $P$ í $D$ þannig að fyrir sérhvern
annan punkt $Q$ í $D$ þá liggur allt línustrikið á milli $P$ og $Q$ í
$D$. 

\end{block}

\end{frame}




\begin{frame}{} 

\begin {block}{Setning \rtask{}}\stepcounter{mycount}
Látum $\Fv$ vera samfellt diffranlegt vigursvið
skilgreint á stjörnusvæði $D$.  Ef $\Fv$ er rótlaust þá er $\Fv$
stigulsvið.  Með öðrum orðum, ef vigursviðið  $\Fv$ er samfellt
diffranlegt og skilgreint á stjörnusvæði $D$ og uppfyllir jöfnurnar
$$\frac{\partial F_1}{\partial y}=
\frac{\partial F_2}{\partial x},\quad
\frac{\partial F_1}{\partial z}=
\frac{\partial F_3}{\partial x},\quad
\frac{\partial F_2}{\partial z}=
\frac{\partial F_3}{\partial y},$$
þá er $\Fv$ stigulsvið.
\end{block}

\end{frame}




\begin{frame}{} 

\begin {block}{Setning \rtask{}}\stepcounter{mycount}
 Lát $\Fv$ vera samfellt diffranlegt vigursvið skilgreint á stjörnusvæði $D$.  Ef $\Fv$ er sundurleitnilaust þá er til vigursvið $\Gv$ þannig að $\Fv=\curl\Gv$.  Vigursviðið $\Gv$ kallast {\em vigurmætti} fyrir $\Fv$.
\end{block}

\end{frame}

\end{document}