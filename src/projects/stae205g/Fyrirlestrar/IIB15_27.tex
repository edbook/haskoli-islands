\def \kaflanr {27}
\lecture[\kaflanr]{\kaflanr. Hagnýtingar í eðlisfræði}{lecture-text}
\date{13.~apríl 2015}
\newcounter{mycount}
\refstepcounter{mycount}

\begin{document}

\begin{frame}
	\maketitle
\end{frame}




\begin{frame}{} 

\begin {block}{Vökvaflæði \rtask{}}
Skoðum vökvaflæði í rúmi.  Hugsum okkur að vökvaflæðið sé líka háð tíma.  Látum $\vv(x,y,z,t)$ tákna hraðavigur agnar sem er í punktinum  $(x,y,z)$ á tíma $t$.  Látum $\delta(x,y,z,t)$ tákna efnisþéttleika (massi per rúmmálseiningu) í punktum $(x,y,z)$ á tíma $t$.  Þá gildir að 
$$\frac{\partial \delta}{\partial t}+\dive(\delta\vv)=0.$$
(Þessi jafna kallast samfelldnijafnan um vökvaflæðið.)
\end{block}

\end{frame}



\begin{frame}{} 

\begin {block}{Vökvaflæði \rtask{}}
 Til viðbótar við $\vv$ og $\delta$ þá skilgreinum við $p(x,y,z,t)$ sem þrýsting og $\Fv$ sem utanaðkomandi kraft, gefinn sem kraftur per massaeiningu.  Þá gildir að $$\delta\frac{\partial \vv}{\partial t}+\delta(\vv\cdot\nabla)\vv=-\nabla p+\delta\Fv.$$
(Þessi jafna er kölluð hreyfijafna flæðisins.)
\end{block}

\end{frame}

\begin{frame}{} 

\begin {block}{Rafsvið \rtask{} - Lögmál Coulombs }
 Látum punkthleðslu $q$ vera í punktinum $\sv=\xi\,\iv+\eta\,\jv+\zeta\,\kv$.   Í punktum $\rv=x\,\iv+y\,\jv+z\,\kv$ er rafsviðið vegna þessarar hleðslu
$$\Ev(\rv)=\frac{q}{4\pi\epsilon_0}\frac{\rv-\sv}{|\rv-\sv|^3}$$
\end{block}
þar sem $\epsilon_0$ er {\emph rafsvörunarstuðull} tómarúms.
\end{frame}


\begin{frame}{} 

\begin {block}{Rafsvið  \rtask{} - Lögmál Gauss (fyrsta jafna Maxwells)}
 Látum $\rho(\xi,\eta,\zeta)$ vera hleðsludreifingu og $\Ev$ rafsviðið vegna hennar.  Þá gildir að 
$$\boxed{\dive\Ev=\frac{\rho}{\epsilon_0}.}$$
\end{block}
\begin {block}{Rafsvið \rtask{}}
 Látum $\rho(\xi,\eta,\zeta)$ vera hleðsludreifingu á takmörkuðu svæði $R$ og $\Ev$ rafsviðið vegna hennar.  Ef við setjum
 $$ \phi(\rv) = -\frac{1}{4 \pi \epsilon_0} \iiint_R \frac{\rho(\sv)}{|\rv-\sv|} dV$$
 þá er $\Ev = \nabla \phi$ og þar með er 
$$\boxed{\curl \Ev= \mathbf{0}.}$$
\end{block}
\end{frame}



\begin{frame}{} 

\begin {block}{Segulsvið \rtask{} - Lögmál Biot-Savart }
 Látum straum $I$ fara eftir ferli $\cal F$.  Táknum segulsviðið með $\Hv$ og látum $\sv=\xi\,\iv+\eta\,\jv+\zeta\,\kv$ vera punkt á ferlinum $\cal F$.  Þá gefur örbútur $d\sv$ úr $\cal F$ af sér segulsvið 
$$d\Hv(\rv)=\frac{\mu_0 I}{4\pi}\frac{d\sv\times(\rv-\sv)}{|\rv-\sv|^3}$$

þar sem $\mu_0$ er {\emph segulsvörunarstuðull} tómarúms. Af þessu sést að 


$$\Hv=\frac{\mu_0 I}{4\pi}\oint_{\cal F}
\frac{d\sv\times(\rv-\sv)}{|\rv-\sv|^3}$$

og sýna má að ef $\rv \notin \mathcal{F}$ þá er $$\curl \Hv = \mathbf{0}.$$

\end{block}

\end{frame}



\begin{frame}{} 

\begin {block}{Segulsvið \rtask{} - Lögmál Ampére}
 Hugsum okkur að straumur $I$ fari upp eftir $z$-ás.  Táknum með $\Hv$ segulsviðið og $H=|\Hv|$.  Í punkti  $\rv=x\,\iv+y\,\jv+z\,\kv$ í fjarlægð $a$ frá $z$-ás er $H=\frac{\mu_0 I}{2\pi a}$ og ef $\cal C$ er lokaður einfaldur ferill sem fer rangsælis einu sinni umhverfis $z$-ásinn þá er 
$$\oint_{\cal C} \Hv\cdot d\rv=\mu_0 I.$$

 Hugsum okkur að $\mathbf{J}(\rv)$ sé straumþéttleiki í punkti $\rv$ (straumur á flatareiningu).  Þá er 
$$ \boxed{\curl \Hv = \mu_0 \mathbf{J}.}$$
\end{block}
Einnig gildir að ef við setjum 
$$\Av(\rv)=\frac{\mu_0}{4\pi}\iiint_{R}
\frac{\mathbf{J}(\mathbf{s})}{|\rv-\sv|}dV,$$
þá er $\Hv=\curl \Av$  og því er $$\boxed{\dive \Hv=0.}$$
\end{frame}

\begin {frame}{Samantekt}
 \begin {align*}
  \dive \Ev &= \frac{\rho}{\epsilon_0} \quad~ \dive \Hv = 0 \\
  \curl \Ev &= \mathbf{0} \qquad \curl \Hv = \mu_0 \mathbf{J}
 \end {align*}

 Jöfnur Maxwells
 \begin {align*}
  \dive \Ev &= \frac{\rho}{\epsilon_0} \qquad ~ \dive \Hv = 0 \\
  \curl \Ev &= -\frac{\partial \Hv}{\partial t} \quad \curl \Hv = \mu_0 \mathbf{J} + \mu_0 \epsilon_0  \frac{\partial\Ev}{\partial t}
 \end {align*}

\end {frame}



\end{document}