% Generated by Sphinx.
\def\sphinxdocclass{report}
\documentclass[letterpaper,10pt,icelandic]{sphinxmanual}
\usepackage[utf8]{inputenc}
\DeclareUnicodeCharacter{00A0}{\nobreakspace}
\usepackage{cmap}
\usepackage[T1]{fontenc}
\usepackage{babel}
\usepackage{amssymb}
\usepackage{amsmath}
\usepackage{times}
\usepackage[Sonny]{fncychap}
\usepackage{longtable}
\usepackage{sphinx}
\usepackage{multirow}


\title{Töluleg greining (STÆ405G)}
\date{29. febrúar, 2016}
\release{0.2}
\author{Benedikt Steinar Magnússon}
\newcommand{\sphinxlogo}{\includegraphics{hi_vnvs_horiz_raunvisindadeild.jpg}\par}
\renewcommand{\releasename}{Útgáfa}
\makeindex

\makeatletter
\def\PYG@reset{\let\PYG@it=\relax \let\PYG@bf=\relax%
    \let\PYG@ul=\relax \let\PYG@tc=\relax%
    \let\PYG@bc=\relax \let\PYG@ff=\relax}
\def\PYG@tok#1{\csname PYG@tok@#1\endcsname}
\def\PYG@toks#1+{\ifx\relax#1\empty\else%
    \PYG@tok{#1}\expandafter\PYG@toks\fi}
\def\PYG@do#1{\PYG@bc{\PYG@tc{\PYG@ul{%
    \PYG@it{\PYG@bf{\PYG@ff{#1}}}}}}}
\def\PYG#1#2{\PYG@reset\PYG@toks#1+\relax+\PYG@do{#2}}

\expandafter\def\csname PYG@tok@kc\endcsname{\let\PYG@bf=\textbf\def\PYG@tc##1{\textcolor[rgb]{0.00,0.44,0.13}{##1}}}
\expandafter\def\csname PYG@tok@vc\endcsname{\def\PYG@tc##1{\textcolor[rgb]{0.73,0.38,0.84}{##1}}}
\expandafter\def\csname PYG@tok@gs\endcsname{\let\PYG@bf=\textbf}
\expandafter\def\csname PYG@tok@se\endcsname{\let\PYG@bf=\textbf\def\PYG@tc##1{\textcolor[rgb]{0.25,0.44,0.63}{##1}}}
\expandafter\def\csname PYG@tok@mb\endcsname{\def\PYG@tc##1{\textcolor[rgb]{0.13,0.50,0.31}{##1}}}
\expandafter\def\csname PYG@tok@nv\endcsname{\def\PYG@tc##1{\textcolor[rgb]{0.73,0.38,0.84}{##1}}}
\expandafter\def\csname PYG@tok@s\endcsname{\def\PYG@tc##1{\textcolor[rgb]{0.25,0.44,0.63}{##1}}}
\expandafter\def\csname PYG@tok@nt\endcsname{\let\PYG@bf=\textbf\def\PYG@tc##1{\textcolor[rgb]{0.02,0.16,0.45}{##1}}}
\expandafter\def\csname PYG@tok@s1\endcsname{\def\PYG@tc##1{\textcolor[rgb]{0.25,0.44,0.63}{##1}}}
\expandafter\def\csname PYG@tok@ni\endcsname{\let\PYG@bf=\textbf\def\PYG@tc##1{\textcolor[rgb]{0.84,0.33,0.22}{##1}}}
\expandafter\def\csname PYG@tok@vg\endcsname{\def\PYG@tc##1{\textcolor[rgb]{0.73,0.38,0.84}{##1}}}
\expandafter\def\csname PYG@tok@mo\endcsname{\def\PYG@tc##1{\textcolor[rgb]{0.13,0.50,0.31}{##1}}}
\expandafter\def\csname PYG@tok@gu\endcsname{\let\PYG@bf=\textbf\def\PYG@tc##1{\textcolor[rgb]{0.50,0.00,0.50}{##1}}}
\expandafter\def\csname PYG@tok@cs\endcsname{\def\PYG@tc##1{\textcolor[rgb]{0.25,0.50,0.56}{##1}}\def\PYG@bc##1{\setlength{\fboxsep}{0pt}\colorbox[rgb]{1.00,0.94,0.94}{\strut ##1}}}
\expandafter\def\csname PYG@tok@mi\endcsname{\def\PYG@tc##1{\textcolor[rgb]{0.13,0.50,0.31}{##1}}}
\expandafter\def\csname PYG@tok@m\endcsname{\def\PYG@tc##1{\textcolor[rgb]{0.13,0.50,0.31}{##1}}}
\expandafter\def\csname PYG@tok@sd\endcsname{\let\PYG@it=\textit\def\PYG@tc##1{\textcolor[rgb]{0.25,0.44,0.63}{##1}}}
\expandafter\def\csname PYG@tok@gh\endcsname{\let\PYG@bf=\textbf\def\PYG@tc##1{\textcolor[rgb]{0.00,0.00,0.50}{##1}}}
\expandafter\def\csname PYG@tok@gr\endcsname{\def\PYG@tc##1{\textcolor[rgb]{1.00,0.00,0.00}{##1}}}
\expandafter\def\csname PYG@tok@nl\endcsname{\let\PYG@bf=\textbf\def\PYG@tc##1{\textcolor[rgb]{0.00,0.13,0.44}{##1}}}
\expandafter\def\csname PYG@tok@kd\endcsname{\let\PYG@bf=\textbf\def\PYG@tc##1{\textcolor[rgb]{0.00,0.44,0.13}{##1}}}
\expandafter\def\csname PYG@tok@w\endcsname{\def\PYG@tc##1{\textcolor[rgb]{0.73,0.73,0.73}{##1}}}
\expandafter\def\csname PYG@tok@no\endcsname{\def\PYG@tc##1{\textcolor[rgb]{0.38,0.68,0.84}{##1}}}
\expandafter\def\csname PYG@tok@ge\endcsname{\let\PYG@it=\textit}
\expandafter\def\csname PYG@tok@k\endcsname{\let\PYG@bf=\textbf\def\PYG@tc##1{\textcolor[rgb]{0.00,0.44,0.13}{##1}}}
\expandafter\def\csname PYG@tok@kr\endcsname{\let\PYG@bf=\textbf\def\PYG@tc##1{\textcolor[rgb]{0.00,0.44,0.13}{##1}}}
\expandafter\def\csname PYG@tok@s2\endcsname{\def\PYG@tc##1{\textcolor[rgb]{0.25,0.44,0.63}{##1}}}
\expandafter\def\csname PYG@tok@err\endcsname{\def\PYG@bc##1{\setlength{\fboxsep}{0pt}\fcolorbox[rgb]{1.00,0.00,0.00}{1,1,1}{\strut ##1}}}
\expandafter\def\csname PYG@tok@c\endcsname{\let\PYG@it=\textit\def\PYG@tc##1{\textcolor[rgb]{0.25,0.50,0.56}{##1}}}
\expandafter\def\csname PYG@tok@ss\endcsname{\def\PYG@tc##1{\textcolor[rgb]{0.32,0.47,0.09}{##1}}}
\expandafter\def\csname PYG@tok@il\endcsname{\def\PYG@tc##1{\textcolor[rgb]{0.13,0.50,0.31}{##1}}}
\expandafter\def\csname PYG@tok@si\endcsname{\let\PYG@it=\textit\def\PYG@tc##1{\textcolor[rgb]{0.44,0.63,0.82}{##1}}}
\expandafter\def\csname PYG@tok@nc\endcsname{\let\PYG@bf=\textbf\def\PYG@tc##1{\textcolor[rgb]{0.05,0.52,0.71}{##1}}}
\expandafter\def\csname PYG@tok@nn\endcsname{\let\PYG@bf=\textbf\def\PYG@tc##1{\textcolor[rgb]{0.05,0.52,0.71}{##1}}}
\expandafter\def\csname PYG@tok@ow\endcsname{\let\PYG@bf=\textbf\def\PYG@tc##1{\textcolor[rgb]{0.00,0.44,0.13}{##1}}}
\expandafter\def\csname PYG@tok@mf\endcsname{\def\PYG@tc##1{\textcolor[rgb]{0.13,0.50,0.31}{##1}}}
\expandafter\def\csname PYG@tok@gt\endcsname{\def\PYG@tc##1{\textcolor[rgb]{0.00,0.27,0.87}{##1}}}
\expandafter\def\csname PYG@tok@sh\endcsname{\def\PYG@tc##1{\textcolor[rgb]{0.25,0.44,0.63}{##1}}}
\expandafter\def\csname PYG@tok@ne\endcsname{\def\PYG@tc##1{\textcolor[rgb]{0.00,0.44,0.13}{##1}}}
\expandafter\def\csname PYG@tok@mh\endcsname{\def\PYG@tc##1{\textcolor[rgb]{0.13,0.50,0.31}{##1}}}
\expandafter\def\csname PYG@tok@sx\endcsname{\def\PYG@tc##1{\textcolor[rgb]{0.78,0.36,0.04}{##1}}}
\expandafter\def\csname PYG@tok@nd\endcsname{\let\PYG@bf=\textbf\def\PYG@tc##1{\textcolor[rgb]{0.33,0.33,0.33}{##1}}}
\expandafter\def\csname PYG@tok@c1\endcsname{\let\PYG@it=\textit\def\PYG@tc##1{\textcolor[rgb]{0.25,0.50,0.56}{##1}}}
\expandafter\def\csname PYG@tok@nb\endcsname{\def\PYG@tc##1{\textcolor[rgb]{0.00,0.44,0.13}{##1}}}
\expandafter\def\csname PYG@tok@gp\endcsname{\let\PYG@bf=\textbf\def\PYG@tc##1{\textcolor[rgb]{0.78,0.36,0.04}{##1}}}
\expandafter\def\csname PYG@tok@o\endcsname{\def\PYG@tc##1{\textcolor[rgb]{0.40,0.40,0.40}{##1}}}
\expandafter\def\csname PYG@tok@cm\endcsname{\let\PYG@it=\textit\def\PYG@tc##1{\textcolor[rgb]{0.25,0.50,0.56}{##1}}}
\expandafter\def\csname PYG@tok@gi\endcsname{\def\PYG@tc##1{\textcolor[rgb]{0.00,0.63,0.00}{##1}}}
\expandafter\def\csname PYG@tok@vi\endcsname{\def\PYG@tc##1{\textcolor[rgb]{0.73,0.38,0.84}{##1}}}
\expandafter\def\csname PYG@tok@sr\endcsname{\def\PYG@tc##1{\textcolor[rgb]{0.14,0.33,0.53}{##1}}}
\expandafter\def\csname PYG@tok@sb\endcsname{\def\PYG@tc##1{\textcolor[rgb]{0.25,0.44,0.63}{##1}}}
\expandafter\def\csname PYG@tok@cp\endcsname{\def\PYG@tc##1{\textcolor[rgb]{0.00,0.44,0.13}{##1}}}
\expandafter\def\csname PYG@tok@gd\endcsname{\def\PYG@tc##1{\textcolor[rgb]{0.63,0.00,0.00}{##1}}}
\expandafter\def\csname PYG@tok@kp\endcsname{\def\PYG@tc##1{\textcolor[rgb]{0.00,0.44,0.13}{##1}}}
\expandafter\def\csname PYG@tok@nf\endcsname{\def\PYG@tc##1{\textcolor[rgb]{0.02,0.16,0.49}{##1}}}
\expandafter\def\csname PYG@tok@kt\endcsname{\def\PYG@tc##1{\textcolor[rgb]{0.56,0.13,0.00}{##1}}}
\expandafter\def\csname PYG@tok@na\endcsname{\def\PYG@tc##1{\textcolor[rgb]{0.25,0.44,0.63}{##1}}}
\expandafter\def\csname PYG@tok@kn\endcsname{\let\PYG@bf=\textbf\def\PYG@tc##1{\textcolor[rgb]{0.00,0.44,0.13}{##1}}}
\expandafter\def\csname PYG@tok@bp\endcsname{\def\PYG@tc##1{\textcolor[rgb]{0.00,0.44,0.13}{##1}}}
\expandafter\def\csname PYG@tok@sc\endcsname{\def\PYG@tc##1{\textcolor[rgb]{0.25,0.44,0.63}{##1}}}
\expandafter\def\csname PYG@tok@go\endcsname{\def\PYG@tc##1{\textcolor[rgb]{0.20,0.20,0.20}{##1}}}

\def\PYGZbs{\char`\\}
\def\PYGZus{\char`\_}
\def\PYGZob{\char`\{}
\def\PYGZcb{\char`\}}
\def\PYGZca{\char`\^}
\def\PYGZam{\char`\&}
\def\PYGZlt{\char`\<}
\def\PYGZgt{\char`\>}
\def\PYGZsh{\char`\#}
\def\PYGZpc{\char`\%}
\def\PYGZdl{\char`\$}
\def\PYGZhy{\char`\-}
\def\PYGZsq{\char`\'}
\def\PYGZdq{\char`\"}
\def\PYGZti{\char`\~}
% for compatibility with earlier versions
\def\PYGZat{@}
\def\PYGZlb{[}
\def\PYGZrb{]}
\makeatother

\renewcommand\PYGZsq{\textquotesingle}

\begin{document}

\maketitle
\tableofcontents
\phantomsection\label{index::doc}


\chapter{Inngangur}
\label{kafli01::doc}\label{kafli01:inngangur}
\emph{He was determined to discover the underlying logic behind the universe. Which was going to be hard, because there wasn't one.}

- Terry Pratchett, Mort


\section{Hvað er töluleg greining?}
\label{kafli01:hva-er-toluleg-greining}

\subsection{Tilraun að svari}
\label{kafli01:tilraun-a-svari}\begin{itemize}
\item {} 
Fagið \emph{töluleg greining} snýst um að búa til, greina og forrita
aðferðir til þess að nálga á lausnum á stærðfræðilegum verkefnum.

\item {} 
Aðferðirnar eru settar fram með reikniritum sem síðan eru forrituð og
það þarf góðan skilning á eiginleikum lausnanna sem verið er að nálga
til þess að geta greint hvernig forritin munu virka.

\item {} 
Greining á reikniritum er aðallega fólgin í skekkjumati og mati á
þeim aðgerðafjölda sem þarf til þess að ná að nálga lausn með
fyrirfram gefinni nákvæmni, þ.e. hagkvæmni og nákvæmni reikniritsins.

\item {} 
Líkanagerð í raunvísindum og verkfræði felur yfirleitt í sér eftirfarandi skref:
\begin{enumerate}
\item {} 
\emph{Greina} kerfið sem um ræðir

\item {} 
\emph{Smíða líkan} sem útskýrir hvernig kerfið hegðar sér, þó yfirleitt með töluverðum einföldunum.

\item {} 
\emph{Herma} kerfið í tölvu eins vel og hægt er. Hér þarf að ná ásættanlegri námkvæmni á þeim tíma sem útreikningar mega taka.

\item {} 
\emph{Túlka} niðurstöðurnar og bera saman við upphaflega kerfið.

\end{enumerate}

Töluleg greining kemur mikið við sögu í lið 3. og einnig í lið 4.

\end{itemize}


\section{Dæmi: Eldflaug}
\label{kafli01:daemi-eldflaug}
Gerum ráð fyrir að við höfum eftirfarandi eldflaug undir höndum:
\begin{itemize}
\item {} 
Eldsneytið dugir í 18 sek., þ.e. \(t\in [0,18]\).

\item {} 
Loftmótstaðan er \(d=0.1v^2\), þar sem \(v(t)\) er hraðinn á tíma \(t\).

\item {} 
Krafturinn sem knýr flaugina er \(T=5000\) N.

\item {} 
Massi eldsneytisins er \(m=180-10t\) kg.

\item {} 
Massi flaugarinnar er \(M = 120 + m = 300 - 10t\) kg.

\end{itemize}

Spurningin er: Í hvaða hæð er eldflaugin þegar eldsneytið klárast?

Úr öðru lögmáli Newtons fæst að \(F = (Mv)'\). Kraftarnir sem verka
á eldflaugina er \(T\) upp á við og loftmótstaðan og
þyngdarkrafturinn niður á við. Þannig fæst
\begin{gather}
\begin{split}(Mv)' = F = T - Mg - d\end{split}\notag
\end{gather}
það er
\begin{gather}
\begin{split}M'v + Mv' = T - Mg -d.\end{split}\notag
\end{gather}
Þetta jafngildir því að
\phantomsection\label{kafli01:eldflaug}\begin{gather}
\begin{split}v' = \frac{T-Mg-d-M'v}{M} = \frac{5000-(300-10t)g-0,1v^2+10v}{300-10t},
\label{eldflaug}\end{split}\notag
\end{gather}
og upphafsskilyrðin eru \(v(0) =0\).

Þar sem \(h' = v\), þá er hæðin á tíma \(t\) gefin með
\(h(t) =\int_0^t v(s)\, ds\). Þegar eldsneytið klárast þá er hæðin
\(h(18) = \int_0^{18} v(s)\, ds\).

Verkefnið er því að finna \(v\), og reikna svo heildið.

{\hyperref[kafli01:eldflaug]{\emph{Diffurjafnan}}} hér að ofan er ólínuleg og ekki aðgreinanleg þannig
að við getum ekki vænst
þess finna lausn með þeim aðferðum sem við höfum þegar lært. Eins er
ekki víst að við getum auðveldlega fundið stofnfall \(h\) fyrir
\(v\) til þess að reikna heildið, jafnvel þótt við hefðum \(v\).

Hins vegar getum við leyst diffurjöfnuna tölulega með aðferðunum úr
{\hyperref[kafli06:upphafsgildisverkefni]{\emph{kafla 6}}},
og heildið reiknum við svo tölulega með aðferðunum úr {\hyperref[kafli05:heildun]{\emph{kafla 5}}}.


\section{Samleitni runa}
\label{kafli01:samleitni-runa}

\subsection{Nokkur atriði um samleitni runa}
\label{kafli01:nokkur-atrii-um-samleitni-runa}
Mörg reiknirit til nálgunar á einhverri rauntölu eru hönnuð þannig að
reiknuð er runa \(x_0,x_1,x_2,\dots\) sem á að nálgast lausnina
okkar.

\index{runa}\index{samleitni}\index{samleitni!línuleg}\index{samleitni!ofurlínuleg}\index{samleitni!ferningssamleitni}\index{samleitni!af stigi \(\alpha\)}\index{markgildi}

\subsection{Skilgreining: Samleitni}
\label{kafli01:skilgreining-samleitni}\label{kafli01:index-0}
\emph{Rauntalnaruna} \((x_n)\) er sögð vera \emph{samleitin} (e. convergent)
að \emph{markgildinu} \(r\) ef um sérhvert \(\varepsilon>0\) gildir
að til er \(N>0\) þannig að
\begin{gather}
\begin{split}|x_n-r|<\varepsilon, \qquad \text{ ef } \quad n\geq N.\end{split}\notag
\end{gather}
Þetta er táknað annað hvort með
\begin{gather}
\begin{split}\lim_{n\to \infty}x_n=r \qquad \text{ eða } \qquad  x_n\to r
    \text{ ef } n\to \infty.\end{split}\notag
\end{gather}
Ef runan \((x_n)\) er samleitin að markgildinu \(r\) þá segjum
við einnig að hún \emph{stefni á} \(r\).

Hugsum okkur nú að \((x_n)\) sé gefin runa sem stefnir á \(r\)
og táknum skekkjuna með \(e_n=r-x_n\).

Runan er sögð vera \emph{línulega samleitin} (e. linear convergence) ef til
er \(\lambda\in ]0,1[\) þannig að
\begin{gather}
\begin{split}\lim_{n\to \infty}\dfrac{|e_{n+1}|}{|e_n|}=\lambda,\end{split}\notag
\end{gather}
\emph{ofurlínulega samleitin} (e. superlinear convergence), ef
\begin{gather}
\begin{split}\lim_{n\to \infty}\dfrac{|e_{n+1}|}{|e_n|}=0,\end{split}\notag
\end{gather}
\emph{ferningssamleitin} (e. quadratic convergence) ef til er \(\lambda>0\) þannig að
\begin{gather}
\begin{split}\lim_{n\to \infty}\dfrac{|e_{n+1}|}{|e_n|^2}=\lambda,\end{split}\notag
\end{gather}
og \emph{samleitin af stigi} \(\alpha\) (e. convergence of order
\(\alpha\)), þar sem \(\alpha> 1\), ef til er \(\lambda>0\)
þannig að
\begin{gather}
\begin{split}\lim_{n\to \infty}\dfrac{|e_{n+1}|}{|e_n|^\alpha}=\lambda.\end{split}\notag
\end{gather}
\begin{notice}{note}{Athugasemd:}
Runa er ofurlínulega samleitin ef hún er samleitin af stigi \(\alpha>1\).

Ferningssamleitin runa er samleitin af stigi 2 þannig að hún er einnig ofurlínulega samleitin.
\end{notice}


\subsection{Skilgreining}
\label{kafli01:skilgreining}
Oft eru notuð veikari hugtök til þess að lýsa samleitni runa (t.d. ef
við getum ekki fundið \(\lambda\) og \(\alpha\) nákvæmlega).

Þannig segjum við að runan \((x_n)\) sé \emph{að minnsta kosti línulega
samleitin} ef til er \(\lambda\in ]0,1[\) og \(N >0\) þannig að
\begin{gather}
\begin{split}|e_{n+1}|\leq \lambda |e_n|, \qquad n\geq N,\end{split}\notag
\end{gather}
ef til er \(\lambda>0\) og \(N>0\) þannig að
\begin{gather}
\begin{split}|e_{n+1}|\leq \lambda |e_n|^2, \qquad n\geq N,\end{split}\notag
\end{gather}
og \emph{að minnsta kosti samleitin af stigi} \(\alpha\), þar sem
\(\alpha> 1\), ef til eru \(\lambda>0\) og \(N>0\) þannig að
\begin{gather}
\begin{split}|e_{n+1}|\leq \lambda |e_n|^\alpha, \qquad n\geq N.\end{split}\notag
\end{gather}

\section{Setning Taylors}
\label{kafli01:setning-taylors}
\emph{Sometimes it's better to light a flamethrower than curse the darkness.}
- Terry Pratchett, Men at Arms: The Play

\index{föll!diffranlegt}\index{föll!afleiða}\index{föll!rúm samfelldra falla}\index{föll!rúm diffranlegra falla}

\subsection{Ritháttur fyrir diffranleg föll}
\label{kafli01:index-1}\label{kafli01:rithattur-fyrir-diffranleg-foll}
Látum nú \(f : I \to \mathbb{C}\) vera fall á bili \(I\) sem
tekur gildi í tvinntölunum. Ef \(f\) er deildanlegt í sérhverjum
punkti í \(I\), þá táknum við afleiðuna með \(f'\). Ef
\(f'\) er deildanlegt í sérhverjum punkti í \(I\), þá táknum við
\emph{aðra afleiðu} \(f\) með \(f''\), og svo framvegis.

Við skilgreinum með þrepun \(f^{(k)}\) fyrir \(k = 0,1,2,
\ldots\) þannig að \(f^{(0)} = f\) og ef \(f^{(k-1)}\) er
deildanlegt í sérhverjum punkti í \(I\), þá er
\(f^{(k)} = (f^{(k-1)})'\).

Við látum \(C^{k}(I)\) tákna línulega rúmið sem samanstendur af
öllum föllum \(f:I \to {\mathbb C}\) þannig að \(f', \ldots, f^{(k)}\) eru til í
sérhverjum punkti í \(I\) og \(f^{(k)}\) er samfellt fall á
\(I\).

\index{Taylor-margliða}

\subsection{Nálgun með Taylor-margliðu}
\label{kafli01:nalgun-me-taylor-margliu}\label{kafli01:index-2}
Ef \(a \in I\), \(m\) er jákvæð heiltala og
\(f \in C^{m}(I)\), þá nefnist margliðan
\begin{gather}
\begin{split}p(x) = f(a) + f'(a)(x-a) + \ldots   + \frac{f^{(m)}(a)}{m!}(x-a)^m\end{split}\notag
\end{gather}
Taylor-margliða fallsins \(f\) í punktinum \(a\) af stigi
\(m\), og er stundum táknuð með \(T_m f(x;a)\).

Athugið að stig margliðunnar \(p\) er minna eða jafnt og \(m\).

\index{setning Taylors}

\subsection{Setning Taylors}
\label{kafli01:index-3}\label{kafli01:id1}
Látum \(I \subseteq {\mathbb R}\) vera bil, 
\(f : I \to {\mathbb C}\) vera fall, \(m \geq 0\) vera heiltölu og gerum ráð
fyrir að \(f \in C^m(I)\) og að \(f^{(m+1)}(x)\) sé til í sérhverjum innri punkti
bilsins \(I\). Þá er til punktur \(\xi\) á milli \(a\) og
\(x\) þannig að
\begin{gather}
\begin{split}f(x) - T_mf(x;a)= \frac{f^{(m+1)}(\xi)}{(m+1)!}(x-a)^{m+1}.\end{split}\notag
\end{gather}
Hægri hliðin er oft táknuð \(R_m(x)\).

\begin{notice}{note}{Athugasemd:}
Þetta þýðir að skekkjan í því að nálga fallið \(f(x)\) með
Taylor-margliðu af stigi \(m\) hagar sér eins og
\((x-a)^{m+1}\).
\end{notice}


\subsection{Viðbót}
\label{kafli01:vibot}
Ef \(f^{(m+1)}\) er samfellt á lokaða bilinu með endapunkta
\(a\) og \(x\), þá er
\begin{gather}
\begin{split}\begin{aligned}
  R_m(x) &= f(x) - T_mf(x;a) \\
  &= \int\limits_a^x \frac{(x-t)^m}{m!}f^{(m+1)}(t) dt \notag \\
  &= (x-a)^{m+1} \int\limits_0^1 \frac{(1-s)^m}{m!} f^{(m+1)}(a + s(x-a)) ds.
\end{aligned}\end{split}\notag
\end{gather}

\subsection{Sýnidæmi: Nálgun á fallgildum \(x-\sin x\)}
\label{kafli01:synidaemi-nalgun-a-fallgildum}
Vitum að \(x \approx \sin x\) ef \(x\) er lítið. Tökum
\(x=0.1\) og hugsum okkur að við séum að reikna á vél með 8 stafa
nákvæmni. Hún gefur
\begin{gather}
\begin{split}\sin 0.1 = 0.099833417\end{split}\notag
\end{gather}
Af því leiðir
\begin{gather}
\begin{split}0.1 - \sin 0.1 = 1.66583\cdot 10^{-4}\end{split}\notag
\end{gather}
Við höfum tapað tveimur markverðum stöfum í nákvæmni.

Ef við notum Taylor-nálgunina fyrir \(\sin(x)\),
\begin{gather}
\begin{split}\sin x = x - \frac{x^3}{3!} + \frac{x^5}{5!}
    - \frac{x^7}{7!} \cdots\end{split}\notag
\end{gather}
og tökum fyrstu þrjá liðina, þ.e. skoðum 6. stigs Taylor-margliðu
fallsins.

\(x-\sin(x)\) er þá u.þ.b.
\begin{gather}
\begin{split}x - \left(x - \frac{x^3}{3!} + \frac{x^5}{5!}\right) = \frac{x^3}{3!} - \frac{x^5}{5!}.\end{split}\notag
\end{gather}
Fallgildið er þá
\begin{gather}
\begin{split}\frac {0.1^3}{3!} - \frac{0.1^5}{5!} = 1.6658334 \cdot 10^{-4}.\end{split}\notag
\end{gather}
Skekkjan er gefin með
\begin{gather}
\begin{split}|R_6(0.1)| = \left|\frac{\sin^{(7)}(\xi)}{7!}0.1^7\right|
    = \left|\frac{-\cos(\xi)}{7!}0.1^7\right|
    \leq \frac{1}{7!}0.1^7 < 0.2\cdot 10^{-10}.\end{split}\notag
\end{gather}
Sem þýðir að allir 8 stafir reiknivélarinnar eru markverðir, þ.e.
allir stafir \(1.6658334 \cdot 10^{-4}\) eru réttir.

\(\sin^{(7)}\) hér að ofan táknar 7. afleiðu \(\sin\), sem er
\(-\cos\).

Ef við tökum \(x = 0.01\) er þetta enn greinilegra. Reiknivélin
gefur
\begin{gather}
\begin{split}\sin(0.01) = 0.0099998333\end{split}\notag
\end{gather}
Þannig að
\begin{gather}
\begin{split}0.01 - \sin 0.01 = 0.1667\cdot 10^{-7}\end{split}\notag
\end{gather}
og við erum bara með 4 markverða stafi.

Hér dugir að taka aðeins þriðja stigs liðinn í Taylor-formúlunni
\begin{gather}
\begin{split}0.01 - \sin (0.01) \approx \frac{0.01^3}{3!}
    = 0.16666667 \cdot 10^{-7},\end{split}\notag
\end{gather}
því skekkjan er
\begin{gather}
\begin{split}R_4(0.01) \leq \frac{0.01^5}{5!} < 10^{-12}\end{split}\notag
\end{gather}
\index{skekkja}\index{skekkja!mæliskekkja}\index{skekkja!aðferðarskekkja}\index{skekkja!reikningsskekkja}\index{skekkja!mannlegar villur}

\section{Skekkjur}
\label{kafli01:index-4}\label{kafli01:skekkjur}
Við allar úrlausnir á verkefnum í tölulegri greininingu þarf að fást við
skekkjur. Þær eru af ýmsum toga:
\begin{itemize}
\item {} 
Gögn eru oft niðurstöður mælinga og þá fylgja þeim \emph{mæliskekkjur}.
Eins getum við þurft að notast við nálganir á föstum sem koma fyrir
(t.d. \(\pi\), Avogadrosar talan, …).

\item {} 
Við nálganir á lausnum á stærðfræðilegum verkefnum verða til
\emph{aðferðarskekkjur}. Þær verða til þegar reikniritin eru hönnuð og
greining á reikniritum snýst fyrst og fremst um mat á
aðferðarskekkjum.

\item {} 
\emph{Reikningsskekkjur} verða til í tölvum á öllum stigum, jafnvel þegar
tölur eru lesnar inn í tugakerfi og þeim snúið yfir í tvíundarkerfi.
Þær verða líka til vegna þess að tölvur geta einungis unnið með
endanlegt mengi af tölum og allar útkomur þarf að nálga innan þess
mengis. Þessar skekkjur nefnast oft \emph{afrúningsskekkjur}.

\item {} 
\emph{Mannlegar villur} eru óumflýjanlegar. Það sem við getum gert er
temja okkur vinnubrögð sem lágmarka líkur á þeim og auðvelda okkur að
finna villur sem við gerum.

\emph{Real stupidity beats artificial intelligence every time.}
-- Terry Pratchett

\end{itemize}

\index{skekkja!algildi}\index{skekkja!hlutfallsleg}

\subsection{Skekkja í nálgun á rauntölu \(r\)}
\label{kafli01:index-5}\label{kafli01:skekkja-i-nalgun-a-rauntolu}
Við getum stillt upp jöfnunum svona
\begin{gather}
\begin{split}r \text{ (rétt gildi) } = x\text{ (nálgunargildi)} +
    e \text{ (skekkja)}\end{split}\notag
\end{gather}
þar sem talan \(x\) er nálgun á tölunni \(r\), og þá nefnist
\begin{gather}
\begin{split}e=r-x\end{split}\notag
\end{gather}
\emph{skekkjan (e. error) í nálgun á} \(r\) \emph{með} \(x\) eða bara
\emph{skekkja}.

\emph{Algildi skekkju (e. absolute error)} er tölugildið \(|e|=|r-x|\)

Ef vitað er að \(r\neq 0\), þá nefnist
\begin{gather}
\begin{split}\dfrac{|e|}{|r|}=\dfrac{|r-x|}{|r|}\end{split}\notag
\end{gather}
\emph{hlutfallsleg skekkja (e. relative error)} í nálgun á \(r\) með
\(x\).

\begin{notice}{warning}{Aðvörun:}
Auðvitað er talan \(r\) sem við leitum að óþekkt (annars
þyrftum við ekki að framkvæma alla þessa reikninga), sem þýðir að við
getum hvergi notað hana í reikningum.
\end{notice}

\index{skekkja!fyrirframmat}

\subsection{Fyrirframmat á skekkju}
\label{kafli01:index-6}\label{kafli01:fyrirframmat-a-skekkju}
Metið er áður en reikningar hefjast hversu umfangsmikla reikninga þarf
að framkvæma til þess að nálgunin náist innan fyrirfram gefinna
skekkjumarka.

Ef lausnin er fundin með ítrekunaraðferð er yfirleitt metið hversu
margar ítrekarnir þarf til þess að nálgun verði innan skekkjumarka.

\index{skekkja!eftirámat}

\subsection{Eftirámat á skekkju}
\label{kafli01:index-7}\label{kafli01:eftiramat-a-skekkju}
Um leið og reikningar eru framkvæmdir er lagt mat á skekkju og
reikningum er hætt þegar matið segir að nálgun sé innan skekkjumarka.
Það gerist yfirleitt þegar gildið sem við reiknum út breytist orðið
lítið í hverju skrefi.

Hér þarf að skipta í tvö tilvik, fyrst skoðum við tilvikið þegar runan er ofurlínulega samleitin
og seinna tilvikið er þegar við vitum aðeins að runan er línulega samleitin, en
þá er matið aðeins flóknara.

\index{samleitni!ofurlínuleg}

\subsection{Ofurlínuleg samleitni -- Eftirámat á skekkju}
\label{kafli01:index-8}\label{kafli01:ofurlinuleg-samleitni-eftiramat-a-skekkju}
Hugsum okkur að við séum að nálga töluna \(r\) með gildum rununnar
\(x_n\), að við höfum reiknað út \(x_0,\dots,x_n\) og viljum fá
mat á skekkjunni \(e_n=r-x_n\) í \(n\)-ta skrefi.

Við reiknum næst út \(x_{n+1}\) og skrifum
\(e_{n+1}=\lambda_ne_n\). Þá er
\begin{gather}
\begin{split}x_{n+1}-x_n = (r-x_n)-(r-x_{n+1})
    = e_n-e_{n+1} = (1-\lambda_n)e_n\end{split}\notag
\end{gather}
og við fáum
\begin{gather}
\begin{split}e_n = \dfrac{x_{n+1}-x_n}{1-\lambda_n}.\end{split}\notag
\end{gather}
Ef við vitum að runan er \emph{ofurlínulega samleitin}, þá stefnir
\(\lambda_n\) á \(0\) og þar með er
\begin{gather}
\begin{split}e_n\approx x_{n+1}-x_n.\end{split}\notag
\end{gather}
Við hættum því útreikningi þegar \(|x_{n+1}-x_n|<\varepsilon\) þar
sem \(\varepsilon\) er fyrirfram gefin tala, sem lýsir þeirri
nákvæmni sem við viljum ná.

\index{samleitni!línuleg}

\subsection{Línuleg samleitni -- Eftirámat á skekkju}
\label{kafli01:linuleg-samleitni-eftiramat-a-skekkju}\label{kafli01:index-9}
Skoðum nú tilvikið ef einu upplýsingarnar sem við höfum er
að runan \(x_n\) sé \emph{að minnsta kosti
línulega samleitin}, þ.e. \(c\in [0,1)\) og \(N\in \mathbb N\)
þannig að
\begin{gather}
\begin{split}|e_{n+1}|\leq c|e_n|, \qquad \text{fyrir } n \geq N.\end{split}\notag
\end{gather}
Þá stefnir \(\lambda_n = e_{n+1}/e_n\) á fasta \(\lambda \leq c\) og við höfum
\begin{gather}
\begin{split}\lambda_n = \dfrac{e_{n+1}}{e_n} =
    \dfrac{1-\lambda_n}{1-\lambda_{n+1}}
    \cdot\dfrac{x_{n+2}-x_{n+1}}{x_{n+1}-x_n}\approx
    \dfrac{x_{n+2}-x_{n+1}}{x_{n+1}-x_n}\end{split}\notag
\end{gather}
Nú þurfum við að átta okkur á því hvernig þetta er nýtt í útreikningum.

Hugsum okkur að við höfum reiknað út \(x_0,\dots,x_n\) og viljum fá
mat á \(e_n\). Við reiknum þá út \(x_{n+1}\) og \(x_{n+2}\)
og síðan hlutfallið \(\kappa_n=(x_{n+2} - x_{n+1})/(x_{n+1} -
x_n)\) sem við notum sem mat á \(\lambda_n\). Eftirámatið á
skekkjunni í ítrekunarskrefi númer \(n\) verður síðan
\begin{gather}
\begin{split}e_n\approx \dfrac{x_{n+1}-x_n}{1-\kappa_n}.\end{split}\notag
\end{gather}
Ef stærðin í hægri hliðinni er komin niður fyrir fyrirfram gefin
skekkjumörk \(\varepsilon\), þá stöðvum við útreikningana.


\subsection{Sýnidæmi}
\label{kafli01:synidaemi}
Okkur er gefin runa af nálgunum á lausn jöfnunnar
\begin{gather}
\begin{split}f(x) = e^x\sin x-x^2 = 0\end{split}\notag
\end{gather}
og eigum að staðfesta hvort nálgunaraðferðin er ferningssamleitin:

\begin{tabulary}{\linewidth}{|L|L|L|L|}
\hline
\textsf{\relax 
\(n\)
} & \textsf{\relax 
\(x_n\)
} & \textsf{\relax 
\(|x_{n+1}-x_n|\)
} & \textsf{\relax 
\(\frac{|x_{n+1}-x_n|}{|x_n-x_{n-1}|^2}\)
}\\
\hline
0
 & 
3.00000000000000
 &  & \\
\hline
1
 & 
2.73251570951922
 & 
0.10052257507862
 & 
1.404
\\
\hline
2
 & 
2.63199313444060
 & 
0.01373904283351
 & 
1.359
\\
\hline
3
 & 
2.61825409160709
 & 
0.00024006192208
 & 
1.273
\\
\hline
4
 & 
2.61801402968501
 & 
0.00000007236005
 & 
1.256
\\
\hline
5
 & 
2.61801395732496
 & 
0.00000000000001
 & 
1.272
\\
\hline\end{tabulary}


Við metum \(e_n\approx |x_{n+1}-x_n|\) og þar af leiðandi er
\begin{gather}
\begin{split}|e_n|/|e_{n-1}|^2\approx |x_{n+1}-x_n|/|x_n-x_{n-1}|^2.\end{split}\notag
\end{gather}
Við sjáum að hlutfallið \(|x_{n+1}-x_n|/|x_n-x_{n-1}|^2\) helst
stöðugt og því ályktum við að aðferðin sé ferningssamleitin.


\subsection{Útreikningur á samleitnistigi}
\label{kafli01:utreikningur-a-samleitnistigi}
Skoðum lítið dæmi um útreikninga á samleitnistigi.

Eftirfarandi runa stefnir á \(\sqrt 3\).

\begin{tabulary}{\linewidth}{|L|L|}
\hline
\textsf{\relax 
\(n\)
} & \textsf{\relax 
\(x_n\)
}\\
\hline
0
 & 
2.000000000000000
\\
\hline
1
 & 
1.666666666666667
\\
\hline
2
 & 
1.727272727272727
\\
\hline
3
 & 
1.732142857142857
\\
\hline
4
 & 
1.732050680431722
\\
\hline
5
 & 
1.732050807565499
\\
\hline\end{tabulary}


Er samleitnistigið \(1.618\)?

Ef ekki, hvert er þá samleitnistigið?

Ef miðað er við að runan \((x_n)\) sé ofurlínulega
samleitin, þá er eðlilegt að taka \(e_n\approx x_{n+1}-x_n\) sem mat
á skekkjunni \(e_n=\sqrt 3-x_n\) í \(n\)-ta ítrekunarskrefinu.

Við byrjum á því að kanna hvernig tilgátan um að samleitnistigið kemur
út á þessum tölum með \(e_n=x_{n+1}-x_n\):

\begin{tabulary}{\linewidth}{|L|L|L|L|}
\hline
\textsf{\relax 
\(n\)
} & \textsf{\relax 
\(x_n\)
} & \textsf{\relax 
\(|e_n|\)
} & \textsf{\relax 
\(|e_n|/|e_{n-1}|^{1.618}\)
}\\
\hline
0
 & 
2.000000000000000
 & 
3.3333\(\cdot 10^{-1}\)
 & \\
\hline
1
 & 
1.666666666666667
 & 
6.0606\(\cdot 10^{-2}\)
 & 
3.5851\(\cdot 10^{-1}\)
\\
\hline
2
 & 
1.727272727272727
 & 
4.8701\(\cdot 10^{-3}\)
 & 
4.5439\(\cdot 10^{-1}\)
\\
\hline
3
 & 
1.732142857142857
 & 
9.2177\(\cdot 10^{-5}\)
 & 
5.0837\(\cdot 10^{-1}\)
\\
\hline
4
 & 
1.732050680431722
 & 
1.2713\(\cdot 10^{-7}\)
 & 
4.3004\(\cdot 10^{-1}\)
\\
\hline
5
 & 
1.732050807565499
 &  & \\
\hline\end{tabulary}


Tveimur síðustu tölunum í aftasta dálki ber ekki nógu vel saman, svo það
er vafasamt hvort talan \(1.618\) er rétta samleitnistigið.

Ef \((x_n)\) er samleitin af stigi \(\alpha\), þá gildir
\(\lim_{n\to \infty}|e_{n+1}|/|e_n|^\alpha=\lambda\), þar sem
\(\lambda>0\). Þar með höfum við nálgunarjöfnu ef \(n\) er nógu
stórt,
\begin{gather}
\begin{split}\dfrac{|e_{n+1}|}{|e_n|^\alpha} \approx
    \dfrac{|e_{n+2}|}{|e_{n+1}|^\alpha}
    \qquad \text{ þá og því aðeins að } \qquad
    \dfrac{|e_{n+1}|}{|e_{n+2}|} \approx
    \bigg|\dfrac{e_{n}}{e_{n+1}} \bigg|^\alpha.\end{split}\notag
\end{gather}
Ef við lítum á þetta sem jöfnu og leysum út \(\alpha\), þá fáum við
\begin{gather}
\begin{split}\alpha_n =
    \dfrac{\ln(|e_{n+1}|/|e_{n+2}|)}{\ln(|e_{n}|/|e_{n+1}|)}.\end{split}\notag
\end{gather}
Við getum reiknað út þrjú gildi á \(\alpha\) úr þeim gögnum sem við
höfum, \(\alpha_0= 1.479\), \(\alpha_1 = 1.573\) og
\(\alpha_2=1.660\).

Ef við endurtökum útreikninga okkar hér að framan með \(1.660\) í
stað \(1.618\), þá fæst

\begin{tabulary}{\linewidth}{|L|L|L|L|}
\hline
\textsf{\relax 
\(n\)
} & \textsf{\relax 
\(p_n\)
} & \textsf{\relax 
\(|e_n|\)
} & \textsf{\relax 
\(|e_n|/|e_{n-1}|^{1.660}\)
}\\
\hline
0
 & 
2.000000000000000
 & 
3.3333\(\cdot 10^{-1}\)
 & \\
\hline
1
 & 
1.666666666666667
 & 
6.0606\(\cdot 10^{-2}\)
 & 
3.7551\(\cdot 10^{-1}\)
\\
\hline
2
 & 
1.727272727272727
 & 
4.8701\(\cdot 10^{-3}\)
 & 
5.1143\(\cdot 10^{-1}\)
\\
\hline
3
 & 
1.732142857142857
 & 
9.2177\(\cdot 10^{-5}\)
 & 
6.3639\(\cdot 10^{-1}\)
\\
\hline
4
 & 
1.732050680431722
 & 
1.2713\(\cdot 10^{-7}\)
 & 
6.3639\(\cdot 10^{-1}\)
\\
\hline
5
 & 
1.732050807565499
 &  & \\
\hline\end{tabulary}


Tölunum neðst í aftasta dálki ber saman með fimm réttum stöfum og því
ályktum við að \(1.660\) sé nær því að vera rétta samleitnistigið.


\section{Meira um skekkjur}
\label{kafli01:meira-um-skekkjur}
\index{markverðir stafir}

\subsection{Skilgreining: Markverðir stafir}
\label{kafli01:skilgreining-markverir-stafir}\label{kafli01:index-10}
Gerum ráð fyrir að \(r\neq 0\), þá segjum við að \(x\) sé
\emph{nálgun á} \(r\) \emph{með} \(t\) \emph{markverðum stöfum (e. significant
digits)} ef
\begin{gather}
\begin{split}\frac{|r-x|}{|r|} \leq 10^{-t}.\end{split}\notag
\end{gather}
Getum útfært þetta aðeins ítarlegra. Ef
\begin{gather}
\begin{split}10^{-(t+1)} < \frac{|r-x|}{|r|} \leq 10^{-t}.\end{split}\notag
\end{gather}
þá segjum við að nálgunin á \(r\) með \(x\) sé rétt með að
minnsta kosti \(t\) markverðum stöfum og að hámarki með \(t+1\)
markverðum stöfum.

Athugið að ef \(e\) er minnsta heila talan þannig að
\(|r|<10^e\), þá gefur seinni ójafnan matið
\begin{gather}
\begin{split}|r-x| = 0.0\dots 0 a_t a_{t+1}\ldots \ \cdot\  10^e,\end{split}\notag
\end{gather}
þar sem núllin aftan við punkt eru \(t\) talsins.

Einnig er hægt að útfæra þetta fyrir aðrar grunntölur en 10.

\index{skekkja!styttingarskekkja}\index{annars stigs jafna}

\subsection{Úrlausn annars stigs jöfnu}
\label{kafli01:urlausn-annars-stigs-jofnu}\label{kafli01:index-11}
Þegar núllstöðvar annars stigs jöfnunnar \(ax^2+bx+c=0\) eru
reiknaðar út úr formúlunni
\begin{gather}
\begin{split}x = \dfrac{-b\pm\sqrt{b^2-4ac}}{2a},\end{split}\notag
\end{gather}
verður til styttingarskekkja ef \(b^2\) er miklu stærra heldur en
\(4ac\) vegna \(|b|\approx\sqrt{b^2-4ac}\). Við komumst hjá
þessum vandræðum með því að líta á margliðuna fullþáttaða
\(a(x-x_1)(x-x_2)\) og notfæra okkur að núllstöðvarnar \(x_1\)
og \(x_2\) uppfylla \(x_1x_2=c/a\).

Ef \(b>0\), þá reiknum við \(x_1\) fyrst út úr formúlunni
\begin{gather}
\begin{split}x_1 = \dfrac{-b-\sqrt{b^2-4ac}}{2a}
    \quad \text{ og  síðan } \quad
    x_2 = \dfrac{c/a}{x_1}.\end{split}\notag
\end{gather}
Ef aftur á móti \(b<0\), þá reiknum við fyrst \(x_1\) út úr
formúlunni
\begin{gather}
\begin{split}x_1 = \dfrac{-b+\sqrt{b^2-4ac}}{2a}
    \qquad \text{ og síðan } \qquad
    x_2 = \dfrac{c/a}{x_1}.\end{split}\notag
\end{gather}
Ef \(b^2\approx 4ac\) þá lendum við í styttingarskekkjum, en við
neyðumst til þess að lifa með þeim.

\index{skekkja!gagnaskekkja}

\subsection{Áhrif gagnaskekkju}
\label{kafli01:index-12}\label{kafli01:ahrif-gagnaskekkju}
Hugsum okkur að við séum að finna nálgun á núllstöð falls
\(x\mapsto f(x,\alpha)\). Við viljum finna nálgun \(x\) á
lausninni \(r=r(\alpha)\) sem uppfyllir
\begin{gather}
\begin{split}f(r,\alpha) = 0\end{split}\notag
\end{gather}
og við lítum á \(\alpha\) sem stika (t.d. náttúrulegur fasti).

Gerum ráð fyrir að \(\alpha_0\) sé nálgun á \(\alpha\) og að við
þekkjum nálgun á \(r(\alpha_0)\) sem er lausn á jöfnunni
\(f(x,\alpha_0)=0\).

Við viljum athuga hversu mikil áhrif nálgun á \(\alpha\) með
\(\alpha_0\) hefur á lausnina okkar, þ.e. við þurfum að meta
skekkjuna \(r(\alpha)-r(\alpha_0)\).

Ef við gefum okkur að \(f\) sé samfellt deildanlegt í grennd um
punktinn \((x_0,\alpha_0)\), þar sem \(x_0=r(\alpha_0)\) og
\({\partial}_xf(x_0,\alpha_0)\neq 0\), þá segir setningin um fólgin
föll að til sé grennd \(I\) um punktinn \(\alpha_0\) í
\({\mathbb R}\) og samfellt deildanlegt fall
\(r:I\to {\mathbb R}\), þannig að \(r(\alpha_0)=x_0\) og
\(f(r(\alpha),\alpha)=0\) fyrir öll \(\alpha\in I\).

Með öðrum orðum má segja að við getum alltaf leyst jöfnuna
\(f(x,\alpha)=0\) með tilliti til \(x\) þannig að út komi lausn
\(x=r(\alpha)\) sem er samfellt diffranlegt fall af \(\alpha\).

Keðjureglan gefur okkur nú gildi afleiðunnar, því af jöfnunni
\(f(r(\alpha),\alpha)=0\) leiðir að fallið
\(I \ni \alpha \mapsto f(r(\alpha),\alpha)\) er fast, þannig að
\begin{gather}
\begin{split}0 =\frac {\partial}{\partial \alpha}f(r(\alpha),\alpha) = f_x'(r(\alpha), \alpha)\cdot r'(\alpha)
    + f_{\alpha}'(r(\alpha),
    \alpha).\end{split}\notag
\end{gather}
Þetta gefur
\begin{gather}
\begin{split}r'(\alpha) = \frac{-f_{\alpha}'(r(\alpha),\alpha)}
        {f_x'(r(\alpha),\alpha)}.\end{split}\notag
\end{gather}
Nú látum við \(e\) tákna skekkjuna í nálguninni á \(\alpha\) með
\(\alpha_0\), \(e=\alpha-\alpha_0\). Þá fáum við skekkjumatið
\begin{gather}
\begin{split}r(\alpha) - r(\alpha_0) \approx r'(\alpha_0)\cdot e
    = \frac{-f_{\alpha}'(r(\alpha_0),\alpha_0)}
        {f_x'(r(\alpha_0),\alpha_0)}\cdot e\end{split}\notag
\end{gather}
og jafnframt mat á hlutfallslegri skekkju
\begin{gather}
\begin{split}\dfrac{|r(\alpha) - r(\alpha_0)|}
    {|r(\alpha)|} \approx \frac{|f_{\alpha}'(r(\alpha_0),\alpha_0)|}
    {|r(\alpha_0)f_x'(r(\alpha_0),\alpha_0)|}\cdot
    |e|.\end{split}\notag
\end{gather}

\subsection{Sýnidæmi}
\label{kafli01:id2}
Við skulum nú líta á það verkefni að finna nálgun á minnstu jákvæðu
lausn jöfnunnar \(\sin(\pi x)=1-e^{-x}\), þar sem við gerum ráð
fyrir því að þurfa að nálga \(\pi\) með \(3.14\).

Okkur eru gefnar niðurstöður úr nálguninni með einhverri aðferð. Við
setjum \(f(x,\alpha)=1-e^{-x}-\sin(\alpha x)\) og fáum

\begin{tabulary}{\linewidth}{|L|L|L|L|}
\hline
\textsf{\relax 
\(n\)
} & \textsf{\relax 
\(x_n\)
} & \textsf{\relax 
\(|x_{n+1}-x_n|\)
} & \textsf{\relax 
\(\frac{|x_{n+1}-x_n|}{|x_n-x_{n-1}|^2}\)
}\\
\hline
0
 &  &  & 
0.8
\\
\hline
1
 & 
0.81276894538752
 & 
0.00014017936338
 & 
0.8597
\\
\hline
2
 & 
0.81262876602414
 & 
0.00000001621651
 & 
0.8253
\\
\hline
3
 & 
0.81262874980763
 & 
0.00000000000000
 & 
0.8444
\\
\hline\end{tabulary}


Hér er \(\alpha=\pi\) og \(\alpha_0=3.14\) og þar með
\(|e|<0.0016\).

Hlutafleiðurnar eru \(f'_x(x,\alpha)=e^{-x}-\alpha\cos(\alpha x)\)
og \(f'_\alpha(x,\alpha)=-x\cos(\alpha x)\).

Við stingum tölunum okkar inn í matið og notum punktinn
\((x_3,\alpha_0)=(0.8126,3.14)\). Það gefur
\begin{gather}
\begin{split}\begin{aligned}
    r(\pi)-r(3.14)&\approx r'(3.14) \cdot e\\
    &\approx
    \dfrac{|0.8126\cdot \cos(0.8126\cdot 3.14)|}{|e^{-0.8126}-3.14
    \cdot \cos(0.8126 \cdot 3.14)|}\
    0.0016 \\
    &\approx 0.4\cdot 10^{-3}\end{aligned}\end{split}\notag
\end{gather}
Þetta mat segir okkur að við eigum að gera ráð fyrir að áhrif
gagnaskekkjunnar séu þau að við fáum lausn með þremur réttum stöfum,
\(r(\pi) \approx 0.813\). Nálgun okkar á minnstu jákvæðu lausn
jöfnunnar \(\sin(\pi
x)=1-e^{-x}\) er því \(0.813\).

\index{O-ritháttur}

\subsection{\(O\)-ritháttur}
\label{kafli01:rithattur}\label{kafli01:index-13}
Látum \(f\) og \(g\) vera tvö föll sem skilgreind eru á bili
\(I \subset \mathbb{R}\) og látum \(c\) vera tölu á \(I\) eða annan hvorn
endapunkt \(I\).

Við segjum að \(f(t)\) \emph{sé stórt O af} \(g(t)\) og skrifum
\begin{gather}
\begin{split}f(t) = O(g(t)), \qquad t \rightarrow c,\end{split}\notag
\end{gather}
ef til er fasti \(C>0\) þannig að ójafnan
\begin{gather}
\begin{split}|f(t)| \leq C|g(t)|\end{split}\notag
\end{gather}
gildi fyrir öll \(t\) í einhverri grennd um \(c\).

Athugið að grennd um \(c=+\infty\) er bil af gerðinni
\(]\alpha,+\infty[\) og grennd um \(c=-\infty\) er bil af
gerðinni \(]-\infty,\alpha[\).


\subsection{\(O\)-ritháttur og skekkja í Taylor-nálgnum}
\label{kafli01:rithattur-og-skekkja-i-taylor-nalgnum}
Oft er \(O\)-ritháttur notaður þegar fjallað er um skekkjur í
Taylor-nálgunum,
\begin{gather}
\begin{split}\begin{aligned}
    f(x) - T_n f(x;c) &= f(x) - f(c) - f'(x-c) - \cdots
    - \frac{f^{(n)}(c)}{n!}(x-c)^n \\
    &= \frac{f^{(n+1)}(\xi)}{(n+1)!}(x-c)^{n+1} =
    O\big((x-c)^{n+1}\big),  \quad x \to c\end{aligned}\end{split}\notag
\end{gather}

\subsection{Sýnidæmi}
\label{kafli01:id3}
Það eru til haugar af dæmum, sem við þekkjum vel.

Setning Taylors gefur okkur:
\begin{gather}
\begin{split}\begin{gathered}
    x - \sin x = O(x^3), \quad x \to 0\\
    x - \frac{x^3}{3!} - \sin x = O(x^5), \quad x \to 0\end{gathered}\end{split}\notag
\end{gather}
\index{O-ritháttur}

\subsection{\(O\)-ritháttur fyrir runur}
\label{kafli01:rithattur-fyrir-runur}\label{kafli01:index-14}
Látum nú \((a_n)\) og \((b_n)\) vera tvær talnarunur. Við segjum
að \(a_n\) \emph{sé stórt O af} \(b_n\) og skrifum
\begin{gather}
\begin{split}a_n = O(b_n),\end{split}\notag
\end{gather}
ef til er fasti \(C>0\) þannig að ójafnan
\begin{gather}
\begin{split}|a_n| \leq C|b_n|\end{split}\notag
\end{gather}
gildi fyrir öll \(n=0,1,2,3,\dots\).


\subsection{Tvö sýnidæmi}
\label{kafli01:tvo-synidaemi}\begin{itemize}
\item {} 
Út frá Taylor-röðinni fyrir \(\cos x\) fáum við að
\begin{gather}
\begin{split}\cos(1/n)-1+1/(2n^2) = O(1/n^4)\end{split}\notag
\end{gather}
\item {} 
Út frá
\begin{gather}
\begin{split}\sqrt{n+1}-\sqrt n = \dfrac{1}{\sqrt{n+1}+\sqrt n} \leq \frac{1}{2\sqrt n}\end{split}\notag
\end{gather}
sjáum við að
\begin{gather}
\begin{split}\sqrt{n+1}-\sqrt n = O\big(\dfrac 1{\sqrt n}\big)\end{split}\notag
\end{gather}
\end{itemize}


\section{Fleytitalnakerfið}
\label{kafli01:fleytitalnakerfi}
\index{fleytitölur}

\subsection{Framsetning á tölum}
\label{kafli01:index-15}\label{kafli01:framsetning-a-tolum}
Ef \(r\) er rauntala frábrugðin \(0\) og \(\beta\) er
náttúrleg tala, \(2\) eða stærri, þá er til einhlýtt ákvörðuð
framsetning á \(r\) af gerðinni
\begin{gather}
\begin{split}r =
    \pm (0.d_1d_2\dots d_kd_{k+1}\dots)_\beta\times \beta^e\end{split}\notag
\end{gather}
þar sem \(e\) er heiltala og \(d_j\) eru heiltölur
\begin{itemize}
\item {} 
\(1\leq d_1<\beta\),

\item {} 
\(0\leq d_j<\beta\), \(j=2,3,4,\dots\).

\end{itemize}

Tölvur reikna ýmist í \emph{tvíundarkerfi} með \(\beta=2\) eða í
\emph{sextánundarkerfi} með \(\beta=16\), en við mannfólkið með okkar tíu
fingur reiknum í \emph{tugakerfi} með \(\beta=10\).

\index{fleytitölur!mantissa}\index{fleytitölur!markverðir stafir}

\subsection{Mantissa}
\label{kafli01:mantissa}\label{kafli01:index-16}
Formerkið og runan
\begin{gather}
\begin{split}\pm(0.d_1d_2\dots d_kd_{k+1}\dots)_\beta =
    \pm\sum_{j=1}^\infty \dfrac{d_j}{\beta^j}\end{split}\notag
\end{gather}
nefnist \emph{mantissa} tölunnar \(r\).

Við skrifum
\begin{gather}
\begin{split}(0.d_1d_2\dots d_k)_\beta =
    \sum_{j=1}^k \dfrac{d_j}{\beta^j}\end{split}\notag
\end{gather}
ef \(d_{k+1} = d_{k+2} = \cdots = 0\) og segjum þá að talan
\(r\) hafi \(k\)-stafa mantissu.


\subsection{Markverðir \(\beta\)-stafir}
\label{kafli01:markverir-stafir}
Ef rauntalan \(x\) er nálgun á \(r\), þá segjum við að \(x\)
sé nálgun á \(r\) með \emph{að minnsta kosti} \(t\) \emph{markverðum}
\(\beta\) \emph{-stöfum} ef
\begin{gather}
\begin{split}\dfrac{|r-x|}{|r|}\leq \beta^{-t}.\end{split}\notag
\end{gather}
Ef við höfum að auki að
\begin{gather}
\begin{split}\beta^{-t-1}<\dfrac{|r-x|}{|r|}\leq \beta^{-t}.\end{split}\notag
\end{gather}
þá segjum við að \(x\) sé nálgun á \(r\) með \(t\)
\emph{markverðum} \(\beta\) \emph{-stöfum}.

Athugið að ef \(e\) er minnsta heila talan þannig að
\(|r|<\beta^e\), þá gefur seinni ójafnan matið
\begin{gather}
\begin{split}|r-x| = (0.0\dots 0a_ta_{t+1}\dots)_\beta \times \beta^e,\end{split}\notag
\end{gather}
þar sem núllin aftan við punkt eru \(t\) talsins.

\index{afrúningur}\index{afskurður}

\subsection{Afrúningur talna}
\label{kafli01:afruningur-talna}\label{kafli01:index-17}
Ef \(r\) er sett fram á stöðluðu \(\beta\)-fleytitöluformi, þá
nefnist talan
\begin{gather}
\begin{split}x = (\pm 0.d_1d_2\dots d_k)_\beta\times \beta^e\end{split}\notag
\end{gather}
\emph{afskurður tölunnar} \(r\) \emph{við} \(k\) \emph{-ta aukastaf} \(r\), en
talan
\begin{gather}
\begin{split}x = \begin{cases}
    \pm (0.d_1d_2\dots d_k)_\beta\times \beta^e, &
    d_{k+1}<\beta/2,\\
    \pm ((0.d_1d_2\dots d_k)_\beta+\beta^{-k})\times \beta^e,
    &d_{k+1}\geq \beta/2.
    \end{cases}\end{split}\notag
\end{gather}
nefnist \emph{afrúningur tölunnar} \(r\) \emph{við} \(k\) \emph{-ta aukastaf}.

Við köllum þessar aðgerðir \emph{afskurð} (e. chopping) og \emph{afrúning}
(e. rounding).


\subsection{Fleytitölukerfi}
\label{kafli01:fleytitolukerfi}
\emph{Fleytitölukerfi} er endanlegt hlutmengi í \({\mathbb  R}\), sem
samanstendur af öllum tölum
\begin{gather}
\begin{split}\pm (0.d_1d_2\dots d_k)_\beta\times \beta^e\end{split}\notag
\end{gather}
þar sem \(d_j\) eru heiltölur eins og áður var lýst, \(k\) er
föst tala og við höfum mörk á veldisvísinum \(m\leq e\leq M\).

Allar tölvur vinna með eitthvert fleytitölukerfi, oftast með grunntölu
\(\beta=2\) eða \(\beta=16\) eins og áður sagði.

Eftir hverja aðgerð í tölvunni þarf að nálga útkomuna með \emph{afskurði} eða
\emph{afrúningu}.

Ef við förum ekki varlega þá getur þetta magnað upp skekkju.

Sjá {\hyperref[kafli01:urlausn-annars-stigs-jofnu]{Úrlausn annars stigs jöfnu}}.


\subsection{IEEE staðlar}
\label{kafli01:ieee-stalar}\begin{itemize}
\item {} 
Single: \(\beta = 2, k=24, m=-125\) og \(M = 128\),

\item {} 
Double: \(\beta = 2, k=53, m=-1021\) og \(M = 1024\).

\end{itemize}


\subsection{Útreikningur í tugakerfi}
\label{kafli01:utreikningur-i-tugakerfi}
Þegar reiknað er í tugakerfi er tölurnar afrúnaðar við \(k\)-ta
aukastaf ef skekkjan í nálgun á þeim er minni en
\(\frac 12\times 10^{-k}\). Ef
\begin{gather}
\begin{split}\dfrac{|r-x|}{|r|}<10^{-k-1}\end{split}\notag
\end{gather}
þá treystum við öllum \(k\) stöfum mantissunnar, en ef
\begin{gather}
\begin{split}\dfrac{|r-x|}{|r|}>10^{-k+q},\end{split}\notag
\end{gather}
þá eru síðustu \(q\) stafir mantissunnar marklausir auk þess sem
vænta má nokkurs fráviks í \(d_{k-q}\).


\chapter{Núllstöðvar}
\label{kafli02::doc}\label{kafli02:nullstovar}
\emph{Build a man a fire, and he'll be warm for a day. Set a man on fire, and he'll
be warm for the rest of his life.}

- Terry Pratchett, Jingo


\section{Nálgun á núllstöð}
\label{kafli02:nalgun-a-nullsto}
\index{núllstöð}

\subsection{Skilgreining}
\label{kafli02:skilgreining}\label{kafli02:index-0}
Munum að talan \(p\in I\) sögð vera \emph{núllstöð} fallsins
\(f:I\to {\mathbb  R}\) ef
\begin{gather}
\begin{split}f(p)=0.\end{split}\notag
\end{gather}

\subsection{Dæmi}
\label{kafli02:daemi}
Það er auðvelt að finna núllstöðvar (\textbf{rót}) annars stigs margliðu
\(ax^2+bx+c\), því
\begin{gather}
\begin{split}ax^2+bx+c = 0\end{split}\notag
\end{gather}
ef
\begin{gather}
\begin{split}x = \frac{-b \pm \sqrt{b^2-4ac}}{2a}.\end{split}\notag
\end{gather}
Svipaðar formúlur eru til fyrir núllstöðvar þriðja og fjórða stigs margliða.
Einnig þekkjum við núllstöðvar hornafalla.


\subsection{Athugasemd}
\label{kafli02:athugasemd}
Almennt er hins vegar erfitt að finna núllstöðvar falla.
Til dæmis er ekki til almenn formúla fyrir núllstöðvar margliða af stigi 5 og hærra
(sjá \href{https://en.wikipedia.org/wiki/Abel–Ruffini\_theorem}{Abel-Ruffini setningin}).

Eins er ekki hægt treysta á það að geta fundið nákvæmlega núllstöðvar almennra falla með því
að nota þekkingu okkar á algebru og stærðfræðigreinginu. Hverjar (og hversu margar) eru t.d. núllstöðvar
\begin{gather}
\begin{split}e^x + x^3?\end{split}\notag
\end{gather}
Aðferðirnar í þessum kafla ganga út á að finna nálgun á núllstöðvum falla og í sumum tilvikum
hjálpa þær okkur einnig að sýna fram á tilvist núllstöðva (sem er ekki alltaf sjálfgefin).


\section{Helmingunaraðferð}
\label{kafli02:helmingunarafer}
Fyrsta aðferðin til að finna núllstöðvar sem við skoðum kallast
helmingunaraðferð (e. bisection method).


\subsection{Milligildissetningin}
\label{kafli02:milligildissetningin}
Ef \(f\) er samfellt á \([a,b]\) og \(y\) er einhver
tala á milli \(f(a)\) og \(f(b)\), þá er til \(c\)
þannig að \(a < c < b\) og \(f(c) = y\).


\subsection{Afleiðing}
\label{kafli02:afleiing}
Svo ef við höfum \(a\) og \(b\) þannig að \(a < b\) og
þannig að \(f(a)\) og \(f(b)\) hafi ólík formerki, þá hefur
\(f\) núllstöð \(p\) á bilinu \([a,b]\).

\index{helmingunaraðferð}
Notum okkur þetta til þess að finna rætur.
\begin{enumerate}
\item {} 
Látum \(x = \frac 12(a+b)\) vera miðpunkt \([a,b]\).

\item {} 
Reiknum \(f(x)\), þá geta þrjú tilvik komið upp:
\begin{enumerate}
\item {} 
\(f(x) = 0\) og leitinni að rót er lokið.

\item {} 
\(f(a)\) og \(f(x)\) hafa sama formerki, þannig að við
leitum að rót á bilinu \([x,b]\).

\item {} 
\(f(x)\) og \(f(b)\) hafa sama formerki, þannig að við
leitum að rót á bilinu \([a,x]\).

\end{enumerate}

\end{enumerate}

Í tilviki (ii) segir milligildissetningin að \(f\) hafi rót á bilinu
\([x,b]\), og í tilviki (iii) er rótin á bilinu \([a,x]\). Þá
getum við farið aftur í skref 1, nema með helmingi minna bil en áður.

Með því að ítreka þetta ferli \(n\) sinnum fáum við minnkandi runu
af bilum
\begin{gather}
\begin{split}[a,b]=[a_1,b_1]\supset [a_2,b_2]\supset \cdots\supset [a_n,b_n].\end{split}\notag
\end{gather}
Billengdin helmingast í hverju skrefi og milligildissetningin segir okkur að það sé núllstöð á öllum bilunum.

Rununa af bilunum
\begin{gather}
\begin{split}[a,b]= [a_1,b_1]\supset \cdots\supset [a_n,b_n]\supset \cdots\end{split}\notag
\end{gather}
skilgreinum við með ítrun og notum til þess rununa \(x_n=\frac 12(a_n+b_n)\).

Setjum \(a_0=a\), \(b_0=b\), og \(x_0=\frac 12(a+b)\).

Gefið er \(x_0,\dots,x_n\). Reiknum \(f(x_n)\).
\begin{enumerate}
\item {} 
Ef \(f(x_n) = 0\), þá er núllstöð fundin og við hættum.

\item {} 
Ef \(f(x_n)\) og \(f(a_n)\) hafa sama formerki, þá setjum við \(a_{n+1}=x_n\), \(b_{n+1}=b_n\), og  \(x_{n+1}=\frac 12(a_{n+1}+b_{n+1})\)

\item {} 
annars setjum við \(a_{n+1}=a_n\), \(b_{n+1}=x_n\) og \(x_{n+1}=\frac 12(a_{n+1}+b_{n+1})\).

\end{enumerate}


\subsection{Skekkjumat í helmingunaraðferð}
\label{kafli02:skekkjumat-i-helmingunarafer}
Ef við látum miðpunktinn \(p_n=\frac 12(a_n+b_n)\) vera
nálgunargildi okkar fyrir núllstöð fallsins \(f\) í bilinu
\([a_n,b_n]\), þá er skekkjan í nálguninni
\begin{gather}
\begin{split}e_n=p-p_n\end{split}\notag
\end{gather}
og við höfum skekkjumatið
\begin{gather}
\begin{split}|e_n|\leq  \dfrac{b_n - a_n}{2}\
= \frac{b_{n-1}-a_{n-1}}{2^2} = \ldots = \dfrac{b_1-a_1}{2^{n}},\end{split}\notag
\end{gather}
það er
\begin{gather}
\begin{split}|e_n| < \dfrac{b-a}{2^{n}}.\end{split}\notag
\end{gather}

\subsection{Fyrirframmat á skekkju}
\label{kafli02:fyrirframmat-a-skekkju}
Nú er auðvelt að meta hversu margar ítrekanir þarf að framkvæma til þess
að nálgunin lendi innan gefinna skekkjumarka.

Ef \(\varepsilon>0\) er gefið og við viljum að
\(|e_n|< \varepsilon\), þá dugir að
\begin{gather}
\begin{split}|e_n|\leq \dfrac{b-a}{2^{n}} <\varepsilon.\end{split}\notag
\end{gather}
Seinni ójafnan jafngildir því að
\begin{gather}
\begin{split}n>\dfrac{\ln\big((b-a)/\varepsilon\big)}{\ln 2}.\end{split}\notag
\end{gather}

\begin{center}
\includegraphics[width=8 cm,keepaspectratio=true]{bisection.png}

\end{center}
\index{fastapunktsaðferð}\index{fastapunktur}

\section{Fastapunktsaðferð}
\label{kafli02:fastapunktsafer}\label{kafli02:index-2}
Næsta aðferð sem við skoðum kallast fastapunktsaðferð (e. fixed point method) og
er til að finna fastapunkta en ekki núllstöðvar. Það er hins vegar hægt að
nota hana til þess að finna núllstöðvar, sjá athugasemd hér að {\hyperref[kafli02:fastapunktar-nullstodvar]{\emph{neðan}}}.


\subsection{Skilgreining}
\label{kafli02:id1}
Látum \(f : [a,b] \to \mathbb R\) vera samfellt fall. Punktur
\(r \in [a,b]\) þannig að
\begin{gather}
\begin{split}f(r) = r\end{split}\notag
\end{gather}
kallast \emph{fastapunktur} fallsins \(f\).

\begin{notice}{note}{Athugasemd:}
Athugum að í fastapunktum skerast graf fallsins \(y=f(x)\) og línan
\(y=x\). Verkefnið að ákvarða fastapunkta fallsins \(r\) er því
jafngilt því að athuga hvar graf \(f\) sker línuna \(y=x\).
\end{notice}


\subsection{Tenging við núllstöðvar}
\label{kafli02:tenging-vi-nullstovar}\label{kafli02:fastapunktar-nullstodvar}
Verkefnið að finna fastapunkta fallsins \(g(x)\) er jafngilt því að
finna núllstöðvar fallsins \(f(x)=g(x)-x\).

Þannig að ef við viljum t.d. finna núllstöð \(f(x) = e^x + x^3\) þá er nóg að finna fastapunkt
fallsins \(g(x) = e^x + x^3 + x\).


\subsection{Reiknirit}
\label{kafli02:reiknirit}
\textbf{Byrjunarskref:}      Valin er tala \(x_0\in [a,b]\).

\textbf{Ítrunarskref:}       Ef \(x_0,\dots,x_n\) hafa verið valin, þá setjum við
\begin{quote}
\begin{gather}
\begin{split}x_{n+1}=f(x_n)\end{split}\notag
\end{gather}\end{quote}

\begin{notice}{note}{Athugasemd:}
Til þess að þetta sé vel skilgreind runa, þá verðum við að gera ráð
fyrir að \(f(x)\in [a,b]\) fyrir öll \(x\in [a,b]\). Þetta
skilyrði er einnig skrifað
\begin{gather}
\begin{split}f([a,b])\subset [a,b].\end{split}\notag
\end{gather}\end{notice}

\begin{notice}{note}{Athugasemd:}
Ef \(f\) er samfellt og runan er samleitin með markgildið \(r\), þá er
\begin{gather}
\begin{split}r=\lim_{n\to \infty}x_{n+1}=\lim_{n\to \infty}f(x_{n})
=f(\lim_{n\to \infty}x_{n})=f(r).\end{split}\notag
\end{gather}
Þetta segir okkur að \emph{ef} við getum séð til þess að runan verði
samleitin, þá er markgildið fastapunktur.
\end{notice}

\index{herping}

\subsection{Skilgreining: Herping}
\label{kafli02:index-3}\label{kafli02:skilgreining-herping}
Fall \(f:[a,b]\to {\mathbb  R}\) er sagt vera \emph{herping} ef til er
fasti \(\lambda\in [0,1[\) þannig að
\begin{gather}
\begin{split}|f(x)-f(y)|\leq \lambda|x-y| \qquad \text{ fyrir öll } x,y\in [a,b].\end{split}\notag
\end{gather}
\begin{notice}{note}{Athugasemd:}
Sérhver herping er samfellt fall.
\end{notice}


\subsection{Setning}
\label{kafli02:setning}
Ef \(f\) er deildanlegt fall á \(]a,b[\), þá gefur
meðalgildissetningin okkur til er \(\xi\) milli \(x\) og
\(y\) þannig að
\begin{gather}
\begin{split}f(x)-f(y)=f'(\xi)(x-y).\end{split}\notag
\end{gather}
Ef til er \(\lambda\in[0,1[\) þannig að \(|f'(x)|\leq \lambda\)
fyrir öll \(x\in [a,b]\), þá er greinilegt að \(f\) er herping.

\index{fastapunktsaðferð!fastapunktssetningin}

\subsection{Fastapunktssetningin}
\label{kafli02:index-4}\label{kafli02:fastapunktssetningin}
Látum \(f : [a,b] \to [a,b]\) vera herpingu. Þá hefur \(f\)
nákvæmlega einn fastapunkt \(r\) á bilinu \([a,b]\) og runan
\((x_n)\) þar sem
\begin{gather}
\begin{split}\begin{aligned}
  x_0 &\in [a,b] \quad \text{ getur verið hvaða tala sem er  og } \\
  x_{n+1} &= f(x_n), \quad n \geq 0,\end{aligned}\end{split}\notag
\end{gather}
stefnir á fastapunktinn.

Sönnunina brjótum við upp í nokkur skref.

\textbf{1. skref, herping hefur í mesta lagi einn fastapunkt}

Sönnum þetta með mótsögn.

Gerum ráð fyrir að \(r\) og \(s\) séu tveir ólíkir fastapunktar
á \([a,b]\). Þá er
\begin{gather}
\begin{split}|r - s| = |f(r) - f(s)|
  \leq \lambda |r - s| < |r - s|\end{split}\notag
\end{gather}
því \(\lambda < 1\). Þetta fær ekki staðist, þannig að fjöldi
fastapunkta er í mesta lagi einn

\textbf{2. skref, fallið} \(f\) \textbf{hefur fastapunkt:}

Látum \(g(x) = f(x) - x\), þá eru núllstöðvar \(g\) nákvæmlega
fastapunktar \(f\).

Þar sem \(a \leq f(x) \leq b\) fyrir öll \(x \in [a,b]\) er
\begin{gather}
\begin{split}\left\{ \begin{array}{c}
      g(a) = f(a) - a \geq 0 \\
      g(b) = f(b) - b \leq 0
  \end{array} \right.\end{split}\notag
\end{gather}
Ef annað hvort \(g(a) = 0\) eða \(g(b) = 0\) höfum við fundið
fastapunkt fallsins \(f\) og við getum hætt.

Ef hins vegar \(g(a) > 0\) og \(g(b) < 0\) þá hefur \(g\)
ólík formerki í endapunktum bilsins \([a,b]\) og hefur því núllstöð
\(r\) á bilinu skv. milligildissetninguninni. Þá er \(r\)
jafnframt fastapunktur \(f\).

Skref 1 og 2 sýna því að fallið \(f\) hefur nákvæmlega einn
fastapunkt á bilinu.

\textbf{3. skref, runan} \((x_n)\) \textbf{er samleitin}

Látum \(r\) vera ótvírætt ákvarðaða fastapunktinn á \([a,b]\).

Við notfærum okkur að \(f\) er herping og að \(r\) er
fastapunktur \(f\), þá fæst að fyrir sérhvert
\(k\in {\mathbb  N}\) þá er
\begin{gather}
\begin{split}|r - x_k| = |f(r) - f(x_{k-1})|  \leq \lambda |r - x_{k-1}|\end{split}\notag
\end{gather}
það er \(|r - x_k| \leq \lambda |r - x_{k-1}|\).

Með því að nota þetta \(n\)-sinnum þá fæst að
\begin{gather}
\begin{split}\begin{aligned}
    |r - x_n|   &\leq \lambda |r - x_{n-1}| & (k=n)\\
    &\leq \lambda^2 |r - x_{n-2}| & (k=n-1)\\
    &\vdots & \vdots\\
    &\leq \lambda^n |r - x_0| & (k=1).
\end{aligned}\end{split}\notag
\end{gather}
Þar sem \(\lambda < 1\) er því
\begin{gather}
\begin{split}\lim\limits_{n \to +\infty} |r - x_n|
\leq \lim\limits_{n \to +\infty} \lambda^n |r - x_0|
= 0,\end{split}\notag
\end{gather}
það er runan \(x_n\) stefnir á \(r\).


\subsection{Fastapunktsaðferð er að minnsta kosti línulega samleitin}
\label{kafli02:fastapunktsafer-er-a-minnsta-kosti-linulega-samleitin}
Af skilgreiningunni á rununni \(x_n\) leiðir beint að
\begin{gather}
\begin{split}|e_{n+1}|=|r-x_{n+1}|=|f(r)-f(x_n)|\leq \lambda|r-x_n|=\lambda|e_n|\end{split}\notag
\end{gather}
sem segir okkur að fastapunktsaðferð sé að minnsta kosti línulega
samleitin ef \(f\) er herping.


\begin{center}
\includegraphics[width=8 cm,keepaspectratio=true]{fixedpoint.png}

\end{center}
\index{sniðilsaðferð}

\section{Sniðilsaðferð}
\label{kafli02:index-5}\label{kafli02:sniilsafer}
Næst er aðferð til að finna núllstöðvar sem kallast \emph{sniðilsaðferð}
(e. \href{https://en.wikipedia.org/wiki/Secant\_method}{secant method})

Gefið er fallið \(f:[a,b]\to {\mathbb  R}\). Við ætlum að ákvarða
núllstöð \(f\), þ.e.a.s. \(p\in [a,b]\) þannig að
\begin{gather}
\begin{split}f(p)=0.\end{split}\notag
\end{gather}
Rifjum upp að \emph{sniðill} við graf \(f\) gegnum punktana
\((\alpha,f(\alpha))\) og \((\beta,f(\beta))\) er gefinn með
jöfnunni
\begin{gather}
\begin{split}y=f(\alpha)+f[\alpha,\beta](x-\alpha)\end{split}\notag
\end{gather}
þar sem hallatalan er
\begin{gather}
\begin{split}f[\alpha,\beta]=\dfrac{f(\beta)-f(\alpha)}{\beta-\alpha}
=\dfrac{f(\alpha)-f(\beta)}{\alpha -\beta}.\end{split}\notag
\end{gather}
Sniðillinn sker \(x\)-ásinn í punkti \(s\) þar sem
\begin{gather}
\begin{split}0=f(\alpha)+f[\alpha,\beta](s-\alpha) \quad  \text{sem jafngildir því að } \quad
s=\alpha-\dfrac{f(\alpha)}{f[\alpha,\beta]}.\end{split}\notag
\end{gather}

\subsection{Reiknirit}
\label{kafli02:id2}
\textbf{Byrjunarskref:} Giskað er á tvö gildi \(x_0\) og \(x_1\).

\textbf{Ítrunarskref:} Fyrir \(n>1\) þá er punkturinn \(x_{n+1}\)
skilgreindur sem skurðpunktur sniðilsins gegnum \((x_{n-1},f(x_{n-1}))\) og
\((x_n,f(x_n))\) við \(x\)-ás, þ.e.
\begin{gather}
\begin{split}x_{n+1}=x_n-\dfrac{f(x_n)}{f[x_n,x_{n-1}]}.\end{split}\notag
\end{gather}

\subsection{Samleitin runa stefnir á núllstöð \(f\)}
\label{kafli02:samleitin-runa-stefnir-a-nullsto}
Gefum okkur að runan \((x_n)\) sé samleitin að markgildinu
\(r\). Meðalgildissetningin segir okkur þá að til sé punktur
\(\eta_n\) á milli \(x_{n-1}\) og \(x_n\) þannig að
\begin{gather}
\begin{split}f[x_n,x_{n-1}]=f'(\eta_n),\end{split}\notag
\end{gather}
og greinilegt er að \(\eta_n\to r\).

Við fáum því
\begin{gather}
\begin{split}r=\lim_{n\to \infty}x_{n+1}=\lim_{n\to \infty}
\bigg(x_n-\dfrac{f(x_n)}{f'(\eta_n)}\bigg) =r-\dfrac{f(r)}{f'(r)}\end{split}\notag
\end{gather}
Þessi jafna jafngildir því að \(f(r)=0\).


\subsection{Skekkjumat í nálgun á \(f(x)\) með \(p_n(x)\)}
\label{kafli02:skekkjumat-i-nalgun-a-me}
Sniðilinn sem við notum er graf 1. stigs margliðunnar
\begin{gather}
\begin{split}p_n(x) = f(x_n) +
        \dfrac{f(x_{n-1})-f(x_n)}{x_{n-1}-x_n}(x-x_n)
        = f(x_n) + f[x_n,x_{n-1}](x-x_n)\end{split}\notag
\end{gather}
Samkvæmt skilgreiningu er \(p_n(x_{n+1}) = 0\) svo \(x_{n+1}\)
uppfyllir jöfnuna
\begin{gather}
\begin{split}x_{n+1} = x_n - \frac{f(x_n)}{f[x_n,x_{n-1}]}.\end{split}\notag
\end{gather}
Við þurfum að vita hver skekkjan er á því að nálga \(f(x)\) með
\(p_n(x)\).

Niðurstaðan er að fyrir sérhvert \(x \in [a,b]\) er til
\(\xi_n\) sem liggur í minnsta bilinu sem inniheldur \(x\),
\(x_n\) og \(x_{n-1}\) þannig að
\begin{gather}
\begin{split}f(x) - p_n(x) = \frac{1}{2}f''(\xi_n)(x-x_n)(x-x_{n-1})\end{split}\notag
\end{gather}
Ljóst er að matið gildir ef \(x=x_{n-1}\) eða \(x=x_n\).

Festum því punktinn \(x\) og gerum ráð fyrir að \(x\neq x_1\) og
\(x\neq x_n\).

Skilgreinum fallið
\begin{gather}
\begin{split}g(t)=f(t)-p_n(t)-\lambda(t-x_n)(t-x_{n-1})\end{split}\notag
\end{gather}
þar sem \(\lambda\) er valið þannig að \(g(x)=0\).

Látum nú \(\alpha<\beta<\gamma\) vera uppröðun á punktunum
\(x_{n-1}\), \(x_n\) og \(x\).

Fallið
\begin{gather}
\begin{split}g(t)=f(t)-p_n(t)-\lambda(t-x_n)(t-x_{n-1})\end{split}\notag
\end{gather}
hefur núllstöð í öllum punktunum þremur.

Meðalgildissetningin gefur þá að \(g'(t)\) hefur eina núllstöð í
punkti á bilinu \(]\alpha,\beta[\) og aðra í \(]\beta,\gamma[\).

Af því leiðir aftur að \(g''(t)\) hefur núllstöð, \(\xi_n\), í
\([\alpha,\gamma]\), sem er minnsta bilið sem inniheldur alla
punktana \(x_{n-1}\), \(x_n\) og \(x\).

Af þessu leiðir
\begin{gather}
\begin{split}0=g''(\xi_n)=f''(\xi_n)-2\lambda \quad \text{þþaa} \quad
\lambda=\tfrac 12 f''(\xi_n).\end{split}\notag
\end{gather}
Nú var \(\lambda\) upprunalega valið þannig að \(g(x)=0\). Þar
með er
\begin{gather}
\begin{split}f(x) - p_n(x) = \frac{1}{2}f''(\xi_n)(x-x_n)(x-x_{n-1}).\end{split}\notag
\end{gather}

\subsection{Skekkjumat í sniðilsaðferð}
\label{kafli02:skekkjumat-i-sniilsafer}
Skoðum hvað af þessu leiðir:

Nú er \(f(r) = 0\) og því
\begin{gather}
\begin{split}-p_n(r) = \frac{1}{2}f''(\xi_n)e_n\cdot e_{n-1}.\end{split}\notag
\end{gather}
Eins er
\begin{gather}
\begin{split}-p_n(r) = -f[x_n,x_{n-1}]e_{n+1}=-f'(\eta_n)e_{n+1},\end{split}\notag
\end{gather}
þar sem \(\eta_n\) fæst úr meðalgildissetningunni og liggur á milli
\(x_n\) og \(x_{n+1}\). Niðurstaðan verður því
\begin{gather}
\begin{split}e_{n+1} = \frac{-\frac{1}{2}f''(\xi_n)}
        {f[x_n, x_{n+1}]}
    e_ne_{n-1} = \frac{-\frac{1}{2}f''(\xi_n)}
        {f'(\eta_n)}e_ne_{n-1}\end{split}\notag
\end{gather}
það er
\begin{gather}
\begin{split}\lim_{n\to \infty}\dfrac{e_{n+1}}{e_ne_{n-1}}=
\lim_{n \to \infty} \frac{-\frac{1}{2}f''(\xi_n)}
        {f'(\eta_n)}
=
\frac{-\frac{1}{2}f''(r)}
        {f'(r)}.\end{split}\notag
\end{gather}

\subsection{Setning}
\label{kafli02:id3}
Ef sniðilsaðferð er samleitin, \(f\in C^2([a,b])\) (tvisvar
diffranlegt) og \(f'(r)\neq 0\), þá er sniðilsaðferðin ofurlínuleg.
\begin{gather}
\begin{split}\lim_{n\to \infty}\dfrac{|e_{n+1}|}{|e_n|} =
\lim_{n\to \infty}\dfrac{|e_{n+1}e_{n-1}|}{|e_ne_{n-1}|}=
\lim_{n \to \infty} \frac{|e_{n-1}\frac{1}{2}f''(r)|}
        {|f'(r)|} = 0\end{split}\notag
\end{gather}
Raunar þá er sniðilsaðferðin samleitin af stigi
\(\alpha = (1+\sqrt 5)/2 \approx 1,618\) og með
\(\lambda = \left(\frac{f''(r)}{2f'(r)}\right)^{\alpha -1}\).

\index{aðferð Newtons}\index{snertill}

\section{Aðferð Newtons}
\label{kafli02:index-6}\label{kafli02:afer-newtons}
Í sniðilsaðferðinni létum við \(x_{n+1}\) vera skurðpunkt sniðils
gegnum \((x_{n-1},f(x_{n-1}))\) og \((x_n,f(x_n))\) við
\(x\)-ás og fengum við rakningarformúluna
\begin{gather}
\begin{split}x_{n+1} = x_n - \frac{f(x_n)}{f[x_n,x_{n-1}]}.\end{split}\notag
\end{gather}
Aðferð Newtons er nánast eins, nema í stað sniðils tökum við snertil í
punktinum \((x_n,f(x_n))\).

Rakningarformúlan er eins, nema hallatalan verður \(f'(x_n)\) í stað
\(f[x_n,x_{n-1}]\)


\subsection{Reiknirit}
\label{kafli02:id4}
\textbf{Byrjunarskref:} Giskað er á eitt gildi \(x_0\).

\textbf{Ítrunarskref:} Gefin eru \(x_0,\dots,x_n\). Punkturinn \(x_{n+1}\) er
skurðpunktur snertils gegnum \((x_n,f(x_n))\) við \(x\)-ás,
\begin{gather}
\begin{split}x_{n+1}=x_n-\dfrac{f(x_n)}{f'(x_n)}.\end{split}\notag
\end{gather}

\subsection{Upprifjun}
\label{kafli02:upprifjun}
Munum að snertill við graf \(f\) í punktinum \(x_n\) er
\begin{gather}
\begin{split}y=f(x_n) + f'(x_n)(x-x_n),\end{split}\notag
\end{gather}
þessi lína sker \(x\)-ásinn (\(y=0\)) þegar
\(x=x_n - \frac{f(x_n)}{f'(x_n)}\).


\subsection{Samleitin runa stefnir á núllstöð \(f\)}
\label{kafli02:id5}
Gefum okkur að runan \((x_n)\) sé samleitin með markgildið
\(r\). Við fáum því
\begin{gather}
\begin{split}r=\lim_{n\to \infty}x_{n+1}=\lim_{n\to \infty}
\bigg(x_n-\dfrac{f(x_n)}{f'(x_n)}\bigg) =r-\dfrac{f(r)}{f'(r)}\end{split}\notag
\end{gather}
Þessi jafna jafngildir því að \(f(r)=0\).

Þannig að ef runan er samleitin þá fáum við núllstöð.


\subsection{Skekkjumat í nálgun á \(f(x)\) með \(p_n(x)\)}
\label{kafli02:id6}
Snertillinn við \(f\) í punktinum \(x_n\) er 1. stigs margliðan
\begin{gather}
\begin{split}p_n(x) = f(x_n) + f'(x_n)(x-x_n)\end{split}\notag
\end{gather}
Samkvæmt skilgreiningu er \(p_n(x_{n+1}) = 0\) svo \(x_{n+1}\)
uppfyllir jöfnuna
\begin{gather}
\begin{split}x_{n+1} = x_n - \frac{f(x_n)}{f'(x_n)}.\end{split}\notag
\end{gather}
Athugum að \(p_n\) er fyrsta Taylor nálgunin við fallið \(f\)
kringum \(x_n\). Setning Taylors gefur að til er \(\xi_n\) sem
liggur á milli \(r\) og \(x_n\) þannig að
\begin{gather}
\begin{split}f(r) - p_n(r) = \frac{1}{2}f''(\xi_n)(r-x_n)^2.\end{split}\notag
\end{gather}

\subsection{Skekkjumat í aðferð Newtons}
\label{kafli02:skekkjumat-i-afer-newtons}
Nú er \(f(r) = 0\) og því
\begin{gather}
\begin{split}-p_n(r) = \frac{1}{2}f''(\xi_n)e_n^2.\end{split}\notag
\end{gather}
Eins er fæst af skilgreiningunni á \(p_n\) að
\begin{gather}
\begin{split}-p_n(r) = -f'(x_n)e_{n+1}\end{split}\notag
\end{gather}
Niðurstaðan verður því
\begin{gather}
\begin{split}e_{n+1} = \frac{-\frac{1}{2}f''(\xi_n)}
        {f'(x_n)}e_n^2\end{split}\notag
\end{gather}

\begin{center}
\includegraphics[width=8 cm,keepaspectratio=true]{newton.png}

\end{center}

\subsection{Setning}
\label{kafli02:id7}
Ef aðferð Newtons fyrir fallið \(f\) er samleitin,
\(f\in C^2([a,b])\) og \(f'(r)\neq 0\), þá fáum við:
\begin{gather}
\begin{split}\lim_{n\to \infty}\dfrac{e_{n+1}}{e_n^2}=\frac{-\frac{1}{2}f''(r)}
        {f'(r)}\end{split}\notag
\end{gather}
Það þýðir að aðferð Newtons er ferningssamleitin.
\begin{gather}
\begin{split}\lim_{n\to \infty}\dfrac{e_{n+1}}{e_n^2}=
\lim_{n\to \infty}\frac{-\frac{1}{2}f''(\xi_n)}{f'(x_n)} =
\frac{-\frac{1}{2}f''(r)}{f'(r)}\end{split}\notag
\end{gather}
\begin{notice}{note}{Athugasemd:}
Athugið að það er ekki sjálfgefið að aðferð Newtons sé samleitin.

Auðvelt er að finna dæmi þar sem vond upphafságiskun \(x_0\) skilar
runu sem er ekki samleitin.
\end{notice}


\section{Samanburður á aðferðum}
\label{kafli02:samanburur-a-aferum}
\label{kafli02:samanburur-a-aferum}
\begin{tabular}{|l|l|l|}
	\hline
	\textsf{\relax 
		Aðferð
	} & \textsf{\relax 
	Samleitni
} & \textsf{\relax 
Stig samleitni
}\\


\hline 
{\hyperref[kafli02:helmingunarafer]{\emph{Helmingunaraðferð}}}
(e. \href{https://en.wikipedia.org/wiki/Bisection\_method}{bisection method})
&  Já, ef \(f(a)f(b)<0\) & 1, línuleg\\\hline
\hyperref[kafli02:fastapunktsafer]{\emph{Fastapunktsaðferð}}
(e. \href{https://en.wikipedia.org/wiki/Fixed-point\_iteration}{fixed point iteration})
& Ekki alltaf. En saml. ef \(f\) er herping  &  amk 1
\\\hline
\hyperref[kafli02:sniilsafer]{\emph{Sniðilsaðferð}}
(e. \href{https://en.wikipedia.org/wiki/Secant\_method}{secant method})
&  Ekki alltaf & \(\approx 1,618\), ef \(f'(r)\neq 0\)
\\\hline
{\hyperref[kafli02:afer-newtons]{\emph{Aðferð Newtons}}}
(e. \href{https://en.wikipedia.org/wiki/Newton\%27s\_method}{Newtons method})
& Ekki alltaf & 2, ef \(f'(r)\neq 0\)\\
\end{tabular}

\begin{notice}{warning}{Aðvörun:}
Þó að aðferð Newtons sé samleitin af stigi 2, en sniðilsaðferðin af
stigi u.þ.b. 1,618, þá er í vissum tilfellum hagkvæmara að nota
sniðilsaðferðina ef það er erfitt að reikna gildin á afleiðunni
\(f'\).
\end{notice}


\chapter{Brúun}
\label{kafli03:bruun}\label{kafli03::doc}
\emph{Over the centuries, mankind has tried many ways of combating the forces of evil...
prayer, fasting, good works and so on. Up until Doom, no one seemed to have thought
about the double-barrel shotgun. Eat leaden death, demon.}
-- Terry Pratchett


\section{Inngangur}
\label{kafli03:inngangur}
\index{brúun}\index{brúunarmargliða}\index{brúunarpunktur}

\subsection{Markmiðið}
\label{kafli03:markmii}\label{kafli03:index-0}
Viðfangsefni þessa kafla er að finna ferla sem ganga gegnum fyrirfram
gefna \emph{brúunarpunkta} \((x_0,y_0),\dots,(x_m,y_m)\) í planinu eða liggja nálægt
punktunum í einhverjum skilningi.

Fyrst viljum við finna graf margliðu \(p\) sem fer gegnum punktana.
Þá þurfum við að gefa okkur að \(x_i\neq x_j\) ef \(i\neq j\).

Við sýnum fram á að það sé alltaf hægt að finna margliðu \(p\) af
stigi \(\leq m\) sem uppfyllir \(p(x_i)=y_i\) í öllum punktum og
að slík margliða sé ótvírætt ákvörðuð.

Hún nefnist \emph{brúunarmargliða} fyrir punktana
\((x_0,y_0),\dots,(x_m,y_m)\).

Við alhæfum þetta verkefni með því að úthluta sérhverjum punkti jákvæðri
heiltölu \(m_i\) og krefjast þess graf margliðunnar fari í gegnum
alla punktana og til viðbótar að allar afleiður \(p^{(j)}\) upp að
stigi \(m_i-1\) taki einnig fyrirfram gefin gildi \(y^{(j)}_i\).


\subsection{Brúun}
\label{kafli03:id1}
Við tilraunir þá fáum við oft aðeins strjálar mælingar, t.d. ef við
mælum hljóðhraða við mismunandi hitastig. Hins vegar þá viljum við vita
hvert sambandið er fyrir öll möguleg hitastig. Brúunin er margliða og
hún skilgreind er fyrir allar rauntölur og {}`{}`brúar{}`{}` því gildin
milli mælipunktanna.

\index{setning Weierstrass}

\subsection{Afhverju margliður?}
\label{kafli03:afhverju-margliur}\label{kafli03:index-1}\begin{itemize}
\item {} 
Einfalt að meta fallgildin fyrir margliður ({\hyperref[kafli03:reiknirit-horners]{Reiknirit Horners}}).

\item {} 
Einfalt að diffra og heilda margliður.

\item {} 
Margliður eru óendanlega oft diffranlegar.

\item {} 
\emph{Setning Weierstrass:} Látum \(f\) vera samfellt fall á bili
\([a,b]\). Fyrir sérhvert \({\varepsilon}> 0\) þá er til
margliða \(p\) þannig að
\begin{gather}
\begin{split}\|f-p\|_\infty := \max_{x\in [a,b]} |f(x)-p(x)| < {\varepsilon}.\end{split}\notag
\end{gather}
\end{itemize}

\begin{notice}{note}{Athugasemd:}
Setning Weierstrass segir að margliður nægja til að nálga samfelld föll.
Það er sama hvað samfellda fall við skoðum það er alltaf til margliða
sem nálgar það eins vel og við viljum á lokuðu bili.
\end{notice}

\index{margliður}\index{margliður!stig}

\subsection{Margliður}
\label{kafli03:margliur}\label{kafli03:index-2}
Fall \(p\) af gerðinni
\begin{gather}
\begin{split}p(x) = a_0 + a_1 x + \ldots + a_m x^m\end{split}\notag
\end{gather}
þar sem \(m\) er heiltala og \(a_0, \ldots, a_m\) eru tvinntölur
nefnist margliða.

Stærsta talan \(j\) þannig að \(a_j \not= 0\) nefnist \emph{stig
margliðunnar} \(p\).

Ef allir stuðlarnir eru 0 þá nefnist \(p\) \emph{núllmargliðan} og við
segjum að stig hennar sé \(-\infty\).

Munum að stuðullinn \(a_j\) við veldið \(x^j\) er gefinn með
formúlunni
\begin{gather}
\begin{split}a_j = \frac{p^{(j)}(0)}{j!}, \quad j = 0,1,2,\ldots,m.\end{split}\notag
\end{gather}
\index{margliður!staðalform}

\subsection{Mismunandi leiðir á framsetningu}
\label{kafli03:index-3}\label{kafli03:mismunandi-leiir-a-framsetningu}
Hægt er að setja sömu margliðuna fram á marga mismunandi vegu, en við
nefnum framsetninguna hér að framan \emph{staðalform margliðunnar} \(p\).

Ef við veljum okkur einhvern punkt \(x_0 \in {{\mathbb  R}}\), þá
getum við skrifað
\begin{gather}
\begin{split}p(x) = b_0 + b_1(x-x_0) + \ldots + b_m(x-x_0)^m\end{split}\notag
\end{gather}
og stuðlarnir \(b_j\) eru gefnir með
\begin{gather}
\begin{split}b_j = \frac{p^{(j)}(x_0)}{j!}, \quad j = 0,1,2,\ldots,m.\end{split}\notag
\end{gather}
Þessi formúla er jafngild þeirri staðreynd að ef \(p\) er margliða
af stigi \(m\). Þá er Taylor-röð \(p\) í sérhverjum punkti
\(x_0 \in {{\mathbb  R}}\) bara margliðan \(p\), og stuðlarnir í
Taylor-röðinni eru gefnir með formúlunum fyrir \(b_j\) að ofan.

\index{margliður!Newton-form}

\subsection{Newton-form margliðu}
\label{kafli03:index-4}\label{kafli03:newton-form-margliu}
Ef við veljum okkur \(m\) punkta \(x_0, \ldots, x_{m-1}\) þá
nefnist framsetning af gerðinni
\begin{gather}
\begin{split}p(x) = c_0 + c_1(x-x_0) + c_2(x-x_0)(x-x_1)
    + \ldots + c_m(x-x_0)\cdots(x-x_{m-1})\end{split}\notag
\end{gather}
\emph{Newton-form} margliðunnar \(p\) miðað við punktana
\(x_0, \ldots,
x_{m-1}\).

\index{reiknirit Horners}

\subsection{Reiknirit Horners}
\label{kafli03:index-5}\label{kafli03:reiknirit-horners}
Við munum mikið fást við margliður á Newton-formi og því er nauðsynlegt
að hafa hraðvirkt reiknirit til þess að reikna út fallgildi \(p\) út
frá þessari framsetningu.

Eitt slíkt reiknirit er nefnt \emph{reiknirit Horners}. Það byggir á því að
nýta sér að þættirnir \((x-x_j)\) eru endurteknir í liðunum
\begin{gather}
\begin{split}(x-x_0), \quad (x-x_0)(x-x_1),
    \quad (x-x_0)(x-x_1)(x-x_2), \quad \ldots\end{split}\notag
\end{gather}
Þar sem við sleppum við að hefja í veldi þá komumst við af með fáar
reikniaðgerðir hér.

Ef \(m = 2\) má skrifa Newton-form \(p\) sem
\begin{gather}
\begin{split}p(x) = c_0 + (x-x_0)(c_1 + (x-x_1) \cdot c_2).\end{split}\notag
\end{gather}
Ef \(m = 3\) er það
\begin{gather}
\begin{split}p(x) = c_0 + (x-x_0)(c_1 + (x-x_1)(c_2 + (x-x_2)c_3))\end{split}\notag
\end{gather}
og ef \(m = 4\) er það
\begin{gather}
\begin{split}p(x) = c_0 + (x-x_0)(c_1 + (x-x_1)(c_2 + (x-x_2)
    (c_3 + c_4(x-x_3)))).\end{split}\notag
\end{gather}
Reikniritið vinnur á þessari stæðu með því að margfalda upp úr svigunum
frá hægri til vinstri.

Skilgreinum tölur \(b_0\), \(b_1\), \(\ldots\) á
eftirfarandi hátt. Fyrst setjum við
\begin{gather}
\begin{split}b_n = c_n.\end{split}\notag
\end{gather}
Fyrir hvert \(k\) frá \(n-1\) niður í 0 þá setjum við
\begin{gather}
\begin{split}b_k = c_k + (a - x_k) b_{k+1}.\end{split}\notag
\end{gather}
Þá er \(b_0 = p(a)\).
\begin{gather}
\begin{split}p(a) =
    \underbrace{
      c_0 + (a-x_0)(
      \underbrace{
        c_1 + (a-x_1)(
          \underbrace{c_2 + (a-x_2)(
        \underbrace{c_3 + (a-x_3)
          \underbrace{c_4}_{b_4}
          }_{b_3})
        }_{b_2})
      }_{b_1})
    }_{b_0}.\end{split}\notag
\end{gather}
\index{brúun!brúunarmargliða}\index{brúun!brúunarverkefni}

\section{Margliðubrúun: Lagrange-form}
\label{kafli03:margliubruun-lagrange-form}\label{kafli03:index-6}

\subsection{Margliðubrúun}
\label{kafli03:margliubruun}
Látum nú \((x_0,y_0), \ldots, (x_m,y_m)\) vera gefna punkta í plani.
Við höfum áhuga á að finna margliðu \(p\) af lægsta mögulega stigi
þannig að
\begin{gather}
\begin{split}p(x_k) = y_k, \quad k = 0, \ldots, m.\end{split}\notag
\end{gather}
Slík margliða nefnist \emph{brúunarmargliða} fyrir punktana
\((x_0,y_0), \ldots, (x_m,y_m)\)

eða \emph{brúunarmargliða gegnum punktana}
\((x_0,y_0), \ldots, (x_m,y_m)\).

Augljóslega verðum við að gera ráð fyrir að \(x\)-hnitin séu ólík,
það er \(x_j \not= x_k\) ef \(j \not= k\).

Verkefnið að finna margliðuna \(p\) nefnist \emph{brúunarverkefni fyrir
punktana} \((x_0,y_0), \ldots, (x_m,y_m)\).


\subsection{Setning: Brúunarmargliðan er ótvírætt ákvörðuð}
\label{kafli03:setning-bruunarmarglian-er-otviraett-akvoru}
Brúunarmargliðan af stigi \(\leq m\) fyrir \((x_0,y_0),\ldots,(x_m,y_m)\) er
ótvírætt ákvörðuð.

Ef \(p(x)\) og \(q(x)\) eru tvær
brúunarmargliður af stigi \(\leq m\) fyrir punktana
\((x_0,y_0), \ldots, (x_m,y_m)\) þá er mismunurinn
\(r(x) = p(x) - q(x)\) margliða af stigi \(\leq m\) með
núllstöðvar \(x_0, \ldots, x_m\). Þetta eru \(m+1\) ólíkir
punktar og því er \(r(x)\) núllmargliðan samkvæmt
\href{http://www.stae.is/fletta/undirst\%C3\%B6\%C3\%B0usetning/algebrunnar}{undirstöðusetningu algebrunnar}.
Þar með er \(p(x) - q(x)\) núllmargliðan, þ.e. \(p(x) = q(x)\).


\subsection{Setning: Brúunarmargliðan er til}
\label{kafli03:setning-bruunarmarglian-er-til}
Til er margliða \(p\) af stigi \(\leq m\) þannig að
\begin{gather}
\begin{split}p(x_0) = y_0, \quad \ldots \quad p(x_n)=y_n.\end{split}\notag
\end{gather}
Við notum þrepun til að sýna fram á tilvistina.

Ef \(m = 0\), þá erum við aðeins með eitt brúunarskilyrði,
\(p(x_0) = y_0\), og fastamargliðan \(p(x) = y_0\) er lausn af
stigi \(\leq 0\).

G.r.f. að við getum leyst öll brúunarverkefni þar sem fjöldi punkta er
\(m\) og sýnum að við getum þá leyst verkefnið fyrir \(m+1\)
punkt.

Látum \(q\) vera brúunarmargliðuna af stigi \(\leq m-1\) fyrir
punktana \((x_0,y_0), \ldots,
(x_{m-1},y_{m-1})\) og \(r\) vera brúunarmargliðuna af stigi
\(\leq m-1\) fyrir punktana \((x_1,y_1), \ldots, (x_m,y_m)\) og
setjum síðan
\begin{gather}
\begin{split}p(x) = \frac{x-x_m}{x_0-x_m}q(x) + \frac{x-x_0}{x_m-x_0}r(x),\end{split}\notag
\end{gather}
\(p(x)\) er greinilega margliða af stigi \(\leq m\). Skoðum nú
gildin á \(p\)
\begin{gather}
\begin{split}\begin{aligned}
  p(x_0) &= 1 \cdot q(x_0) + 0\cdot r(x_0) = y_0, \\
  p(x_k) &= \frac{x_k-x_m}{x_0-x_m}y_k
  + \frac{x_k-x_0}{x_m-x_0}y_k = y_k,\qquad k = 1, \ldots, m-1,\\
  p(x_m) &= 0 \cdot q(x_m) + 1 \cdot r(x_m) = y_m.\end{aligned}\end{split}\notag
\end{gather}
Þar með er \(p\) brúunarmargliðan sem uppfyllir \(p(x_j)=y_j\)
fyrir \(j=0,\dots,m\) og við höfum leyst brúunarverkefnið fyrir
\(m+1\) punkt.

\index{margliður!Lagrange-form}

\subsection{Lagrange-form brúunarmargliðunnar}
\label{kafli03:index-7}\label{kafli03:lagrange-form-bruunarmargliunnar}
Sönnunin á undan er í raun rakningarformúla til þess að reikna út gildi
brúunarmargliðunnar \(p\) fyrir punktana
\((x_0,y_0),\dots,(x_m,y_m)\).

Hægt er að skrifa lausnina niður beint
\begin{gather}
\begin{split}p(x)=y_0\ell_0(x)+y_1\ell_1(x)+\cdots+y_m\ell_m(x),\end{split}\notag
\end{gather}
þar sem \(\ell_0,\dots,\ell_m\) er ákveðinn grunnur fyrir rúm allra
margliða \({\cal P}_m\) af stigi \(\leq m\) og nefnast
\emph{Lagrange-margliður fyrir punktasafnið}
\((x_0,y_0),\dots,(x_m,y_m)\).


\subsection{Lagrange-margliður, tilfellin \(m=0,1,2\)}
\label{kafli03:lagrange-margliur-tilfellin}\begin{itemize}
\item {} 
\(m=0\) Ef \(m = 0\) þá er \(p(x) = y_0\) fastamargliða
eins og við höfum séð.

\item {} 
\(m=1\) Ef \(m = 1\), þá blasir við að lausnin er
\begin{gather}
\begin{split}p(x) = y_0 \frac{(x-x_1)}{(x_0-x_1)}
  + y_1 \frac{(x-x_0)}{(x_1-x_0)},\end{split}\notag
\end{gather}
sem er margliða af stigi \(\leq 1\) (þ.e. lína) sem leysir
brúunarverkefnið.

\item {} 
\(m=2\) Á hliðstæðan hátt fáum við fyrir \(m = 2\) að
\begin{gather}
\begin{split}p(x) = y_0 \frac{(x-x_1)(x-x_2)}{(x_0-x_1)(x_0-x_2)}
  + y_1 \frac{(x-x_0)(x-x_2)}{(x_1-x_0)(x_1-x_2)}
  + y_2 \frac{(x-x_0)(x-x_1)}{(x_2-x_0)(x_2-x_1)}\end{split}\notag
\end{gather}
leysir brúunarverkefnið.

\end{itemize}


\subsection{Lagrange-margliður almenna tilfellið}
\label{kafli03:lagrange-margliur-almenna-tilfelli}
Almennt fæst lausnin
\begin{gather}
\begin{split}\label{p}
  p(x) = y_0 \, \ell_{0}(x) + y_1 \, \ell_{1}(x)
  + \ldots + y_m \, \ell_{m}(x)\end{split}\notag
\end{gather}
þar sem
\begin{gather}
\begin{split}\ell_{k} = \prod\limits_{\stackrel{j=0}{j\not=k}}^m
  \frac{(x-x_j)}{(x_k-x_j)}\end{split}\notag
\end{gather}
\begin{notice}{note}{Athugasemd:}\begin{gather}
\begin{split}\label{l}
  \ell_{k}(x_i) = \left\{ \begin{array}{cc}
      1 & \text{ef } i = k \\
      0 & \text{ef } i \not= k
  \end{array} \right.\end{split}\notag
\end{gather}\end{notice}

Allar margliðurnar \(\ell_{k}\) eru af stigi \(m\) og því er
\(p\) af stigi \(\leq m\). Nú er augljóst útfrá ({[}p{]}) og ({[}l{]})
að \(p\) er lausn brúunarverkefnisins.


\subsection{Sýnidæmi}
\label{kafli03:synidaemi}
Finnið brúunarmargliðuna gegnum punktana \((1,1)\), \((2,3)\)
og \((3,6)\) með því að nota Lagrange-margliður.

Reiknum fyrst margliðurnar \(\ell_{0}\), \(\ell_{1}\) og
\(\ell_{2}\):
\begin{gather}
\begin{split}\begin{aligned}
  \ell_{0} &= \frac{(x-2)(x-3)}{(1-2)(1-3)}
  = \frac{(x-2)(x-3)}{2} \\
  \ell_{1} &= \frac{(x-1)(x-3)}{(2-1)(2-3)}
  = -(x-1)(x-3) \\
  \ell_{2} &= \frac{(x-1)(x-2)}{(3-1)(3-2)}
  = \frac{(x-1)(x-2)}{2}\end{aligned}\end{split}\notag
\end{gather}
Þá fæst að brúunarmargliðan \(p\) er
\begin{gather}
\begin{split}p(x) = 1 \cdot \frac{(x-2)(x-3)}{2}
  - 3 \cdot (x-1)(x-3)
  + 6 \cdot \frac{(x-1)(x-2)}{2}\end{split}\notag
\end{gather}
Þetta er greinilega annars stigs margliða og auðvelt er að sannfæra sig
um að \(p(1) = 1\), \(p(2) = 3\) og \(p(3) = 6\).

\index{brúun!Newton-form}

\section{Margliðubrúun: Newton-form}
\label{kafli03:index-8}\label{kafli03:margliubruun-newton-form}

\subsection{Formúla fyrir \(c_0, \ldots, c_m\)}
\label{kafli03:formula-fyrir}
Nú ætlum við að leiða út formúlu fyrir stuðlunum
\(c_0, \ldots, c_m\) í Newton-formi brúunarmargliðunnar \(p\)
miðað við röð brúunarpunktanna \(x_0, \ldots, x_{m-1}\).

Athugum að \(c_m = a_m\), þar sem \(a_m\) er stuðullinn við
veldið \(x^m\) í staðalframsetningunni á \(p\).

Til þess að reikna út \(c_0, \ldots, c_m\) þurfum við að reikna út
með skipulegum hætti stuðulinn við veldið \(x^j\) í
brúunarmargliðunni gegnum punktana
\((x_i,y_i), \ldots, (x_{i+j},y_{i+j})\), fyrir öll
\(i = 0, \ldots, m\) og \(j = 0, \ldots, m-i\). Við táknum
þennan stuðul með \(y[x_i, \ldots, x_{i+j}]\).

\begin{notice}{warning}{Aðvörun:}
Verkefnið er háð röð punktanna, þ.e. framsetningin (Newton-formið) á
margliðunni breytist eftir röð punktanna.
En auðvitað er margliðan og gildin á henni alltaf þau sömu

\emph{Dæmi:} Skoðum margliðuna \(p(x) = 2-7x+5x^2\).

Ef \(x_0=0\) og \(x_1=2\) þá er Newton-form hennar
\begin{gather}
\begin{split}p(x) = 3 + 3(x-0) + 5(x-0)(x-2).\end{split}\notag
\end{gather}
En ef \(x_0=2\) og \(x_1=0\) þá er Newton-form hennar
\begin{gather}
\begin{split}p(x) = 8 + 3(x-2) + 5(x-2)(x-0).\end{split}\notag
\end{gather}\end{notice}

\index{mismunakvóti}

\subsection{Mismunakvótar}
\label{kafli03:mismunakvotar}\label{kafli03:index-9}
Skilgreinum mismunakvóta \(y[x_i,\ldots,x_{i+j}]\) fyrir
punktasafnið \((x_i,y_i),\ldots,(x_{i+j},y_{i+j})\) á eftirfarandi
hátt:
\begin{itemize}
\item {} 
\(j=0\): \(y[x_i] = y_i\).

\item {} 
\(j=1\): \(y[x_i,x_{i+1}] = \frac{y_{i+1}-y_i}{x_{i+1}-x_i}\)

\item {} 
\(j=2\):
\(y[x_i,x_{i+1},x_{i+2}] = \frac{y[x_{i+1},x_{i+2}] - y[x_i,x_{i+1}]}{x_{i+2}-x_i}\).

\item {} 
\(j>2\):
\(y[x_i,\ldots,x_{i+j}] = \frac{y[x_{i+1},\ldots,x_{i+j}] - y[x_i,\ldots,x_{i+j-1}]}{x_{i+j}-x_i}\).

\end{itemize}

\begin{notice}{note}{Athugasemd:}
Stærðin \(y[x_{n-1},x_n]\) hefur komið fyrir áður
hjá okkur þegar við fjölluðum um \emph{Sniðilsaðferð}, enda er sniðill
ekkert annað en brúunarmargliða fyrir tvö punkta í planinu.
\end{notice}


\subsection{Upprifjun á tilvistarsönnuninni}
\label{kafli03:upprifjun-a-tilvistarsonnuninni}
Þrepunarskrefið í tilvistarsönnuninni
fyrir brúunarmargliður gefur okkur nú hvernig mismunakvótarnir nýtast okkur.

Látum \(q\) vera brúunarmargliðuna af stigi \(\leq m-1\) fyrir
punktana \((x_0,y_0), \ldots,
(x_{m-1},y_{m-1})\) og \(r\) vera brúunarmargliðuna af stigi
\(\leq m-1\) fyrir punktana \((x_1,y_1), \ldots, (x_m,y_m)\) og
setjum síðan
\begin{gather}
\begin{split}p(x) = \frac{x-x_m}{x_0-x_m}q(x) + \frac{x-x_0}{x_m-x_0}r(x)\end{split}\notag
\end{gather}
Gerum nú ráð fyrir að stuðullinn við veldið \(x^{m-1}\) í
\(q(x)\) sé \(y[x_0, \ldots, x_{m-1}]\) og stuðullinn við veldið
\(x^{m-1}\) í \(r(x)\) sé \(y[x_1, \ldots, x_m]\).

Við sjáum þá að stuðullinn við veldið \(x^m\) í \(p(x)\) er
\begin{gather}
\begin{split}\frac{y[x_0, \ldots, x_{m-1}]}{x_0-x_m} +
  \frac{y[x_1, \ldots, x_m]}{x_m - x_0}
  = y[x_0, \ldots, x_m]\end{split}\notag
\end{gather}
\begin{notice}{note}{Athugasemd:}
Fyrir \(m=0\) gildir að \(p(x) = y_0 = y[x_0]\).
\end{notice}

\index{brúun!mismunakvótatafla}

\subsection{Mismunakvótatöflur fyrir \(m=0,1,2,3\)}
\label{kafli03:mismunakvotatoflur-fyrir}\label{kafli03:index-10}
Mismunakvótar eru venjulega reiknaðir út í svokölluðum
\emph{mismunakvótatöflum}.

Ef \(m = 0\) er mismunakvótataflan aðeins ein lína
\begin{gather}
\begin{split}\begin{array}{c|c|c}
    i & x_i & y[x_i] \\
    \hline
    0 & x_0 & y[x_0] = y_0
  \end{array}\end{split}\notag
\end{gather}
Ef \(m = 1\) er taflan
\begin{gather}
\begin{split}\begin{array}{c|c|cc}
    i & x_i & y[x_i] & y[x_i,x_{i+1}] \\
    \hline
    0 & x_0 & y[x_0] = y_0 & y[x_0,x_1] \\
    1 & x_1 & y[x_1] = y_1 &
  \end{array}\end{split}\notag
\end{gather}
og margliðan er
\begin{gather}
\begin{split}p(x) = y[x_0] + y[x_0,x_1](x-x_0).\end{split}\notag
\end{gather}
Ef \(m = 2\) verður taflan
\begin{gather}
\begin{split}\begin{array}{c|c|ccc}
    i & x_i & y[x_i] & y[x_i,x_{i+1}] & y[x_i,x_{i+1},x_{i+2}] \\
    \hline
    0 & x_0 & y[x_0] = y_0 & y[x_0,x_1] & y[x_0,x_1,x_2] \\
    1 & x_1 & y[x_1] = y_1 & y[x_1,x_2] & \\
    2 & x_2 & y[x_2] = y_2 &  &
  \end{array}\end{split}\notag
\end{gather}
og margliðan er
\begin{gather}
\begin{split}p(x) = y[x_0] + y[x_0,x_1](x-x_0)
  + y[x_0,x_1,x_2](x-x_0)(x-x_1).\end{split}\notag
\end{gather}
Skoðum loks tilfellið \(m = 3\)
\begin{gather}
\begin{split}\begin{array}{c|c|cccc}
    i & x_i & y[x_i] & y[x_i,x_{i+1}] & y[x_i,x_{i+1},x_{i+2}]
    & y[x_i,x_{i+1},x_{i+2},x_{i+3}] \\
    \hline
    0 & x_0 & y[x_0] = y_0 & y[x_0,x_1] & y[x_0,x_1,x_2]
    & y[x_0,x_1,x_2,x_3] \\
    1 & x_1 & y[x_1] = y_1 & y[x_1,x_2] & y[x_1,x_2,x_3] & \\
    2 & x_2 & y[x_2] = y_2 & y[x_2,x_3] & & \\
    3 & x_3 & y[x_3] = y_3 & & &
  \end{array}\end{split}\notag
\end{gather}
Brúunarmargliðan fæst svo með því að nota stuðlana úr fyrstu línu
töflunnar:
\begin{gather}
\begin{split}\begin{aligned}
  p(x) = &y[x_0] + y[x_0,x_1](x-x_0)
  + y[x_0,x_1,x_2](x-x_0)(x-x_1) \\
  &+ y[x_0,x_1,x_2,x_3](x-x_0)(x-x_1)(x-x_2)\end{aligned}\end{split}\notag
\end{gather}

\subsection{Sýnidæmi}
\label{kafli03:id2}
Við skulum reikna út aftur brúunarmargliðuna gegnum \((1,1)\),
\((2,3)\) og \((3,6)\).

Stillum fyrst upp mismunakvótatöflu
\begin{gather}
\begin{split}\begin{array}{cc||ccc}
    i & x_i & y[x_i] & y[x_i,x_{i+1}] & y[x_i,x_{i+1},x_{i+2}] \\
    \hline
    0 & 1   &  1     &    &   \\
    1 & 2   &  3     &    &   \\
    2 & 3   &  6     &    &
  \end{array}\end{split}\notag
\end{gather}
Fyllum svo út í hana með að ganga á hvern dálk á fætur öðrum
\begin{gather}
\begin{split}\begin{array}{cc||ccc}
    i & x_i & y[x_i] & y[x_i,x_{i+1}] & y[x_i,x_{i+1},x_{i+2}] \\
    \hline
    0 & 1 & 1 & \dfrac{3-1}{2-1} = 2 & \dfrac{3-2}{3-1} = 1/2  \\
    1 & 2 & 3 & \dfrac{6-3}{3-2} = 3 & \\
    2 & 3 & 6 & &
  \end{array}\end{split}\notag
\end{gather}
Lesum út brúunarmargliðuna \(p\) með að ganga á efstu línuna:
\begin{gather}
\begin{split}p(x) = 1 + 2 \cdot (x-1) + \frac{1}{2} \cdot (x-1)(x-2).\end{split}\notag
\end{gather}
Reiknum út brúunarmargliðuna gegnum \((3,1)\), \((1,-3)\),
\((5,2)\) og \((6,4)\). Stillum upp og fyllum út í
mismunakvótatöflu:
\begin{gather}
\begin{split}\begin{array}{cc||cccc}
    i & x_i & y[x_i], & y[x_i,x_{i+1}], &
    y[x_i,x_{i+1},x_{i+2}], & y[x_i,\ldots,x_{i+3}] \\
    \hline
    1 & 3 & 1 & \frac{-3-1}{1-3} = 2 & \frac{5/4-2}{5-3} = -3/8 &
    \frac{3/20-(-3/8)}{6-3} = 7/40 \\
    2 & 1 & -3 & \frac{2-(-3)}{5-1} = 5/4 &
    \frac{2-5/4}{6-1} = 3/20 & \\
    3 & 5 & 2 & \frac{4-2}{6-5} = 2 & & \\
    4 & 6 & 4 & & &
  \end{array}\end{split}\notag
\end{gather}
Nú getum við lesið brúunarmargliðuna okkar úr töflunni með að ganga á
efstu línuna, við fáum
\begin{gather}
\begin{split}p(x) = 1 + 2(x-3) - \frac 38 (x-3)(x-1)
  + \frac 7{40} (x-3)(x-1)(x-5)\end{split}\notag
\end{gather}
\begin{notice}{note}{Verkefnalisti}

Umraða
\end{notice}


\subsection{Samantekt}
\label{kafli03:samantekt}
Ef gefnir eru punktar \((x_0,y_0), \ldots, (x_m,y_m)\) í
\({{\mathbb  R}}^2\), þar sem \(x_i\neq x_j\) ef
\(i\neq j\), þá er til nákvæmlega ein margliða \(p\) af stigi
\(\leq m\) þannig að
\begin{gather}
\begin{split}p(x_k) = y_k, \quad k = 0, \ldots, m\end{split}\notag
\end{gather}
Newton-form margliðunnar \(p\) með tilliti til punktanna
\(x_0,\dots,x_{m-1}\) er
\begin{gather}
\begin{split}p(x)=y[x_0]+y[x_0,x_1](x-x_0)+\cdots+y[x_0,\dots,x_m](x-x_0)\cdots(x-x_m)\end{split}\notag
\end{gather}
þar sem mismunakvótarnir eru reiknaðir með rakningarformúlunum
\(y[x_i]=y_i\) og
\begin{gather}
\begin{split}y[x_i,\ldots,x_{i+j}]
  = \frac{y[x_{i+1},\ldots,x_{i+j}] - y[x_i,\ldots,x_{i+j-1}]}
  {x_{i+j} - x_i}, \qquad i=0,\dots,m, \quad j=0,\dots,m-i.\end{split}\notag
\end{gather}

\subsection{Samantekt – Newton-form}
\label{kafli03:samantekt-newton-form}
Venja er að setja mismunakvótana upp í töflu og stuðlarnir í
Newton-forminu raða sér í fyrstu línu töflunnar:
\begin{gather}
\begin{split}\begin{array}{c|c|cccccc}
    i & x_i & y[x_i] & y[x_i,x_{i+1}] & y[x_i,x_{i+1},x_{i+2}]
    & y[x_i,\dots,x_{i+3}] &y[x_i,\dots,x_{i+4}] &\dots  \\
    \hline
    0 & x_0 & y[x_0] = y_0 & y[x_0,x_1] & y[x_0,x_1,x_2]
    & y[x_0,x_1,x_2,x_3]&y[x_0,x_1,x_2,x_3,x_4]& \dots \\
    1 & x_1 & y[x_1] = y_1 & y[x_1,x_2] & y[x_1,x_2,x_3] &
    y[x_1,x_2,x_3,x_4]&\dots \\
    2 & x_2 & y[x_2] = y_2 & y[x_2,x_3] &y[x_2,x_3,x_4]&\dots & \\
    3 & x_3 & y[x_3] = y_3 & y[x_3,x_4] &\dots & & \\
    4 & x_4 & y[x_4] = y_4 & \dots &  \\
\vdots & \vdots &\vdots
  \end{array}\end{split}\notag
\end{gather}

\subsection{Samantekt – Lagrange-margliður}
\label{kafli03:samantekt-lagrange-margliur}
Lagrange-form brúunarmargliðunnar er
\begin{gather}
\begin{split}p(x)=\sum_{k=0}^m y_k\ell_{k}(x)\end{split}\notag
\end{gather}
þar sem \(\ell_{k}\) eru Lagrange-margliðurnar með tilliti til
punktanna \(x_0,\dots,x_m\),
\begin{gather}
\begin{split}  \ell_{k}(x) = \prod_
      {\substack{j=0\\ j\neq k}}^m\dfrac{(x-x_j)}{(x_k-x_j)}
      = \dfrac{(x-x_0)\cdots(x-x_{k-1})
          (x-x_{k+1})\cdots(x-x_m)}
      {(x_k-x_0)\cdots(x_k-x_{k-1})
          (x_k-x_{k+1})\cdots(x_k-x_m)}.\end{split}\notag\\\begin{split}En þær uppfylla\end{split}\notag
\end{gather}\begin{gather}
\begin{split}\ell_{k}(x_i) = \left\{ \begin{array}{cc}
      1 & \text{ef } i = k \\
      0 & \text{ef } i \not= k
  \end{array} \right.\end{split}\notag
\end{gather}

\section{Samantekt}
\label{kafli03:id3}

\subsection{Lagrange-margliður}
\label{kafli03:lagrange-margliur}\begin{itemize}
\item {} 
Auðvelt að finna margliðuna

\item {} 
Dýrara að reikna fallgildin

\end{itemize}


\subsection{Newton-margliður}
\label{kafli03:newton-margliur}\begin{itemize}
\item {} 
Erfiðara að finna margliðuna

\item {} 
Auðvelt að finna fallgildin (reiknirit Horners)

\end{itemize}

\index{brúun!margfaldir punktar}

\section{Margliðubrúun: Margfaldir punktar}
\label{kafli03:margliubruun-margfaldir-punktar}\label{kafli03:index-12}
Látum \(a_1, \ldots, a_k\) vera ólíka punkta í
\({{\mathbb  R}}\), \(m_1, \ldots, m_k\) vera jákvæðar heiltölur
og hugsum okkur að gefnar séu rauntölur
\begin{gather}
\begin{split}y_i^{(j)}, \quad j = 0, \ldots, m_i-1, \quad i = 1, \ldots, k.\end{split}\notag
\end{gather}
Við viljum finna margliðu \(p\) af lægsta mögulega stigi þannig að
margliðan \(p=p^{(0)}\) og afleiður hennar \(p^{(j)}\) uppfylli
\begin{gather}
\begin{split}p^{(j)}(a_i) = y_i^{(j)},
  \quad j = 0, \ldots, m_i-1, \quad i = 1, \ldots, k\end{split}\notag
\end{gather}
Við nefnum verkefnið að finna slíka margliðu \(p\) \emph{alhæft
brúunarverkefni}, og margliða sem uppfyllir þessi skilyrði nefnist
\emph{brúunarmargliða fyrir brúunarverkefnið} sem lýst er með gefnu
skilyrðunum.

\index{brúunarpunktur!einfaldur}\index{brúunarpunktur!tvöfaldur}\index{brúunarpunktur!stig}

\subsection{Margfeldni punktanna}
\label{kafli03:index-13}\label{kafli03:margfeldni-punktanna}
Við segjum að \(a_i\) sé \emph{einfaldur brúunarpunktur} ef
\(m_i=1\), \emph{tvöfaldur brúunarpunktur} ef \(m_i=2\) o.s.frv.

Við skilgreinum nú töluna
\begin{gather}
\begin{split}m = m_1 + m_2 + \ldots + m_k - 1.\end{split}\notag
\end{gather}
Brúunarmargliðan okkar \(p\) á að vera af stigi \(\leq m\), og
fjöldi skilyrða sem við setjum á hana eru \(m+1\).

\begin{notice}{note}{Athugasemd:}
Tilfellið \(k=m+1\), \(m_j=1\) er það sem við skoðuðum hér á undan.
\end{notice}


\subsection{Sértilfelli}
\label{kafli03:sertilfelli}
Tvö sértilfelli þekkjum við nú þegar.
\begin{enumerate}
\item {} 
\emph{Allir punktar eins:} Ef allir punktarnir eru einfaldir, þá er
alhæfða brúunarverkefnið sama verkefni og brúunarverkefnið sem við
leystum í {\hyperref[kafli03:margliubruun-lagrange-form]{Margliðubrúun: Lagrange-form}} og {\hyperref[kafli03:margliubruun-newton-form]{Margliðubrúun: Newton-form}}.
\begin{gather}
\begin{split}p^{(0)}(a_i) = p(a_i) = y_i^{(0)},\end{split}\notag
\end{gather}
og lausnin var leidd út með
\begin{gather}
\begin{split}x_0=a_1,\dots,x_m=a_k \quad \text{ og } \quad
y_0=y_1^{(0)},\dots,y_m=y_k^{(0)}.\end{split}\notag
\end{gather}
\item {} 
\emph{Einn punktur:} Ef aftur á móti \(k = 1\), þá er lausn gefin
með Taylor-margliðunni af röð \(m\) í punktinum \(a_1\)
\begin{gather}
\begin{split}p(x) = y_1^{(0)} + \frac{y'}{1!}(x - a_1) + \frac{y''}{2!}(x - a_1)^2 +
  \ldots + \frac{y_1^{(m)}}{m!}(x - a_1)^m.\end{split}\notag
\end{gather}
\end{enumerate}


\subsection{Upprifjun}
\label{kafli03:upprifjun}
Munum að ef \(p\) er margliða og \(p(a)=0\) þá er \(p\)
deilanleg með \((x-a)\). Það er, hægt er að skrifa
\begin{gather}
\begin{split}p(x) = (x-a)q(x),\end{split}\notag
\end{gather}
þar sem \(q\) er margliða af stigi sem er einu lægra en stig
\(p\) (sjá \href{http://m.xn--st-2ia.is/fletta/undirst\%C3\%B6\%C3\%B0usetning/algebrunnar?device=desktop}{Undirstöðusetning algebrunnar}).


\subsection{Ótvíræðni lausnarinnar}
\label{kafli03:otviraeni-lausnarinnar}
Nú ætlum við að sýna fram á að til sé nákvæmlega ein margliða
\(p(x)\) af stigi \(\leq m\) sem uppfyllir
\begin{gather}
\begin{split}p^{(j)}(a_i) = y_i^{(j)},
  \quad j = 0, \ldots, m_i-1, \quad i = 1, \ldots, k\end{split}\notag
\end{gather}
Byrjum á að sýna að það er í mesta lagi ein margliða sem uppfyllir þetta.

Gerum ráð fyrir að
\(p(x)\) og \(q(x)\) séu tvær margliður af stigi \(\leq m\)
sem uppfylla öll þessi skilyrði.

Þá uppfyllir margliðan \(r(x) = p(x) - q(x)\) að
\begin{gather}
\begin{split}r^{(j)}(a_i) = 0, \quad j = 0, \ldots, m_i-1,
  \quad i = 1, \ldots,k\end{split}\notag
\end{gather}
Af þessu leiðir að \(r(x)\) er deilanlegt með \((x-a_i)^{m_i}\)
en samanlagt stig þessara þátta er \(m_1 + \ldots + m_k = m + 1\).

Nú er stig \(r(x)\) minna eða jafnt \(m\) svo þetta getur aðeins
gerst ef \(r(x)\) er núllmargliðan.

Við höfum því að \(p(x) = q(x)\) og ályktum að við höfum nákvæmlega
eina lausn á brúunarverkefninu ef við getum sýnt fram á tilvist á lausn.


\subsection{Tilvist á lausn}
\label{kafli03:tilvist-a-lausn}
Nú beitum við sams konar röksemdafærslu og í byrjum kaflans til þess að sýna
fram á tilvist á lausn, þ.e. við notum þrepun.

\textbf{Smíðum margliðuna:}

Ef \(m = 0\), þá er lausnin fastamargliðan
\(p(x) = y_1^{(0)}=y_0\).

Gerum nú ráð fyrir að við getum fundið brúunarmargliðu af stigi
\(\leq m-1\) fyrir sérhvert alhæft brúunarverkefni þar sem
samanlagður fjöldi skilyrðanna er \(m\).

Lítum nú aftur á upprunalega brúunarverkefnið þar sem fjöldi skilyrðanna
er \(m+1\). Skilgreinum tvær runur af punktum
\begin{gather}
\begin{split}(x_0,x_1,\ldots,x_m) =
  (\underbrace{a_1, \ldots, a_1}_{m_1 \, \text{sinnum}},
  \underbrace{a_2, \ldots , a_2}_{m_2 \, \text{sinnum}},
  \ldots ,
  \underbrace{a_k, \ldots , a_k}_{m_k \, \text{sinnum}})\end{split}\notag
\end{gather}
og
\begin{gather}
\begin{split}(y_0,y_1,\ldots,y_m) =
  (y_1^{(0)}, \ldots, y_1^{(m_1-1)}, \ldots,
  y_k^{(0)}, \ldots, y_k^{(m_k-1)})\end{split}\notag
\end{gather}
Við höfum séð að í því tilfelli að við höfum einn punkt, \(k=1\),
\(x_0 = x_1 = \ldots = x_m = a_1\) er lausnin gefin með
Taylor-margliðu í \(a_1\).

Við megum því gera ráð fyrir punktarnir séu a.m.k. tveir,
\(k\geq 2\). Það gefur að \(x_0 \not= x_m\).

Látum \(q(x)\) vera margliðuna af stigi \(\leq m-1\) sem
uppfyllir sömu skilyrði og \(p\), nema það síðasta um að
\(q^{(m_k-1)}(a_k)\) þurfi að vera \(y_k^{(m_k-1)}\) .

og látum \(r(x)\) vera margliðuna sem uppfyllir öll
brúunarskilyrðin, nema síðasta skilyrðið í fyrsta punkti um að
\(r^{(m_1-1)}(a_1)\) sé jafnt \(y_1^{(m_1-1)}\).

Setjum síðan
\begin{gather}
\begin{split}p(x) = \frac{x-x_m}{x_0-x_m}q(x)
  + \frac{x-x_0}{x_m-x_0}r(x)
  = \frac{x-a_k}{a_1-a_k}q(x)
  + \frac{x-a_1}{a_k-a_1}r(x)\end{split}\notag
\end{gather}
\textbf{Sýnum að gefin fallgildi eru tekin}

Nú þurfum við að staðfesta að öll skilyrðin séu uppfyllt.

Við byrjum á því að taka \(j=0\) sem svarar til þess að \(p\)
taki fyrirfram gefin fallgildi,
\begin{gather}
\begin{split}\begin{aligned}
  p(a_1) &= \frac{a_1-a_k}{a_1-a_k}q(a_1)
  + \frac{a_1-a_1}{a_k-a_1}r(a_1) = q(a_1) = y_1^{(0)}\\
  p(a_i) &= \frac{a_i-a_k}{a_1-a_k}q(a_i)
  + \frac{a_i-a_1}{a_k-a_1}r(a_i)
  = \left( \frac{a_i-a_k}{a_1-a_k}
    + \frac{a_i-a_1}{a_k-a_1} \right) y_i^{(0)} \\
    &= y_i^{(0)},  \qquad \text{fyrir } i=2,\ldots,k-1,\\
  p(a_k) &= \frac{a_k-a_k}{a_1-a_k}q(a_k)
  + \frac{a_k-a_1}{a_k-a_1}r(a_k) = r(a_k) = y_k^{(0)}.\end{aligned}\end{split}\notag
\end{gather}
Sýnum svo að gildin á afleiðum \(p\) séu rétt.

Rifjum upp margliðuna \(p\):
\begin{gather}
\begin{split}p(x) = \frac{x-x_m}{x_0-x_m}q(x)
  + \frac{x-x_0}{x_m-x_0}r(x)
  = \frac{x-a_k}{a_1-a_k}q(x)
  + \frac{x-a_1}{a_k-a_1}r(x)\end{split}\notag
\end{gather}
Afleiður hennar eru
\begin{gather}
\begin{split}p^{(j)}(x) = \frac{(x-a_k)}{(a_1-a_k)}q^{(j)}(x)
  + \frac{(x-a_1)}{(a_k-a_1)}r^{(j)}(x)
  + j \frac{\left( q^{(j-1)}(x)-r^{(j-1)}(x)\right)}{a_k-a_1}\end{split}\notag
\end{gather}
Ef nú \(m_i > 1\) þá er \(q^{(j-1)}(a_i) = y^{(j-1)}(a_i) =
r^{(j-1)}(a_i)\) fyrir \(j = 1, \ldots, m_i-1\) og því kemur alltaf
\(0\) út úr síðasta liðnum ef við setjum inn \(x = a_i\), fyrir
öll \(i = 1, \ldots, k\).

Af þessu sést að afleiður \(p\) uppfylla skilyrðin
\begin{gather}
\begin{split}p^{(j)}(a_i) = y^{(j)}_i, \qquad \text{fyrir } j=0,\ldots,m_i-1,
  \quad i=1,\ldots,k.\end{split}\notag
\end{gather}

\subsection{Samantekt}
\label{kafli03:id4}
Við höfum því sannað eftirfarandi:

Ef gefnar eru
\begin{itemize}
\item {} 
rauntölur \(a_1,\dots,a_k\), með \(a_j\neq a_k\) ef
\(j\neq k\),

\item {} 
jákvæðar heiltölur \(m_1,\dots,m_k\),

\item {} 
rauntölur \(y_i^{(j)}\), fyrir \(j=0,\dots, m_i-1\),
\(i=1,\dots,k\),

\end{itemize}

og talan \(m\) er skilgreind með \(m=m_1+\cdots+m_k-1\), þá er
til nákvæmlega ein margliða \(p\) af stigi \(\leq m\) þannig að
\begin{gather}
\begin{split}p^{(j)}(a_i)=y_i^{(j)}, \qquad j=0,\dots, m_i-1, \quad i=1,\dots,k.\end{split}\notag
\end{gather}

\subsection{Brúunarmargliðan fundin}
\label{kafli03:bruunarmarglian-fundin}
Ef skilgreindar eru runurnar
\begin{gather}
\begin{split}(x_0,\dots,x_m)=(a_1,\dots,a_1,a_2,\dots,a_2,\dots,a_k,\dots,a_k)\end{split}\notag
\end{gather}
þar sem \(a_1\) kemur fyrir \(m_1\) sinnum, \(a_2\) kemur
fyrir \(m_2\) sinnum o.s.frv., og
\begin{gather}
\begin{split}(y_0,\dots,y_m)=(y_1^{(0)},\cdot\cdot,y_1^{(m_1-1)},y_2^{(0)},\cdot\cdot,y_2^{(m_2-1)},
\cdots,y_k^{(0)},\cdot\cdot,y_k^{(m_k-1)}),\end{split}\notag
\end{gather}
þá er Newton-form margliðunnar \(p\) með tilliti til punktanna
\(x_0,\dots,x_{m-1}\) gefið með
\begin{gather}
\begin{split}p(x)=y[x_0]+y[x_0,x_1](x-x_0)+\cdots+y[x_0,\dots,x_m](x-x_0)\cdots(x-x_{m-1})\end{split}\notag
\end{gather}
þar sem mismunakvótarnir \(y[x_i,\ldots,x_{i+j}]\) eru reiknaðir með
rakningarformúlu þannig að \(y[x_i]=y_i\) og
\begin{gather}
\begin{split}y[x_i,\ldots,x_{i+j}]
  = \begin{cases}\dfrac{y[x_{i+1},\ldots,x_{i+j}] - y[x_i,\ldots,x_{i+j-1}]}
  {x_{i+j} - x_i}, &\text{ ef } x_i\neq x_{i+j},\\
\dfrac{y^{(j)}_i}{j!}, &\text{ ef } x_i=x_{i+j}.
\end{cases}\end{split}\notag
\end{gather}

\section{Skynsamlegir skiptipunktar og Chebyshev margliður}
\label{kafli03:skynsamlegir-skiptipunktar-og-chebyshev-margliur}

\subsection{Um val á brúunarpunktum}
\label{kafli03:um-val-a-bruunarpunktum}
Látum \((t_0,y_0),\dots,(t_n,y_n)\) vera punkta í plani og gerum ráð
fyrir að \(a=t_0<t_1<\cdots<t_n=b\).

Við höfum nú lært að ákvarða margliðu \(p\) af stigi \(\leq n\)
sem tekur gildin \(y_i\) í punktunum \(t_i\).

Ef punktarnir liggja á grafi fallsins \(f\) og nota á margliðuna til
þess að nálga fallgildi \(f\), þá getur það verið ýmsum erfiðleikum
bundið þegar stig hennar stækkar. Til dæmis getur komið fram
óstöðugleiki í útreikningum þannig að örlítil frávik í \(x\) geta
leitt til mikilla frávika í \(p(x)\), og þá hugsanlega í mikilli
skekkju á \(f(x)-p(x)\).


\subsection{Dæmi um óheppilega skiptipunkta}
\label{kafli03:daemi-um-oheppilega-skiptipunkta}
Skoðum dæmi þar sem við brúum fallið \(f(x) = 1/(25x^2+1)\) í 9
brúnarpunktum sem eru jafndreifðir á bilinu \([-1,1]\).

\includegraphics{vond_bruun1.png}

Hér sjáum við ,,þægilegt'' fall þar sem brúunarmargliðan gefur afskaplega
vonda nálgun.


\subsection{Val á brúunarpunktum}
\label{kafli03:val-a-bruunarpunktum}
Það er ekki sjálfgefið að við getum valið í hvaða brúunarpunkta við
notum, t.d. ef þeir ákvarðast af mælingum. Ef við hins vegar getum valið
þá óhindrað, þá vaknar sú spurning hvernig er best að gera það?

\index{staðall!\(\ell_\infty\)}\index{staðall!\(\ell_2\)}

\subsection{Skilgreining}
\label{kafli03:skilgreining}\label{kafli03:index-14}
Fyrst þurfum við að útskýra betur hvað við eigum við með ,,best”. Við
munum bara notast við tvær leiðir hér til að mæla skekkjuna, en það er
\(\ell_\infty\) og \(\ell_2\) staðlarnir, fyrir samfellt fall
\(h\) á bilinu \([a,b]\) þá eru þeir skilgreindir með
\begin{gather}
\begin{split}\|h\|_\infty  = \max_{x\in[a,b]} |h(x)|,\end{split}\notag
\end{gather}
og
\begin{gather}
\begin{split}\|h\|_2 = \left( \int_a^b h(x)^2\, dx \right)^\frac 12\end{split}\notag
\end{gather}
\begin{notice}{note}{Athugasemd:}
Það má líta þannig á þetta að \(\ell_\infty\) staðallinn
mæli hámarksskekkju og \(\ell_2\) mæli einhvers konar,,meðaltalsskekkju'',
þar sem meðaltalið er reiknað með heildi.
\end{notice}


\subsection{Verkefnið}
\label{kafli03:verkefni}
Verkefnið er því eftirfarandi: Fyrir gefið fall \(f(x)\) á bili
\([a,b]\) og fast \(n\), þá viljum við finna
\(x_0,\ldots,x_n\) sem lágmarka annað hvort
\begin{gather}
\begin{split}\|f-p\|_\infty \quad \text{ eða } \quad \|f-p\|_2.\end{split}\notag
\end{gather}
Þar sem \(p\) er brúunarmargliðan fyrir brúunarpunktana
\((x_i,f(x_i))\).

Byrjum á að skoða \(\ell_\infty\) tilvikið.

\index{margliður!Chebyshev}

\subsection{Skilgreining: Chebyshev margliður}
\label{kafli03:index-15}\label{kafli03:skilgreining-chebyshev-margliur}
Fyrir náttúrlega tölu \(n\) þá skilgreinum við
\emph{Chebyshev margliðuna} \(T_n\) á \([-1,1]\) með
\begin{gather}
\begin{split}T_n(x) = \cos(n \arccos(x)).\end{split}\notag
\end{gather}
Með því að setja inn \(n=0\) og \(n=1\) þá fæst að
\begin{gather}
\begin{split}T_0(x) = 1 \quad \text{ og } \quad T_1(x) = x,\end{split}\notag
\end{gather}
og með hornafallareglunum fæst að
\begin{gather}
\begin{split}T_{n+1}(x) = 2xT_n(x) - T_{n-1}(x), n \geq 1.\end{split}\notag
\end{gather}
Af jöfnunni hér á undan þá fæst með þrepun að
\begin{itemize}
\item {} 
\(T_n(x)\) er margliða af stigi \(n\).

\item {} 
Forystustuðull \(T_n\) er \(2^{n-1}\).

\item {} 
\(T_n\) er jafnstæð ef \(n\) er slétt og oddstæð ef \(n\)
er oddatala.

\end{itemize}


\subsection{Setning}
\label{kafli03:setning}
Chebyshev margliðan \(T_n\) hefur \(n\) einfaldar
núllstöðvar á bilinu \([-1,1]\) og þær eru gefnar með
\begin{gather}
\begin{split}x_j = \cos\left(\frac{2j+1}{2n}\pi\right),\qquad j=0,1,2,3,\ldots,n-1.\end{split}\notag
\end{gather}
Auk þess eru útgildi \(T_n\) á \([-1,1]\) staðsett í
\begin{gather}
\begin{split}z_j = \cos\left( \frac{j\pi}{n}\right),\qquad j=0,1,2,\ldots,n,\end{split}\notag
\end{gather}
og fallgildin þar uppfylla \(T_n(z_j) = (-1)^j\)

\index{margliður!staðlaðar}

\subsection{Staðlaðar Chebyshev margliður}
\label{kafli03:index-16}\label{kafli03:stalaar-chebyshev-margliur}
Margliða er kölluð \emph{stöðluð} ef forystustuðull hennar er 1.

\emph{Stöðluðu Chebyshev margliðurnar} \(\tilde T\) eru skilgreindar á eftirfarandi hátt
\begin{gather}
\begin{split}\tilde T(x) =
    \begin{cases}
      T_0(x) & \text{ef } n = 0 \\
      2^{1-n}T_n(x)   & \text{ef } n\geq 1              \end{cases}\end{split}\notag
\end{gather}
Fyrir sérhverja staðlaða margliðu \(q\) af stigi
\(n\) þá er
\begin{gather}
\begin{split}\frac 1{2^{n-1}} = \max_{x\in [-1,1]} T_n(x) \leq \max_{x\in[-1,1]} |q(x)|.\end{split}\notag
\end{gather}
Þ.e. af öllum stöðluðum margliðum þá eru stöðluðu Chebyshev margliðurnar
,,minnstar” á bilinu \([-1,1]\).


\subsection{Skynsamlegir skiptipunktar fyrir bilið \([-1,1]\)}
\label{kafli03:skynsamlegir-skiptipunktar-fyrir-bili}
Við vitum að skekkjan í því að nálga fallið \(f\) með
brúunarmargliðu \(p\) með brúunarpunkta \(x_0,\ldots,x_n\) er
\begin{gather}
\begin{split}f(x)-p(x) = \frac{f^{(n+1)}(\xi)}{(n+1)!}\, (x-x_0)(x-x_1)\cdots (x-x_n),\end{split}\notag
\end{gather}
þar sem \(\xi\) er á minnsta bilinu sem inniheldur \(x\) og
\(x_0,x_1,\ldots,x_n\). Ef við skoðum jöfnuna að ofan þá sjáum við
að þar sem \(n\) og \(f\) (og þar með \(f^{(n+1)}\)) er fast
þá er stæðan \((x-x_0)\cdots(x-x_n)\) það eina sem við höfum
einhverja stjórn á.

Með því að nota Chebyshev margliðurnar þá getum við lágmarkað þennan
hluta skekkjunnar.

Athugið að \((x-x_0)\cdots (x-x_n)\) er stöðluð margliða af stigi
\(n+1\). Þannig að samkvæmt því sem kom fram hér að
ofan þá lágmörkum við framlag hennar
til skekkjunnar með \((x-x_0)\cdots (x-x_n) = \tilde T_{n+1}\),
þ.e. með því að velja
\begin{gather}
\begin{split}x_i = \cos\left(\frac{2i+1}{2(n+1)}\pi\right), \qquad i=0,1,\ldots,n.\end{split}\notag
\end{gather}
Hæsta gildi \(\tilde T_{n+1}\) er \(\frac 1{2^n}\), sem þýðir að
við fáum skekkjumatið
\begin{gather}
\begin{split}\|f(x)-p(x)\|_\infty \leq \frac{\|f^{(n+1)}\|_\infty}{2^n(n+1)!}.\end{split}\notag
\end{gather}

\subsection{Dæmi um óheppilega skiptipunkta skoðað aftur}
\label{kafli03:daemi-um-oheppilega-skiptipunkta-skoa-aftur}
Skoðum aftur fallið \(f(x) = 1/(25x^2+1)\), en í stað þess að taka 9
jafndreifaða brúunarpunkta á bilinu \([-1,1]\), þá skulum við nota
Chebyshev margliðurnar til að finna 9 punkta á bilinu.

\includegraphics{vond_bruun2.png}


\subsection{Skynsamlegir skiptipunktar fyrir bil \([a,b]\)}
\label{kafli03:skynsamlegir-skiptipunktar-fyrir-bil}
Hér á undan miðaðist allt við að finna brúunarmargliðu fyrir fallið
\(f\) á bilinu \([-1,1]\). Ef við viljum skoða almennt bil
\([a,b]\) þá byrjum við á athuga að fallið
\(\eta:[-1,1]\to [a,b]\),
\begin{gather}
\begin{split}\eta(t) = \frac{b-a}2 t + \frac{b+a}2\end{split}\notag
\end{gather}
skilgreinir línulega vörpun (hliðrun og stríkkun) frá \([-1,1]\)
yfir á \([a,b]\). Athugið að vörpunin sendir \(-1\) í \(a\)
og \(1\) í \(b\).

Með því að taka rætur stöðluðu Chebyshev margliðunnar
\(\tilde T_{n+1}\) og varpa þeim með \(\eta\) yfir á bilið
\([a,b]\) þá fáum við þá punkta \(x_0,\ldots,x_n \in [a,b]\) sem
lágmarka \((x-x_0)\cdots (x-x_n)\) á bilinu \([a,b]\),
\begin{gather}
\begin{split}x_i = \eta\left(\cos\left(\frac{2i+1}{2(n+1)}\pi\right)\right)
    = \frac{b-a}2 \cos\left(\frac{2i+1}{2(n+1)}\pi\right) + \frac{b+a}2,\end{split}\notag
\end{gather}
\(i=0,1,2,\ldots,n\).


\subsection{Lágmörkun á skekkju með tilliti til \(\ell_2\)}
\label{kafli03:lagmorkun-a-skekkju-me-tilliti-til}
Nú skulum við skipta um staðal, þannig að í stað þess að lágmarka
\(\|f-p\|_\infty\) þá skulum við reyna að lágmarka
\begin{gather}
\begin{split}\|f-p\|_2 = \left(\int_a^b (f-p)^2\, dx\right)^{1/2}\end{split}\notag
\end{gather}
Við vitum að skekkjan í því að nálga fallið \(f\) með
brúunarmargliðu \(p\) með brúunarpunkta \(x_0,\ldots,x_n\) er
\begin{gather}
\begin{split}f(x)-p(x) = \frac{f^{(n+1)}(\xi)}{(n+1)!}\, (x-x_0)(x-x_1)\cdots (x-x_n),\end{split}\notag
\end{gather}
þar sem \(\xi\) er á minnsta bilinu sem inniheldur \(x\) og
\(x_0,x_1,\ldots,x_n\).

Eins og áður þá sjáum við að stæðan \((x-x_0)\cdots(x-x_n)\) það
eina sem við getum stjórnað með því að velja brúunarpunktana
\(x_j\).

\index{margliður!Legendre}

\subsection{Skilgreining: Legendre margliðurnar}
\label{kafli03:skilgreining-legendre-margliurnar}\label{kafli03:index-17}
Fyrir náttúrlega tölu \(n\) þá skilgreinum við
\emph{Legendre margliðurnar} svona
\begin{gather}
\begin{split}\begin{aligned}
   P_0(x) &= 1,\\
   P_1(x) &= x,\\
   P_n(x) &= \frac{2n-1}n x P_{n-1}(x) - \frac{n-1}n P_{n-2}(x).
  \end{aligned}\end{split}\notag
\end{gather}
Af skilgreiningunni hér á undan þá sjáum við að
\begin{itemize}
\item {} 
\(P_n(x)\) er margliða af stigi \(n\).

\item {} 
Forystustuðull \(P_n\) er
\(\frac {2n-1}n \cdot \frac {2n-3}{n-2} \cdots \frac 32 \cdot 1\).

\item {} 
\(P_n\) er jafnstæð ef \(n\) er slétt og oddstæð ef \(n\)
er oddatala.

\end{itemize}


\subsection{Setning}
\label{kafli03:id5}\begin{gather}
\begin{split}\int_{-1}^1 P_j(x) P_k(x)\, dx =
    \begin{cases}
     0, & \text{ef } j\neq k\\
     \frac{2}{2j+1}, & \text{ef } j=k.
    \end{cases}\end{split}\notag
\end{gather}
Einnig gildir að ef \(q\) er margliða af stigi minna en \(n\) þá er
\begin{gather}
\begin{split}\int_{-1}^1 q(x)P_n(x)\, dx = 0.\end{split}\notag
\end{gather}
Þetta segir okkur að Legendre margliðurnar eru hornréttar (með tilliti
til innfeldisins sem heildið skilgreinir).


\subsection{Setning}
\label{kafli03:id6}
\(P_n\) hefur \(n\) ólíkar núllstöðvar sem liggja
allar á \([-1,1]\).


\subsection{Skilgreining: Staðlaðar Legendre margliður}
\label{kafli03:skilgreining-stalaar-legendre-margliur}
Eins og þegar við fengumst við Chebyshev margliðurnar þá skilgreinum við
\emph{stöðluðu Legendre margliðurnar} \(\tilde P_n\) með því að deila upp
í \(P_n\) með forrystustuðlunum \(P_n\).

\begin{notice}{note}{Athugasemd:}
Setningarnar þrjár hér undan gilda um \(\tilde P\) alveg eins og \(P\).
\end{notice}


\subsection{Setning: Lágmörkun með Legendre margliðunum}
\label{kafli03:setning-lagmorkun-me-legendre-margliunum}
Ef \(p\) er stöðluð margliða af stigi \(n+1\) þá er
\(\|p\|_2\geq \|\tilde P_{n+1}\|_2\).

Skilgreinum
\(q = p-\tilde P_{n+1}\), sem þýðir að \(q\) er margliða af
stigi minna en \(n+1\). Nú er
\begin{gather}
\begin{split}\begin{aligned}
   \|p\|_2^2 &= \|\tilde P_{n+1} + q\|_2^2 \\
   &= \int_{-1}^1 (\tilde P_{n+1}(x) + q(x))^2\, dx \\
   &= \int_{-1}^1 \tilde P_{n+1}(x)^2 + 2q(x)\tilde P_{n+1}(x) + q(x)^2\, dx\\
   &= \|\tilde P_{n+1}\|_2^2 + 2\int_{-1}^1 q(x)\tilde P_{n+1}(x)\, dx + \|q\|_2^2\\
   &= \|\tilde P_{n+1}\|_2^2 +  \|q\|_2^2 \geq \|\tilde P_{n+1}\|_2^2
  \end{aligned}\end{split}\notag
\end{gather}
því \(\int_{-1}^1 q(x)\tilde P_{n+1}(x)\, dx=0\) og
\(\|q\|_2 \geq 0\).

Af síðustu setningu sjáum við að til þess að lágmarka
\begin{gather}
\begin{split}\|f(x)-p(x)\|_2 = \left\|\frac{f^{(n+1)}(\xi)}{(n+1)!}\, (x-x_0)(x-x_1)\cdots (x-x_n) \right\|_2,\end{split}\notag
\end{gather}
þá veljum við \(x_1,\ldots,x_n\) þannig að
\((x-x_0)(x-x_1)\cdots (x-x_n) = \tilde P_{n+1}\).
Þ.e. \(x_j\) þurfa að vera rætur stöðluðu Legendre margliðunnar af
stigi \(n+1\).


\subsection{Núllstöðvar \(P_n\), fyrir \(n=1,\ldots,10\)}
\label{kafli03:nullstovar-fyrir}
Ólíkt Chebyshev margliðunum þá er ekki hlaupið að því að finna rætur
\(\tilde P_{n+1}\). Þannig að við þurfum að reikna þær tölulega og
geyma í töflu.

\includegraphics{legendre.png}


\subsection{Dæmi um óheppilega skiptipunkta skoðað aftur}
\label{kafli03:id7}
Skoðum enn einu sinni fallið \(f(x) = 1/(25x^2+1)\), en í stað þess
að taka 9 jafndreifaða brúunarpunkta á bilinu \([-1,1]\), þá skulum
við nota Legendre margliðurnar til að finna 9 punkta á bilinu.

\includegraphics{vond_bruun3.png}


\subsection{Athugasemd um \(\ell_\infty\) og \(\ell_2\)}
\label{kafli03:athugasemd-um-og}
\begin{notice}{note}{Athugasemd:}
Það má líta þannig á þetta að \(\ell_\infty\) staðallinn
mæli hámarksskekkju
og \(\ell_2\) mæli einhvers konar,,heildarskekkju'',
þar sem skekkjan er reiknað með heildinu hér á undan og svarar því hér um
bil til flatarmálsins á milli fallsins og brúunarmargliðunnar.
\end{notice}
\begin{itemize}
\item {} 
\(\ell_\infty\) staðallinn mælir hámarksskekkju, þannig að með
því nota Chebyshev margliðurnar þá erum við að reyna að lágmarka mestu skekkju
á bilinu.

\item {} 
\(\ell_2\) mæli einhvers konar,,heildarskekkju'',
þar sem skekkjan er reiknað með heildi.
Þannig að með því að nota Legendre margliðurnar þá erum við
í einhverjum skilningi að lágmarka flatarmál.

\end{itemize}

\index{brúun!skekkjumat}

\section{Skekkjumat}
\label{kafli03:index-18}\label{kafli03:skekkjumat}

\subsection{Nálgun á föllum með margliðum}
\label{kafli03:nalgun-a-follum-me-marglium}
Lítum nú aftur á almenna brúunarverkefnið og gefum okkur að tölurnar
\(y_i^{(j)}\) séu af gerðinni \(f^{(j)}(a_i)\) þar sem
\(f : I \to {{\mathbb  R}}\) er fall á bili \(I\) sem inniheldur
alla punktana \(a_1, \ldots, a_k\).

Þá snýst brúunarverkefnið um að finna margliðu af stigi \(\leq m\)
sem uppfyllir
\begin{gather}
\begin{split}p^{(j)}(a_i) = f^{(j)}(a_i), \quad
  j = 0, \ldots, m_i-1, \quad i = 1, \ldots, k.\end{split}\notag
\end{gather}
Við vitum að lausn þess er ótvírætt ákvörðuð. Ef við notum Newton form
lausnarinnar, þá táknum við mismunakvótana með
\begin{gather}
\begin{split}f[x_i,\ldots,x_{i+j}]\end{split}\notag
\end{gather}
í stað
\begin{gather}
\begin{split}y[x_i,\ldots,x_{i+j}]\end{split}\notag
\end{gather}

\subsection{Nálgun á fallgildum}
\label{kafli03:nalgun-a-fallgildum}
Runurnar \((x_0,\ldots,x_m)\) og \((y_0,\ldots,y_m)\) eru
skilgreindar með
\begin{gather}
\begin{split}(x_0,x_1,\ldots,x_m) =
  (\underbrace{a_1, \ldots, a_1}_{m_1 \, \text{sinnum}},
  \underbrace{a_2, \ldots , a_2}_{m_2 \, \text{sinnum}},
  \ldots ,
  \underbrace{a_k, \ldots , a_k}_{m_k \, \text{sinnum}})\end{split}\notag
\end{gather}
og
\begin{gather}
\begin{split}\begin{gathered}
  (y_0,y_1,\ldots,y_m) =
  (f^{(0)}(a_1), \ldots, f^{(m_1-1)}(a_1),
f^{(0)}(a_2), \ldots, f^{(m_2-1)}(a_2) \\ \ldots,
  f^{(0)}(a_k), \ldots, f^{(m_k-1)}(a_k))
  \label{bru.margfald.5}\end{gathered}\end{split}\notag
\end{gather}

\subsection{Skekkjumat}
\label{kafli03:id8}
Nú tökum við punkt \(x \in I\) og spyrjum um skekkjuna
\(f(x) - p(x)\) í nálgun á \(f(x)\) með \(p(x)\). Ef
\(x\) er einn punktana \(a_1, \ldots, a_k\), þá er
\(p(x) = f(x)\) og skekkjan þar með 0, svo við skulum gera ráð fyrir
að \(x \not= a_i\), \(i = 1, \ldots, k\).

Við bætum nú \((x,f(x))\) sem einföldum brúunarpunkti við alhæfða
brúunar verkefnið og fáum sem lausn \(q(t)\) á þessu aukna verkefni.
Margliðan \(q\) er af stigi \(\leq m+1\). Við notum táknið
\(t\) fyrir breytu, því \(x\) er frátekið.

Þá uppfyllir \(q(t)\) að \(q(x) = f(x)\) auk allra skilyrðanna
\begin{gather}
\begin{split}q^{(j)}(a_i) = p^{(j)}(a_i) = f^{(j)}(a_i)\end{split}\notag
\end{gather}
í verkefninu sem við byrjuðum með.

Við getum þá skrifað (sjá \emph{Newton-margliður} til hliðsjónar)
\begin{gather}
\begin{split}\begin{aligned}
  q(t) &= p(t) + f[x_0,\ldots,x_m,x](t-x_0)\cdots(t-x_m) \\
  &= p(t) + f[x_0,\ldots,x_m,x](t-a_1)^{m_1}\cdots(t-a_k)^{m_k}.\end{aligned}\end{split}\notag
\end{gather}
Þegar við gefum breytunni \(t\) gildið \(x\), þá fáum við
\(q(x) = f(x)\) og því fæst formúla fyrir skekkjunni
\begin{gather}
\begin{split}f(x) - p(x)
  = f[x_0,\ldots,x_m,x](x-a_1)^{m_1}\cdots(x-a_k)^{m_k}\end{split}\notag
\end{gather}
Nú ætlum við að finna leið til þess að meta skekkjuliðinn. Til þess
þurfum við að gefa okkur að \(f\) hafi að minnsta kosti \(m+1\)
afleiðu.


\subsection{Tilfellið þegar við höfum aðeins einn punkt}
\label{kafli03:tilfelli-egar-vi-hofum-aeins-einn-punkt}
Munum nú að í tilfellinu þegar við erum bara með einn punkt \(a_1\),
þá erum við með \(m+1\) skilyrði
\begin{gather}
\begin{split}p^{(j)}(a_1)=f^{(j)}(a_1), \qquad j=0,\dots,m\end{split}\notag
\end{gather}
og við fáum að \(p\) er Taylor-margliða fallsins \(f\) í
punktinum \(a_1\). Þá er \(x_0=\cdots=x_m=a_1\) og við fáum
\begin{gather}
\begin{split}f(x) - p(x)
  = f[x_0,\ldots,x_m,x](x-a_1)^{m+1}\end{split}\notag
\end{gather}
Nú segir setning Taylors okkur að til sé punktur \(\xi\) milli
\(a_1\) og \(x\) þannig að
\begin{gather}
\begin{split}f(x) - p(x)
  = \dfrac{f^{(n+1)}(\xi)}{(n+1)!}(x-a_1)^{m+1}\end{split}\notag
\end{gather}
Við getum því dregið þá ályktun að í þessu sértilfelli er
\begin{gather}
\begin{split}f[x_0,\ldots,x_m,x]=\dfrac{f^{(n+1)}(\xi)}{(n+1)!}\end{split}\notag
\end{gather}
Það kemur í ljós að þetta er almenn regla sem gildir fyrir \emph{öll}
alhæfðu brúunarverkefnin.

\textbf{Tilfellið} \(m=1\) \textbf{er meðalgildisreglan}

Munum að tilfellið \(m=1\) er meðalgildisreglan
\begin{gather}
\begin{split}f[a_1,x]=\dfrac{f(x)-f(a_1)}{x-a_1}=f'(\xi).\end{split}\notag
\end{gather}

\subsection{Margfeldni núllstöðva}
\label{kafli03:margfeldni-nullstova}
Samfellt fall \(\varphi\) á bili \(I\) er sagt hafa núllstöð \emph{af
stigi að minnsta kosti} \(m>0\) í punktinum \(a\in I\), ef til
er samfellt fall \(\psi\) á \(I\) þannig að
\begin{gather}
\begin{split}\varphi(x)=(x-a)^m\psi(x)\end{split}\notag
\end{gather}
Við segjum að \(\varphi\) hafi núllstöð af \emph{margfeldni} \(m\) ef
\(\psi(a)\neq0\).

Athugið að ef \(\varphi\) er deildanlegt \(I\) með samfellda
afleiðu, þá er \(\psi\) deildanlegt með samfellda afleiðu í
\(I\setminus\{a\}\) og við höfum
\begin{gather}
\begin{split}\begin{aligned}
  \varphi'(x)&=m(x-a)^{m-1}\psi(x)+(x-a)^m\psi'(x)\\
&= (x-a)^{m-1} \big(m\psi(x)+(x-a)\psi'(x)\big)\end{aligned}\end{split}\notag
\end{gather}
Ef afleiðan \(\psi'\) er takmörkuð í grennd um \(a\), þá sjáum
við á þessari formúlu að \(\varphi'\) hefur núllstöð af stigi að
minnsta kosti \(m-1\) í \(a\).

Hugsum okkur nú að við séum með \(a_1,\dots,a_k\) ólíka punkta í
bilinu \(I\) og að \(m_1,\dots,m_k\) séu jákvæðar náttúrlegar
tölur.

Ef fallið \(\varphi\) hefur núllstöðvar í öllum punktunum
\(a_j\) og núllstöðin \(a_j\) er af stigi að minnsta kosti
\(m_j\). Við segjum að þá hafi \(\varphi\) \emph{að minnsta kosti}
\begin{gather}
\begin{split}n=m_1+\cdots+m_k\end{split}\notag
\end{gather}
\emph{núllstöðvar taldar með margfeldni}.

Eins þá segjum við að \(\varphi\) hafi \(n\) núllstöðvar í
\(\{a_1,\dots,a_k\}\) \emph{taldar með margfeldni} ef \(\varphi\)
hefur núllstöðvar í öllum punktum \(a_1,\dots,a_k\) og samanlögð
margfeldni þeirra er \(n\)

Hugsum okkur nú að fallið \(\varphi\) hafi núllstöð af stigi
\(m_j\) í punktunum \(a_j\) fyrir öll \(j=1,\dots,k\) og að
\(n=m_1+\cdots+m_k\).

Til einföldunar gerum við ráð fyrir að
\begin{gather}
\begin{split}a_1<a_2<\cdots<a_k.\end{split}\notag
\end{gather}
Þá gefur meðalgildissetningin að \(\varphi'\) hefur að minnsta kosti
eina núllstöð á sérhverju bilanna
\begin{gather}
\begin{split}]a_1,a_2[, \  ]a_2,a_3[, \ \dots \  ]a_{k-1},a_k[\end{split}\notag
\end{gather}
Þau eru samanlagt \(k-1\) talsins. Að auki vitum við að
\(\varphi'\) hefur núllstöðvar af stigi að minnsta kosti
\(m_j-1\) í punktinum \(a_j\). Ef við leggjum þetta saman, þá
fáum við að \(\varphi'\) hefur núllstöðvar af margfeldni að minnsta
kosti
\begin{gather}
\begin{split}k-1+(m_1-1)+\cdots+(m_k-1)=n-1\end{split}\notag
\end{gather}
í minnsta lokaða bilinu sem inniheldur alla punktana
\(a_1,\dots,a_k\).


\subsection{Setning: Skekkjumat}
\label{kafli03:setning-skekkjumat}
Nú ætlum við að sýna fram á að fyrir föll \(f\) sem eru
\((m+1)\) sinnum samfellt deildanleg að til sé \(\xi\) á minnsta
bili sem inniheldur \(a_1, \ldots, a_k\) og \(x\) þannig að
\begin{gather}
\begin{split}f[x_0,\ldots,x_m,x] = \frac{f^{(m+1)}(\xi)}{(m+1)!}\end{split}\notag
\end{gather}
Við skilgreinum fallið
\begin{gather}
\begin{split}g(t) = f(t) - p(t) - \lambda w(t),\end{split}\notag
\end{gather}
þar sem
\begin{gather}
\begin{split}w(t) = (t-a_1)^{m_1} \cdots (t-a_k)^{m_k}\end{split}\notag
\end{gather}
og talan \(\lambda\) er valin þannig að \(g(x) = 0\).

Nú er \(p^{(j)}(a_i)=f^{(j)}(a_i)\) fyrir \(j=0,\dots,m_i-1\),
þá gefur setning Taylors okkur að \(g\) hefur núllstöð af stigi
\(m_i\) í sérhverjum punktanna \(a_i\). Auk þess hefur \(g\)
núllstöð í \(x\). Samanlagt eru þetta að minnsta kosti \(m+2\)
núllstöðvar taldar með margfeldni.

Höfum:
\begin{itemize}
\item {} 
\(g\) hefur að minnsta kosti \(m+2\) núllstöðvar taldar með margfeldni,

\item {} 
\(g'\) hefur að minnsta kosti \(m+1\) núllstöð talda með margfeldni,

\item {} 
\(g''\) hefur að minnsta kosti \(m\) núllstöðvar taldar með margfeldni

\item {} 
og þannig áfram, þar til við ályktum að

\item {} 
\(g^{(m+1)}\) hefur að minnsta kosti eina núllstöð.

\end{itemize}

Tökum eina slíka og köllum hana \(\xi\).

Munum að
\begin{gather}
\begin{split}g(t) = f(t) - p(t) - \lambda w(t),\end{split}\notag
\end{gather}
þar sem
\begin{gather}
\begin{split}w(t) = (t-a_1)^{m_1} \cdots (t-a_k)^{m_k}=t^{m+1}+b_mt^m+\cdots+b_1t+b_0\end{split}\notag
\end{gather}
Margliðan \(p\) hefur stig \(\leq m\) svo
\(p^{(m+1)}(x) = 0\) fyrir öll \(x\)

og margliðan \(w\) er af stigi \(m+1\) með stuðul \(1\) við
hæsta veldið, svo \(w^{(m+1)}(t) = (m+1)!\). Við höfum því
\begin{gather}
\begin{split}0 = g^{(m+1)}(\xi) = f^{(m+1)}(\xi) - \lambda \cdot (m+1)!\end{split}\notag
\end{gather}
sem jafngildir því að
\begin{gather}
\begin{split}\lambda =\dfrac{f^{(m+1)}(\xi)}{(m+1)!}\end{split}\notag
\end{gather}
Við setjum nú inn \(t=x\) sem gefur
\begin{gather}
\begin{split}0=g(x) = f(x) - p(x) - \lambda w(x),\end{split}\notag
\end{gather}
og við fáum þar með formúlu fyrir skekkjunni á nálgun á \(f(x)\) með
alhæfðu brúunarmargliðunni \(p(x)\),
\begin{gather}
\begin{split}f(x) - p(x) =\lambda w(x) = \dfrac{f^{(m+1)}(\xi)}{(m+1)!}
(x-a_1)^{m_1} \cdots (x-a_k)^{m_k}\end{split}\notag
\end{gather}

\subsection{Samantekt}
\label{kafli03:id9}
Ef gefið er fall \(f:I\to {{\mathbb  R}}\) á bili \(I\),
\(a_1,\dots,a_k\) í \(I\), með \(a_j\neq a_k\) ef
\(j\neq k\), jákvæðar heiltölur \(m_1,\dots,m_k\), talan
\(m\) er skilgreind með \(m=m_1+\cdots+m_k-1\), og gert er ráð
fyrir að \(f\in C^{m+1}(I)\), þá er til nákvæmlega ein margliða
\(p\) af stigi \(\leq m\) þannig að
\begin{gather}
\begin{split}p^{(j)}(a_i)=f^{(j)}(a_i), \qquad j=0,\dots, m_i-1, \quad i=1,\dots,k.\end{split}\notag
\end{gather}
Newton-form margliðunnar \(p\) er gefið með
\begin{gather}
\begin{split}p(x)=f[x_0]+f[x_0,x_1](x-x_0)+\cdots+f[x_0,\dots,x_m](x-x_0)\cdots(x-x_{m-1})\end{split}\notag
\end{gather}
þar sem mismunakvótarnir \(f[x_i,\dots,x_{i+j}]\) eru skilgreindir
sem \(y[x_i,\dots,x_{i+j}]\) út frá gögnunum \(y^{(j)}_i\).
Fyrir sérhvert \(x\) í \(I\) er skekkjan \(f(x)-p(x)\) í
nálgun á \(f(x)\) með \(p(x)\) gefin með
\begin{gather}
\begin{split}f(x)-p(x)=f[x_0,\dots,x_m,x](x-a_1)^{m_1}\cdots(x-a_k)^{m_k}.\end{split}\notag
\end{gather}
Fyrir sérhvert \(i=1,\dots,k\) og \(j=0,\dots,m-i\) þá gildir að
til er tala \(\xi\) á minnsta bilinu sem inniheldur
\(x_i\dots,x_{i+j}\) þannig að
\begin{gather}
\begin{split}f[x_i,\dots,x_{i+j}]=\dfrac{f^{(j)}(\xi)}{j!},\end{split}\notag
\end{gather}
því gildir sérstaklega að til er tala \(\xi\) á minnsta bilinu sem
inniheldur \(a_1,\dots,a_k\) og \(x\) þannig að
\begin{gather}
\begin{split}f(x)-p(x)=\dfrac{f^{(m+1)}(\xi)}{(m+1)!}(x-a_1)^{m_1}\cdots(x-a_k)^{m_k}.\end{split}\notag
\end{gather}

\subsection{Sýnidæmi}
\label{kafli03:id10}
Látum \(f(x)=x^2\ln x\).
\begin{enumerate}
\item {} 
Setjið upp mismunakvótatöflu til þess að reikna út brúunarmargliðu
\(p\) af stigi \(\leq 3\) fyrir fallið \(f\), sem hefur tvo
tvöfalda brúunarpunkta \(a_1=1\) og \(a_2=2\). Skrifið upp
Newton-form margliðunnar \(p\).

\item {} 
Reiknið út \(p(1.3)\). Notið aðferðarskekkju fyrir margliðubrúun
til þess að meta skekkjuna \(f(1.3)-p(1.3)\) að ofan og neðan og
fáið þannig bil þar sem rétta gildið liggur. Veljið miðpunkt bilsins sem
nálgunargildi fyrir \(f(1.3)\) og afrúnið gildið miðað við mörk
bilsins.

\item {} 
Látum nú \(q\) vera brúnunarmargliðuna af stigi \(\leq 4\) sem
uppfyllir sömu skilyrði og gefin eru í fyrsta lið að viðbættu því að
\(a_2=2\) á að vera þrefaldur brúunarpunktur. Sýnið hvernig hægt er
að ákvarða mismunakvótatöfluna fyrir \(q\) með því að stækka töfluna
í \textbf{a)}. Ákvarðið síðan \(q\) og reiknið út \(q(1.3)\).

\end{enumerate}

\textbf{1. og 2.}
Til þess að spara pláss skulum leysa fyrsta og þriðja lið báða í einu
með því að reikna strax út mismunakvótatöfluna fyrir fjórða stigs margliðuna í
3. lið. Punktarnir \(x_0,\dots,x_4\) eru þá \(1,1,2,2,2\) og
við höfum gefin fallgildin
\begin{gather}
\begin{split}f(1)=f[x_0]=f[x_1]=0 \qquad  \text{ og } \qquad
f(2)=f[x_2]=f[x_3]=f[x_4].\end{split}\notag
\end{gather}
Í 1. lið eru punktarnir tvöfaldir svo við höfum gefin gildi
afleiðunnar \(f'(x)=2x\ln x+x\) í punktunum \(1\) og \(2\).
\begin{gather}
\begin{split}f'(1)=f[1,1]=f[x_0,x_1]=1 \ \text{ og } \
f'(2)=f[2,2]=f[x_2,x_3]=4\ln 2+2.\end{split}\notag
\end{gather}
Í 3. lið er gildið á 2. afleiðu \(f''(x)=2\ln x +3\) gefið í
punktinum \(2\). Það gefur okkur
\begin{gather}
\begin{split}f''(2)/2!=f[2,2,2]=f[x_2,x_3,x_4]=\ln 2+\tfrac 32.\end{split}\notag
\end{gather}
Við setjum þessi gildi inn í mismunakvótatöfluna og fyllum hana út með
því að taka mismunakvóta milli allra gilda
\begin{gather}
\begin{split}\begin{matrix}
i&x_i & f[x_i] &f[x_i,x_{i+1}] & f[x_i,x_{i+1},x_{i+2}] &
f[x_i,\dots,x_{i+3}]&f[x_i,\dots,x_{i+4}] \\\hline
0&1&0     &1       &4\ln 2-1        &-4\ln 2+3& 5\ln 2-\tfrac 72\\
1&1&0     &4\ln 2  &2               &\ln 2-\tfrac 12\\
2&2&4\ln 2&4\ln 2+2&\ln 2+\tfrac 32\\
3&2&4\ln 2&4\ln 2+2\\
4&2&4\ln 2
\end{matrix}\end{split}\notag
\end{gather}
Margliðan í 1. lið er því
\begin{gather}
\begin{split}p(x)=(x-1)+(4\ln 2-1)(x-1)^2+(-4\ln 2+3)(x-1)^2(x-2).\end{split}\notag
\end{gather}
en í 3. lið er hún
\begin{gather}
\begin{split}q(x)=p(x)+(5\ln 2-\tfrac 72)(x-1)^2(x-2)^2\end{split}\notag
\end{gather}
\textbf{2.} Við stingum gildinu \(x=1.3\) inn í margliðuna og fáum
\(p(1.3)=0.445206074\). Skekkjan er
\begin{gather}
\begin{split}f(x)-p(x)=\dfrac{f^{(4)}(\xi)}{4!}(x-1)^2(x-2)^2\end{split}\notag
\end{gather}
þar sem \(\xi\) er einhver punktur á bilinu \([1,2]\).

Við þurfum því að meta fjórðu afleiðuna,
\begin{gather}
\begin{split}\begin{aligned}
f(x) &=x^2\ln x, \quad
f'(x)=2x\ln x+x, \quad
f''(x)=2\ln x+3, \quad
\\
f'''(x)& =2/x, \quad
f^{(4)}(x)=-2/x^2.\end{aligned}\end{split}\notag
\end{gather}
Ef \(x\in [1,2]\), þá höfum við matið
\(-2\leq f^{(4)}(x)\leq -\tfrac 12\).

Af ójöfnunum \(-2\leq f^{(4)}(x)\leq -\tfrac 12\) leiðir síðan að
\begin{gather}
\begin{split}\alpha=\dfrac{-2\cdot(0.3)^2\cdot(-0.7)^2}{24}\leq f(1.3)-p(1.3)
\leq \dfrac{-0.5\cdot(0.3)^2\cdot(-0.7)^2}{24}=\beta.\end{split}\notag
\end{gather}
Við reiknum út úr báðum brotunum
\begin{gather}
\begin{split}\alpha=-0.003675 \qquad \text{ og } \qquad
\beta=-0.00091875.\end{split}\notag
\end{gather}
þar með er \(f(1.3)\) á bilinu milli \(p(1.3)+\alpha=0.441531\)
og \(p(1.3)+\beta=0.444287\).

Nálgunargildi okkar á að vera miðpunktur þessa bils og algildi
skekkjunnar verður þá hálf billengdin. Það færir okkur nálgunina
\(f(1.3) \approx 0.442909\) og skekkjuna \(\pm 0.0014\). Réttur
afrúningur er \(f(1.3)=0.44\).

Við eigum aðeins eftir að reikna út gildi margliðunnar \(q\) í
punktinum \(1.3\). Út úr mismunakvótatöflunni fáum við
\begin{gather}
\begin{split}q(x)=p(x)+(5\ln 2-\tfrac 72)(x-1)^2(x-2)^2\end{split}\notag
\end{gather}
sem gefur okkur gildið
\begin{gather}
\begin{split}q(1.3)=0.445206074-0.001511046=0.4436950278\end{split}\notag
\end{gather}
Til samanburðar höfum við rétt gildi
\begin{gather}
\begin{split}f(1.3)=0.443395606950060\ldots.\end{split}\notag
\end{gather}
\index{brúun!splæsibrúun}\index{splæsibrúun}

\section{Splæsibrúun}
\label{kafli03:splaesibruun}\label{kafli03:index-19}
Látum \((t_0,y_0),\dots,(t_n,y_n)\) vera punkta í plani og gerum ráð
fyrir að \(a=t_0<t_1<\cdots<t_n=b\).

Við höfum nú lært að ákvarða margliðu \(p\) af stigi \(\leq n\)
sem tekur gildin \(y_i\) í punktunum \(t_i\).

Ef punktarnir liggja á grafi fallsins \(f\) og nota á margliðuna til
þess að nálga fallgildi \(f\), þá getur það verið ýmsum erfiðleikum
bundið þegar stig hennar stækkar eins og við sáum í byrjun kaflans
{\hyperref[kafli03:skynsamlegir-skiptipunktar-og-chebyshev-margliur]{Skynsamlegir skiptipunktar og Chebyshev margliður}}.
Lausnin þar var að reyna að velja brúunarpunktana skynsamlega. Ef við
hins vegar getum ekki valið brúunarpunktana eftir eigin höfði þá erum
við í vandræðum og þurfum við að leita annarra leiða.


\subsection{Almennt um splæsibrúun}
\label{kafli03:almennt-um-splaesibruun}
Splæsibrúun er leið út úr þessum vandræðum.

Með henni er fundið samfellt fall \(S\) sem brúar gefnu punktana,
\(S(t_i)=y_i\), og er þannig að einskorðun þess við hlutbilin
\([t_i,t_{i+1}]\) er gefið með margliðu af stigi \(\leq m\), þar
sem \(m\) er fyrirfram gefin tala.

Algengast er að nota \(m=3\).

\index{splæsibrúun!fyrsta stigs}

\subsection{Fyrsta stigs splæsibrúun:}
\label{kafli03:index-20}\label{kafli03:fyrsta-stigs-splaesibruun}
Ef stigið \(m\) er \(1\), þá erum við einfaldlega að draga
línustrik milli punktanna og sjáum í hendi okkar að lausnin er
\begin{gather}
\begin{split}S(x) = \begin{cases}
        S_0(x) = \dfrac {y_1-y_0}{t_1-t_0}(x-t_0)+y_0,
            & x \in [t_0,t_1],\\
        S_1(x) = \dfrac {y_2-y_1}{t_2-t_1}(x-t_1)+y_1,
            & x \in [t_1,t_2],\\
        \vdots & \\
        S_{n-1}(x) = \dfrac {y_n-y_{n-1}}{t_n-t_{n-1}}
            (x-t_{n-1})+y_{n-1}, &x \in [t_{n-1},t_n].
    \end{cases}\end{split}\notag
\end{gather}
Þessi aðferð er ekki mikið notuð því hún er ósannfærandi fyrir
deildanleg föll.

\index{splæsibrúun!þriðja stigs}

\subsection{Þriðja stigs splæsibrúun}
\label{kafli03:index-21}\label{kafli03:rija-stigs-splaesibruun}
Algengast er að framkvæma splæsibrúun með þriðja stigs margliðum.

Við skulum tákna einskorðun \(S\) við hlutbilið
\([t_i,t_{i+1}]\) með \(S_i\) og skrifa
\begin{gather}
\begin{split}S_i(x) = a_i+b_i(x-t_i)+c_i(x-t_i)^2+d_i(x-t_i)^3,
        \qquad x\in [t_i,t_{i+1}).\end{split}\notag
\end{gather}
Við ætlum að leiða út jöfnur fyrir stuðlunum \(a_i, b_i, c_i\) og
\(d_i\). Kröfurnar sem við setjum eru:
\begin{enumerate}
\item {} 
\(S\) er tvisvar sinnum samfellt deildanlegt á öllu bilinu \([a,b]\)

\item {} 
\(S\) taki gildin \(y_i\) í punktunum \(t_i\)

\end{enumerate}

Setjum til einföldunar \(h_i = t_{i+1}-t_i\) fyrir \(i = 0,
\ldots, n-1\).

Skilyrðin tvö má því skrifa sem eftirfarandi jöfnuhneppi:

Á hverju hlutbili \([t_i,t_{i+1}]\) höfum við:
\begin{gather}
\begin{split}S_i(x) = a_i+b_i(x-t_i)+c_i(x-t_i)^2+d_i(x-t_i)^3,
        \qquad x\in [t_i,t_{i+1}),\end{split}\notag
\end{gather}
sem þýðir að skilyrðin tvö má skrifa sem
\begin{gather}
\begin{split}\begin{aligned}
    a_i &=& &S_i(t_i)& &=& y_i
        &, \quad (1) \\
    a_i + b_ih_i + c_ih_i^2 + d_ih_i^3 &=& &S_i(t_{i+1})
        = S_{i+1}(t_{i+1})& &=& a_{i+1}
        &, \quad (2) \\
    b_i + 2c_ih_i + 3d_ih_i^2 &=& &S_i'(t_{i+1})
        = S_{i+1}'(t_{i+1})& &=& b_{i+1}
        &, \quad (3) \\
    2c_i + 6d_ih_i &=& &S_i''(t_{i+1})
        = S_{i+1}''(t_{i+1})& &=& 2c_{i+1}
        &, \quad (4)\end{aligned}\end{split}\notag
\end{gather}
Í (1) höfum við \(i = 0,\ldots,n\) og í (2)-(4) höfum við
\(i=0,\ldots,n-2\).

Samtals: \((n+1)+3(n-1)=4n-2\) línulegar jöfnur til þess að ákvarða
\(4n\) óþekktar stærðir.

Það er því ljóst að okkur vantar tvö skilyrði til þess að geta fengið
ótvírætt ákvarðaða lausn.

Fyrstu jöfnurnar gefa strax gildi \(a_i\) og (4) gefur að
\begin{gather}
\begin{split}d_i = \frac{c_{i+1}-c_i}{3h_i}, \quad i=0,\ldots,n-2\end{split}\notag
\end{gather}
Ef við setjum þetta inn í (2) og (3) fæst
\begin{gather}
\begin{split}\begin{aligned}
    a_{i+1} = a_i + b_ih_i + \frac{c_{i+1}+c_i}{3}h_i^2
        &, \quad i=0,\ldots,n-2 \\
    b_{i+1} = b_i + (c_{i+1} + c_i)h_i
        &, \quad i=0,\ldots,n-2\end{aligned}\end{split}\notag
\end{gather}
Þegar við leysum fyrri jöfnuna fyrir \(b_i\) fæst
\begin{gather}
\begin{split}b_i = \frac{a_{i+1}-a_i}{h_i}-\frac{c_i+c_{i+1}}{3}h_i
        , \quad i=0,\ldots,n-2\end{split}\notag
\end{gather}
og ef við setjum þetta inn í seinni jöfnuna fæst á endanum að
\begin{gather}
\begin{split}h_{i-1}c_{i-1} + 2(h_{i-1}+h_i)c_i + h_ic_{i+1} =
    \frac{3}{h_i}(a_{i+1}-a_i)
        - \frac{3}{h_{i-1}}(a_i-a_{i-1})
    , \quad i=1,\ldots,n-1\end{split}\notag
\end{gather}

\subsection{Jöfnuhneppið}
\label{kafli03:jofnuhneppi}\[
\left[ \begin{array}{cccccc}
.\text{?}.  & .\text{?}.       &&&& \\ 
h_0 & 2(h_0+h_1) & h_1 &&&\\
& h_1        & 2(h_1+h_2) & h_2 &&\\
&&&&&\\
&            & \ddots      & \ddots & \ddots &\\
&&&&&\\
&  &  & h_{n-2}  & 2(h_{n-2} + h_{n-1}) & h_{n-1}
\\ 
&  &  &   & .?.    & .?.
\end{array} \right]
\left[ \begin{array}{c}
c_0 \\ 
c_1 \\
c_2 \\
\\
\vdots \\
\\
c_{n-1} \\ 
c_n
\end{array} \right]
\\
= 3\left[ \begin{array}{c}
.?. \\
\dfrac{a_2-a_1}{h_1} - \dfrac{a_1-a_0}{h_0} \\
\dfrac{a_3-a_2}{h_2} - \dfrac{a_2-a_1}{h_1} \\
\vdots \\
\dfrac{a_n-a_{n-1}}{h_{n-1}}
- \dfrac{a_{n-1}-a_{n-2}}{h_{n-2}}
\\ 
\end{array} \right]\]
En það vantar í þetta einhver skilyrði á \(c_0\) og \(c_n\).

Þegar þau hafa verið sett inn, þá getum við leyst þetta hneppi, reiknað svo
gildi \(b_i\) og \(d_i\) og þá höfum við fundið splæsifallið
okkar.

Það eru til margar leiðir til að ákvarða \(c_0\) og \(c_n\), en
fjórar eru algengastar.


\subsection{Tilfelli 1: Ekki-hnúts endaskilyrði}
\label{kafli03:tilfelli-1-ekki-hnuts-endaskilyri}
Ef við höfum engar upplýsingar um fallið \(f\) í \(t_1\) og
\(t_{n-1}\) liggur beint við að krefjast þess að \(S'''\) sé
samfellt þar, sem þýðir að \(d_0 = d_1\) og
\(d_{n-2} = d_{n-1}\). Með að nota jöfnurnar fyrir \(d_i\) má
skrifa þetta sem
\begin{gather}
\begin{split}\begin{aligned}
    h_1c_0 - (h_0 + h_1)c_1 + h_0c_2 = 0 \\
    h_{n-1}c_{n-2}-(h_{n-2}+h_{n-1})c_{n-1}+h_{n-2}c_n = 0\end{aligned}\end{split}\notag
\end{gather}
og þessar jöfnur, ásamt hinum, má leysa til að ákvarða \(c_i\)-in.


\subsection{Tilfelli 2: Þvinguð endaskilyrði}
\label{kafli03:tilfelli-2-vingu-endaskilyri}
Ef hallatala fallsins \(f\) er þekkt í endapunktum bilsins er
eðlilegt að nota þær upplýsingar við ákvörðun splæsifallsins. Gerum því
ráð fyrir að \(f'(t_0) = A\) og \(f'(t_n) = B\). Skilyrðið
\(S'(t_0) = A\) gefur þá að
\begin{gather}
\begin{split}A = \frac{a_1-a_0}{h_0} - \frac{2c_0+c_1}{3}h_0,\end{split}\notag
\end{gather}
eða
\begin{gather}
\begin{split}2h_0c_0 + h_0c_1 =
    3 \left( \frac{a_1-a_0}{h_0} - A \right)\end{split}\notag
\end{gather}
og \(S'(t_n) = B\) gefur
\begin{gather}
\begin{split}B = b_{n-1} + 2c_{n-1}h_{n-1} + 3d_{n-1}h_{n-1}^2\end{split}\notag
\end{gather}
og með að nota formúlurnar fyrir \(b_{n-1}\) og \(d_{n-1}\)
fæst
\begin{gather}
\begin{split}c_{n-1}h_{n-1} + 2c_nh_{n-1} =
    3 \left( B  - \frac{a_n-a_{n-1}}{h_{n-1}} \right).\end{split}\notag
\end{gather}

\subsection{Tilfelli 3: Náttúrleg endaskilyrði}
\label{kafli03:tilfelli-3-natturleg-endaskilyri}
Einfaldasta lausnin er að setja \(c_0 = c_n = 0\), en það jafngildir
því að \(S''(t_0) = S''(t_n) = 0\).


\subsection{Tilfelli 4: Lotubundið endaskilyrði}
\label{kafli03:tilfelli-4-lotubundi-endaskilyri}
Hugsum okkur að við viljum framlengja \(S\) í tvisvar samfellt
deildanlegt \((b-a)\)-lotubundið fall á \({{\mathbb  R}}\). Það
setur skilyrðin
\begin{gather}
\begin{split}y_0 = S(t_0) = S_(t_n) = y_n, \quad
    S'(t_0) = S'(t_n), \quad
    \text{ og } \quad
    S''(t_0) = S''(t_n)\end{split}\notag
\end{gather}
Fljótséð er að \(S''(t_0) = S''(t_n)\) þýðir að \(c_0 = c_n\),
eða
\begin{gather}
\begin{split}c_0 - c_n = 0.\end{split}\notag
\end{gather}
Þetta er fyrri jafnan sem við þurfum.

Nú gefur \(S'(t_0) = S'(t_n)\) að
\begin{gather}
\begin{split}b_0 = b_{n-1} + 2c_{n-1}h_{n-1} + 3d_{n-1}h_{n-1}^2\end{split}\notag
\end{gather}
og með að setja inn formúlurnar fyrir \(b_0, b_{n-1}, d_{n-1}\) og
nota að \(c_0 = c_n\) fæst jafnan
\begin{gather}
\begin{split}h_0c_1 + 2h_{n-1}c_{n-1} + (2h_0 + 2h_{n-1})c_n
    = 3 \left( \frac{a_1-a_0}{h_0}
        - \frac{a_n-a_{n-1}}{h_{n-1}} \right).\end{split}\notag
\end{gather}
\index{brúun!ferlar}

\subsection{Teikning á ferlum í planinu}
\label{kafli03:teikning-a-ferlum-i-planinu}\label{kafli03:index-22}
Hægt er að nota brúun til þess að nálga ferla í \(\mathbb R^n\).
Skoðum tilvikið \(n=2\).

Gerum nú ráð fyrir að við höfum gefna punkta \((x_0,y_0),\dots,(x_n,y_n) \in \mathbb R^2\)
og að við viljum finna samfelldan splæsiferil í gegnum þá. Þetta er gert
í nokkrum skrefum:
\begin{enumerate}
\item {} 
Ákveðið er stikabil \([a,b]\) og skiptingu á því
\begin{gather}
\begin{split}a=t_0<t_1<\cdots<t_n=b\end{split}\notag
\end{gather}
til dæmis \([0,n]\) og skiptinguna
\begin{gather}
\begin{split}0=t_0<t_1=1<\cdots<t_n=n.\end{split}\notag
\end{gather}
\item {} 
Ákveðið er hvaða endaskilyrði eiga við.

\item {} 
Búin eru til tvö splæsiföll \(R(t)\) fyrir punktasafnið
\(x_0,\dots,x_n\) og \(S(t)\) fyrir punktasafnið
\(y_0,\dots,y_n\).

\item {} 
Stikaferillinn \([a,b]\ni t\mapsto (R(t),S(t))\) er síðan teiknaður, en hann uppfyllir
\((R(t_j),S(t_j))=(x_j,y_j)\), \(j=0,\dots,n\).

\end{enumerate}

\begin{notice}{warning}{Aðvörun:}
Athugið að hér er \(t\) breytan okkar en \(x\) og \(y\) eru gildin sem við
viljum að ferillinn taki.

Þetta er frábrugðið því þegar við skoðum graf af einni breytu en þá er
\(x\) venjulega breytan og \(y\) gildin sem viljum taka.
\end{notice}

\index{aðferð minnstu fervika}

\section{Aðferð minnstu fervika}
\label{kafli03:index-23}\label{kafli03:afer-minnstu-fervika}
Látum \((x_1,y_1),\dots,(x_m,y_m)\) vera safn punkta í plani með
\(x_j\in
[a,b]\) fyrir öll \(j\) og látum \(f_1,\dots,f_n\) vera raungild
föll á \([a,b]\).

Við viljum finna það fall \(f\) af gerðinni
\begin{gather}
\begin{split}f(x)=c_1f_1(x) + \cdots + c_nf_n(x)\end{split}\notag
\end{gather}
með stuðla \(c_1, \ldots, c_n\) þannig að punktarnir
\((x_j,f(x_j))\) nálgi gefna punktasafnið sem best og þá er átt við
að ferningssummuna
\begin{gather}
\begin{split}\sum_{i=1}^m\big(y_i-f(x_i)\big)^2\end{split}\notag
\end{gather}
verði eins lítil og mögulegt er.

\index{jafna bestu línu}

\subsection{Jafna bestu línu}
\label{kafli03:index-24}\label{kafli03:jafna-bestu-linu}
Flestir hafa heyrt talað um bestu línu gegnum punktasafn, hún fæst með
að taka hér \(f_1(x) = 1\) og \(f_2(x) = x\), en lítið mál er að
finna einnig besta fleygboga, bestu margliðu af fyrirfram ákveðnu stigi
eða einhverja aðra samantekt falla gegnum punktasafnið.


\subsection{Smávegis línuleg algebra}
\label{kafli03:smavegis-linuleg-algebra}
Til þess að finna þessi gildi á stuðlunum \(c_i\) er heppilegt að
notfæra sér nokkrar niðurstöður úr línulegri algebru. Fyrir gefin gildi
á \(c_1,\dots,c_n\) setjum við
\begin{gather}
\begin{split}b_i = f(x_i) = c_1f(x_i) + \cdots + c_n f_n(x_i),
    \qquad i=1,\dots,m,\end{split}\notag
\end{gather}
og skilgreinum síðan dálkvigrana
\begin{gather}
\begin{split}b = [b_1,\dots,b_m]^T,\qquad
    y = [y_1,\dots,y_m]^T,\quad \text{ og } \quad
    c = [c_1,\dots,c_n]^T,\qquad\end{split}\notag
\end{gather}
Þá er \(Ac=b\), þar sem \(A\) er \(m\times n\) fylkið
\begin{gather}
\begin{split}A = \left[\begin{matrix}
        f_1(x_1)& f_2(x_1) & \dots & f_n(x_1) \\
        f_1(x_2)& f_2(x_2) & \dots & f_n(x_2) \\
        \vdots &\vdots &\ddots &\vdots \\
        f_1(x_m)& f_2(x_m) & \dots & f_n(x_m)
    \end{matrix}\right].\end{split}\notag
\end{gather}
Verkefnið snýst nú um að finna þann vigur \(c\in {{\mathbb  R}}^n\)
sem lágmarkar
\begin{gather}
\begin{split}\sum_{i=1}^m \big(y_i-b_i\big)^2
    = \| y - b \|^2 = \| y - Ac \|^2\end{split}\notag
\end{gather}
þar sem \(\|\cdot\|\) táknar \href{https://en.wikipedia.org/wiki/Euclidean\_distance}{evklíðska fjarlægðina}
(staðalinn) á \({{\mathbb  R}}^m\).

Vigrar af gerðinni \(b= Ac\) spanna dálkrúm fylkisins \(A\) og
þá má skrifa sem línulegar samantektir af gerðinni
\begin{gather}
\begin{split}b = c_1A_1 + \cdots + c_nA_n\end{split}\notag
\end{gather}
þar sem \(A_j\) er dálkur númer \(j\).

Verkefnið snýst um að finna þann vigur í dálkrúminu sem næstur er
\(y\). Vigurinn \(b\) er næstur \(y\) ef og aðeins ef
\(y-b\) er hornréttur á alla vigra dálkrúmsins.

Þessi skilyrði má fá með innfeldi
\begin{gather}
\begin{split}A_j \cdot (y-b) = 0, \qquad j = 1, \ldots , n\end{split}\notag
\end{gather}
Með fylkjarithætti fæst ein jafna
\begin{gather}
\begin{split}A^T (y-b) = 0.\end{split}\notag
\end{gather}
Setjum nú inn \(b=Ac\). Þá ákvarðast \(c\) af hneppinu
\begin{gather}
\begin{split}A^T(y-Ac) = 0\end{split}\notag
\end{gather}
sem jafngildir
\begin{gather}
\begin{split}(A^TA)c = A^Ty\end{split}\notag
\end{gather}
Við þurfum því aðeins að leysa þetta jöfnuhneppi
\begin{gather}
\begin{split}(A^TA)c = A^Ty\end{split}\notag
\end{gather}
fyrir \(c\) til að finna stuðlana okkar. Ef fylkið \(A^TA\)
hefur andhverfu, þá fæst alltaf ótvírætt ákvörðuð lausn \(c\).

Ef fylkið \(A^TA\) hefur ekki andhverfu eða að það hefur ákveðu sem
er mjög nálægt \(0\), þá þurfum við að beita flóknari brögðum. Við
komum að því síðar.

\index{jafna bestu línu}

\subsection{Jafna bestu línu}
\label{kafli03:id11}\label{kafli03:index-25}
Algengt er að menn vilji finna beina línu sem best fellur að
punktasafninu \((x_1,y_1,)\dots,(x_m,y_m)\). Þá er \(n=2\) og
við tökum lausnagrunninn \(f_1(x)=1\) og \(f_2(x)=x\).

Fylkið er þá
\begin{gather}
\begin{split}A = \left[\begin{matrix}
        1& x_1\\
        1& x_2 \\
        \vdots &\vdots \\
        1& x_m
    \end{matrix}\right].\end{split}\notag
\end{gather}
og þar með
\begin{gather}
\begin{split}A^TA = \left[\begin{matrix}
        m& \sum_{j=1}^mx_j\\
        \sum_{j=1}^mx_j& \sum_{j=1}^mx_j^2
    \end{matrix}\right].
\qquad \text{ og } \qquad
    A^Ty = \left[\begin{matrix}
         \sum_{j=1}^my_j\\
        \sum_{j=1}^mx_jy_j
    \end{matrix}\right].\end{split}\notag
\end{gather}
\begin{notice}{note}{Athugasemd:}
Það er auðvelt að leysa þetta tilvik því við höfum einfalda formúlu fyrir andhverfum
\(2 \times 2\) fylkja (svo lengi sem ákveðan er ekki 0).

Sjá \href{https://en.wikipedia.org/wiki/Invertible\_matrix\#Inversion\_of\_2.C3.972\_matrices}{Wikipedia}.
\end{notice}

\index{jafna besta fleygboga}

\subsection{Jafna bestu annars stigs margliðu}
\label{kafli03:jafna-bestu-annars-stigs-margliu}\label{kafli03:index-26}
Ef við viljum finna bestu annars stigs margliðu gegnum punktasafnið, þá
er \(n=3\) og við tökum lausnagrunninn \(f_1(x)=1\),
\(f_2(x)=x\) og \(f_3(x)=x^2\).

Þetta val gefur fylkið
\begin{gather}
\begin{split}A = \left[\begin{matrix}
        1& x_1 & x_1^2\\
        1& x_2 & x_2^2\\
        \vdots &\vdots &\vdots\\
        1& x_m& x_m^2
    \end{matrix}\right].\end{split}\notag
\end{gather}
Fylkið \(A^TA\) er þá \(3\times 3\) og vigurinn \(A^Ty\) er
dálkvigur með \(3\) hnit.


\subsection{Sýnidæmi: besta annars stigs margliða}
\label{kafli03:synidaemi-besta-annars-stigs-marglia}
Gefin eru mæligildin

\[
\begin{array}{l|l|l|l|l|l|l|l|}
\hline
x & 0 &1 & 2 & 3 & 4 & 5 & 6\\
\hline
y & 2.7 & -0.5 & -1.7 & -1.9 & -1.5 & 0.2 & 2.3\\
\hline\end{array}
\]

Beitið aðferð minnstu fervika til þess að finna þá annars stigs margliðu
sem best fellur að þessum gögnum Teiknið upp gögnin og graf marliðunnar.

Við leitum hér að þremur tölum \(c_1\), \(c_2\) og
\(c_3\) þannig að annars stigs margliðan
\(f(x)=c_1f_1(x)+c_2f_2(x)+c_3f_3(x)\) falli sem best að gögnunum.
Grunnföllin þrjú eru \(f_1(x)=1\), \(f_2(x)=x\) og
\(f_3(x)=x^2\).

Í þessu dæmi er fylkið \(A\) gefið með
\begin{gather}
\begin{split}A=\left[\begin{matrix}
1 & 0&0\\
1 & 1&1\\
1&2&4\\
1&3&9\\
1&4&16\\
1&5&25\\
1&6&36
\end{matrix}\right],\end{split}\notag
\end{gather}
því stak númer \((i,j)\) í A er gefið með \(A_{ij} = f_j(x_i)\).

Nú látum við matlab um afganginn

\begin{Verbatim}[commandchars=\\\{\}]
\PYGZpc{}  Matlab forrit sem teiknar upp bestu margliðunálgun á gefnum gögnum
x=[0; 1; 2; 3; 4; 5; 6]
y=[2.7; \PYGZhy{}0.5; \PYGZhy{}1.7; \PYGZhy{}1.9; \PYGZhy{}1.5; 0.2; 2.3 ]
m=length(x);

\PYGZpc{} Við leitum að bestu margliðu af stigi 2 eða lægri
\PYGZpc{} og því eru  grunnföllin eru 3 talsins.
n=3;

\PYGZpc{} Stuðlafylkið er A=(a\PYGZus{}\PYGZob{}ij\PYGZcb{}), a\PYGZus{}\PYGZob{}ij\PYGZcb{}=x\PYGZus{}i\PYGZca{}\PYGZob{}j\PYGZhy{}1\PYGZcb{}
A(1:m,1)=ones(m,1);
A(1:m,2)=x;
for j=3:n
    A(1:m,j)=A(1:m,j\PYGZhy{}1).*x;
end
\PYGZpc{} Reiknum úr úr normaljöfnuhneppinu A\PYGZca{}TAc=A\PYGZca{}Ty:
c=(A\PYGZsq{}*A)\PYGZbs{}(A\PYGZsq{}*y);

\PYGZpc{} Teikning undirbúin
N=100;
X=linspace(min(x),max(x),N);

\PYGZpc{} Hliðrun í reikniriti horners er 0
\PYGZpc{}
hlidrun=zeros(n,1);
for j=1:N
    Y(j)=horner(c, hlidrun, X(j));
end
figure
plot(x,y,\PYGZsq{}*\PYGZsq{},X,Y)
xlabel(\PYGZsq{}x\PYGZsq{}), ylabel(\PYGZsq{}y\PYGZsq{})
title(\PYGZsq{}Adferd minnstu fervika fyrir marglidu af stigi 2\PYGZsq{})
print
\end{Verbatim}

Hér kemur myndin sem beðið var um:

\includegraphics{synidaemi_minnstu_fervik.png}


\chapter{Töluleg diffrun}
\label{kafli04::doc}\label{kafli04:toluleg-diffrun}
\emph{Nanny's philosophy of life was to do what seemed like a good idea at the time, and do it as hard as possible. It had never let her down.}
-- Terry Pratchett, Maskerade


\section{Inngangur}
\label{kafli04:inngangur}
\index{töluleg diffrun}\index{töluleg heildun}

\subsection{Töluleg diffrun og heildun}
\label{kafli04:index-0}\label{kafli04:toluleg-diffrun-og-heildun}
Deildun og heildun eru meginaðgerðir stærðfræðigreiningarinnar.

Þess vegna er nauðsynlegt að geta nálgað
\begin{gather}
\begin{split}f'(a),f''(a),f'''(a),\dots \quad
  \text{ og } \quad
  \int_a^b f(x)\, dx,\end{split}\notag
\end{gather}
þar sem \(f\) er fall sem skilgreint er á bili \(I\) sem
inniheldur \(a\) og \(b\).


\subsection{Meginhugmynd í öllum nálgunaraðferðunum}
\label{kafli04:meginhugmynd-i-ollum-nalgunaraferunum}
Látum \(p\) vera margliðu sem nálgar \(f\), og látum
\(r(x)=f(x)-p(x)\) tákna skekkjuna í nálgun á \(f(x)\) með
\(p(x)\). Þá er
\begin{gather}
\begin{split}f'(x)=p'(x)+r'(x), \quad f''(x)=p''(x)+r''(x), \dots\end{split}\notag
\end{gather}
og
\begin{gather}
\begin{split}\int_a^b f(x)\, dx=\int_a^b p(x)\, dx+\int_a^b r(x)\, dx.\end{split}\notag
\end{gather}
Nú þurfum við að gera tvennt:
\begin{enumerate}
\item {} 
Finna heppilegar nálgunarmargliður og reikna út
\begin{gather}
\begin{split}p'(a), \ p''(a),\dots, \qquad \int_a^b p(x)\, dx\end{split}\notag
\end{gather}
\item {} 
Meta skekkjurnar
\begin{gather}
\begin{split}r'(a), \ r''(a), \dots \int_a^b r(x)\, dx\end{split}\notag
\end{gather}
\end{enumerate}

Byrjum á að leiða út nokkrar nálgunarformúlur með skekkjumati.

\index{afleiður}\index{töluleg diffrun!frammismunur}

\section{Aðferðirnar}
\label{kafli04:aferirnar}\label{kafli04:index-1}
Látum \(f : I \to \mathbb R\) vera fall á bili
\(I \subset \mathbb R\) og \(a\) vera punkt í \(I\). Afleiða
\(f\) í punktinum \(a\) er skilgreind með
\begin{gather}
\begin{split}f'(a) = \lim\limits_{h \to 0}
  \frac{f(a+h)-f(a)}{h}\end{split}\notag
\end{gather}
ef markgildið er til. Við skrifum því oft
\begin{gather}
\begin{split}f'(a) \approx \frac{f(a+h)-f(a)}{h}\end{split}\notag
\end{gather}
Þessi nálgun er kölluð \emph{frammismunur} því oftast hugsar maður sér að
\(h > 0\) og þá er \(a+h\) lítið skref áfram frá \(a\).

Við þurfum skekkjumat fyrir þessa formúlu ef við eigum að geta notað
hana.


\subsection{Frammismunur}
\label{kafli04:frammismunur}
Við fáum mat á skekkjuna í nálguninni með að skoða Taylor-margliðu
\(f\) í \(a\). Samkvæmt setningu Taylors er til \(\xi\) á
milli \(a\) og \(a+h\) þannig að
\begin{gather}
\begin{split}f(a+h) = f(a) + f'(a)h + \frac{1}{2} f''(\xi)h^2.\end{split}\notag
\end{gather}
Þá fæst að skekkjan í nálgun á \(f'(a)\) með
\begin{gather}
\begin{split}\frac{f(a+h)-f(a)}{h} = f[a,a+h]\end{split}\notag
\end{gather}
er
\begin{gather}
\begin{split}e = f'(a) - \frac{f(a+h)-f(a)}{h} = -\frac{1}{2} f''(\xi) h\end{split}\notag
\end{gather}
Með öðrum orðum
\begin{gather}
\begin{split}\min_{t\in [0,h]} -\frac 12 f''(t)h \leq e \leq
\max_{t\in [0,h]} -\frac 12 f''(t)h.\end{split}\notag
\end{gather}
Við sjáum því að \(e=O(h)\) þegar \(h \to 0\).

\index{töluleg diffrun!bakmismunur}

\subsection{Bakmismunur}
\label{kafli04:bakmismunur}\label{kafli04:index-2}
Við getum sett \(a-h\) í stað \(a+h\) í skilgreininguna á
afleiðu. Þá fæst svokallaður \emph{bakmismunur}
\begin{gather}
\begin{split}f'(a) \approx \frac{f(a)-f(a-h)}{h}\end{split}\notag
\end{gather}
og ljóst er að sama skekkjumat gengur fyrir þessa nálgun og fyrir nálgun
með frammismun.

\index{töluleg diffrun!miðsettur mismunakvóti}

\subsection{Miðsettur mismunakvóti}
\label{kafli04:index-3}\label{kafli04:misettur-mismunakvoti}
Lítum nú á þriðja stigs Taylor nálgun
\begin{gather}
\begin{split}\begin{aligned}
  f(a+h)&=f(a)+f'(a)h+\tfrac 12 f''(a)h^2+\tfrac 16 f'''(\alpha)h^3,\\
  f(a-h)&=f(a)-f'(a)h+\tfrac 12 f''(a)h^2-\tfrac 16 f'''(\beta)h^3,\end{aligned}\end{split}\notag
\end{gather}
þar sem \(\alpha\) er á milli \(a\) og \(a+h\) og
\(\beta\) er á milli \(a\) og \(a-h\).

Tökum nú mismuninn og fáum
\begin{gather}
\begin{split}f(a+h)-f(a-h)=f'(a)\cdot 2h+\tfrac 16\big(f'''(\alpha)+f'''(\beta)\big)h^3\end{split}\notag
\end{gather}
Ef \(f'''\) er samfellt fall, þá gefur milligildissetningin okkur að
til er \(\xi\) á milli \(\alpha\) og \(\beta\) þannig að
\(f'''(\xi)=\tfrac 12 (f'''(\alpha)+f'''(\beta))\)

Niðurstaðan verður
\begin{gather}
\begin{split}f'(a)=\dfrac{f(a+h)-f(a-h)}{2h}-\tfrac 16f'''(\xi)h^2.\end{split}\notag
\end{gather}
Þannig að skekkjan er
\begin{gather}
\begin{split}e = -\frac 16 f'''(\xi) h^2,\end{split}\notag
\end{gather}
og jafnframt er \(e = O(h^2)\) þegar \(h\to 0\).


\begin{center}
\includegraphics[width=12cm,keepaspectratio=true]{./afleida.png}
\end{center}


\index{töluleg diffrun!miðsetttur mismunakvóti fyrir aðra afleiðu}

\subsection{Miðsettur mismunakvóti fyrir aðra afleiðu}
\label{kafli04:misettur-mismunakvoti-fyrir-ara-afleiu}\label{kafli04:index-4}
Við getum útfært þessa sömu hugmynd til þess að reikna út aðra afleiðu,
en þá byrjum við með fjórða stigs Taylor-nálgun
\begin{gather}
\begin{split}\begin{aligned}
  f(a+h)&=f(a)+f'(a)h+\tfrac 12 f''(a)h^2+\tfrac 16 f'''(a)h^3
+\tfrac 1{24}f^{(4)}(\alpha)h^4,\\
  f(a-h)&=f(a)-f'(a)h+\tfrac 12 f''(a)h^2-\tfrac 16 f'''(a)h^3
+\tfrac 1{24}f^{(4)}(\beta)h^4,\end{aligned}\end{split}\notag
\end{gather}
þar sem \(\alpha\) er á milli \(a\) og \(a+h\) og
\(\beta\) er á milli \(a\) og \(a-h\).

Nú leggjum við saman og fáum
\begin{gather}
\begin{split}f(a+h)+f(a-h)=2f(a) +f''(a)h^2+\tfrac
1{24}\big(f^{(4)}(\alpha)+f^{(4)}(\beta)\big)h^4.\end{split}\notag
\end{gather}
Nú þurfum við að gefa okkur að \(f^{(4)}\) sé samfellt fall, þá
gefur milligildissetningin okkur að til er \(\xi\) á milli
\(\alpha\) og \(\beta\) þannig að
\(f^{(4)}(\xi)=\tfrac 12 (f^{(4)}(\alpha)+f^{(4)}(\beta))\).

Niðurstaðan verður
\begin{gather}
\begin{split}f''(a)=\dfrac{f(a+h)+f(a-h)-2f(a)}{h^2}-\tfrac 1{12}f^{(4)}(\xi)h^2\end{split}\notag
\end{gather}
Með Taylor-margliðum má leiða út fleiri nálgunarformúlur fyrir afleiður.

Við ætlum ekki að halda lengra í þessa átt heldur snúa okkur að almennu
aðferðinni.


\section{Skekkjumat}
\label{kafli04:skekkjumat}

\subsection{Almennt um nálganir á afleiðum}
\label{kafli04:almennt-um-nalganir-a-afleium}
Ef \(x_0,\ldots, x_n\) eru punktar í \(I\) (hugsanlega með
endurtekningum) og \(p\) er margliðan sem brúar \(f\) í þeim, þá
er
\begin{gather}
\begin{split}f(x) = p(x) + r(x),\end{split}\notag
\end{gather}
þar sem skekkjuliðurinn \(r(x)\) er gefinn með formúlunni
\begin{gather}
\begin{split}r(x)=f[x_0,\ldots,x_n,x](x-x_0)\cdots(x-x_n)\end{split}\notag
\end{gather}
Ef við tökum \(p'(a)\) sem nálgun á \(f'(a)\) er skekkjan
\begin{gather}
\begin{split}r'(a) =  f'(a) - p'(a).\end{split}\notag
\end{gather}
\index{töluleg diffrun!skekkjumat}

\subsection{Skekkjumat}
\label{kafli04:index-5}\label{kafli04:id1}
Munið að formúlan fyrir afleiðu af margfeldi margra þátta er
\begin{gather}
\begin{split}\begin{gathered}
  (\varphi_1\varphi_2\varphi_3\cdots\varphi_m)'(a)\\
=\varphi_1'(a)\varphi_2(a)\varphi_3(a)\cdots\varphi_m(a)
+\varphi_1(a)\varphi_2'(a)\varphi_3(a)\cdots\varphi_m(a)
+\cdots\\
\cdots+\varphi_1(a)\varphi_2(a)\cdots \varphi_{m-1}(a)\varphi_m'(a)\\\end{gathered}\end{split}\notag
\end{gather}
Horfum nú á skekkjuliðinn \(r(x)\). Hann er svona margfeldi með
\(\varphi_1(x)=f[x_0,\dots,x_n,x]\), \(\varphi_2(x)=x-x_0\),
\(\varphi_3(x)=x-x_1\) o.s.frv.

Athugum nú að ef \(a\) er einn af gefnu punktunum \(x_k\), þá er
\(\varphi_{k+2}(x)=(x-x_k)\) sem gefur \(\varphi_{k+2}(x_k)=0\)
og \(\varphi_{k+2}'(x_k)=1\).

Þetta segir okkur að ef við tökum \(a=x_k\), þá eru allir liðirnir í
summunni í hægri hliðinni \(0\) nema einn, þ.e. við sitjum eftir með
þann sem inniheldur \({\varphi}_{k+2}'\).

Niðurstaðan verður því að skekkjan í nálgun á \(f'(a)\) með
\(p'(a)\) er
\begin{gather}
\begin{split}\begin{aligned}
  f'(a) - p'(a) &= r'(a)
=f[x_0,\dots,x_n,x_k]
\prod_{\stackrel{j=0}{j \not= k}} (x_k-x_j)\\
&=\dfrac{f^{(n+1)}(\xi)}{(n+1)!}
  \prod_{\stackrel{j=0}{j \not= k}} (a-x_j)\end{aligned}\end{split}\notag
\end{gather}
þar sem \(a=x_k\).

Hér notuðum við skekkjumatið fyrir Newton aðferðina sem
segir að til er \(\xi\) á minnsta bilinu sem inniheldur
\(x_0,\ldots,x_n,x_k\) sem uppfyllir
\begin{gather}
\begin{split}f[x_0,\ldots,x_n,x_k] = \frac{f^{(n+1)}(\xi)}{(n+1)!}.\end{split}\notag
\end{gather}

\subsection{Frammismunur}
\label{kafli04:id2}
Nálgum \(f\) með fyrsta stigs brúunarmargliðunni gegnum punktana
\((a,f(a))\) og \((a+h,f(a+h))\) (þ.e. \(x_0 = a\) og
\(x_1 = a+h\)),
\begin{gather}
\begin{split}f(x)=f[a]+f[a,a+h](x-a)+f[a,a+h,x](x-a)(x-a-h)\end{split}\notag
\end{gather}
Af þessu leiðir formúlan sem við vorum áður komin með
\begin{gather}
\begin{split}f'(a)=f[a,a+h]+f[a,a+h,a](a-a-h)
  =\dfrac{f(a+h)-f(a)}h-\tfrac 12 f''(\xi)h\end{split}\notag
\end{gather}
Þar sem \(\xi\) er á milli \(a\) og \(a+h\) og uppfyllir að
\(f[a,a+h,a]=f[a,a,a+h]=\tfrac 12f''(\xi)\). Hér erum við að
notafæra okkur aftur skekkjumatið sem við sönnuðum í kaflanum um
brúunarmargliður.


\subsection{Miðsettur mismunakvóti}
\label{kafli04:id3}
Tökum þriggja punkta brúunarformúlu með \(a-h\), \(a+h\) og
\(a\). Þá er
\begin{gather}
\begin{split}\begin{aligned}
  f(x)&=f[a-h]+f[a-h,a+h](x-a+h)\\
  &+f[a-h,a+h,a](x-a+h)(x-a-h)\\
  &+f[a-h,a+h,a,x](x-a+h)(x-a-h)(x-a)\end{aligned}\end{split}\notag
\end{gather}
Athugum að afleiðan af annars stigs þættinum
\begin{gather}
\begin{split}x\mapsto (x-a+h)(x-a-h)=(x-a)^2-h^2\end{split}\notag
\end{gather}
er \(0\) í punktinum \(a\) og því er
\begin{gather}
\begin{split}\begin{aligned}
  f'(a)&=f[a-h,a+h]+f[a-h,a+h,a,a](-h^2)\\
  &=\dfrac{f(a+h)-f(a-h)}{2h}-\tfrac 16 f'''(\xi)h^2 \end{aligned}\end{split}\notag
\end{gather}
Hér nýttum við okkur að til er \(\xi\) á milli \(a-h\) og
\(a+h\) þannig að \(f[a-h,a+h,a,a]=\tfrac 16 f'''(\xi)\).


\subsection{Miðsettur mismunakvóti fyrir aðra afleiðu}
\label{kafli04:id4}
Áfram heldur leikurinn. Nú skulum við leiða aftur út formúluna fyrir
nálgun á \(f''(a)\) með miðsettum mismunakvóta

Þá tökum við þriggja punkta brúunarformúlu með \(a-h\), \(a+h\)
og \(a\) með \(a\) tvöfaldan. Þá er
\begin{gather}
\begin{split}\begin{aligned}
  f(x)&=f[a-h]+f[a-h,a+h](x-a+h)\\
  &+f[a-h,a+h,a](x-a+h)(x-a-h)\\
  &+f[a-h,a+h,a,a](x-a+h)(x-a-h)(x-a)\\
  &+f[a-h,a+h,a,a,x](x-a+h)(x-a-h)(x-a)^2\end{aligned}\end{split}\notag
\end{gather}
Gætum þess að halda liðnum \((x-a)\). Þá fáum við
\begin{gather}
\begin{split}\begin{aligned}
  f(x)&=f[a-h]+f[a-h,a+h](x-a+h)\\
  &+f[a-h,a+h,a]\big((x-a)^2-h^2)\big)\\
  &+f[a-h,a+h,a,a]\big((x-a)^3-h^2(x-a))\big)\\
  &+f[a-h,a+h,a,a,x]\big((x-a)^4-h^2(x-a)^2)\big)\end{aligned}\end{split}\notag
\end{gather}
Nú þurfum við að reikna aðra afleiðu í punktinum \(a\). Athugum að
önnur afleiða af annars stigs þættinum
\begin{gather}
\begin{split}x\mapsto (x-a+h)(x-a-h)=(x-a)^2-h^2\end{split}\notag
\end{gather}
er fastafallið \(2\), önnur afleiða af þriðja stigs liðnum
\begin{gather}
\begin{split}x\mapsto (x-a)^3-h^2(x-a)\end{split}\notag
\end{gather}
er \(0\) í punktinum \(a\) og önnur afleiða af fjórða stigs
liðnum
\begin{gather}
\begin{split}x\mapsto (x-a)^4-h^2(x-a)^2\end{split}\notag
\end{gather}
er fastafallið \(-2h^2\).

Við höfum því
\begin{gather}
\begin{split}f''(a)=2f[a-h,a+h,a]+f[a-h,a+h,a,a,a](-2h^2)\end{split}\notag
\end{gather}
Nú er til punktur \(\xi\) á minnsta bili sem inniheldur \(a-h\),
\(a+h\) og \(a\) þannig að \(f[a-h,a+h,a,a,a]=\tfrac
1{24}f^{(4)}(\xi)\).

Við þurfum að reikna út fyrri mismunakvótann
\begin{gather}
\begin{split}\begin{aligned}
  f[a-h,a+h,a]&=f[a-h,a,a+h]=\dfrac{f[a,a+h]-f[a-h,a]}{2h}\\
  &=\dfrac 1{2h}\bigg(\dfrac{f(a+h)-f(a)}h-\dfrac{f(a)-f(a-h)}h\bigg)\\
  &=\dfrac{f(a+h)+f(a-h)-2f(a)}{2h^2}  \end{aligned}\end{split}\notag
\end{gather}
Við höfum því leitt aftur út formúluna
\begin{gather}
\begin{split}f''(a)=\dfrac{f(a+h)+f(a-h)-2f(a)}{h^2}-\tfrac
  1{12}f^{(4)}(\xi)h^2\end{split}\notag
\end{gather}
\index{töluleg diffrun!Richardson útgiskun}

\section{Richardson útgiskun}
\label{kafli04:richardson-utgiskun}\label{kafli04:index-6}
Það ætti að vera ljóst að töluleg deildun er nokkuð óstöðug aðferð því
ef skrefastærðin \(h\) er lítil eru tölurnar
\(f(a+h), f(a), f(a-h)\) nálægt hver annarri og við getum lent í
styttingarskekkjum.

Því er ekki hægt að búast við að fá alltaf betri nálgun á \(f'(a)\)
við að minnka skrefalengdina \(h\).

Leiðin er Richardson útgiskun (e. extrapolation), sem er aðferð til að
bæta nálganir.

Til eru mjög almennar útgáfur þessarar aðferðar en við munum aðeins
skoða þau sértilfelli sem nýtast okkur mest.


\subsection{Útleiðsla á miðsettum mismunakvóta}
\label{kafli04:utleisla-a-misettum-mismunakvota}
Við skulum byrja á að að leiða aftur út formúluna fyrir miðsettann
mismunakvóta til að fá betri upplýsingar um skekkjuliðinn. Fyrir fall
\(f\) sem er nógu oft deildanlegt má beita Taylor til að skrifa
\begin{gather}
\begin{split}\begin{aligned}
  f(a+h) &= f(a) + f'(a)h   + \ldots
  + \frac{f^{(2n)}(a)}{(2n!)}h^{2n}
  + \frac{f^{(2n+1)}(a)}{(2n+1)!)}h^{2n+1} + O(h^{2n+2}) \\
  f(a-h) &= f(a) - f'(a)h
    + \ldots
  + \frac{f^{(2n)}(a)}{(2n!)}h^{2n}
  - \frac{f^{(2n+1)}(a)}{(2n+1)!)}h^{2n+1} + O(h^{2n+2})\end{aligned}\end{split}\notag
\end{gather}
Ef við drögum seinni jöfnuna frá þeirri fyrri fæst
\begin{gather}
\begin{split}f(a+h)-f(a-h) = 2f'(a)h + 2\frac{f'''(a)}{3!}h^3
  + \ldots + 2\frac{f^{(2n+1)}(a)}{(2n+1)!}h^{2n+1} + O(h^{2n+2})\end{split}\notag
\end{gather}
svo ef við einangrum \(f'(a)\) sjáum við að
\begin{gather}
\begin{split}f'(a) = R_1(h)
  + a_2 h^2 + a_4 h^4 + \ldots + a_{2n} h^{2n} + O(h^{2n+1})\end{split}\notag
\end{gather}
þar sem
\begin{gather}
\begin{split}R_1(h) = \frac{f(a+h)-f(a-h)}{2h}
  \quad \text{og} \quad
  a_k = -\frac{f^{(k+1)}(a)}{(k+1)!},
  \quad k = 2,4,\ldots,2n.\end{split}\notag
\end{gather}

\subsection{Helmingun á skrefinu}
\label{kafli04:helmingun-a-skrefinu}
Hér er minnsta veldi í skekkjuliðnum \(h^2\), svo nálgunin
\(f'(a)
\approx R_1(h)\) er \(O(h^2)\), eins og við höfum reyndar séð áður.
Helmingum nú skrefalengdina \(h\), þá fæst
\begin{gather}
\begin{split}f'(a) = R_1(h/2) + a_2 \left(\frac{h}{2}\right)^2
  + a_4 \left(\frac{h}{2}\right)^4 + \ldots
  + a_{2n} \left(\frac{h}{2}\right)^{2n} + O(h^{2n+1}).\end{split}\notag
\end{gather}
Nú berum við saman þessi tvö skref:
\begin{gather}
\begin{split}\begin{aligned}
  f'(a) &= R_1(h/2) + \tfrac 14 a_2 h^2
  + a_4 \left(\frac{h}{2}\right)^4 + \ldots
  + a_{2n} \left(\frac{h}{2}\right)^{2n} + O(h^{2n+1}),\\
  f'(a) &= R_1(h)
  + a_2 h^2 + a_4 h^4 + \ldots + a_{2n} h^{2n} + O(h^{2n+1})\\\end{aligned}\end{split}\notag
\end{gather}
Margföldum efri jöfnuna með \(4\) og drögum þá síðari frá. Þá
stendur eftir
\begin{gather}
\begin{split}\begin{aligned}
  3f'(a) &= 4 R_1(h/2) - R_1(h)
  + a_4 \left( \frac{4}{2^4} - 1 \right)h^4 \\
  &+ a_6 \left( \frac{4}{2^6} - 1 \right)h^6
  + \ldots
  + a_{2n} \left( \frac{4}{2^{2n}} - 1 \right)h^{2n}
  + O(h^{2n+1})\end{aligned}\end{split}\notag
\end{gather}

\subsection{Fjórða stigs nálgun}
\label{kafli04:fjora-stigs-nalgun}
Nú erum við komin með nýja formúlu:
\begin{gather}
\begin{split}f'(a) = R_2(h) + b_4 h^4 + b_6 h^6 + \ldots + b_{2n} h^{2n}
  + O(h^{2n+1})\end{split}\notag
\end{gather}
þar sem
\begin{gather}
\begin{split}R_2(h) = \frac{4 R_1(h/2) - R_1(h)}{3}
  \quad \text{og} \quad
  b_k = \frac{a_k}{3} \cdot \left(\frac{4}{2^k}-1\right),
  \  k = 4,6,\ldots,2n.\end{split}\notag
\end{gather}
Ef við berum þetta saman við jöfnuna sem við byrjuðum með
\begin{gather}
\begin{split}f'(a) = R_1(h)
  + a_2 h^2 + a_4 h^4 + \ldots + a_{2n} h^{2n} + O(h^{2n+1})\end{split}\notag
\end{gather}
þá sjáum við að minnsta veldi í skekkjuliðnum er \(h^4\), svo
nálgunin \(f'(a)
\approx R_2(h)\) uppfyllir
\begin{gather}
\begin{split}f'(a) - R_2(h) = O(h^4)\end{split}\notag
\end{gather}
og er því betri nálgun en áður.

Þetta ferli heitir \emph{Richardson útgiskun}.


\subsection{Hægt er að halda áfram útgiskun}
\label{kafli04:haegt-er-a-halda-afram-utgiskun}
Næsta takmark er að eyða liðnum \(b_4h^4\) úr þessari formúlu með
því að líta á
\begin{gather}
\begin{split}f'(a) = R_2(h/2) + b_4 \left(\frac{h}{2}\right)^4
  + b_6 \left(\frac{h}{2}\right)^6 + \ldots
  + b_{2n} \left(\frac{h}{2}\right)^{2n} + O(h^{2n+1})\end{split}\notag
\end{gather}
Síðan stillum við þessari jöfnu upp með þeirri síðari
\begin{gather}
\begin{split}\begin{aligned}
  f'(a) &= R_2(h/2) + \tfrac 1{16}b_4 h^4
  + \tfrac 1{64}b_6 h^6 + \ldots
  + \tfrac 1{2^{2n}}b_{2n} h^{2n} + O(h^{2n+1})\\
  f'(a) &= R_2(h) + b_4 h^4 + b_6 h^6 + \ldots + b_{2n} h^{2n}
  + O(h^{2n+1})\end{aligned}\end{split}\notag
\end{gather}
Margföldum fyrri jöfnuna með \(16\) og drögum þá síðari frá
\begin{gather}
\begin{split}\begin{aligned}
  15f'(a) &= 16 R_2(h/2) - R_2(h)
  + b_6 \left( \frac{16}{2^6} - 1 \right) h^6 \\
  &+ b_8 \left( \frac{16}{2^8} - 1 \right) h^8
  + \ldots
  + b_{2n} \left( \frac{16}{2^{2n}} - 1 \right) h^{2n}
  + O(h^{2n+1}).\end{aligned}\end{split}\notag
\end{gather}

\subsection{Sjötta stigs skekkja}
\label{kafli04:sjotta-stigs-skekkja}\begin{gather}
\begin{split}\begin{aligned}
  15f'(a) &= 16 R_2(h/2) - R_2(h)
  + b_6 \left( \frac{16}{2^6} - 1 \right) h^6 \\
  &+ b_8 \left( \frac{16}{2^8} - 1 \right) h^8
  + \ldots
  + b_{2n} \left( \frac{16}{2^{2n}} - 1 \right) h^{2n}
  + O(h^{2n+1}).\end{aligned}\end{split}\notag
\end{gather}
Því er
\begin{gather}
\begin{split}f'(a) = R_3(h) + c_6 h^6 + c_8 h^8 \ldots + c_{2n} h^{2n}
  + O(h^{2n+1})\end{split}\notag
\end{gather}
þar sem
\begin{gather}
\begin{split}R_3(h) = \frac{16 R_2(h/2) - R_2(h)}{15},
  \quad \text{og} \quad
  c_k = \frac{b_k}{15} \cdot \left( \frac{16}{2^k} - 1 \right),
  \quad k = 6,8,\ldots,2n.\end{split}\notag
\end{gather}
Nýja nálgunin uppfyllir
\begin{gather}
\begin{split}f'(a) - R_3(h) = O(h^6)\end{split}\notag
\end{gather}
og er því enn betri en áður, en við þurfum líka að reikna út
\(R_1(h/4)\) til að reikna \(R_2(h/2)\).


\subsection{Almenn rakningarformúla}
\label{kafli04:almenn-rakningarformula}
Richardson-útgiskunin heldur áfram og út kemur
\begin{gather}
\begin{split}R_{i+1}(h) = \frac{4^i R_i(h/2) - R_i(h)}{4^i-1}
  = R_i(h/2) + \frac{R_i(h/2)-R_i(h)}{4^i-1}\end{split}\notag
\end{gather}
fyrir \((i+1)\)-tu Richardson útgiskun og \(R_{i+1}(h)\)
uppfyllir að
\begin{gather}
\begin{split}f'(a) - R_{i+1}(h) = O(h^{2i+2}),\end{split}\notag
\end{gather}
en á móti kemur að til að reikna út \(R_{i+1}(h)\) þurfum við að
hafa reiknað út tölurnar

\begin{DUlineblock}{0em}
\item[] \(R_1(h)\), \(R_1(h/2)\), \(\ldots\), \(R_1(h/2^i)\)
auk
\item[] \(R_2(h)\), \(R_2(h/2)\), …, \(R_2(h/2^{i-1})\) og svo
framvegis að
\item[] \(\qquad \vdots\)
\item[] \(R_i(h)\) og \(R_i(h/2)\).
\end{DUlineblock}

Eins og áður sagði fara styttingarskekkjur á endanum að segja til sín í
útreikningum á \(R_1(h)\), svo einhver takmörk eru fyrir hversu
margar Richardson útgiskanir er hægt að framkvæma.


\subsection{Reiknirit}
\label{kafli04:reiknirit}
Útreikningarnir að ofan eru yfirleitt settir fram í töflu
\begin{gather}
\begin{split}\begin{array}{ccccc}
    D(1,1) &   &   &   &   \\
    D(2,1) & D(2,2) &  &  &  \\
    D(3,1) & D(3,2) & D(3,3) & & \\
    \vdots & \vdots & \vdots & \ddots & \\
    D(n,1) & D(n,2) & D(n,3) & \ldots & D(n,n)
  \end{array}\end{split}\notag
\end{gather}
þar sem \(D(i,j) = R_j(h/2^{i-j})\) og þar með
\begin{gather}
\begin{split}D(i,j) = \begin{cases}
    \dfrac{f(a+h/2^{i-1})-f(a-h/2^{i-1})}{2\cdot h/2^{i-1}}, & j = 1 \\
    D(i,j-1) + \dfrac{D(i,j-1)-D(i-1,j-1)}{4^{j-1}-1}, & j > 1
  \end{cases}\end{split}\notag
\end{gather}
sem gerir okkur auðvelt að forrita Richardson útgiskun.


\subsection{Skekkjumat}
\label{kafli04:id5}
Finnum nú eftirámat fyrir \(D(i,j)\) með stærðunum
\(D(i,j-1)\) og \(D(i-1,j-1)\). Hér á eftir er
\(R_j(h/2)\) í hlutverki \(D(i,j-1)\) og \(R_i(h)\) í
hlutverki \(D(i-1,j-1)\)
(\(h\) er helmingað þegar við förum niður um eina línu).

Munum að \(R_i(h)\) uppfyllir að
\begin{gather}
\begin{split}f'(a) = R_j(h) + Kh^{2j} + O(h^{2j+1})\end{split}\notag
\end{gather}
fyrir eitthvert \(K\) í \(\mathbb R\) og að
\begin{gather}
\begin{split}f'(a) = R_j(h/2) + K \left( \frac{h}{2} \right)^{2j}
  + O(h^{2j+1})\end{split}\notag
\end{gather}
Ef við tökum mismun á hægri og vinstri hliðum þessara jafna, þá fáum við
\begin{gather}
\begin{split}0 = R_j(h) - R_j(h/2) + K \left(1 - \frac{1}{2^{2j}}\right)h^{2j}
  + O(h^{2j+1})\end{split}\notag
\end{gather}
og ef við einangrum \(K\) fæst
\begin{gather}
\begin{split}K = -\frac{4^{j}}{h^{2j}} \cdot \frac{R_j(h)-R_j(h/2)}{4^{j}-1} +
O(h^{2j+1}).\end{split}\notag
\end{gather}

\subsection{Útleiðsla á fyrirframmati}
\label{kafli04:utleisla-a-fyrirframmati}
Þá er skekkjan í nálgun á \(f'(a)\) með \(R_j(h/2)\) jöfn
\begin{gather}
\begin{split}\begin{aligned}
  e_j(h/2) &= f'(a) - R_j(h/2) \\
  &= K\left(\frac{h}{2}\right)^{2j} + O(h^{2j+1}) \\
  &= -\frac{R_j(h)-R_j(h/2)}{4^{j}-1} + O(h^{2j+1}) \\
  &\approx -\frac{R_j(h)-R_j(h/2)}{4^{j}-1}.\end{aligned}\end{split}\notag
\end{gather}
Þar sem \(R_j(h/2)\) er nálgun á \(f'(a)\) af stigi
\(O(h^{2j+1})\), en \(R_{j+1}(h)\) er nálgun á \(f'(a)\) af
stigi \(O(h^{2i+3})\) getum við slegið á \(e_{j+1}(h)\) með
\(e_j(h/2)\). Ef við lækkum vísinn \(j+1\) um einn gefur það
okkur matið
\begin{gather}
\begin{split}e_j(h) \approx \frac{R_{j-1}(h)-R_{j-1}(h/2)}{4^{j-1}-1} =
  \frac{D(i,j-1)-D(i-1,j-1)}{4^{j-1}-1}\end{split}\notag
\end{gather}
sem er einmitt liðurinn í rakningarformúlunni fyrir \(D(i,j)\).


\subsection{Sýnidæmi}
\label{kafli04:synidaemi}
Látum \(f(x)=x/(x^2+4)^{2/3}\) og \(a=-1\). Byrjum með
\(h=1\) og notum svo rakningarformúluna til þess að fylla út
útgiskunartöfluna.

\begin{tabulary}{\linewidth}{|L|L|L|L|L|}
\hline
\textsf{\relax 
\(h\)
} & \textsf{\relax 
\(D(i,1)\)
} & \textsf{\relax 
\(D(i,2)\)
} & \textsf{\relax 
\(D(i,3)\)
} & \textsf{\relax 
\(D(i,4)\)
}\\
\hline
1 .
 & 
0.25000000
 &  &  & \\
\hline
0.5
 & 
0.25151838
 & 
0.25202451
 &  & \\
\hline
0.25
 & 
0.25104655
 & 
0.25088928
 & 
0.25081360
 & \\
\hline
0.125
 & 
0.25086355
 & 
0.25080254
 & 
0.25079676
 & 
0.25079649
\\
\hline\end{tabulary}


Niðustaðan er: \(f'(-1)\approx   0.2507964\), með eftirámat á
skekkju \(-3\cdot 10^{-7}\).

Rétt gildi er \(0.25079647217924889177\).
\phantomsection\label{kafli05:heildun}
\index{töluleg heildun}

\chapter{Töluleg heildun}
\label{kafli05:toluleg-heildun}\label{kafli05::doc}\label{kafli05:index-0}
\emph{So much universe, and so little time.}
-- Terry Pratchett

Gerum ráð fyrir að \(x_0,x_1, \ldots, x_n\) séu punktar á bilinu
\([a,b]\) og að við þekkjum gildi \(f\) í þessum punktum. Þá
getum við fundið brúunarmargliðuna \(p_n\) gegnum punktana
\((x_k,f(x_k))\) og skrifað
\begin{gather}
\begin{split}f(x) = p_n(x) + r_n(x),\end{split}\notag
\end{gather}
þar sem skekkjan \(r_n\) er gefin með
\begin{gather}
\begin{split}r_n(x) = f[x_0,\ldots,x_n,x](x-x_0)\cdots(x-x_n).\end{split}\notag
\end{gather}
Nú er auðvelt að reikna heildi margliða, svo við nálgum heildi
\(f\) með
\begin{gather}
\begin{split}\int\limits_a^b f(x) dx \approx
  I_n(f) := \int\limits_a^b p_n(x) dx\end{split}\notag
\end{gather}
og skekkjan í þessari nálgun er gefin með
\begin{gather}
\begin{split}e_n = \int\limits_a^b r_n(x) dx.\end{split}\notag
\end{gather}
Þessi aðferð er kölluð \emph{Newton-Cotes-heildun}.

\index{töluleg heildun!Newton-Cotes}

\section{Aðferðirnar}
\label{kafli05:aferirnar}\label{kafli05:index-1}

\subsection{Newton-Cotes heildun}
\label{kafli05:newton-cotes-heildun}
Hugsum okkur að brúunarpunktarnir \(x_0, \ldots, x_n\) séu ólíkir.
Þá getum við skrifað \(p_n\) með Lagrange-margliðum
\begin{gather}
\begin{split}p_n(x) = \sum\limits_{k=0}^n f(x_k) \ell_k(x),
  \quad
  \ell_k(x) = \prod\limits_{\stackrel{j=0}{j \not= k}}^n
  \frac{(x-x_j)}{(x_k-x_j)},\end{split}\notag
\end{gather}
og þá er heildi \(p_n\) jafnt
\begin{gather}
\begin{split}\int\limits_a^b p_n(x) dx =
  \sum\limits_{k=0}^n f(x_k) A_k,
  \quad \text{þar sem} \quad
  A_k = \int\limits_a^b \ell_k(x) dx.\end{split}\notag
\end{gather}
Athugið að gildi \(A_k\) veltur aðeins á brúunarpunktunum
\(x_0, \ldots,
x_n\) en ekki gildum \(f(x_k)\). Ef það á að heilda mörg föll yfir
sama bil er því hægt að reikna gildi \(A_k\) í eitt skipti fyrir öll
og endurnýta þau svo.


\subsection{Sýnidæmi}
\label{kafli05:synidaemi}
Metum heildi \(f(x) = e^{-x}\cos(x)\) og
\(g(x) = \sin (\frac{x^2}{2})\) yfir bilið \([0,2]\) með að nota
skiptipunktana \(x_0 = 0\), \(x_1 = 1\) og \(x_2 = 2\).
Lagrange-margliðurnar sem við eiga eru
\begin{gather}
\begin{split}\ell_0(x) = \frac{(x-1)(x-2)}{2}, \quad
  \ell_1(x) = -x(x-2), \quad
  \ell_2(x) = \frac{x(x-1)}{2}\end{split}\notag
\end{gather}
svo við fáum að
\begin{gather}
\begin{split}\begin{gathered}
  A_0 = \frac{1}{2} \int\limits_0^2 (x-1)(x-2) dx = \frac{1}{3},
  \qquad
  A_1 = -\int\limits_0^2 x(x-2) dx = \frac{4}{3}, \\
  A_2 = \frac{1}{2} \int\limits_0^2 x(x-1) dx = \frac{1}{3}.\end{gathered}\end{split}\notag
\end{gather}
Nú eru stuðlarnir fundnir og því fáum við
\begin{gather}
\begin{split}\begin{aligned}
  \int\limits_0^2 f(x) dx &\approx
  f(0)\frac{1}{3} + f(1)\frac{4}{3} + f(2)\frac{1}{3}\\
  &= \frac{1 + 4e^{-1}\cos(1) + e^{-2}\cos(2)}{3}
  \approx 0.59581\end{aligned}\end{split}\notag
\end{gather}
og
\begin{gather}
\begin{split}\begin{aligned}
  \int\limits_0^2 g(x) dx &\approx
  g(0)\frac{1}{3} + g(1)\frac{4}{3} + g(2)\frac{1}{3}\\
 & = \frac{4\sin(1/2) + \sin(2)}{3}
  \approx 0.91972.\end{aligned}\end{split}\notag
\end{gather}
Gildi heildanna eru \(\int\limits_0^2 f(x) dx \approx 0.58969\) og
\(\int\limits_0^2 g(x) dx \approx 0.99762\) með 5 réttum aukastöfum
svo nálgunargildin verða að teljast nokkuð góð miðað við hversu lítið
fór í þau.

\index{töluleg heildun!trapisureglan}

\subsection{Trapisureglan}
\label{kafli05:trapisureglan}\label{kafli05:index-2}
Nú ætlum við að leiða út formúlur fyrir helstu reglum fyrir nálgun á
heildum. Sú fyrsta er \emph{trapisuregla}.

Veljum \(x_0 = a\) og \(x_1 = b\) sem skiptipunktana okkar. Þá
er graf \(p_1\) línustrikið gegnum \((a,f(a))\) og
\((b,f(b))\),
\begin{gather}
\begin{split}p_1(x) = f(a) \ell_0(x) + f(b) \ell_1(x)
  = f(a)\frac{b-x}{b-a} + f(b) \frac{x-a}{b-a}\end{split}\notag
\end{gather}
og vigtirnar eru
\begin{gather}
\begin{split}A_0 = \int\limits_a^b \ell_0(x) = \frac{b-a}{2} = A_1,\end{split}\notag
\end{gather}
svo
\begin{gather}
\begin{split}\int\limits_a^b f(x) dx \approx
  \frac{b-a}{2}\left(f(a)+f(b)\right).\end{split}\notag
\end{gather}
Trapisureglan er kölluð þessu nafni því með henni nálgum við heildi
\(f\) með flatarmáli trapisunnar sem hefur hornpunktana
\((a,0)\), \((b,0)\), \((b,f(b))\) og \((a,f(a))\).

\index{töluleg heildun!miðpunktsreglan}

\subsection{Miðpunktsreglan}
\label{kafli05:index-3}\label{kafli05:mipunktsreglan}
Enn einfaldari er \emph{miðpunktsreglan}, þá veljum við aðeins einn
skiptipunkt, \(x_0 = \frac{1}{2}(a+b)\), og brúunarmargliðan verður
fastamargliðan \(p_0(x) = f(x_0)\). Þá er
\begin{gather}
\begin{split}\int\limits_a^b f(x) dx \approx (b-a)\, f\left(\frac{a+b}{2}\right)\end{split}\notag
\end{gather}
\index{töluleg heildun!regla Simpsons}

\subsection{Regla Simpsons}
\label{kafli05:index-4}\label{kafli05:regla-simpsons}
Nú veljum við þrjá skiptipunkta, \(x_0 = a\), \(x_1 = b\) og
\(x_2 =
\frac{1}{2}(a+b)\). Til einföldunar skulum við hliðra fallinu \(f\)
um miðpunkt bilsins \(m=\tfrac{1}{2}(a+b)\).

Við skilgreinum \(\alpha=\tfrac 12(b-a)\) og
\(g(x) = f\big(x+m\big)\)

Þá hliðrast \(a\), \(m\) og \(b\) yfir í \(-\alpha\),
\(0\) og \(\alpha\) og
\begin{gather}
\begin{split}\int\limits_{-\alpha}^{\alpha} g(x) dx =
  \int\limits_a^b f(x) dx.\end{split}\notag
\end{gather}
Lagrange margliðurnar og vigtirnar eru
\begin{gather}
\begin{split}\begin{aligned}
  l_0(x) &= \frac{(x-\alpha)x}{(-\alpha-\alpha)(-\alpha - 0)}
  = \frac{(x-\alpha)x}{2\alpha^2} \\
  A_0 &= \int_{-\alpha}^{\alpha} l_0(x)\,dx = \frac{\alpha}{3} \\
  l_1(x) &= \frac{(x-(-\alpha))(x-0)}{(\alpha - ( -\alpha))(\alpha - 0)}
  = \frac{(x+\alpha)x}{2\alpha^2}\\
  A_1 &= \int_{-\alpha}^{\alpha} l_1(x)\,dx = \frac{\alpha}{3}\\
  l_2(x) &= \frac{(x-(\alpha))(x-\alpha)}{0-(-\alpha)(0-\alpha)}
  = \frac{(x+\alpha)(x-\alpha)}{-\alpha^2}\\
  A_2 &= \int_{\alpha}^{\alpha} l_2(x)\,dx = \frac{4\alpha}{3}\end{aligned}\end{split}\notag
\end{gather}
Nálgunarformúlan verður þá
\begin{gather}
\begin{split}\begin{aligned}
  \int_a^b f(x) \, dx = \int\limits_{-\alpha}^{\alpha} g(x) \, dx
  &\approx \frac{\alpha}{3}g(-\alpha) + \frac{\alpha}{3}g(\alpha)
  + \frac{4\alpha}{3}g(0)\\
  &=(b-a)\left( \frac{1}{6}f(a) + \frac{4}{6}f
    \left( \frac{a+b}{2}\right) + \frac{1}{6} f(b)  \right)\end{aligned}\end{split}\notag
\end{gather}
Ef við tökum brúunarmargliðu gegnum \(a\), \(b\) og
\(\frac{1}{2}(a+b)\) með \(\frac{1}{2}(a+b)\) tvöfaldan þá fáum
við 3. stigs brúunarmargliðu
\begin{gather}
\begin{split}p_3(x) = p_2(x) + g[-\alpha, \alpha, 0, 0](x+\alpha)(x-\alpha)x\end{split}\notag
\end{gather}
Heildið yfir seinni liðinn hægra megin er 0 því margliðan
\((x+a)(x-a)x\) er oddstæð, en heildið yfir fyrri liðinn er
\begin{gather}
\begin{split}\frac \alpha3(g(-\alpha) + 4g(0) + g(\alpha)).\end{split}\notag
\end{gather}
Út kemur því Simpson-regla.


\begin{center}
\includegraphics[width=12cm,keepaspectratio=true]{./heildun.png}
\end{center}


\index{töluleg heildun!samsett}

\section{Samsettar útgáfur}
\label{kafli05:index-5}\label{kafli05:samsettar-utgafur}
\emph{Sometimes the truth is arrived at by adding all the little lies together and
deducting them from the totality of what is known.}
-- Terry Pratchett, Going Postal


\subsection{Inngangur}
\label{kafli05:inngangur}
Þar sem Newton-Cotes heildun notar brúunarmargliður fylgja henni nokkur
vandamál.

Ef okkur finnst nákvæmnin í nálguninni vera of lítil getum við ekki
búist við að hún batni við að fjölga skiptipunktum; þá hækkar stig
margliðunnar líklega sem orsakar sveiflukenndari hegðun.

Eins er ekki gott að halda sig við margliður af lægra stigi; ef bilið
sem á að heilda yfir er stórt væri mikil tilviljun að 1., 2. eða 3.
stigs brúunarmargliða nálgaði fallið vel á öllu bilinu.

Lausnin á þessu vandamáli er í sama anda og fyrir splæsibrúun. Við
veljum skiptingu
\begin{gather}
\begin{split}a  =x_0 < x_1 < \ldots < x_n = b\end{split}\notag
\end{gather}
á bilinu \([a,b]\).

Um heildi gildir að
\begin{gather}
\begin{split}\int\limits_a^bf(x)\, dx = \sum\limits_{k=1}^n \ \ \int\limits_{x_{k-1}}^{x_k} f(x) \, dx\end{split}\notag
\end{gather}
svo við getum nálgað heildi \(f\) á sérhverju litlu hlutbili
\([x_{k-1},x_k]\) með að heilda brúunarmargliðu af lágu stigi og
lagt öll gildin saman til að fá nálgun á heildi \(f\) yfir allt
bilið.

Þegar ákveðin regla er notuð til að nálga heildi \(f\) á sérhverju
hlutbili er þetta kölluð \emph{samsetta} útgáfa reglunnar. Einfalt er að
leiða út samsettar útgáfur reglanna að ofan.

\index{töluleg heildun!samsetta trapisureglan}

\subsection{Samsetta trapisureglan}
\label{kafli05:index-6}\label{kafli05:samsetta-trapisureglan}
Á sérhverju hlutbili er
\begin{gather}
\begin{split}\int\limits_{x_{k-1}}^{x_k} f(x) \, dx
  \approx
  \frac{x_k-x_{k-1}}{2}(f(x_{k-1}) + f(x_k))\end{split}\notag
\end{gather}
svo
\begin{gather}
\begin{split}\int\limits_a^b f(x) \, dx
  \approx
  \sum\limits_{k=1}^n \frac{x_k-x_{k-1}}{2}(f(x_{k-1}) + f(x_k)).\end{split}\notag
\end{gather}
Ef öll hlutbilin eru jafn löng og \(h = x_k-x_{k-1}\), þá fæst
\begin{gather}
\begin{split}\begin{gathered}
  \int\limits_a^b f(x) \, dx \\
  \approx
  h\left( \frac{1}{2}f(a) + f(a+h) + f(a+2h)
    + \cdots + f(a+(n-1)h) + \frac{1}{2}f(b) \right).\end{gathered}\end{split}\notag
\end{gather}

\begin{center}
\includegraphics[width=12cm,keepaspectratio=true]{./samsett_trapisuregla.png}
\end{center}


\index{töluleg heildun!samsetta miðpunktsreglan}

\subsection{Samsetta miðpunktsreglan}
\label{kafli05:index-7}\label{kafli05:samsetta-mipunktsreglan}
Fljótséð er að
\begin{gather}
\begin{split}\int\limits_a^b f(x) \, dx
  \approx
  \sum\limits_{k=1}^n (x_k-x_{k-1})f
  \left(
    \frac{x_{k-1}+x_k}{2}
  \right)\end{split}\notag
\end{gather}
Ef öll hlutbilin eru jafn löng verður formúlan
\begin{gather}
\begin{split}\int\limits_a^b f(x) \, dx
  \approx
  h \sum\limits_{k=1}^n f \left(\frac{x_{k-1}+x_k}{2}\right)\end{split}\notag
\end{gather}
\index{töluleg heildun!samsett regla Simpsons}

\subsection{Samsett regla Simpsons}
\label{kafli05:samsett-regla-simpsons}\label{kafli05:index-8}
Hér er venjan að velja \(2n+1\) jafndreifða skiptipunkta og fá
\(n\) jafn stór hlutbil. Þá er \(h = \frac{b-a}{2n}\),
\(x_k = a + kh\) fyrir \(k =
0,\ldots,2n\) og hlutbilin eru \([x_{2k-2},x_{2k}]\) fyrir
\(k = 1,
\ldots, n\).

Á hverju hlutbili er
\begin{gather}
\begin{split}\int\limits_{x_{2k-2}}^{x_{2k}} f(x) \, dx
  \approx
  2h \left(
    \frac{1}{6} f(x_{2k-2}) + \frac{4}{6} f(x_{2k-1})
    + \frac{1}{6} f(x_{2k})
  \right)\end{split}\notag
\end{gather}
svo að
\begin{gather}
\begin{split}\begin{aligned}
  \int\limits_a^b f(x) \, dx
  \approx &
  \sum\limits_{k=1}^n
  \bigg(
    \frac{h}{3}
    \Big(
      f(x_{2k-2}) + 4f(x_{2k-1}) + f(x_{2k})
    \Big)
  \bigg) \\
  = &
  \frac{h}{3}
  \Big(
    f(a) + 4f(a+h) + 2f(a+2h)+ 4f(a+3h) + 2f(a+4h) \\
    &+ \cdots + 2f(a+(2n-2)h) + 4f(a+(2n-1)h) + f(b).
  \Big)\end{aligned}\end{split}\notag
\end{gather}
\index{töluleg heildun!skekkjumat}

\section{Skekkjumat}
\label{kafli05:skekkjumat}\label{kafli05:index-9}

\subsection{Inngangur}
\label{kafli05:id1}
Rifjum upp grunnhugmyndina að baki nálgunarformúlunum. Við veljum
brúunarpunkta \(x_0, \ldots, x_n\) í \([a,b]\), látum
\(p_n\) vera tilsvarandi brúunarmargliðu og skrifum
\begin{gather}
\begin{split}f(x) = p_n(x) + r_n(x)\end{split}\notag
\end{gather}
þar sem \(r_n(x) = f[x_0, \ldots , x_n, x](x-x_0) \cdots (x-x_n)\).
Þá er nálgunin
\begin{gather}
\begin{split}\int_a^b f(x)\,dx \approx \int_a^b p_n(x)\,dx\end{split}\notag
\end{gather}
með skekkjuna
\begin{gather}
\begin{split}\int_a^b r_n(x)\,dx\end{split}\notag
\end{gather}
Nú viljum við meta skekkjuheildið.

\index{Meðalgildissetningin fyrir heildi}

\subsection{Meðalgildissetningin fyrir heildi}
\label{kafli05:mealgildissetningin-fyrir-heildi}\label{kafli05:index-10}
Við skekkjumatið í þessum kafla munum við þurfa að nota eftirafarandi
setningu nokkrum sinnum.

Ef \(G:[a,b] \to {{\mathbb  R}}\) er samfellt fall og
\({\varphi}\) er heildanlegt fall sem skiptir ekki um formerki á
bilinu \([a,b]\) þá er til tala \(\eta \in [a,b]\) þannig að
\begin{gather}
\begin{split}\int_a^b G(x){\varphi}(x)\, dx = G(\eta) \int_a^b {\varphi}(x)\, dx.\end{split}\notag
\end{gather}

\subsection{Trapisureglan}
\label{kafli05:id2}
Við getum hliðrar sérhverju bili \([a,b]\) yfir í \([-\alpha,\alpha]\)
þar sem \(\alpha = (b-a)/2\), því er nóg fyrir okkur að skoða samhverf
bil af gerðinni \([-\alpha,\alpha]\). Þetta er það sama og við gerðum
þegar \href{https://notendur.hi.is/~bsm/stae405/kafli05.html\#regla-simpsons}{regla Simpsons}
var leidd út.

Samkvæmt \href{https://notendur.hi.is/~bsm/stae405/kafli03.html\#id9}{3.7.7} þá er
\begin{gather}
\begin{split}r_1(x) = f[-\alpha, \alpha, x](x+\alpha)(x-\alpha)\end{split}\notag
\end{gather}
Athugum að
\begin{gather}
\begin{split}(x+\alpha)(x-\alpha) = (x^2 - \alpha^2)\end{split}\notag
\end{gather}
skiptir ekki um formerki á bilinu \(]-\alpha, \alpha[\). Þá gefur
meðalgildissetningin fyrir heildi að til er \(\eta \in [a,b]\)
þannig að
\begin{gather}
\begin{split}\begin{aligned}
  \int_a^b r_1(x)\,dx
  &= f[-\alpha, \alpha, \eta]
  \int_{-\alpha}^{\alpha}(x^2 - \alpha^2)\,dx\\
  &= \frac{f''(\xi)}{2!} \left( - \frac{4}{3}\alpha^3 \right)\\
  &= \frac{-f''(\xi)}{2!}\frac{(b-a)^3}{6}, \qquad \xi \in [a,b]\end{aligned}\end{split}\notag
\end{gather}
Niðurstaða:
\begin{gather}
\begin{split}\int_a^b f(x)\,dx = (b-a)
  \left( \frac{1}{2} f(a) + \frac{1}{2}f(b) \right)
  - \frac{1}{12} f''(\xi)(b-a)^3\end{split}\notag
\end{gather}

\subsection{Samsetta trapisureglan}
\label{kafli05:id5}
Ef við lítum á samsettu trapisuregluna með jafna skiptingu þar sem
hlutbilin eru \([x_i,
x_{i+1}]\), þá fáum við fyrir hvert hlutbil skekkjuna
\begin{gather}
\begin{split}- \frac{h^3}{12}f''(\xi_i), \qquad \xi_i \in [x_i, x_{i+1}]\end{split}\notag
\end{gather}
Ef við leggjum skekkjurnar saman og beitum milligildissetningunni á
\(f''\) þá fæst að til er \(\xi \in [a,b]\) þannig að
\(f''(\xi) = \sum_{i=1}^n f''(\xi_i)/n\).
Þá fáum við a'
\begin{gather}
\begin{split}\int_a^b f(x)\,dx = T(h) - \frac{h^2}{12}(b-a)f''(\xi), \qquad
  \xi \in [a,b]\end{split}\notag
\end{gather}
að því gefnu að \(f\in C^2 [a,b]\).

Athugið að hér er \(T(h)\) útkoman úr samsettu Trapisureglunni með jafna
skiptingu \(h = \frac{b-a}n\).


\subsection{Miðpunktsregla}
\label{kafli05:mipunktsregla}
Til einföldunar skoðum við áfram bilið \([-\alpha,\alpha]\). Veljum
miðpunktinn sem tvöfaldan brúunarpunkt
\begin{gather}
\begin{split}\begin{aligned}
  &p_1(x) = f(0) + f'(0)x\\
  &r_1(x) = f[0,0,x]x^2\end{aligned}\end{split}\notag
\end{gather}
Athugum að heildið af \(f'(0)x\) yfir \([-\alpha,\alpha]\) er 0.
Nú skiptir \(x^2\) ekki um formerki og því gefur meðalgildisreglan
fyrir heildi að til er \(\eta \in [-\alpha,\alpha]\) þannig að
\begin{gather}
\begin{split}\begin{aligned}
  \int_a^b r_1(x)\,dx
  &= \int_{-\alpha}^{\alpha} f[0,0,x]x^2 \,dx\\
  &= f[0,0,\eta]\int_{-\alpha}^\alpha x^2\,dx\\
  &= \frac{f''(\xi)}{2!}2\frac{\alpha^3}{3}\\
  &= \frac{(b-a)^3}{24}\cdot f''(\xi)\end{aligned}\end{split}\notag
\end{gather}
Þar sem \(\xi\) fæst úr \href{https://notendur.hi.is/~bsm/stae405/kafli03.html\#id9}{skekkjumatinu fyrir brúunarmargliður}.


\subsection{Samsetta miðpunktsreglan}
\label{kafli05:id6}
Fyrir hvert bil fáum við skekkjulið:
\begin{gather}
\begin{split}\frac{h^3}{24}\cdot f''(\xi_i)\end{split}\notag
\end{gather}
Leggjum saman skekkjuliðina og beitum milligildissetningunni, þá fæst að
til er \(\xi\) þannig að:
\begin{gather}
\begin{split}\int_a^b f(x)\,dx = h \sum_{i=1}^n
  f\left(a+ (i - \frac{1}{2})h\right) + \frac{b-a}{24}f''(\xi)h^2\end{split}\notag
\end{gather}

\subsection{Regla Simpsons}
\label{kafli05:id7}\begin{gather}
\begin{split}\int_a^b f(x)\,dx \approx (b-a)
  \left(
    \frac{1}{6}f(a) + \frac{4}{6}f
    \left( \frac{1}{2}(a+b) \right) + \frac{1}{6}f(b)
  \right)\end{split}\notag
\end{gather}
Leiddum út þessa formúlu með því að taka brúunarmargliðu \(p_3(x)\)
með punktana \(-\alpha, \alpha, 0, 0\). Skekkjan er
\begin{gather}
\begin{split}f(x) - p_3(x) = f[-\alpha, \alpha, 0, 0, x]
  (x+\alpha)(x-\alpha)x^2\end{split}\notag
\end{gather}
þar með er skekkjan í formúlu Simpsons:
\begin{gather}
\begin{split}\int_{-\alpha}^{\alpha}f[-\alpha, \alpha, 0, 0, x]
  (x+\alpha)(x-\alpha)x^2 \,dx\end{split}\notag
\end{gather}
Fallið \(x\mapsto (x+\alpha)(x-\alpha)x^2 = (x^2 - \alpha^2)x^2\) er
\(\leq 0\) á \([-\alpha, \alpha]\). Þar með gefur
meðalgildissetningin fyrir heildi að til er
\(\eta \in [-\alpha, \alpha]\) þannig að skekkjan er
\begin{gather}
\begin{split}\begin{gathered}
  f[-\alpha, \alpha, 0, 0, \eta]
  \int_{-\alpha}^{\alpha}(x^2 - \alpha^2)x^2 \,dx \\
  = \frac{f^{(4)}(\xi)}{4!}\cdot \frac{(-4)}{15}\cdot \alpha^5
  = \frac{-f^{(4)}(\xi)}{90}\left(\frac{b-a}{2}\right)^5, \qquad
  \xi \in [a,b]\end{gathered}\end{split}\notag
\end{gather}
Þar sem \(\xi\) fæst úr
\href{https://notendur.hi.is/~bsm/stae405/kafli03.html\#id9}{skekkjumatinu fyrir brúunarmargliður}.


\subsection{Samsett regla Simpsons}
\label{kafli05:id9}
Skiptum \([a,b]\) í \(n\) jafnlöng bil og látum \(h\) vera
helming hlutbillengdarinnar,
\begin{gather}
\begin{split}h = \frac{(b-a)}{2n}.\end{split}\notag
\end{gather}
Þá er
\begin{gather}
\begin{split}\begin{aligned}
  \int\limits_a^b f(x) \, dx
  \approx &
  \sum\limits_{k=1}^n
  \bigg(
    \frac{h}{3}
    \Big(
      f(x_{2k-2}) + 4f(x_{2k-1}) + f(x_{2k})
    \Big)
  \bigg) \\
  = &
  \frac{h}{3}
  \Big(
    f(a) + 4f(a+h) + 2f(a+2h)+ 4f(a+3h) + 2f(a+4h) \\
    &+ \cdots + 2f(a+(2n-2)h) + 4f(a+(2n-1)h) + f(b)
  \Big)\end{aligned}\end{split}\notag
\end{gather}
Ef við beitum skekkjumatinu á sérhvert bilanna þá fáum við
\begin{gather}
\begin{split}\frac{-f^{(4)}(\xi_i)}{90}h^5\end{split}\notag
\end{gather}
sem skekkju með \(\xi_i \in [x_i, x_i+1]\). Heildarskekkjan verður
\begin{gather}
\begin{split}-\sum_{i=1}^n \frac{f^{(4)}(\xi_i)}{90}h^5
  = \frac{-h^5}{90}\cdot \sum_{i=1}^n f^{(4)}(\xi_i)\end{split}\notag
\end{gather}
Nú gefur meðalgildisreglan að til er \(\xi \in [a,b]\) þannig að
\begin{gather}
\begin{split}f^{(4)}(\xi) = \frac{1}{n} \sum_{i=1}^n f^{(4)}(\xi_i)\end{split}\notag
\end{gather}
Nú er \(nh = \frac{(b-a)}{2}\) þar með er skekkjan:
\begin{gather}
\begin{split}\frac{-h^5}{90}\cdot nf^{(4)}(\xi)
  = \frac{-(b-a)}{180}f^{(4)}(\xi)\cdot h^4\end{split}\notag
\end{gather}
Ef við táknum útkomuna úr samsettu Simpsonsreglunni fyrir
\(h=\frac{b-a}{2n}\) með \(S(h)\) þá fæst að til er
\(\xi \in [a,b]\) þannig að
\begin{gather}
\begin{split}\int_a^b f(x)\,dx = S(h) - \frac{(b-a)}{180}f^{(4)}(\xi)h^4\end{split}\notag
\end{gather}

\section{Romberg-útgiskun}
\label{kafli05:romberg-utgiskun}
Á sama hátt og við gátum bætt nálgun okkar á afleiðu falls með að nota
\href{https://notendur.hi.is/~bsm/stae405/kafli04.html\#richardson-utgiskun}{Richardson útgiskun}
getum við bætt nálgun á heildi.

Aðferðin virkar í aðalatriðum eins fyrir heildi og afleiður, en til að
fá sem bestar upplýsingar um samleitni hennar skulum við leiða út
formúluna fyrir trapisureglunni aftur.


\subsection{Euler-Maclauren-formúlan}
\label{kafli05:euler-maclauren-formulan}
Fyrir samfellt fall \(f : [0,1] \to \mathbb R\) sem er
\(2n\)-sinnum samfellt deildanlegt gildir Euler-Maclauren formúlan
\begin{gather}
\begin{split}\begin{aligned}
  \int\limits_0^1 f(t) \, dt
  =&  \frac{1}{2}\left( f(0) + f(1) \right)
  + \sum\limits_{k=1}^{n-1} A_{2k}
  \left( f^{(2k-1)}(0) - f^{(2k-1)}(1)\right) \\
  & - A_{2n}f^{(2n)}(\xi), \qquad \xi \in [0,1]\end{aligned}\end{split}\notag
\end{gather}
Hér eru stuðlarnir \(A_k\) þannig að \(k!A_k\) verði
Bernoulli-talan númer \(k\). Þessar tölur eru stuðlar í veldaröðinni
\begin{gather}
\begin{split}\frac{x}{e^x -1} = \sum\limits_{k=0}^{\infty}A_kx^k\end{split}\notag
\end{gather}
\begin{notice}{note}{Athugasemd:}
Það þarf að hafa töluvert fyrir því að sanna þessa formúlu og því sleppum
við því hér.
\end{notice}


\subsection{Afleiðing af Euler-Maclaurin-formúlunni}
\label{kafli05:afleiing-af-euler-maclaurin-formulunni}
Látum nú \(f : [a,b] \to \mathbb R\) vera \(2n\)-sinnum samfellt
deildanlegt fall. Ef við búum til skiptingu
\(a= x_0 < x_1 < \cdots <
x_n = b\) með jöfn hlutbil \(h = x_{i+1} - x_i\) og beitum síðan
Euler-Maclauren formúlunni á \(g(t) = f(x_i + ht)\) fæst
\begin{gather}
\begin{split}\begin{aligned}
\int_{x_i}^{x_{i+1}} f(x)\,dx
= & h\int_0^1 \underbrace{f(x_i + ht)}_{g(t)}\,dt \\
= & h \left( \frac{1}{2}f(x_i) + \frac{1}{2}f(x_{i+1})\right) \\
& +    \sum_{k=1}^{n-1}A_{2k}h^{2k}\left( f^{(2k-1)}(x_i) -
f^{(2k-1)}(x_{i+1}) \right) - A_{2n}h^{2n+1}f^{(2n)}(\xi_i), \end{aligned}\end{split}\notag
\end{gather}
þar sem \(\xi_i \in [x_i, x_{i+1}]\).

Nú innleiðum við
\begin{gather}
\begin{split}\begin{aligned}
T(h)
&:= \sum_{i=0}^{n-1}
h \left( \frac{1}{2} f(x_i) +
\frac{1}{2}f(x_{i+1}) \right)\\
&= h\left( \frac{1}{2}f(a) + f(a+h)
+ \cdots + f(a+(n-1)h) + \frac{1}{2}f(a+nh)\right)\end{aligned}\end{split}\notag
\end{gather}
og fáum síðan:
\begin{gather}
\begin{split}\begin{aligned}
  \int\limits_a^b f(x)\, dx
  = & T(h) + \sum_{k=1}^{n-1}A_{2k}h^{2k}
  \left( f^{(2k-1)}(a) - f^{(2k-1)}(b) \right) \\
  & - A_{2n}h^{2n+1} \sum_{i=0}^{n-1} f^{(2n)}(\xi_i)\end{aligned}\end{split}\notag
\end{gather}
Nú gefur milligildissetningin að til er \(\xi \in [a,b]\) þannig að
\begin{gather}
\begin{split}\frac{1}{n} \sum\limits_{k=0}^{n-1} f^{(2n)}(\xi_i)
  = f^{(2n)}(\xi)\end{split}\notag
\end{gather}
Notum okkur nú að \(nh = b-a\) og fáum að
\begin{gather}
\begin{split}\begin{aligned}
  \int\limits_a^b f(x) \, dx
  = & T(h) + \sum_{k=1}^{n-1}A_{2k}h^{2k}
  \left( f^{(2k-1)}(a) - f^{(2k-1)}(b) \right) \\
  & - A_{2n} h^{2n}(b-a)f^{(2n)}(\xi).\end{aligned}\end{split}\notag
\end{gather}
Niðurstaðan er að samsetta trapisureglan er
\begin{gather}
\begin{split}\int\limits_a^b f(x) \, dx
  = T(h) + c_2h^2 + c_4h^4 + \cdots + c_{2m-2}h^{2m-2}
  + c_{2n}h^{2m}f^{(2m)}(\xi)\end{split}\notag
\end{gather}

\subsection{Ítrekun á samsettu trapisureglunni með helmingun}
\label{kafli05:itrekun-a-samsettu-trapisureglunni-me-helmingun}
Hugsum okkur nú að við viljum reikna út \(T(h_j)\) fyrir
\(h_j =(b-a)/
2^j\), \(j = 1,2,\ldots\) og að við viljum nýta öll fallgildi í
\(T(h_{j-1})\) til að reikna út \(T(h_j)\). Rakningarformúlan er
\begin{gather}
\begin{split}T(h_j) = \frac{1}{2} T(h_{j-1}) + h_j \sum_{k=1}^{2^{j-1}} f(a+(2k-1)h_j)\end{split}\notag
\end{gather}
Athugið að hér er bilinu \([a,b]\) skipt í \(2^j\) hlutbil.

\begin{notice}{warning}{Aðvörun:}
Þetta er gott að nota ef forrita á Romberg-heildun til þess að
spara útreikninga þegar fyrsti dálkurinn er reiknaður.
Það er hins vegar ekki nauðsynlegt að nota þetta og þetta tengist ekki
beint Romberg aðferðinni.
\end{notice}


\subsection{Reikniritið fyrir Romberg-heildun}
\label{kafli05:reikniriti-fyrir-romberg-heildun}
Romberg-heildun er hugsuð nákvæmlega eins og Richardson-útgiskunin: Við
reiknum út línu fyrir línu í töflunni:
\begin{gather}
\begin{split}\begin{array}{cccccc}
    i\\
    1 & R(1,1)\\
    2 & R(2,1) & R(2,2)\\
    3 & R(3,1) & R(3,2) & R(3,3)\\
    4 & R(4,1) & R(4,2) & R(4,3) & R(4,4)\\
    \vdots & \vdots & \vdots & \vdots & \vdots & \ddots
  \end{array}\end{split}\notag
\end{gather}
þar sem
\begin{gather}
\begin{split}\begin{aligned}
  &R(i,1) = T(h_i) \qquad i = 1,2,\ldots\\
  &R(i,j) = \frac{4^{j-1} R(i,j-1) - R(i-1,j-1)}{4^{j-1} - 1}.\end{aligned}\end{split}\notag
\end{gather}
Með þessu fæst
\(\int\limits_a^b f(x)\, dx = R(k,k) + O(h_k^{2k})\), þar sem
\(k\) er síðasta línan sem við reiknum í töflunni að ofan.


\subsection{Skekkjumat í Romberg heildun}
\label{kafli05:skekkjumat-i-romberg-heildun}
Skekkjumatið er hægt að finna með nákvæmlega sama hætti í fyrir
Richardson útgiskuna. Þ.e. við getum notað síðustu viðbót sem eftirámat
fyrir skekkjuna, þetta mat er
\begin{gather}
\begin{split}e \approx \frac{1}{4^{j-1}-1}\left( R(i,j-1) - R(i-1,j-1)\right)\end{split}\notag
\end{gather}
þegar þessi stærð er komin niður fyrir fyrirfram gefin skekkjumörk er
hætt.

Einnig er hægt að nota
\begin{gather}
\begin{split}e \approx \frac{1}{2^{j-1}}\left( R(i,j-1) - R(i-1,j-1)\right),\end{split}\notag
\end{gather}
sem gefur heldur varfærnislegra mat.

\begin{notice}{note}{Athugasemd:}
Athugið að það er ekki nauðsynlegt að hafa \(h_1\) sem allt bilið
\([a,b]\), það er ekkert sem kemur í veg fyrir það að við byrjum
með \(h_1 = \frac{b-a}{m}\), og helmingum svo;
\(h_2 = \frac{b-a}{2m}\), \(h_3 = \frac{b-a}{4m}\),
\(\ldots\).
Þannig að almennt þá er \(h_j=\frac{b-a}{2^{j-1}m}\).
\end{notice}
\phantomsection\label{kafli06:upphafsgildisverkefni}
\index{upphafsgildisverkefni}

\chapter{Upphafsgildisverkefni}
\label{kafli06::doc}\label{kafli06:index-0}\label{kafli06:id1}
\emph{In the beginning there was nothing, which exploded.}
-- Terry Pretchett


\section{Inngangur}
\label{kafli06:inngangur}
\index{upphafsgildisverkefni!fyrsta stigs}

\subsection{Fyrsta stigs afleiðujafna með upphafsgildi}
\label{kafli06:index-1}\label{kafli06:fyrsta-stigs-afleiujafna-me-upphafsgildi}
Gefið fall \(f\) á einhverju svæði \(U\) í
\(\mathbb{R}^2\) sem inniheldur \((t_0,x_0)\) þá er
\emph{fyrsta stigs afleiðujafna með upphafsgildi} verkefni á forminu
\begin{gather}
\begin{split}\begin{cases}
x' = f(t,x),\\
x(t_0) = x_0.
\end{cases}\end{split}\notag
\end{gather}
Við segjum að fall \(x(t)\) sé lausn á þessu verkefni ef \(x\) er
skilgreint á bili \(I\), sem er þannig að
\begin{itemize}
\item {} 
\(t_0 \in I\),

\item {} 
\((t,x(t)) \in U\) fyrir öll \(t \in I\),

\item {} 
\(x'(t) = f(t,x(t))\) fyrir öll \(t \in I\), og

\item {} 
\(x(t_0) = x_0\).

\end{itemize}


\subsection{Útskýring og dæmi}
\label{kafli06:utskyring-og-daemi}
Ástæðan fyrir því að við notum \(t\) fyrir breytuna og \(x\) fyrir fall
hér, er að breytan \(t\) lýsir oft tíma og lausnin getur þá t.d. verið staðsetning
sem fall af tíma.

Eðlilegt er að hugsa um afleiðujöfnuna þannig að fallið \(f\) geymi upplýsingar
um þau lögmál sem kerfið að hagar sér eftir, upphafsskilyrðið \(x(t_0)=x_0\)
segir í hvaða stöðu kerfið er þegar það er sett af stað og
lausnin \(x(t)\) lýsir hegðun kerfisins með tíma.

Í upphafi námskeiðsins,
\href{https://notendur.hi.is/~bsm/stae405/kafli01.html\#daemi-eldflaug}{dæmi 1.2}, skoðuðum
við dæmi um eldflaug sem skotið er á loft. Þar er notum við reyndar \(v\) í stað
\(x\) því jafnan lýsir hraða (e. velocity) en ekki staðsetningu.
Jafnan var
\begin{gather}
\begin{split}v' = \frac{5000-(300-10t)g-0,1v^2+10v}{300-10t},   \qquad v(0)=0.\end{split}\notag
\end{gather}
Hér er því \(f(t,v) = \frac{5000-(300-10t)g-0,1v^2+10v}{300-10t}\), \(t_0=0\)
og \(x_0 = 0\).

\index{upphafsgildisverkefni!tilvist og ótvíræðni}

\subsection{Tilvist og ótvíræðni lausna}
\label{kafli06:tilvist-og-otviraeni-lausna}\label{kafli06:index-2}\begin{gather}
\begin{split}\begin{cases}
x' = f(t,x)\\
x(t_0) = x_0
\end{cases}\end{split}\notag
\end{gather}
Ef \(f\) er samfellt, þá er alltaf til lausn á einhverju bili
\(I\). (\href{https://en.wikipedia.org/wiki/Peano\_existence\_theorem}{Setning Peano})

Ef \(f\) uppfyllir Lipschitz-skilyrði með tilliti til \(x\),
þ.e.a.s. til er fasti \(C\) þannig að
\begin{gather}
\begin{split}|f(t,x_1) - f(t,x_2)| \leq C|x_1 - x_2|\end{split}\notag
\end{gather}
fyrir öll \((t,x_1)\) og \((t,x_2)\) í grennd um
\((t_0, x_0)\) þá er lausnin ótvírætt ákvörðuð. (\href{https://en.wikipedia.org/wiki/Picard\%E2\%80\%93Lindel\%C3\%B6f\_theorem}{Setning Picard})


\subsection{Upphafsgildisverkefni fyrir hneppi}
\label{kafli06:upphafsgildisverkefni-fyrir-hneppi}
Með því að nota vigra og vigurgild föll má útfæra upphafsgildisverkefni fyrir hneppi:
\begin{gather}
\begin{split}\begin{cases}
{\mbox{${\bf x}$}}' ={\mbox{${\bf f}$}}(t,{\mbox{${\bf x}$}})\\
{\mbox{${\bf x}$}}(t_0) = {\mbox{${\bf x}$}}_0
\end{cases}\end{split}\notag
\end{gather}
þar sem
\begin{gather}
\begin{split}{\mbox{${\bf f}$}}: U \rightarrow {{\mathbb  R}}^n, \qquad U\subset \mathbb{R}^{n+1}, \quad
(t_0,{\mbox{${\bf x}$}}_0) \in U.\end{split}\notag
\end{gather}
Athugið að við skrifum \({\mbox{${\bf x}$}}(t)\) og
\({\mbox{${\bf f}$}}(t,{\mbox{${\bf x}$}})\) sem dálkvigra,
\begin{gather}
\begin{split}{\mbox{${\bf x}$}}(t) = [x_1(t), x_2(t), \ldots , x_n(t)]^T
\quad \text{  og } \quad
{\mbox{${\bf f}$}}(t,{\mbox{${\bf x}$}}) = [f_1(t,{\mbox{${\bf x}$}}), \ldots , f_n(t, {\mbox{${\bf x}$}})]^T\end{split}\notag
\end{gather}
\index{upphafsgildisverkefni!jafngild hneppi}

\subsection{Jöfnur af stigi \(>1\) og jafngild hneppi}
\label{kafli06:index-3}\label{kafli06:jofnur-af-stigi-og-jafngild-hneppi}
Aðferðirnar sem við munum skoða eru eingöngu fyrir fyrsta stigs afleiðujöfnur,
sem þýðir að jöfnurnar sem við leysum
innihalda bara \(x'\) en ekki \(x'',x''',\ldots\).
Hins vegar þá getum við leyst afleiðujöfnur af hærra stigi með því að umrita þær yfir í jafngilt
fyrsta stigs hneppi.

Ef við höfum
\(m\)-stigs diffurjöfnu
\begin{gather}
\begin{split}\begin{aligned}
u^{(m)} &= g(t,u, u',u'',\ldots , u^{(m-1)})\\
u(t_0) &= u_0, \quad u'(t_0) = u_1, \quad \ldots, \quad  u^{m-1}(t_0) = u_{(m-1)}\end{aligned}\end{split}\notag
\end{gather}
þar sem \(g\) er gefið fall og \(t_, u_0, \ldots , u_{m-1}\) eru
gefnar tölur þá er jafngilt hneppi er fengið með því að setja
\begin{gather}
\begin{split}\begin{aligned}
x_1 =& u, \\
x_2 =& u', \\
x_3 =& u'', \\
\vdots& \\
x_m =& u^{(m-1)}\end{aligned}.\end{split}\notag
\end{gather}
Þá fæst hneppið
\begin{gather}
\begin{split}{\bf x}' =
\begin{pmatrix}
x_1' &= x_2 x_1(t_0) \\
x_2' &= x_3 x_2(t_0 \\
\vdots & \vdots\\
x_{m-1}' &= x_m \\
x_m' &= g(t,x_1, \ldots , x_m)
\end{pmatrix} =
{\bf f}(t,{\bf x})\end{split}\notag
\end{gather}
með upphafsskilyrðið \({\bf x}(t_0)^T = [u_0,u_1,u_2,\ldots,u_m]^T\).

Fyrsta hnitið í lausn hneppisins, \(x_1\), gefur þá lausn, \(u\)
á upprunalegu \(m\)-ta stigs afleiðujöfnunni.


\subsection{Tilvist og ótvíræðni lausna á hneppum}
\label{kafli06:tilvist-og-otviraeni-lausna-a-hneppum}
Tilvistar- og ótvíræðnisetningar Peanos og Picards eru þær sömu fyrir
hneppi
\begin{gather}
\begin{split}\begin{cases}
{\mbox{${\bf x}$}}' ={\mbox{${\bf f}$}}(t,{\mbox{${\bf x}$}})\\
{\mbox{${\bf x}$}}(t_0) = {\mbox{${\bf x}$}}_0
\end{cases}\end{split}\notag
\end{gather}
Við þurfum bara að setja norm \(\|\cdot\|\) í stað tölugildis
\(|\cdot|\) í öllum ójöfnum og þar með talið í Lipschitz-skilyrðinu.

\index{upphafsgildisverkefni!nálgunargildi}\index{tímaskref}\index{skekkja}

\subsection{Ritháttur}
\label{kafli06:index-4}\label{kafli06:rithattur}
Til einföldunar á rithætti skulum við skrifa lausnarvigurinn
\({\mbox{${\bf x}$}}\) og vörpunina \({\mbox{${\bf f}$}}\) sem
\(x\) og \(f\) og láta eins og við séum að leysa fyrsta stigs
afleiðujöfnu.

Við veljum gildi \(t_0 < t_1 < \cdots < t_j<\cdots\) og reiknum út
\emph{nálgunargildi} \(w_j\) á gildi lausnarinnar \(x(t_j)\) í
punktinum \(t_j\). Gildið \(w_0=x(t_0)\) er rétta upphafsgildi
lausnarinnar

Talan \(t_j\) kallast \(j\)-ti \emph{tímapunkturinn} og talan
\(h_j=t_j-t_{j-1}\) nefnist \(j\)-ta \emph{tímaskrefið}.

\emph{Skekkjan} á tíma \(t_j\) er þá \(e_j = x(t_j)-w_j\).


\subsection{Grunnhugmyndin í nálgunaraðferðum}
\label{kafli06:grunnhugmyndin-i-nalgunaraferum}
Ef við heildum lausn afleiðujöfnunnar yfir tímabilið \([t,t+h]\), þá
fáum við að hún uppfyllir jöfnuna
\begin{gather}
\begin{split}x(t+h)=x(t)+\int_t^{t+h}f(\tau,x(\tau))\, d\tau
=x(t)+h\int_0^1f(t+sh,x(t+sh))\, ds.\end{split}\notag
\end{gather}
Ef við setjum \(t=t_{j-1}\) inn í þessa jöfnu, þá fáum við
\begin{gather}
\begin{split}\dfrac{x(t_j)-x(t_{j-1})}{h_j}=\int_0^1f(t_{j-1}+sh_j,x(t_{j-1}+sh_j))\, ds\end{split}\notag
\end{gather}
Nálgunaraðferðirnar snúast allar um að gera einhvers konar nálgun á
heildinu í hægri hliðinni
\begin{gather}
\begin{split}\int_0^1f(t_{j-1}+sh_j,x(t_{j-1}+sh_j))\, ds
  \approx \varphi(f,t_0,\dots,t_j,w_0,\dots,w_j)\end{split}\notag
\end{gather}
og leysa síðan \(w_j\) út úr jöfnunni
\begin{gather}
\begin{split}\dfrac{w_j-w_{j-1}}{h_j}=\varphi(f,t_0,\dots,t_j,w_0,\dots,w_j)\end{split}\notag
\end{gather}
\index{upphafsgildisverkefni!beinar/óbeinar aðferðir}

\subsection{Beinar og óbeinar aðferðir}
\label{kafli06:index-5}\label{kafli06:beinar-og-obeinar-aferir}
Nálgunaraðferð sem byggir á jöfnunni
\begin{gather}
\begin{split}\dfrac{w_j-w_{j-1}}{h_j}=\varphi(f,t_{0},\dots,t_j,w_{0},\dots,w_j)\end{split}\notag
\end{gather}
er nefnist \emph{bein aðferð} (e. explicit method) ef \(w_j\) kemur ekki
fyrir í í hægri hliðinni.

Annars nefnist hún \emph{óbein aðferð} eða \emph{fólgin aðferð} (e. implicit
method).

Ef aðferðin er bein og við höfum reiknað út \(w_0,\dots,w_{j-1}\),
þá fáum við rakningarformúlu, þannig að \(w_j\approx x(t_j)\) er
reiknað út
\begin{gather}
\begin{split}w_j=w_{j-1}+h_j\varphi(f,t_{0},\dots,t_j,w_{0},\dots,w_{j-1})\end{split}\notag
\end{gather}
\index{upphafsgildisverkefni!skref}

\subsection{Eins skrefs aðferðir og fjölskrefaaðferðir}
\label{kafli06:index-6}\label{kafli06:eins-skrefs-aferir-og-fjolskrefaaferir}
Nálgunaraðferð sem byggir á jöfnunni
\begin{gather}
\begin{split}\dfrac{w_j-w_{j-1}}{h_j}=\varphi(f,t_{j-1},t_j,w_{j-1},w_j)\end{split}\notag
\end{gather}
er nefnist \emph{eins skrefs aðferð} (e. one step method) og er þá vísað til
þess að fallið í hægri hliðinni er einungis háð gildum á síðasta
tímaskrefinu.

er af gerðinni
\begin{gather}
\begin{split}\dfrac{w_j-w_{j-1}}{h_j}=\varphi(f,t_{j-2},t_{j-1},t_j,w_{j-2},w_{j-1},w_j)\end{split}\notag
\end{gather}
Almennt er \(k\) \emph{-skrefa aðferð} af gerðinni
\begin{gather}
\begin{split}\dfrac{w_j-w_{j-1}}{h_j}=\varphi(f,t_{j-k},\dots,t_j,w_{j-k},\dots,w_j)\end{split}\notag
\end{gather}
\emph{Fjölskrefaðferð} er \(k\)-skrefa aðferð með \(k\geq 2\).


\section{Aðferðir með fasta skrefastærð}
\label{kafli06:aferir-me-fasta-skrefastaer}
\index{upphafsgildisverkefni!aðferð Eulers}

\subsection{Aðferð Eulers}
\label{kafli06:index-7}\label{kafli06:afer-eulers}
Rifjum upp að lausnin uppfyllir
\begin{gather}
\begin{split}\begin{aligned}
  x(t+h) - x(t) &= \int\limits_t^{t+h} x'(\tau) \, d\tau
  = \int\limits_t^{t+h} f(\tau,x(\tau)) \, d\tau\\
&= h\int\limits_0^{1} f(t+sh,x(t+sh)) \, ds\end{aligned}\end{split}\notag
\end{gather}
Billengdin í síðasta heildinu er \(1\), svo við tökum einföldustu
nálgum sem hugsast getur en það er gildið í vinstri endapunkti
\(f(t,x(t))\). Fyrir lítil \(h\) fæst því
\begin{gather}
\begin{split}x(t+h) \approx x(t) + hf(t,x(t)).\end{split}\notag
\end{gather}
Við þekkjum \(w_0=x(t_0)\), svo með þessu getum við fikrað okkur
áfram og fengið runu nálgunargilda \(w_0, w_1, w_2, \ldots\) þannig
að
\begin{gather}
\begin{split}w_j = w_{j-1} + h_{j} f(t_{j-1},w_{j-1}).\end{split}\notag
\end{gather}

\subsection{Aðferð Eulers: Matlab-forrit}
\label{kafli06:afer-eulers-matlab-forrit}
\begin{Verbatim}[commandchars=\\\{\}]
function w = euler(f,t,alpha);
\PYGZpc{}   function w = euler(f,t,alpha)
\PYGZpc{} Aðferð Eulers fyrir afleiðujöfnuhneppi
\PYGZpc{}         x\PYGZsq{}(t)=f(t,x(t)), x(0)=alpha.
\PYGZpc{} Inn fara: f \PYGZhy{} fallið f
\PYGZpc{}           t \PYGZhy{} vigur með skiptingu á t\PYGZhy{}ás.
\PYGZpc{}           alpha \PYGZhy{} upphafsgildið í t(1).
\PYGZpc{} Út koma:  w \PYGZhy{} fylki með nálgunargildunum.

N = length(t);
m = length(alpha);
w = zeros(m,N);
w(:,1) = alpha;
for j=2:N
   w(:,j) = w(:,j\PYGZhy{}1)+(t(j)\PYGZhy{}t(j\PYGZhy{}1))*f(t(j\PYGZhy{}1),w(:,j\PYGZhy{}1));
end
\end{Verbatim}


\subsection{Aðferð Eulers: Dæmi}
\label{kafli06:afer-eulers-daemi}
Prófum aðferð Eulers á afleiðujöfnunni
\begin{gather}
\begin{split}x' = \frac tx, \qquad x(0) = 1\end{split}\notag
\end{gather}
Við sjáum að rétt lausn er \(x(t) = \sqrt{t^2+ 1}\).

Notum 101 jafndreifð tímagildi á bilinu {[}0,5{]}. Þá er skekkjan

\begin{Verbatim}[commandchars=\\\{\}]
\PYGZgt{}\PYGZgt{} f = @(t,x) t./x;
\PYGZgt{}\PYGZgt{} t=linspace(0,5,101);
\PYGZgt{}\PYGZgt{} w=euler(f,t,1);
\PYGZgt{}\PYGZgt{} plot(t,sqrt(t.\PYGZca{}2+1) \PYGZhy{} w)
\PYGZgt{}\PYGZgt{} title(\PYGZsq{}Skekkja í aðferð Eulers\PYGZsq{}); xlabel(\PYGZsq{}t\PYGZsq{}); ylabel(\PYGZsq{}x\PYGZhy{}w\PYGZsq{});
\end{Verbatim}

\includegraphics{7euler.png}

\index{upphafsgildisverkefni!endurbætt aðferð Eulers}

\subsection{Endurbætt aðferð Eulers}
\label{kafli06:endurbaett-afer-eulers}\label{kafli06:index-8}
Í aðferð Eulers nálguðum við heildið
\(\int_0^1 f(t+sh,x(t+sh))\, ds\) með margfeldi af billengdinni og
fallgildinu í vinstri endapunkti.

Við getum endurbætt þessa nálgun með því að taka einhverja nákvæmari
tölulega nálgun á heildinu til dæmis miðpunktsaðferð

Nálgunarformúlan verður þá
\begin{gather}
\begin{split}\int_0^1f(t+sh,x(t+sh))\, ds \approx f(t+\tfrac 12h,x(t+\tfrac 12 h)).\end{split}\notag
\end{gather}
Nú er vandamálið að við höfum nálgað \(x(t_{j-1})\) með
\(w_{j-1}\) en höfum ekkert nálgunargildi á
\(x(t_{j-1}+\frac 12 h_j)\).

Við grípum þá til fyrsta stigs Taylor nálgunar
\begin{gather}
\begin{split}\begin{aligned}
x(t_j+\tfrac 12 h_j)&=x(t_{j-1})+x'(t_{j-1})\big(\tfrac 12 h_j \big)
+\tfrac 12x''(\xi)\big(\tfrac 12 h_j \big)^2\\
&\approx w_{j-1}+\tfrac 12 h_jf(t_{j-1},w_{j-1}).\end{aligned}\end{split}\notag
\end{gather}
Endurbætt aðferð Eulers er þá í tveim skrefum; við reiknum
\begin{gather}
\begin{split}\tilde w_j = w_{j-1} + \tfrac 12 h_j f(t_{j-1},w_{j-1})\end{split}\notag
\end{gather}
og fáum svo nálgunargildið
\begin{gather}
\begin{split}w_j = w_{j-1} + h_jf\left(
    t_{j-1}+\tfrac 12 h_j,\tilde w_j\right)\end{split}\notag
\end{gather}
\index{upphafsgildisverkefni!aðferð Heun}

\subsection{Aðferð Heun}
\label{kafli06:afer-heun}\label{kafli06:index-9}
Lítum nú á aðra aðferð þar sem við nálgum heildið með trapisuaðferð.
\begin{gather}
\begin{split}\int_0^1f(t+sh,x(t+sh))\, ds \approx
\tfrac 12 \big(f(t,x(t))+f(t+h,x(t+h))\big).\end{split}\notag
\end{gather}
Af þessu leiðir að nálgunarformúlan á að vera
\begin{gather}
\begin{split}w_j=w_{j-1}+\tfrac 12h_j\big(f(t_{j-1},w_{j-1})+f(t_j,w_j)\big)\end{split}\notag
\end{gather}
Þetta er greinilega óbein aðferð svo við verðum að byrja á nálgun á
\(w_j\), með
\begin{gather}
\begin{split}w_j\approx x(t_j)=x(t_{j-1}+h_j)\approx x(t_{j-1})+h_jx'(t_{j-1})
=x(t_{j-1})+h_jf(t_{j-1},w_{j-1})\end{split}\notag
\end{gather}
Þetta nýja afbrigði af aðferð Eulers nefnist \emph{aðferð Heun}. Hún er í
tveim skrefum: Við reiknum fyrst
\begin{gather}
\begin{split}\tilde w_j = w_{j-1} + h_jf(t_{j-1},w_{j-1})\end{split}\notag
\end{gather}
og fáum svo nálgunargildið
\begin{gather}
\begin{split}w_j = w_{j-1} + \tfrac 12h_j
\big(f(t_{j-1},w_{j-1})+f(t_j,\tilde w_j)\big)\end{split}\notag
\end{gather}
\index{upphafsgildisverkefni!Runge-Kutta aðferðir}\index{upphafsgildisverkefni!forsagnar- og leiðréttingarskref}

\subsection{Forsagnar- og leiðréttingarskref}
\label{kafli06:forsagnar-og-leirettingarskref}\label{kafli06:index-10}
Endurbætt aðferð Eulers og aðferð Heun eru leiðir til þess að vinna úr
óbeinum aðferðum, þar sem rakningarformúlan fyrir nálgunargildin er af
gerðinni
\begin{gather}
\begin{split}w_j=w_{j-1}+h_j\varphi(f,t_{j-1},t_j,w_{j-1},w_j)\end{split}\notag
\end{gather}
og okkur vantar eitthverja nálgun á \(w_j\) til þess að stinga inn í
hægri hlið þessarar jöfnu. Við skiptum þessu tvö skref:

Við beitum einhverri beinni aðferð til þess að reikna út
\begin{gather}
\begin{split}\tilde w_j=w_{j-1}+h_j\psi(f,t_{j-1},t_j,w_{j-1})\end{split}\notag
\end{gather}
Setjum
\begin{gather}
\begin{split}w_j=w_{j-1}+h_j\varphi(f,t_{j-1},t_j,w_{j-1},\tilde w_j).\end{split}\notag
\end{gather}
Svona aðferðir kallast \emph{Runge-Kutta aðferðir}. Fyrra skrefið, þegar
\(\tilde w_j\) er reiknað út kallast \emph{forsagnarskref} og
seinna skrefið kallast \emph{leiðréttingarskref}.

\index{upphafsgildisverkefni!annars stigs Runge-Kutta}

\subsection{Annars stigs Runge-Kutta-aðferð}
\label{kafli06:annars-stigs-runge-kutta-afer}\label{kafli06:index-11}
Lítum aftur á verkefnið
\begin{gather}
\begin{split}\left\{
    \begin{array}{l}
      x'(t) = f(t,x(t)) \\
      x(t_0) = x_0
    \end{array}
  \right.\end{split}\notag
\end{gather}
og skoðum 2. stigs Taylor liðun á lausninni \(x\) í punkti
\(t\). Innleiðum fyrst smá rithátt til styttingar, setjum
\begin{gather}
\begin{split}x = x(t), \quad f'_t = \frac{\partial f}{\partial t}(t,x(t)), \quad
  f = f(t,x(t)), \quad f'_x = \frac{\partial f}{\partial x}(t,x(t)).\end{split}\notag
\end{gather}
Keðjureglan gefur
\begin{gather}
\begin{split}x''(t)=\dfrac d{dt}f(t,x(t))=f'_t+f'_xx'(t)=f'_t+f\,f'_x.\end{split}\notag
\end{gather}
Taylor-liðun lausnarinnar er
\begin{gather}
\begin{split}\begin{aligned}
  x(t+h) &= x + hx'(t) + \frac{1}{2} h^2 x''(t) + O(h^3) \\
  &= x + hf + \frac{1}{2} h^2 ( f'_t + f f'_x ) + O(h^3) \\
  &= x + \frac{1}{2}hf + \frac{1}{2}h( f + hf'_t + (hf)f'_x) + O(h^3)\end{aligned}\end{split}\notag
\end{gather}
Nú sjáum við að síðasti liðurinn er 1. stigs Taylor liðun \(f\) með
miðju \((t,x)\) skoðuð í punktinum \((t+h,x+hf)\), því
\begin{gather}
\begin{split}f(t+h,x + hf) = f + hf'_t + (hf) f'_x + O(h^2)\end{split}\notag
\end{gather}
og þar með er
\begin{gather}
\begin{split}x(t+h) = x(t) + \frac{1}{2} hf(t,x) + \frac{1}{2} hf(t+h,x+hf) + O(h^3).\end{split}\notag
\end{gather}
Þessi formúla liggur til grundvallar 2. stigs Runge-Kutta-aðferð: Með
henni fáum við nálgunarrunu \(w_0, w_1, w_2, \ldots\) þannig að
\(w_0=x(0)\) og
\begin{gather}
\begin{split}w_j = w_{j-1} + \tfrac{1}{2}(F_1 + F_2), \quad j = 1,2,\ldots\end{split}\notag
\end{gather}
þar sem
\begin{gather}
\begin{split}F_1 = h_jf(t_{j-1},w_{j-1}),
  \quad \text{og} \quad
  F_2 = h_jf(t_j,w_{j-1}+F_1)\end{split}\notag
\end{gather}
og eins og alltaf er \(w_j \approx x(t_j)\).

\index{upphafsgildisverkefni!klassíska Runge-Kutta}

\subsection{Klassíska (fjórða stigs) Runge-Kutta aðferðin}
\label{kafli06:index-12}\label{kafli06:klassiska-fjora-stigs-runge-kutta-aferin}
Algengasta Runge-Kutta aðferðin er klassíska Runge-Kutta aðferðin. Þetta
er fjórða stigs aðferð, sem þýðir að staðarskekkjan er \(O(h^5)\) og
heildarskekkjan er \(O(h^4)\), .
\begin{gather}
\begin{split}w_{j} = w_{j-1} + \frac 16(k_1 + 2k_2 + 2k_3 + k_4),\end{split}\notag
\end{gather}
þar sem
\begin{gather}
\begin{split}\begin{aligned}
  k_1 &= hf(t_{j-1},w_{j-1}) \\
  k_2 &= hf\left(t_{j-1} + \frac h2,w_{j-1}+ \frac{k_1}2\right) \\
  k_3 &= hf\left(t_{j-1} + \frac h2,w_{j-1}+ \frac{k_2}2\right) \\
  k_4 &= hf(t_{j-1} + h,w_{j-1}+ k_3).
 \end{aligned}\end{split}\notag
\end{gather}
Ef \(f(t,x)\) er bara fall af \(t\), þ.e. óháð \(x\), þá
svarar þetta til þess að meta heildið \({\varphi}\) með
Simpson-reglunni.


\subsection{Klassíska Runge-Kutta aðferðin: Dæmi}
\label{kafli06:klassiska-runge-kutta-aferin-daemi}
Skoðum nú sama dæmi og þegar við prófuðum aðferð Eulers.

Þá gefa eftirfarandi skipanir mynd af skekkjunni.

\begin{Verbatim}[commandchars=\\\{\}]
\PYGZgt{}\PYGZgt{} f = @(t,x) t./x;
\PYGZgt{}\PYGZgt{} [w,t]=rk4(f,0,1,5,100);
\PYGZgt{}\PYGZgt{} plot(t,sqrt(t.\PYGZca{}2+1) \PYGZhy{} w)
\end{Verbatim}

\includegraphics{7rk4.png}

Þetta er töluvert betra en aðferð Eulers sem skilaði skekkju af stærðargráðunni
\(10^{-2}\).


\section{Skekkjumat, samleitni og stöðugleiki}
\label{kafli06:skekkjumat-samleitni-og-stougleiki}
\begin{Verbatim}[commandchars=\\\{\}]
+++Mr. Jelly! Mr. Jelly!+++
+++Error At Address: 14, Treacle Mine Road, Ankh\PYGZhy{}Morpork+++
+++MELON MELON MELON+++
+++Divide By Cucumber Error. Please Reinstall Universe And Reboot +++
+++Whoops! Here Comes The Cheese! +++
+++Oneoneoneoneoneoneone+++
\end{Verbatim}

--villuskilaboð tölvunnar Hex í Interesting Times eftir Terry Pratchett

\index{upphafsgildisverkefni!staðarskekkja}\index{upphafsgildisverkefni!heildarskekkja}

\subsection{Skekkja}
\label{kafli06:index-13}\label{kafli06:skekkja}
Fyrir eins skrefs aðferð skilgreinum við \emph{staðarskekkju} við tímann
\(t_n\) sem
\begin{gather}
\begin{split}\tau_n = \dfrac{x(t_n)-x(t_{n-1})}{h_n} -
\varphi(f,t_{n-1},t_n,x(t_{n-1}),x(t_{n}))\end{split}\notag
\end{gather}
Hér er réttu lausninni stungið inn í nálgunarformúluna. Munum að hún
uppfyllir
\begin{gather}
\begin{split}\dfrac{x(t_n)-x(t_{n-1})}{h_n}
=\int_0^1 f(t_{n-1}+sh_n,x(t_{n-1}+sh_n))\, ds\end{split}\notag
\end{gather}
Viljum geta metið \(\tau_n\) sem fall af \(h_n\), t.d.
\begin{gather}
\begin{split}\tau_n = O(h_n^k)\end{split}\notag
\end{gather}
Almennt batna aðferðir eftir því sem veldisvísirinn \(k\) í
staðarskekkjunni verður stærri.

Staðarskekkja er hlutfallsleg skekkja við að fara úr \(w_{n-1}\)
yfir í \(w_n\). Einnig má skoða uppsafnaða skekkju frá
upphafstímanum \(t_0\), hún er skilgreind með
\(e_n = x(t_j)-w_j\) og kallast \emph{heildarskekkja}.


\subsection{Staðarskekkja í aðferð Eulers}
\label{kafli06:staarskekkja-i-afer-eulers}
Aðferð Eulers er sett fram með formúlunni
\begin{gather}
\begin{split}w_n=w_{n-1}+h_nf(t_{n-1},w_{n-1})\end{split}\notag
\end{gather}
Staðarskekkjan er því
\begin{gather}
\begin{split}\begin{aligned}
  \tau_n&=\dfrac{x(t_n)-x(t_{n-1})}{h_n}-f(t_{n-1},x(t_{n-1}))\\
&=\dfrac{x(t_n)-x(t_{n-1})-x'(t_{n-1})h_n}{h_n}\\
&=\dfrac{\tfrac 12 x''(\xi_{n})h_{n-1}^2}{h_n}
=\tfrac 12 x''(\xi_{n})h_{n-1}=O(h_n)\end{aligned}\end{split}\notag
\end{gather}
Aðferð Eulers er því fyrsta stigs aðferð.


\subsection{Stýring á staðarskekkju og breytileg skrefastærð}
\label{kafli06:styring-a-staarskekkju-og-breytileg-skrefastaer}
Hingað til þá höfum við ekki fengið neinar upplýsingar til að finna heppilegustu skrefastærð.
Eftir því sem skrefastærðin er minni er staðarskekkjan sennilega minni, en þá komumst við
hægar yfir og það er hætta á að heildarskekkjan hækki við að taka mörg skref.
Í {\hyperref[kafli06:aferir-me-breytilega-skrefastaer]{Aðferðir með breytilega skrefastærð}} munum við reyna að stilla
skrefastærðina þannig að við tökum eins stór skref og mögulegt en þó
þannig að staðarskekkjan sé ekki of há.
Þá munum við þurfa eftirfarandi útleiðslu.

Hugsum okkur að við höfum tvær beinar nálgunaraðferðir
\begin{gather}
\begin{split}w_{n} = w_{n-1} + h_n\varphi(f,t_{n-1},t_n,w_{n-1})\end{split}\notag
\end{gather}
og
\begin{gather}
\begin{split}\tilde w_{n} = w_{n-1} + h_n\tilde\varphi(f,t_{n-1},t_n,w_{n-1})\end{split}\notag
\end{gather}
Skilgreinum tilsvarandi staðarskekkjur
\begin{gather}
\begin{split}\tau_n(h_n) = k_1h_n^{\alpha_1} + o(h_i^{\alpha_1})\end{split}\notag
\end{gather}
og
\begin{gather}
\begin{split}\tilde\tau_n(h_n) = k_2h_n^{\alpha_2} + o(h_i^{\alpha_2}),\end{split}\notag
\end{gather}
þar sem \(\alpha_2>\alpha_1\). Við tímann \(t_{n-1}\) hafa
nálgunargildin \(w_0,\ldots,w_{n-1}\) hafi verið valin samkvæmt
fyrri aðferðinni.

Meiningin að velja næsta tímapunkt \(t_n\) og þar með tímaskref
\(h_n\) þannig að \(\tau_n(h_n)\leq \delta\), en að
\(\tau_n(h_n)\) haldi sig sem næst \(\delta\), þar sem
\(\delta\) er gefið efra mark á staðarskekkjunni í fyrri
aðferðinni.

Stærðin \(\delta\) er kölluð \emph{þolmörk} (e. tolerance) fyrir
staðarskekkjuna og er oft táknuð með \(TOL\).

Við byrjum á að setja \(h=h_{n}\) inn í báðar aðferðirnar og bera
útkomurnar saman
\begin{gather}
\begin{split}w_{n} = w_{n-1} + h\varphi(f,t_{n-1},t_{n-1}+h,w_{n-1})\end{split}\notag
\end{gather}\begin{gather}
\begin{split}\tilde w_{n} = \tilde w_{n-1} +
h\tilde\varphi(f,t_{n-1},t_{n-1}+h,w_{n-1})\end{split}\notag
\end{gather}
Við látum \(\hat w_{n}\) tákna rétt gildi lausnarinnar á
upphafsgildisverkefninu
\begin{itemize}
\item {} 
\(x'(t)=f(t,x(t))\),

\item {} 
\(x(t_{n-1})=w_{n-1}\),

\end{itemize}

í punktinum \(t_{n-1}+h\).

Þá höfum við
\begin{gather}
\begin{split}\begin{aligned}
 \tau_n(h)&=\dfrac{\hat
w_{n}-w_{n-1}}{h}-\varphi(f,t_{n-1},t_{n-1}+h,w_{n-1})\\
&=\dfrac{\hat
w_{n}-w_{n-1}-h\varphi(f,t_{n-1},t_{n-1}+h,w_{n-1})}{h}
=\dfrac {\hat w_{n}-w_{n}}{h}\end{aligned}\end{split}\notag
\end{gather}
og eins fæst
\begin{gather}
\begin{split}\begin{aligned}
\tilde \tau_n(h)
&=\dfrac{\hat
w_{n}-w_{n-1}}{h}-\tilde \varphi(f,t_{n-1},t_{n-1}+h,w_{n-1})\\
&=\dfrac{\hat
w_{n}-w_{n-1}-h\tilde \varphi(f,t_{n-1},t_{n-1}+h,w_{n-1})}{h}
=\dfrac {\hat w_{n}-\tilde w_{n}}{h}. \end{aligned}\end{split}\notag
\end{gather}
Nú tökum við mismuninn og skilgreinum
\begin{gather}
\begin{split}\begin{aligned}
\varepsilon
&= \left|\frac{\tilde w_{n}-w_{n}}{h}\right|=|\tau_n(h)-\tilde
  \tau_n(h)|\\
&=|k_1|h^{\alpha_1}+o(h^{\alpha_1}) \approx |k_1|h^{\alpha_1}  \end{aligned}\end{split}\notag
\end{gather}
Munum að hér er skreflengdin \(h=h_{n}\). Þessi nálgunarformúla
gefur okkur möguleika á því að meta fastann
\begin{gather}
\begin{split}|k_1|\approx
\dfrac\varepsilon{h_{n}^{\alpha_1}}.\end{split}\notag
\end{gather}

\subsection{Mat á skrefastærð}
\label{kafli06:mat-a-skrefastaer}
Segjum nú að við viljum halda staðarskekkjunni innan markanna
\(\delta/2\) og hafa skreflengdina í næsta skrefi
\(h_{n}=qh_{n-1}\), þá höfum við nálgunarjöfnuna
\begin{gather}
\begin{split}|\tau_n(qh_{n-1})|\approx |k_1|(qh_{n-1})^{\alpha_1}=
\varepsilon {q^{\alpha_1}} \approx  \frac{\delta} 2.\end{split}\notag
\end{gather}
Við tökum
\begin{gather}
\begin{split}q = \left(\frac{\delta}{2\varepsilon}\right)^{1/{\alpha_1}}\end{split}\notag
\end{gather}
veljum síðan skrefstærðina \(h_n = qh_{n-1}\) og reiknum út næsta
gildi
\begin{gather}
\begin{split}w_{n} = w_{n-1} + h_n\varphi(f,t_{n-1},t_n,w_{n-1})\end{split}\notag
\end{gather}
\index{upphafsgildisverkefni!breytileg skrefastærð}

\section{Aðferðir með breytilega skrefastærð}
\label{kafli06:aferir-me-breytilega-skrefastaer}\label{kafli06:index-14}
Dæmi um aðferðir sem notast við breytilega skrefastærð.
\begin{itemize}
\item {} 
Einfaldast væri að nota Heun aðferðina (annars stigs) til að meta
skrefastærðina í Euler aðferðinni (fyrsta stigs).

\item {} 
Algengasta aðferðin er Runge-Kutta-Fehlberg (RKF45) sem notar
5. stigs nálgun til þess að meta staðarskekkjuna í 4. stigs aðferð.

\item {} 
Endurbót á RKF45 er Runge-Kutta-Verner (RKV56) sem notar 6. stigs
aðferð til að meta skekkjuna í 5. stigs aðferð.

\item {} 
Fleiri aðferðir: Bogacki–Shampine (3. og 2. stigs), Cash–Karp (5. og
4. stigs) og Dormand–Prince (5. og 4. stigs).

\end{itemize}

\index{upphafsgildisverkefni!Runge-Kutta-Fehlberg (RKF45)}

\subsection{Reiknirit fyrir Runge-Kutta-Fehlberg (RKF45)}
\label{kafli06:index-15}\label{kafli06:reiknirit-fyrir-runge-kutta-fehlberg-rkf45}\begin{gather}
\begin{split}\begin{aligned}
  \tilde w_j &= w_{j-1} \frac{16}{135} k_1 + \frac{6656}{12825}k_3 + \frac{28561}{56430}k_4
  - \frac{9}{50}k_5 + \frac{2}{55}k_6\\
  w_j &= w_{j-1} + \frac{25}{216}k_1 + \frac{1408}{2565}k_3 + \frac{2197}{4104}k_4 - \frac 15 k_5
 \end{aligned}\end{split}\notag
\end{gather}
þar sem
\begin{gather}
\begin{split}\begin{aligned}
  k_1 &= hf(t_{j-1},w_{j-1}) \\
  k_2 &= hf\left( t_{j-1}+\frac 14h, w_{j-1}+\frac 14k_1          \right)\\
  k_3 &= hf\left( t_{j-1}+\frac 38h, w_{j-1}+\frac 3{32}k_1 + \frac 9{32}k_2\right)\\
  k_4 &= hf\left( t_{j-1}+\frac{12}{13}h, w_{j-1} + \frac{1932}{2197}k_1
  - \frac{7200}{2197}k_2 + \frac{7296}{2197}k_3 \right)\\
  k_5 &= hf\left( t_{j-1} +h, w_{j-1} + \frac{439}{216}k_1 - 8k_2+\frac{3680}{513}k_3
  -\frac{845}{4104}k_4\right)\\
  k_6 &= hf\left( t_{j-1} +\frac 12h, w_{j-1} - \frac 8{27}k_1 + 2k_2 -\frac{3544}{2565}k_3
  +\frac{1859}{4104}k_4 - \frac{11}{40}k_5\right)\\
 \end{aligned}\end{split}\notag
\end{gather}

\subsection{Runge-Kutte-Fehlberg (RKF45) prófuð}
\label{kafli06:runge-kutte-fehlberg-rkf45-profu}
Höldum áfram með dæmi sem við beittum aðferð Eulers og
klassísku Runge-Kutta hér á undan

Þá gefur eftirfarandi mynd af skekkjunni. Hér er 0.01 minnsta leyfilega
skrefastærðin, 0.1 stærsta leyfilega skrefastærðin og þolmörkin eru
\(10^{-10}\).

\begin{Verbatim}[commandchars=\\\{\}]
\PYGZgt{}\PYGZgt{} f = @(t,x) t./x;
\PYGZgt{}\PYGZgt{} [w,t] = rkf45(f,0,1,5,[0.01,0.1,1E\PYGZhy{}10]);
\PYGZgt{}\PYGZgt{} plot(t,sqrt(t.\PYGZca{}2+1) \PYGZhy{} w)
\end{Verbatim}

\includegraphics{7rkf45.png}

Hér á undan þá notðum við þolmörkin \(10^{-10}\) sem skilaði okkur
103 misstórum tímagildum á bilinu \([0,5]\). Svona getum við teiknað
upp stærðina á tímaskrefunum.

\begin{Verbatim}[commandchars=\\\{\}]
\PYGZgt{}\PYGZgt{} plot(t(2:end)\PYGZhy{}t(1:end\PYGZhy{}1),\PYGZsq{}*\PYGZsq{})
\end{Verbatim}

\includegraphics{7rkf45t.png}

\index{upphafsgildisverkefni!fjölskrefaaðferðir}

\section{Fjölskrefaaðferðir}
\label{kafli06:index-16}\label{kafli06:fjolskrefaaferir}
Þær aðferðir sem við höfum séð eiga allar sameiginlegt að ákvarða
nálgunargildi \(w_{n}\) aðeins út frá gildinu \(w_{n-1}\) næst á
undan. Hægt er að nota fleiri gildi \(w_{n-1}\), \(w_{n-2}\),
\(\ldots\) og fá þannig betri nákvæmni, en aðferðirnar verða að sama
skapi flóknari í notkun.

Eins og alltaf höfum við verkefnið
\begin{gather}
\begin{split}\left\{
    \begin{array}{l}
      x'(t) = f(t,x(t)) \\
      x(t_0) = w_0
    \end{array}
  \right.\end{split}\notag
\end{gather}
og viljum nálga gildi lausnarinnar \(x\) á bili \([a,b]\) þar
sem \(a =t_0\) eða \(b = t_0\). Látum \(t_0\), \(t_1\),
\(\ldots\), \(t_n\) vera skiptingu á bilinu \([a,b]\) og
gerum til einföldunar ráð fyrir að hún hafi jafna billengd
\(h=t_{j} - t_{j-1}\) fyrir \(j= 1, \ldots, n\).

\index{upphafsgildisverkefni!Adams-Bashforth}

\subsection{\(k\)-skrefa Adams-Bashforth aðferð}
\label{kafli06:index-17}\label{kafli06:skrefa-adams-bashforth-afer}
Við vitum að lausnin \(x\) uppfyllir
\begin{gather}
\begin{split}x(t_{n}) - x(t_{n-1}) =
  \int\limits_{t_{n-1}}^{t_n} f(t,x(t)) \, dt\end{split}\notag
\end{gather}
Skrifum nú
\begin{gather}
\begin{split}f(t,x(t)) = P_{k-1}(t) + R_{k-1}(t)\end{split}\notag
\end{gather}
þar sem
\begin{gather}
\begin{split}P_{k-1}(t) = \sum\limits_{j=1}^k f(t_{n-j},x(t_{n-j})) \cdot
  \ell_{k-1,j}(t)\end{split}\notag
\end{gather}
er brúunarmargliðan gegnum punktana \((t_{n-k},x(t_{n-k}))\),
\((t_{n+1-k},x(t_{n+1-k}))\), \(\ldots\),
\((t_{n-1},x(t_{n-1}))\), þ.e. gegnum síðustu \(k\) punkta á
undan \((t_n,x(t_n))\).

Þetta eru \(k\) punktar og því er aðferðin kölluð \(k\)-skrefa
aðferð.

Munum að til er \(\xi\) þannig að
\begin{gather}
\begin{split}R_{k-1}(t) = \frac{f^{(k)}(\xi,x(\xi))}{k!}
  \prod\limits_{j=1}^m (t-t_{n-j}).\end{split}\notag
\end{gather}
Við nálgum nú heildið af \(f\) yfir bilið \([t_{n-1},t_n]\) með
heildi \(P_{k-1}\) og fáum
\begin{gather}
\begin{split}w_{i+1} = w_i +
  \int\limits_{t_i}^{t_{i+1}} P_{k-1}(t) \, dt\end{split}\notag
\end{gather}
og með beinum útreikningum má sjá að skekkjan í þessari nálgun er
\(O(h^{k+1})\). Þessir útreikninga flækjast auðvitað eftir því sem
\(k\) stækkar.

Augljóslega getum við ekki notað \(k\) skrefa Adams-Bashforth
aðferðir um leið og við sjáum upphafsgildisverkefni, því við þurfum
\(k\) ágiskunargildi \(w_0, w_1, \ldots, w_{k-1}\) til að byrja
að nota aðferðina. Þessi gildi má fá með hverri sem er af aðferðunum sem
við höfum séð hingað til.

Ákveðin sértilfelli Adams-Bashforth aðferðanna eru meira notuð en önnur,
það eru tveggja, þriggja og fjögurra skrefa aðferðirnar. Áhugasömum
verður ekki skotaskuld úr að leiða út formúlurnar fyrir þær, en við
birtum bara niðurstöðurnar.

Til styttingar skilgreinum við \(f_j = f(t_j,w_j)\).


\subsection{Tveggja skrefa Adams-Bashforth-aðferð}
\label{kafli06:tveggja-skrefa-adams-bashforth-afer}
Þegar gildin \(w_{n-1}\) og \(w_{n-2}\) hafa verið fundin fæst
næsta nálgunargildi með
\begin{gather}
\begin{split}w_{n} = w_{n-1} + h\big(\tfrac 32 f_{n-1} - \tfrac 12 f_{n-2}\big)\end{split}\notag
\end{gather}
og skekkjan í nálguninni er \(O(h^3)\).


\subsection{Forrit fyrir tveggja skrefa Adams-Bashforth-aðferð}
\label{kafli06:forrit-fyrir-tveggja-skrefa-adams-bashforth-afer}
Aðferðin er útfærð í forritinu hér að neðan; það skýrir sig að mestu
sjálft en við skulum taka eftir þrennu:

(i) Við krefjumst þess að notandinn gefi nálgunargildi á x(t(2)), þetta
gerum við því til eru margar mismunandi aðferðir til að fá slíkt gildi
og þær henta mis vel hverju sinni.

(ii) Við gerum ekki sérstaklega ráð fyrir að jafnt bil sé á milli
stakanna í vigrinum t þó við höfum gert það hingað til. Það var aðeins
gert til að einfalda útreikninga; aðferðin virkar nákvæmlega eins ef það
er ekki jafnt bil á milli stakanna, svo sjálfsagt er að forrita hana
þannig.

(iii) Við lágmörkum fjölda skipta sem við reiknum gildi f með að geyma
alltaf gildið frá síðustu ítrun og nota það aftur, þetta getur sparað
nokkurn tíma í útreikningum ef f er flókið fall.

\begin{Verbatim}[commandchars=\\\{\}]
function w = adams\PYGZus{}bashforth\PYGZus{}2(f,t,x1,x2)
\PYGZpc{}   w = adams\PYGZob{}\PYGZus{}\PYGZcb{}bashforth\PYGZob{}\PYGZus{}\PYGZcb{}2(f,t,x1,x2)
\PYGZpc{} Nálgar lausn upphafsgildisverkefnisins
\PYGZpc{}   x\PYGZsq{} = f(t,x)
\PYGZpc{}   x(t(1)) = x1
\PYGZpc{} í punktunum í t með 2ja þrepa Adams\PYGZhy{}Bashforth aðferð.
\PYGZpc{} Stakið x2 er nálgunargildi á x(t(2)).

N = length(t);  M = length(x1); w = zeros(M,N);
\PYGZpc{} Upphafsstillum gildi f(t,x) og w
fx1 = f(t(1),x1); fx2 = f(t(2),x2);
w(:,1) = x1; w(:,2) = x2;
for i=3:N
  \PYGZpc{} Reiknum nálgunargildi
  h = t(i)\PYGZhy{}t(i\PYGZhy{}1);
  w(:,i) = w(:,i\PYGZhy{}1) + (h/2)*(3*fx2 \PYGZhy{} fx1);
  fx1 = fx2; fx2 = f(t(i),w(:,i));
end
\end{Verbatim}


\subsection{Þriggja skrefa Adams-Bashforth}
\label{kafli06:riggja-skrefa-adams-bashforth}
Gefin \(w_{n-1}\), \(w_{n-2}\) og \(w_{n-3}\) fæst næsta
nálgunargildi með
\begin{gather}
\begin{split}w_{n} = w_{n-1} + {h}(\tfrac{23}{12} f_{n-1} - \tfrac {16}{12}
  f_{n-2} + \tfrac 5{12} f_{n-2})\end{split}\notag
\end{gather}
og staðarskekkjan er \(O(h^4)\)


\subsection{Fjögurra skrefa Adams-Bashforth}
\label{kafli06:fjogurra-skrefa-adams-bashforth}
Þegar við þekkjum \(w_{n-1}\), \(w_{n-2}\), \(w_{n-3}\) og
\(w_{n-4}\) reiknum við næsta gildi með
\begin{gather}
\begin{split}w_{n} = w_{n-1} + h\big(\tfrac{55}{24}f_{n-1} - \tfrac{59}{24}f_{n-2} +
\tfrac {37}{24}f_{n-3} -\tfrac 9{24}f_{n-4}\big)\end{split}\notag
\end{gather}
og skekkjan í nálguninni er \(O(h^5)\).


\section{Greining á samleitni og stöðugleika}
\label{kafli06:greining-a-samleitni-og-stougleika}
Lítum aftur á upphafsgildisverkefnið okkar
\begin{gather}
\begin{split}\begin{cases}
  x'(t)=f(t,x(t)),\\
x(t_0)=w_0.
\end{cases}\end{split}\notag
\end{gather}
Við hugsum okkur að nálgun sé fundin í tímapunktunum
\begin{gather}
\begin{split}a=t_0<t_1<t_2<\cdots<t_N=b.\end{split}\notag
\end{gather}
Við táknum nálgunargildi á \(x(t_j)\) með \(w_j\). Það er gefið
með
\begin{gather}
\begin{split}w_n=w_{n-1}+h_n\varphi(f,t_{0},\dots,t_n,w_{0},\dots,w_{n})\end{split}\notag
\end{gather}
þar sem fallið \(\varphi(f,t_{0},\dots,t_n,w_{0},\dots,w_{n})\) er
skilgreint með einhverjum hætti.

Við köllum þetta \emph{nálgunaraðferðina sem fallið} \(\varphi\) \emph{gefur af
sér.}


\subsection{Skekkja}
\label{kafli06:id3}
\emph{Skekkja} (e. error) eða \emph{heildarskekkja} (e. total error) í nálgun á
\(x(t_n)\) með \(w_n\) er
\begin{gather}
\begin{split}e_n=x(t_n)-w_n,\end{split}\notag
\end{gather}
og \emph{staðarskekkja} (e. local truncation error) nálgunaraðferðarinnar við tímann \(t_n\)
er
\begin{gather}
\begin{split}\tau_n=\dfrac{x(t_n)-x(t_{n-1})}{h_n}
-\varphi(f,t_{0},\dots,t_n,x(t_{0}),\dots,x(t_{n}))\end{split}\notag
\end{gather}
\begin{notice}{note}{Athugasemd:}
Munið að hér er \emph{rétta lausnin} sett inn í nálgunaraðferðina.
\end{notice}

\index{upphafsgildisverkefni!samleitni}

\subsection{Samleitni}
\label{kafli06:samleitni}\label{kafli06:index-18}
Hugsum okkur nú að fjöldi tímapunktanna \(N\) stefni á óendanlegt.
Við segjum að nálgunaraðferðin \(\varphi\) sé \emph{samleitin} ef
\begin{gather}
\begin{split}\lim_{N\to \infty} \max\limits_{1\leq n\leq N} |e_n|=0\end{split}\notag
\end{gather}
þar sem \(e_n=x(t_n)-w_n\) táknar skekkjuna í \(n\)-ta
tímaskrefinu.

\index{upphafsgildisverkefni!samræmi}

\subsection{Samræmi}
\label{kafli06:samraemi}\label{kafli06:index-19}
Við segjum að nálgunaraðferðin \(\varphi\) \emph{samræmist}
upphafsgildisverkefninu ef um sérhvern tímapunkt \(t_{n-1}\) gildir
að
\begin{gather}
\begin{split}\begin{gathered}
\lim_{h_n\to 0}\tau_n\\
=\lim_{t_n\to t_{n-1}}\bigg(\dfrac{x(t_n)-x(t_{n-1})}{t_n-t_{n-1}}
-\varphi(f,t_{0},\dots,t_n,x(t_{0}),\dots,x(t_{n}))\bigg)
=0  \end{gathered}\end{split}\notag
\end{gather}

\subsection{Samræmi endurbættu Euler-aðferðarinnar}
\label{kafli06:samraemi-endurbaettu-euler-aferarinnar}
Munum að endurbætta Euler-aðferðin er
\begin{gather}
\begin{split}w_n=w_{n-1}+h_nf(t_{n-1}+\tfrac 12 h_n,w_{n-1}+\tfrac 12 hf(t_{n-1},w_{n-1}))\end{split}\notag
\end{gather}
sem gefur staðarskekkjuna
\begin{gather}
\begin{split}\begin{gathered}
\tau_n=\dfrac{x(t_{n-1}+h_n)-x(t_{n-1})}{h_n}\\
-f(t_{n-1}+\tfrac 12 h_n,x(t_{n-1})+\tfrac 12 h_nf(t_{n-1},x(t_{n-1}))).
  \end{gathered}\end{split}\notag
\end{gather}
Nú hugsum við okkur að \(t_{n-1}\) sé haldið föstu og látum
billengdina \(h_n=t_n-t_{n-1}\) stefna á \(0\). Þá fæst
\begin{gather}
\begin{split}\lim_{h_n\to 0} \tau_n= x'(t_{n-1})-f(t_{n-1},x(t_{n-1}))=0\end{split}\notag
\end{gather}
Þetta segir okkur að endurbætta Euler-aðferðin samræmist
upphafsgildisverkefninu.


\subsection{Samræmi beinna eins skrefs aðferða}
\label{kafli06:samraemi-beinna-eins-skrefs-afera}
Þessi röksemdafærla alhæfist á allar beinar eins skrefs aðferðir, því
staðarskekkja þeirra er
\begin{gather}
\begin{split}\tau_n=\dfrac{x(t_{n-1}+h_n)-x(t_{n-1})}{h_n}
-\varphi(f,t_{n-1},t_{n-1}+h_n,x(t_{n-1}))\end{split}\notag
\end{gather}
Nú er eðlilegt að gefa sér að \(\varphi\) sé samfellt fall og þá
verður markgildið af staðarskekkjunni
\begin{gather}
\begin{split}\begin{gathered}
x'(t_{n-1})-\varphi(f,t_{n-1},t_{n-1},x(t_{n-1}))\\
=f(t_{n-1},x(t_{n-1}))-\varphi(f,t_{n-1},t_{n-1},x(t_{n-1})).\end{gathered}\end{split}\notag
\end{gather}
Eins skrefs aðferðin sem fallið \(\varphi\) gefur af sér er því
stöðug ef og aðeins ef
\begin{gather}
\begin{split}\varphi(f,t_{n-1},t_{n-1},x(t_{n-1}))
=f(t_{n-1},x(t_{n-1})).\end{split}\notag
\end{gather}
\index{upphafsgildisverkefni!stöðugleiki}

\subsection{Stöðugleiki}
\label{kafli06:index-20}\label{kafli06:stougleiki}
Gerum nú ráð fyrir að upphafsgildinu \(w_0\) sé breytt í
\(\tilde w_0\) og að \(\tilde x(t)\) uppfylli
\begin{gather}
\begin{split}\begin{cases}
  \tilde x'(t)=f(t,\tilde x(t)),\\
\tilde x(t_0)=\tilde w_0.
\end{cases}\end{split}\notag
\end{gather}
Lítum síðan á tilsvarandi nálgunarrunu
\begin{gather}
\begin{split}\tilde w_n=\tilde w_{n-1}+h_n\varphi(f,t_0,\dots,t_n,\tilde
w_0,\dots,\tilde w_n).\end{split}\notag
\end{gather}
Við segjum að nálgunaraðferðin sem \(\varphi\)
gefur af sér sé \emph{stöðug} ef til er fall \(k(t)>0\) þannig að
\begin{gather}
\begin{split}|\tilde w_n-w_n|\leq k(t_n)|\tilde w_0-w_0|, \qquad n=1,2,3\dots.\end{split}\notag
\end{gather}
\index{Lipschitz-samfelldni}

\subsection{Lipschitz-samfelldni}
\label{kafli06:index-21}\label{kafli06:lipschitz-samfelldni}
Rifjum nú upp að við gerum ráð fyrir að fallið \(f(t,x)\) sé
skilgreint á svæði \(D\) sem inniheldur
\begin{gather}
\begin{split}\{(t,x)\in {{\mathbb  R}}^2 \, ;\, a\leq t\leq b, x\in {{\mathbb  R}}\}.\end{split}\notag
\end{gather}
Við segjum að \(f\) sé \emph{Lipschitz samfellt á} \(D\) \emph{með tilliti
til} \(x\) ef til er fasti \(C_f\) þannig að
\begin{gather}
\begin{split}|f(t,x)-f(t,y)|\leq C_f|x-y|, \qquad x,y\in {{\mathbb  R}}.\end{split}\notag
\end{gather}
Hugsum okkur að \(\varphi(f,s,t,x)\) sé fall sem gefur af sér beina
eins skrefs nálgunaraðferð fyrir upphafsgildisverkefnið
\(x'(t)=f(t,x(t))\) með \(x(t_0)=w_0\).

Við segjum að \(\varphi\) sé \emph{Lipschitz-samfellt með tilliti til}
\(x\) ef um sérhvert Lipschitz-samfellt fall \(f\), tölur
\(s,t\in [a,b]\) og \(x,y\in {{\mathbb  R}}\) gildir að til er
fasti \(L_\varphi\) þannig að
\begin{gather}
\begin{split}|\varphi(f,s,t,x)-\varphi(f,s,t,y)|\leq L_\varphi|x-y|, \qquad x,y\in {{\mathbb  R}}.\end{split}\notag
\end{gather}

\subsection{Setning um stöðugleika og samleitni}
\label{kafli06:setning-um-stougleika-og-samleitni}
Gefum okkur jafna skiptingu á tímabilinu \([a,b]\),
\(t_n=a+nh\), þar sem \(n=0,1,2,\dots,N\) og \(h=(b-a)/N\).

Ef fallið \(\varphi\) er Lipschitz-samfellt með tilliti til
\(x\) með Lipschitz-fastann \(L_\varphi\), þá gildir:
\begin{enumerate}
\item {} 
Eins skrefs aðferðin sem \(\varphi\) gefur af sér er stöðug,
\begin{gather}
\begin{split}|\tilde w_n-w_n|\leq e^{L_\varphi(t_n-a)}|\tilde w_0-w_0|, \qquad
n=1,2,3,\dots.\end{split}\notag
\end{gather}
\item {} 
Ef til eru fastar \(c\) og \(p\) þannig að staðarskekkjan
uppfyllir \(|\tau_n|\leq c\, h^p\), fyrir öll
\(n=1,2,3,\dots\) og \(h\in ]0,h_0]\), þá er aðferðin
samleitin og við höfum
\begin{gather}
\begin{split}|e_n|=|x(t_n)-w_n|\leq \dfrac{ch^p}{L_\varphi}
\bigg(e^{L_\varphi(t_n-a)}-1\bigg).\end{split}\notag
\end{gather}
\end{enumerate}

\index{jaðargildisverkefni}

\chapter{Jaðargildisverkefni}
\label{kafli07::doc}\label{kafli07:index-0}\label{kafli07:jaargildisverkefni}
\emph{The pen is mightier than the sword if the sword is very short, and the pen is very sharp.}
-- Terry Pratchett


\section{Inngangur}
\label{kafli07:inngangur}

\subsection{Jaðargildisverkefni}
\label{kafli07:id1}
Við ætlum að finna nálgunarlausnir á verkefnum af gerðinni
\begin{gather}
\begin{split}\begin{gathered}
    y''=f(x,y,y'), \qquad a\leq x\leq b,\\
\alpha_1y(a)+\alpha_2 y'(a)=\alpha_3,\\
\beta_1 y(b)+\beta_2y'(b)=\beta_3.
  \end{gathered}\end{split}\notag
\end{gather}
Lausn á verkefninu er þá fall \(y(x):[a,b]\to \mathbb R\) sem er þannig að \(y\)
uppfyllir
\begin{itemize}
\item {} 
afleiðujöfnuna \(y''(x) = f(x,y(x),y'(x))\),

\item {} 
jaðarskilyrðin \(\alpha_1y(a)+\alpha_2 y'(a)=\alpha_3\) í \(x=a\) og

\item {} 
jaðarskilyrðin \(\beta_1 y(b)+\beta_2y'(b)=\beta_3\) í \(x=b\).

\end{itemize}

Afleiðujafnan er sögð vera línuleg ef \(f\) er á forminu
\begin{gather}
\begin{split}y''=p(x)y'+q(x)y+r(x), \qquad x\in [a,b].\end{split}\notag
\end{gather}
\index{jaðargildisverkefni!Dirichlet-jaðarskilyrði}\index{jaðargildisverkefni!Neumann-jaðarskilyrði}\index{jaðargildisverkefni!Robin-jaðarskilyrði}

\subsection{Þrjár tegundir jaðarskilyrða}
\label{kafli07:rjar-tegundir-jaarskilyra}\label{kafli07:index-1}
Venjulega eru jaðarskilyrðin flokkuð í þrjá flokka.

\begin{tabular}{lll}
	(i)  &Dirichlet-jaðarskilyrði: &  $y(a)=\alpha$, \ \  $y(b)=\beta$\\
	&(Fallsjaðarskilyrði:) \\ 
	(ii)&Neumann-jaðarskilyrði: 
	& $y'(a)=\alpha$, \ \  $y'(b)=\beta$\\
	&(Afleiðujaðarskilyrði:)\\
	&(Flæðisjaðarskilyrði:)\\
	(iii)&Robin-jaðarskilyrði: 
	&$\alpha_1y(a)+\alpha_2 y'(a)=\alpha_3$ \\ 
	&(Blandað jaðarskilyrði:)&$\beta_1 y(b)+\beta_2y'(b)=\beta_3$\\
	&&$(\alpha_1,\alpha_2)\neq (0,0)$
\end{tabular}


\begin{notice}{note}{Athugasemd:}
Athugið að Robin jaðarskilyrði með \(\alpha_2=0\) (eða
\(\beta_2=0\)) er Dirichlet skilyrði með \(\alpha=\alpha_3/\alpha_1\) (eða
\(\beta=\beta_3/\beta_1\)).

Athugið að Robin jaðarskilyrði með \(\alpha_1=0\) (eða
\(\beta_1=0\))
er Neumann skilyrði með \(\alpha=\alpha_3/\alpha_2\) (eða
\(\beta=\beta_3/\beta_2\)).
\end{notice}


\section{Dirichlet-jaðarskilyrði}
\label{kafli07:dirichlet-jaarskilyri}
\index{jaðargildisverkefni!skiptipunktar}

\subsection{Skiptipunktar}
\label{kafli07:skiptipunktar}\label{kafli07:index-2}
Gefum okkur jafna skiptingu á bilinu \([a,b]\), \(x_j=a+hj\),
\(j=0,\ldots,N\) þar sem \(h=(b-a)/N\). Þá er
\begin{gather}
\begin{split}a=x_0<x_1<x_2<\cdots<x_{N-1}<x_N=b.\end{split}\notag
\end{gather}
Við nefnum \(x_j\) \emph{skiptipunkta} skiptingarinnar.

Punktarnir \(a=x_0\) og \(b=x_N\) nefnast \emph{endapunktar}
skiptingarinnar og \(x_j\), með \(j=1,\dots,N-1\), nefnast
\emph{innri punktar} skiptingarinnar.

Við ætlum aðeins að nálga lausnir á línulegum jöfnum
\begin{gather}
\begin{split}y''=p(x)y'+q(x)y+r(x), \qquad x\in [a,b],\end{split}\notag
\end{gather}
Við munum reiknum út nálgun á réttu lausninni \(y(x)\) í skiptipunktunum \(x_j\).
Rétta gildið í punktinum \(x_j\) táknum við með \(y_j\) og
nálgunargildið með \(w_j\),
\begin{gather}
\begin{split}y_j=y(x_j)\approx w_j.\end{split}\notag
\end{gather}
Eins skrifum við
\begin{gather}
\begin{split}p_j=p(x_j), \qquad q_j=q(x_j), \qquad  r_j=r(x_j).\end{split}\notag
\end{gather}
\begin{notice}{warning}{Aðvörun:}
Ólíkt upphafsgildisverkefnunum í kaflanum á undan þá táknum við breytuna
hér með \(x\) og fallgildið með \(y\). Þetta er eðlilegur ritháttur
hér því í jaðargildisverkefnum þá er \(y\) oftast fall
af staðsetningu en ekki tíma, t.d. hiti í röri, sveigja burðarbita, o.s.fr.
\end{notice}

\index{jaðargildisverkefni!línulegar afleiðujöfnur}

\subsection{Línulegar afleiðujöfnur}
\label{kafli07:index-3}\label{kafli07:linulegar-afleiujofnur}
Nú leiðum við út nálgunarjöfnur, eina fyrir hvern innri skiptipunkt. Við
byrjum á því að stinga punkti \(x_j\) inn í afleiðujöfnuna
\begin{gather}
\begin{split}\big\{ y''(x)= p(x)y'(x)+q(x)y(x) + r(x)\big\}_{x=x_j}.\end{split}\notag
\end{gather}
Næst skiptum við afleiðunum \(y''\) og \(y'\) út fyrir
miðsettan mismunakvóta fyrir aðra afleiðuna og
miðsettan mismunakvóta fyrir fyrstu afleiðuna. Þá fæst
\begin{gather}
\begin{split}\dfrac{y_{j+1}-2y_j+y_{j-1}}{h^2} +O(h^2)
=p_j\dfrac{y_{j+1}-y_{j-1}}{2h}+q_jy_j+r_j+ O(h^2).\end{split}\notag
\end{gather}
Nú fellum við niður leifaliðina og setjum nálgunargildin í stað réttu
gildanna
\begin{gather}
\begin{split}\dfrac{w_{j+1}-2w_j+w_{j-1}}{h^2}
=p_j\dfrac{w_{j+1}-w_{j-1}}{2h}+q_jw_j+r_j\end{split}\notag
\end{gather}
Hér fáum við eina jöfnu fyrir sérhvern innri skiptipunkt
\(j=1,\dots,N-1\).


\subsection{Dirichlet-jaðarskilyrði}
\label{kafli07:id2}
Við erum komin með \(N-1\) nálgunarjöfnu til þess að finna
\(N+1\) nálgunargildi \(w_0,\dots,w_N\) fyrir
\(y_0,\dots,y_N\).

Ef við erum að leysa línulegt jaðargildisverkefni með
Dirichlet-jaðarskilyrðum,
\begin{gather}
\begin{split}\begin{gathered}
    y''=p(x)y'+q(x)y+r(x), \qquad a\leq x\leq b,\\
y(a)=\alpha \quad \text{ og } \quad y(b)=\beta,
  \end{gathered}\end{split}\notag
\end{gather}
þá fæst nálgunin með því að leysa línulega jöfnuhneppið
\begin{gather}
\begin{split}\begin{aligned}
w_0&=\alpha,\\
\dfrac{w_{j+1}-2w_j+w_{j-1}}{h^2}
&=p_j\dfrac{w_{j+1}-w_{j-1}}{2h}+q_jw_j+r_j, \qquad j=1,\dots,N-1,\\
w_N&=\beta.  \end{aligned}\end{split}\notag
\end{gather}

\subsection{Jafngild framsetning á hneppinu}
\label{kafli07:jafngild-framsetning-a-hneppinu}
Við lítum aftur á línulegu nálgunarjöfnurnar
\begin{gather}
\begin{split}\dfrac{w_{j+1}-2w_j+w_{j-1}}{h^2}
=p_j\dfrac{w_{j+1}-w_{j-1}}{2h}+q_jw_j+r_j.\end{split}\notag
\end{gather}
Margföldum alla liði með \(-h^2\) og röðum síðan óþekktu stærðunum
vinstra mengin jafnaðarmerkisins. Þá fæst línulega jöfnuhneppið
\begin{gather}
\begin{split}\big(-1-\tfrac 12 h p_j\big)w_{j-1}
+\big(2+h^2q_j\big) w_j
+\big(-1+\tfrac 12 h p_j\big)w_{j+1}
=-h^2\, r_j\end{split}\notag
\end{gather}
fyrir \(j=1,2,3,\dots,N-1\).


\subsection{Línulega jöfnuhneppið á fylkjaformi}
\label{kafli07:linulega-jofnuhneppi-a-fylkjaformi}
Nú er hægt að skrifa jöfnuhneppið á fylkjaformi
\begin{gather}
\begin{split}A{\mbox{${\bf w}$}}={\mbox{${\bf b}$}}\end{split}\notag
\end{gather}
Hér er
\begin{gather}
\begin{split}A=\left[\begin{matrix}
  1&0\\
  l_1&d_1&u_1\\
  &l_2&d_2&u_2\\
  &&\cdot&\cdot&\cdot \\
  &&&\cdot&\cdot&\cdot \\
  &&&&\cdot&\cdot&\cdot \\
  &&&&&l_{N-2}&d_{N-2}&u_{N-2} \\
  &&&&&&l_{N-1}&d_{N-1}&u_{N-1} \\
  &&&&&&&0&1
  \end{matrix}\right]\end{split}\notag
\end{gather}
þar sem stuðlarnir \(l_j\), \(d_j\) og \(u_j\) eru gefnir
með
\begin{gather}
\begin{split}\begin{aligned}
  l_j&=-1-\tfrac 12 hp_j\\
d_j&=2+h^2q_j\\
u_j&=-1+\tfrac 12 hp_j\end{aligned}\end{split}\notag
\end{gather}
og vigrarnir eru
\begin{gather}
\begin{split}{\mbox{${\bf w}$}}=\left[
  \begin{matrix}
w_0\\ w_1\\ w_2\\ \cdot\\ \cdot\\ \cdot\\
w_{N-2}\\ w_{N-1}\\ w_N
\end{matrix}\right]
\qquad \text{ og } \qquad
{\mbox{${\bf b}$}}=\left[
\begin{matrix}
\alpha \\ -h^2r_1\\ -h^2r_2\\ \cdot \\ \cdot\\ \cdot\\
-h^2r_{N-2}\\ -h^2r_{N-1}\\ \beta
\end{matrix}\right]\end{split}\notag
\end{gather}
Við þekkjum allar tölurnar í \(A\) og \(\bf b\), þannig að
við getum leyst jöfnuhneppið og með því fundið
nálgunargildin \(\bf w\).

\index{jaðargildisverkefni!felugildi}\index{jaðargildisverkefni!felupunktar}

\section{Neumann og Robin -jaðarskilyrði}
\label{kafli07:index-4}\label{kafli07:neumann-og-robin-jaarskilyri}

\subsection{Felugildi}
\label{kafli07:felugildi}
Við skulum gera ráð fyrir að rétta lausnin \(y(x)\) uppfylli blandað
jaðarskilyrði í \(x=a\),
\begin{gather}
\begin{split}\alpha_1y(a)+\alpha_2 y'(a)=\alpha_3.\end{split}\notag
\end{gather}
Til þess að líkja eftir afleiðujöfnunni í punktinum \(x=a\) þá
hugsum við okkur að við bætum við einum skiptipunkti \(x_{-1}=a-h\) og
látum \(w_f\) tákna ímyndað gildi lausnarinnar í \(x_{-1}\).

Svona punktur \(x_{-1}\) utan við skiptinguna er kallaður
\emph{felupunktur} við skiptinguna og ímyndað gildi \(w_f\) í felupunkti
er kallað \emph{felugildi}.

Takið eftir því að lausnin er ekki til í felupunktinum, en við reiknum
eins og \(w_f\) sé gildi hennar þar.

Mismunajafnan sem líkir eftir afleiðujöfnunni í punktinum \(x_0\) er
\begin{gather}
\begin{split}\big(-1-\tfrac 12 hp_0\big)w_f+\big(2+h^2 q_0\big)w_0
+\big(-1+\tfrac 12 hp_0\big)w_1=-h^2r_0\end{split}\notag
\end{gather}
Mismunajafnan sem líkir eftir jaðarskilyrðinu er
\begin{gather}
\begin{split}\alpha_1w_0+\alpha_2 \dfrac{w_1-w_f}{2h}=\alpha_3.\end{split}\notag
\end{gather}

\subsection{Jafna fyrir felugildið}
\label{kafli07:jafna-fyrir-felugildi}
Jafnan sem líkir eftir jaðarskilyrðinu er:
\begin{gather}
\begin{split}\alpha_1w_0+\alpha_2 \dfrac{w_1-w_f}{2h}=\alpha_3.\end{split}\notag
\end{gather}
Út úr henni leysum við
\begin{gather}
\begin{split}w_f=w_1-\dfrac{2h}{\alpha_2}\big(\alpha_3-\alpha_1w_0\big)\end{split}\notag
\end{gather}
Við stingum síðan þessu gildi inn í jöfnuna sem líkir eftir
afleiðujöfnunni
\begin{gather}
\begin{split}\big(-1-\tfrac 12 hp_0\big)w_f+\big(2+h^2 q_0\big)w_0
+\big(-1+\tfrac 12 hp_0\big)w_1=-h^2r_0\end{split}\notag
\end{gather}
Út fæst fyrsta jafna hneppisins
\begin{gather}
\begin{split}\bigg(2+h^2q_0-\big(2+hp_0\big)h\dfrac{\alpha_1}{\alpha_2}\bigg)w_0
-2w_1=-h^2r_0-\big(2+hp_0\big)h\dfrac{\alpha_3}{\alpha_2}.\end{split}\notag
\end{gather}
Með því að innleiða felupunkt \(x_{N+1}=b+h\) hægra megin við
skiptinguna, tilsvarandi felugildi \(w_f\) og leysa saman tvær
jöfnur þá fáum við síðustu jöfnu hneppisins
\begin{gather}
\begin{split}-2w_{N-1}
+\bigg(2+h^2q_N+\big(2-hp_N\big)h\dfrac{\beta_1}{\beta_2}\bigg)w_N
=-h^2r_N-\big(2-hp_N\big)h\dfrac{\beta_3}{\beta_2}\end{split}\notag
\end{gather}
Við erum því aftur komin með \((N+1)\times (N+1)\)-jöfnuhneppi.


\subsection{Hneppið á fylkjaformi}
\label{kafli07:hneppi-a-fylkjaformi}\begin{gather}
\begin{split}A{\mbox{${\bf w}$}}={\mbox{${\bf b}$}}\end{split}\notag
\end{gather}\begin{gather}
\begin{split}A=\left[\begin{matrix}
a_{11}&a_{12}\\
l_1&d_1&u_1\\
&l_2&d_2&u_2\\
&&\cdot&\cdot&\cdot \\
&&&\cdot&\cdot&\cdot \\
&&&&\cdot&\cdot&\cdot \\
&&&&&l_{N-2}&d_{N-2}&u_{N-2} \\
&&&&&&l_{N-1}&d_{N-1}&u_{N-1} \\
&&&&&&&a_{N+1,N}&a_{N+1,N+1}
\end{matrix}\right]\end{split}\notag
\end{gather}
Þar sem stuðlarnir \(l_j\), \(d_j\) og \(u_j\) fyrir
\(j=1,2,3\dots,N-1\) eru þeir sömu og áður.
\begin{gather}
\begin{split}\begin{aligned}
  l_j&=-1-\tfrac 12 hp_j\\
d_j&=2+h^2q_j\\
u_j&=-1+\tfrac 12 hp_j\end{aligned}\end{split}\notag
\end{gather}

\subsection{Fyrsta og síðasta lína hneppisins}
\label{kafli07:fyrsta-og-siasta-lina-hneppisins}\begin{gather}
\begin{split}\begin{aligned}
a_{11}&=
\begin{cases}
  1,&\text{Dirichlet í } x=a: \alpha_1\neq 0, \alpha_2=0,\\
d_0&\text{Neumann í } x=a:  \alpha_1=0, \alpha_2\neq 0,\\
d_0+2hl_0\alpha_1/\alpha_2&\text{Robin í } x=a:  \alpha_2\neq 0.
\end{cases} \\
a_{12}&=
\begin{cases}
  0,&\text{Dirichlet í } x=a: \alpha_1\neq 0, \alpha_2=0,\\
-2,&\text{annars}.
\end{cases}
\\
a_{N+1,N+1}&=
\begin{cases}
  1,&\text{Dirichlet í } x=b: \beta_1\neq 0, \beta_2=0,\\
d_N&\text{Neumann í } x=b:  \beta_1=0, \beta_2\neq 0,\\
d_N-2hu_N\beta_1/\beta_2&\text{Robin í } x=a:  \beta_2\neq 0.
\end{cases}
\\
a_{N+1,N}&=
\begin{cases}
  0,&\text{Dirichlet í } x=b: \beta_1\neq 0, \beta_2=0,\\
-2&\text{annars}.
\end{cases}
  \end{aligned}\end{split}\notag
\end{gather}

\subsection{Hægri hlið hneppisins}
\label{kafli07:haegri-hli-hneppisins}\begin{gather}
\begin{split}{\mbox{${\bf b}$}}=\left[
\begin{matrix}
b_1 \\ -h^2r_1\\ -h^2r_2\\ \cdot \\ \cdot\\ \cdot\\
-h^2r_{N-2}\\ -h^2r_{N-1}\\ b_{N+1}
\end{matrix}\right]\end{split}\notag
\end{gather}\begin{gather}
\begin{split}b_{1}=
\begin{cases}
  \alpha=\alpha_3/\alpha_1,
&\text{Dirichlet í } x=a: \alpha_1\neq 0, \alpha_2=0,\\
-h^2r_0+2hl_0\alpha_3/\alpha_2
&\text{Neumann í } x=a:  \alpha_1=0, \alpha_2\neq 0,\\
-h^2r_0+2hl_0\alpha_3/\alpha_2&\text{Robin í } x=a:  \alpha_2\neq 0.
\end{cases}\end{split}\notag
\end{gather}\begin{gather}
\begin{split}b_{N+1}=
\begin{cases}
  \beta=\beta_3/\beta_1,
&\text{Dirichlet í } x=a: \beta_1\neq 0, \beta_2=0,\\
-h^2r_N-2hu_N\beta_3/\beta_2&\text{Neumann í } x=a:  \beta_1=0, \beta_2\neq 0,\\
-h^2r_N-2hu_N\beta_3/\beta_2&\text{Robin í } x=a:  \beta_2\neq 0.
\end{cases}\end{split}\notag
\end{gather}
\index{jaðargildisverkefni!samantekt}

\subsection{Samantekt}
\label{kafli07:index-5}\label{kafli07:samantekt}
Gildi lausnarinnar \(y(x)\) á línulega jaðargildisverkefninu
\begin{gather}
\begin{split}\begin{gathered}
    y''=p(x)y'+q(x)y+r(x), \qquad a\leq x\leq b,\\
\alpha_1y(a)+\alpha_2 y'(a)=\alpha_3,\\
\beta_1 y(b)+\beta_2y'(b)=\beta_3
  \end{gathered}\end{split}\notag
\end{gather}
í punktunum \(x_j=a+jh\), þar sem \(h=(b-a)/N\) og
\(j=0,\dots,N\), eru nálguð með
\begin{gather}
\begin{split}w_j\approx y(x_j)=y_j\end{split}\notag
\end{gather}
Dálkvigurinn
\begin{gather}
\begin{split}{\mbox{${\bf w}$}}=[w_0,w_1,\dots,w_N]^T\end{split}\notag
\end{gather}
er lausn á línulegu jöfnuhneppi
\(A{\mbox{${\bf w}$}}={\mbox{${\bf b}$}}\).

Stuðlum \((N+1)\times(N+1)\) fylkisins \(A\) og
\((N+1)\)-dálkvigursins \({\mbox{${\bf b}$}}\) hefur verið lýst
hér að framan.


\chapter{Jöfnuhneppi}
\label{kafli08::doc}\label{kafli08:jofnuhneppi}

\textbf{Í VINNSLU}


\chapter{Eigingildisverkefni}
\label{kafli09:eigingildisverkefni}\label{kafli09::doc}

\textbf{Í VINNSLU}


\chapter{Monte Carlo hermanir}
\label{kafli10::doc}\label{kafli10:monte-carlo-hermanir}

\textbf{Í VINNSLU}


\chapter{Viðauki}
\label{vidauki:viauki}\label{vidauki::doc}
\emph{The trouble with having an open mind, of course, is that people will insist on coming along and trying to put things in it.}
-- Terry Pratchett

\begin{itemize}
	
	\item Þessi skrá (pdf) má finna á \\ \url{http://notendur.hi.is/bsm/stae405/stae405.pdf}
	
	\item Rafræna útgáfu af þessum nótum má finna á \\ \url{http://notendur.hi.is/bsm/stae405}
	
	\item Heimasíða námskeiðsins á Uglu \\ \url{https://ugla.hi.is/kv/index2.php?sid=219&namsknr=09104320160}
	
    \item Heimasíða námskeiðsins á PiazzaUglu \\ \url{https://www.piazza.com/hi.is/spring2016/st405g}
	
\end{itemize}

\newpage

\bigskip\hrule{}\bigskip



\chapter{Kennsluáætlun}
\label{vidauki:kennsluaaetlun}
\emph{Why bother with a cunning plan when a simple one will do?}
-- Terry Pratchett, Thud!

% \begin{longtable}{|p{0.317\linewidth}|p{0.317\linewidth}|p{0.317\linewidth}|}
\begin{center}
	\begin{tabular}{c|l|c}
		Dags. &Efni&Kaflar\\
		\hline
		08.01.16 & 1.  Inngangur  &  1.1-1.7\\\hline
		13.01.16 & 2. Núllstöðvar & 2.1-2.3\\
		14.01.16 & \textbf{Heimadæmi 1} & \\
		15.01.16 & 2. Núllstöðvar & 2.4-2.6\\\hline
		20.01.16 & 3. Brúun & 3.1-3.2 \\
		21.01.16 & \textbf{Heimadæmi 2} & \\
		22.01.16 & 3. Brúun & 3.3\\\hline
		27.01.16 & 3. Brúun & 3.4\\
		28.01.16 & \textbf{Heimadæmi 3} & \\
		29.01.16 & 3. Brúun & 3.5-3.6 \\
		& \textbf{Verkefni I kynnt} & \\\hline
		03.02.16 & 3. Brúun & 3.7 \\
		04.02.16 & \textbf{Heimadæmi 4} &\\
		05.02.16 & 3. Brúun & 3.8-3.9\\\hline
		10.02.16 & 4. Töluleg diffrun & 4.1-4.2\\
		12.02.16 & Samantekt & \\	
		& \textbf{Verkefni I skilað}\\\hline
		17.02.16 & 4. Töluleg diffrun & 4.3 \\
		18.02.16 & \textbf{Heimadæmi 5} & \\
		19.02.16 & 5. Töluleg heildun & 5.1\\\hline
		24.02.16 & 5. Töluleg heildun & 5.2\\
		25.02.16 & \textbf{Heimadæmi 6}&\\
		26.02.16 & 5. Töluleg heildun 5.3-5.4& \\
		& \textbf{Verkefni II kynnt} & \\\hline
		02.03.16 & 6. Upphafsgildisverkefni & 6.1-6.2\\
		03.03.16 & \textbf{Heimadæmi 7}&\\
		04.03.16 & 6. Upphafsgildisverkefni & 6.3-6.4\\\hline
		09.03.16 & 6. Upphafsgildisverkefni & 6.5-6.6\\
		11.03.16 & Hagnýtingar & Ítarefni \\
		& \textbf{Verkefni II skilað} & \\
		16.03.16 & 7. Jaðargildisverkefni & 7.1-7.2\\
		17.03.16 & \textbf{Heimadæmi 8} & \\
		18.03.16 & 7. Jaðargildisverkefni & 7.3\\\hline
		23.03.16 & \emph{Páskafrí} & \\
		25.03.16 & \emph{Páskafrí} & \\\hline
		30.03.16 & 8. Jöfnuhneppi & \\
		31.03.16 & \textbf{Heimadæmi 9} &\\
		01.04.16 & 8. Jöfnuhneppi & \\\hline
		06.04.16 & 8. Jöfnuhneppi & \\
		07.04.16 & \textbf{Heimadæmi 10}\\
		08.04.16 & 9. Eigingildisverkefni &\\\hline
		13.04.16 & 10. Monte Carlo hermanir &\\
		15.04.16 & Samantekt & \\\hline
		20.04.16 & Prófundirbúningur & \\\hline
		% } &  &  \multirow{2}{*}{
		% 10.  Monte Carlo hermanir
		% 
		% Samantekt
		% } &  &  \multirow{2}{*}{}\\
		% \hline\\
		% \hline & 
		% 20.04.16
		%  &  & 
		% Prófundirbúningur
		%  &  & \\
		% \hline\end{longtable}
	\end{tabular}
\end{center}


\bigskip\hrule{}\bigskip

\newpage

\section{Skipulag námskeiðsins}
\label{vidauki:skipulag-namskeisins}
\emph{Always be wary of any helpful item that weighs less than its operating manual.}
-- Terry Pratchett, Jingo


\subsection{Kennsla}
\label{vidauki:kennsla}
Fyrirlestrar verða á miðvikudögum klukkan 8:20-9:50 í HT-105, Háskólatorgi, og á föstudögum klukkan 10:00-11:30 í HT-105, Háskólatorgi.
Kennari er Benedikt Steinar Magnússon \textless{}\href{mailto:bsm@hi.is}{bsm@hi.is}\textgreater{}.

Aðstoðarkennarar eru Halla Björg Sigurþórsdóttir, Jón Áskell Þorbjarnarson, Jónas Grétar Jónasson, Páll Ásgeir Björnsson og Pétur Rafn Bryde.


\subsection{Kennsluefni}
\label{vidauki:kennsluefni}
Kennslubókin eru þessar nótur, \href{http://notendur.hi.is/bsm/stae405/}{http://notendur.hi.is/bsm/stae405/}. sem einnir er hægt að nálgast á pdf-formi. Auk þess set ég forritaskrár og annað ítarefni á Uglu eftir því sem við á.

Þeir sem vilja ítarlegri kennslubækur bendi ég á
\begin{itemize}
\item {} 
\emph{A Friendly Introduction to Numerical Analysis} eftir Brian Bradie.

\item {} 
\emph{Numerical Mathematics and Computing} eftir Ward Cheney og David Kincaid

\item {} 
\emph{Introduction to applied numerical analysis} eftir R. W. Hamming.
\begin{quote}

``The purpose of computing is insight, not numbers'' -R. W. Hamming
\end{quote}

\end{itemize}


\subsection{Matlab/Octave}
\label{vidauki:matlab-octave}
Við munum forrita töluvert í námskeiðinu. Til þess notum við annað hvort \emph{Matlab}, \href{http://www.mathworks.com/products/matlab/}{http://www.mathworks.com/products/matlab/} eða \emph{Octave}, \href{http://www.gnu.org/software/octave/}{http://www.gnu.org/software/octave/}.

Á heimasíðu Kristjáns Jónassonar, \href{http://notendur.hi.is/jonasson/matlab/}{http://notendur.hi.is/jonasson/matlab/} finnið þið leiðbeiningar um uppsetningu á Matlab. En fyrir þá sem kjósa frjálsan hugbúnað eru leiðbeiningar fyrir Octave hér \href{http://www.gnu.org/software/octave/download.html}{http://www.gnu.org/software/octave/download.html}.

Octave er að mestu leyti sambærilegt við Matlab, bæði ritháttur og svo styður það einnig \emph{m}-skrárnar úr Matlab. Sjá nánar \href{http://en.wikibooks.org/wiki/MATLAB\_Programming/Differences\_between\_Octave\_and\_MATLAB}{Differences\_between\_Octave\_and\_MATLAB}.

Þið hafið fullkomið val um það hvort þið notið Matlab eða Octave við að leysa verkefni námskeiðsins.

Ítarefni fyrir Matlab/Octave:
\begin{itemize}
\item {} 
Inngangur að Matlab/Octave fyrir línulega algebru, \href{https://notendur.hi.is/~bsm/linalg/}{https://notendur.hi.is/\textasciitilde{}bsm/linalg/}.

\item {} 
Kennslubók Kristjáns Jónassonar um Matlab fæst í \href{http://www.boksala.is/matlab-forritunarmal-fyrir-visindalega-utreikning.html}{Bóksölu Stúdenta}.

\item {} 
\href{http://en.wikibooks.org/wiki/Matlab}{http://en.wikibooks.org/wiki/Matlab}.

\item {} 
Hjálpin í Matlab og Octave er einnig mjög gagnleg.

\end{itemize}


\subsection{Heimadæmi og dæmatímar}
\label{vidauki:heimadaemi-og-daematimar}
Dæmatímarnir í námskeiðunu verða notaðir bæði til að reikna dæmi á töflu og sem stoðtímar fyrir heimadæmin. Alls verða lögð fyrir 10 heimadæmi. Þeim á að skila á fimmtudögum fyrir klukkan 16:00 í hólf viðkomandi dæmatímakennara. Heimadæmin er að finna á vikublöðum sem verða sett viku fyrr í möppuna \emph{Vikublöð} á Uglu.

\textbf{TIL AÐ ÖÐLAST PRÓFTÖKURÉTT ÞARF AÐ SKILA AÐ MINNSTA KOSTI 7 AF 10 HEIMADÆMUM.}

Undanþágur frá þessari reglu fást eingöngu fyrir atbeina Náms- og starfsráðgjafar Háskólans.

Merkt verður við heimadæmin á Uglu undir \emph{Verkefni og hlutapróf} og eru nemendur beðnir um að fylgjast með skráningunni þar og ganga úr skuggu um að allt sé rétt skráð.


\subsection{Verkefni}
\label{vidauki:verkefni}
Á misserinu verða tvö viðamikil forritunarverkefni.
Í kennsluáætlun stendur í hvaða fyrirlestrum þau verða kynnt og hvenær á að skila þeim. Verkefnin eigið þið að leysa í hóp, tvö eða þrjú saman. Allir í hópnum eiga að vera virkir og taka þátt í að leysa alla liði verkefnisins. Forritað er í Matlab eða Octave.

Í vikunum sem skila á verkefunum þá munum við nota dæmatímana sem stoðtíma fyrir verkefnin. Skila á verkefnunum á föstudögum kl. 16:00, fyrra verkefninu 12. febrúar og seinna verkefninu 11. mars.

Matlab/Octave-forritin eiga að vera hluti af úrlausn og skal þeim skilað ásamt skýrslu í gegnum Uglu.

Ef við finnum sömu forritin í fleiri en einni úrlausn, þá lítum við á það sem svindl og lækkum einkunn hjá öllum sem skráðir eru fyrir þeim lausnum og eftir atvikum tilkynnum deildarforseta og setjum í farveg innan sviðsins (sbr. \href{http://www.hi.is/adalvefur/reglur\_fyrir\_haskola\_islands\#51}{51. gr. rgl. 569/2009 HÍ}).


\subsection{Lokapróf}
\label{vidauki:lokaprof}
Prófið verður 3 tímar og skiptist í fræðilegar krossaspurningar og skrifleg dæmi. Formúlublöð sem fylgja prófinu eru einu
skriflegu hjálpargögnin sem leyfð verða.  Þið \textbf{eigið} að taka með ykkur reiknivélar. Prófað verður bæði úr efni fyrirlestranna og úr dæmareikningi. Nauðsynlegt er að ná prófinu með einkunn 5.  Verkefnaeinkunn gildir
30\% af lokaeinkunn.


\subsection{Námsmat}
\label{vidauki:namsmat}
Til þess að standast námskeiðið þarf eftirfarandi:
\begin{itemize}
\item {} 
Skila að minnsta kosti 7 af 10 heimadæmum.

\item {} 
Skila báðum verkefnunum.

\item {} 
Ná lokaprófinu með einkunn 5.

\item {} 
Lokaeinkunn (lokapróf 70\%, verkefnaeinkunn 30\%) þarf að vera að minnsta kosti 5.

\end{itemize}

Þau ykkar sem hafa prófrétt frá síðasta ári haldið verkefnaeinkunn. En þið þurfið að tilkynna mér það með tölvupósti sem fyrst.


\section{Frágangur heimadæma}
\label{vidauki:fragangur-heimadaema}
\emph{Let grammar, punctuation, and spelling into your life! Even the most energetic and wonderful mess has to be turned into sentences.}
-- Terry Pratchett
\begin{itemize}
\item {} 
Skrifið upp \textbf{dæmið} og lausnina snyrtilega

\item {} 
Vísið í setningar og niðurstöður sem þið notið

\item {} 
Notið ekki rökfræðitákn eins og \(\Leftarrow\), \(\Rightarrow\), \(\Leftrightarrow\), \(\wedge\), \(\vee\)

\item {} 
Textinn á að vera samfelldur og læsilegur (lesið hann sjálf yfir)

\item {} 
Skýrt svar/niðurstaða

\end{itemize}

\emph{If you trust in yourself. . .and believe in your dreams. . .and follow your star. . . you'll still get beaten by people who spent their time working hard and learning things and weren't so lazy.}
-- Terry Pratchett, The Wee Free Men
\newpage

\section{Gagnlegir tenglar}
\label{vidauki:gagnlegir-tenglar}
\emph{She got on with her education. In her opinion, school kept on trying to interfere with it.}
-- Terry Pratchett, Soul Music
\begin{itemize}
\item {} 
Orðasafn Íslenska stærðfræðafélagsins \href{http://stae.is/os}{http://stæ.is/os}

\item {} 
\href{http://mathworld.wolfram.com/topics/NumericalMethods.html}{http://mathworld.wolfram.com/topics/NumericalMethods.html}

\item {} 
\href{http://en.wikipedia.org}{http://en.wikipedia.org}

\item {} 
\href{https://en.wikibooks.org/wiki/Octave\_Programming\_Tutorial}{Octave Programming Tutorial}

\item {} 
\href{http://www.lehman.edu/academics/cmacs/documents/refcard-a4.pdf}{Octave Quick Reference (pdf)}

\item {} 
\href{http://se.mathworks.com/help/pdf\_doc/matlab/getstart.pdf?s\_tid=int\_tut}{Getting Started with Matlab (pdf)}

\item {} 
\href{https://matlabacademy.mathworks.com/R2015b/}{Matlab Academy}

\item {} 
\href{http://se.mathworks.com/academia/student\_center/tutorials/launchpad.html}{Matlab Tutorials and Learning Resources}

\end{itemize}
\begin{itemize}
\item {} 
\emph{genindex}

\end{itemize}



\renewcommand{\indexname}{Atriðaskrá}
\printindex
\end{document}
