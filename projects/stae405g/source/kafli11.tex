\documentclass[icelandic,a4paper,12pt]{article}
\usepackage{beamerarticle}



\mode<presentation>
{
  \usetheme{boxes}
  %\useoutertheme{infolines}
  % með efnisyfirliti: Szeged, Frankfurt 
  % án efnisyfirlits: Pittsburgh
  % áhugavert: CambridgeUS, Boadilla
  %\setbeamercovered{transparent} %gegnsætt
  \setbeamercovered{invisible}

\defbeamertemplate*{footline}{infolines theme}
{
  \leavevmode%
  \hbox{%
  \begin{beamercolorbox}[wd=.333333\paperwidth,ht=2.25ex,dp=1ex,center]{author in head/foot}%
  %  \usebeamerfont{author in head/foot}\insertshortauthor~~\beamer@ifempty{\insertshortinstitute}{}{(\insertshortinstitute)}
  \end{beamercolorbox}%
  \begin{beamercolorbox}[wd=.333333\paperwidth,ht=2.25ex,dp=1ex,center]{title in head/foot}%
   % \usebeamerfont{title in head/foot}\insertshorttitle
  \end{beamercolorbox}%
  \begin{beamercolorbox}[wd=.333333\paperwidth,ht=2.25ex,dp=1ex,right]{date in head/foot}%
    %\usebeamerfont{date in head/foot}\insertshortdate{}\hspace*{2em}
    \insertshortlecture.\insertframenumber{} / \insertshortlecture.\inserttotalframenumber\hspace*{2ex} 
  \end{beamercolorbox}}%
  \vskip0pt%
}
\resetcounteronoverlays{rtaskno} %Does not increase counter rtaskno on \pause in beamer


}


\usepackage[english,icelandic]{babel}
\usepackage[utf8]{inputenc}
\usepackage{t1enc}
\usepackage{graphicx}
\usepackage{amsmath}
\usepackage{amssymb}
\usepackage{mathrsfs}
\usepackage{verbatim}
\usepackage{esint}


% RAGNAR SIGURÐSSON
%\usepackage[T1]{fontenc} 
%\usepackage[icelandic]{babel}
\usepackage{latexsym,amssymb,amsmath}
%\usepackage[utf8]{inputenc}
%\usepackage{graphicx}
\usepackage{epstopdf}
\usepackage{verbatim}
\usepackage{array,tabularx,arydshln}
\setbeamertemplate{theorems}[numbered]


\newtheorem{setning}{Setning}
\newtheorem{hjalpar}{Hjálparsetning}
\theoremstyle{definition}
\newtheorem{rithattur}{Ritháttur}
\newtheorem{skilgreining}{Skilgreining}
\newtheorem{daemi}{Dæmi}
\newtheorem{ath}{Athugasemd}

\newcommand\Wider[2][3em]{%
\makebox[\linewidth][c]{%
  \begin{minipage}{\dimexpr\textwidth+#1\relax}
  \raggedright#2
  \end{minipage}%
  }%
}

%counter used for blocks
\newcounter{rtaskno}
\DeclareRobustCommand{\rtask}[1]{%
   \refstepcounter{rtaskno}%
   \kaflanr.\thertaskno\label{#1}}

\newcommand{\C}{{\mathbb  C}}
\newcommand{\Z}{{\mathbb Z}}
\newcommand{\R}{{\mathbb  R}}
\newcommand{\N}{{\mathbb  N}}
\newcommand{\Q}{{\mathbb Q}}
\renewcommand{\phi}{\varphi}
\renewcommand{\epsilon}{\varepsilon}
\newcommand{\p}{{\partial}}
\renewcommand{\d}{{\partial}}

% RAGNAR SIGURÐSSON
\newcommand{\nin}{\mbox{$\;\not\in\;$}}
\newcommand{\dive}{\mbox{${\rm\bf div\,}$}}
\newcommand{\curl}{\mbox{${\rm\bf curl\,}$}}
\newcommand{\grad}{\mbox{${\rm\bf grad\,}$}}
\newcommand{\spann}{\mbox{${\rm Span}$}}
\newcommand{\tr}{\mbox{${\rm tr}$}}
\newcommand{\rank}{\mbox{${\rm rank}$}}
\newcommand{\image}{\mbox{${\rm image}$}}
\newcommand{\nullity}{\mbox{${\rm null}$}}
\newcommand{\proj}{\mbox{${\rm proj}$}}
\newcommand{\id}{\mbox{${\rm id}$}}
%\newcommand{\R}{\mbox{${\bf R}$}}
%\newcommand{\C}{\mbox{${\bf C}$}}
\newcommand{\Rn}{\mbox{${\bf R}^n$}}
\newcommand{\Rm}{\mbox{${\bf R}^m$}}
\newcommand{\Rk}{\mbox{${\bf R}^k$}}
\newcommand{\Av}{\mbox{${\bf A}$}}
\newcommand{\av}{\mbox{${\bf a}$}}
\newcommand{\uv}{\mbox{${\bf u}$}}
\newcommand{\vv}{\mbox{${\bf v}$}}
\newcommand{\wv}{\mbox{${\bf w}$}}
\newcommand{\xv}{\mbox{${\bf x}$}}
\newcommand{\zv}{\mbox{${\bf z}$}}
\newcommand{\yv}{\mbox{${\bf y}$}}
\newcommand{\bv}{\mbox{${\bf b}$}}
\newcommand{\cv}{\mbox{${\bf c}$}}
\newcommand{\dv}{\mbox{${\bf d}$}}
\newcommand{\ev}{\mbox{${\bf e}$}}
\newcommand{\fv}{\mbox{${\bf f}$}}
\newcommand{\gv}{\mbox{${\bf g}$}}
\newcommand{\hv}{\mbox{${\bf h}$}}
\newcommand{\iv}{\mbox{${\bf i}$}}
\newcommand{\jv}{\mbox{${\bf j}$}}
\newcommand{\kv}{\mbox{${\bf k}$}}
\newcommand{\pv}{\mbox{${\bf p}$}}
\newcommand{\nv}{\mbox{${\bf n}$}}
\newcommand{\qv}{\mbox{${\bf q}$}}
\newcommand{\rv}{\mbox{${\bf r}$}}
\newcommand{\sv}{\mbox{${\bf s}$}}
\newcommand{\tv}{\mbox{${\bf t}$}}
\newcommand{\ov}{\mbox{${\bf 0}$}}
\newcommand{\Fv}{\mbox{${\bf F}$}}
\newcommand{\Gv}{\mbox{${\bf G}$}}
\newcommand{\Uv}{\mbox{${\bf U}$}}
\newcommand{\Nv}{\mbox{${\bf N}$}}
\newcommand{\Hv}{\mbox{${\bf H}$}}
\newcommand{\Ev}{\mbox{${\bf E}$}}
\newcommand{\Sv}{\mbox{${\bf S}$}}
\newcommand{\Tv}{\mbox{${\bf T}$}}
\newcommand{\Bv}{\mbox{${\bf B}$}}
\newcommand{\Oa}{\mbox{$(0,0)$}}
\newcommand{\Ob}{\mbox{$(0,0,0)$}}
\newcommand{\Onv}{\mbox{$[0,0,\ldots,0]$}}
\newcommand{\an}{\mbox{$(a_1,a_2, \ldots,a_n)$}}
\newcommand{\xn}{\mbox{$(x_1,x_2, \ldots,x_n)$}}
\newcommand{\xnv}{\mbox{$[x_1,x_2, \ldots,x_n]$}}
\newcommand{\vnv}{\mbox{$[v_1,v_2, \ldots,v_n]$}}
\newcommand{\wnv}{\mbox{$[w_1,w_2, \ldots,w_n]$}}
\newcommand{\tvint}{\int\!\!\!\int}
\newcommand{\thrint}{\int\!\!\!\int\!\!\!\int}
\renewcommand{\ast}{{\operatorname{\text{astand}}}}


\usepackage{caption}
%\usepackage{pgfpages}
% \pgfpagesuselayout{2 on 1}[a4paper,border shrink=5mm]

\def\lecturename{Stærðfræðigreining IIB}
\title{\insertlecture}
\author{Sigurður Örn Stefánsson, \href{mailto:sigurdur@hi.is}{sigurdur@hi.is}}
\institute
{
  Verkfræði- og náttúruvísindasvið\\
  Háskóli Íslands
}
\subtitle{Stærðfræðigreining IIB, STÆ205G}
%\subject{\lecturename}

\mode<article>
{
	\usepackage[colorlinks=false,
	pdfauthor={Sigurður Örn Stefánson},
	%pdftitle={Töluleg greining}
	]{hyperref}
  %\usepackage{times}
  %\usepackage{mathptmx}
  \usepackage[left=1.5cm,right=4cm,top=1.5cm,bottom=3cm]{geometry}
}

% Beamer version theme settings

%\useoutertheme[height=0pt,width=2cm,right]{sidebar}
%\usecolortheme{rose,sidebartab}
%\useinnertheme{circles}
%\usefonttheme[only large]{structurebold}


\setbeamercolor{sidebar right}{bg=black!15}
\setbeamercolor{structure}{fg=blue}
\setbeamercolor{author}{parent=structure}

\setbeamerfont{title}{series=\normalfont,size=\LARGE}
\setbeamerfont{title in sidebar}{series=\bfseries}
\setbeamerfont{author in sidebar}{series=\bfseries}
\setbeamerfont*{item}{series=}
\setbeamerfont{frametitle}{size=}
\setbeamerfont{block title}{size=\small}
\setbeamerfont{subtitle}{size=\normalsize,series=\normalfont}


\setbeamertemplate{sidebar right}
{
  {\usebeamerfont{title in sidebar}%
    \vskip1.5em%
    \hskip3pt%
    \usebeamercolor[fg]{title in sidebar}%
    \insertshorttitle[width=2cm-6pt,center,respectlinebreaks]\par%
    \vskip1.25em%
  }%
  {%
    \hskip3pt%
    \usebeamercolor[fg]{author in sidebar}%
    \usebeamerfont{author in sidebar}%
    \insertshortauthor[width=2cm-2pt,center,respectlinebreaks]\par%
    \vskip1.25em%
  }%
  \hbox to2cm{\hss\insertlogo\hss}
  \vskip1.25em%
  \insertverticalnavigation{2cm}%
  \vfill
  \hbox to 2cm{\hfill\usebeamerfont{subsection in
      sidebar}\strut\usebeamercolor[fg]{subsection in
      sidebar}\insertshortlecture.\insertframenumber\hskip5pt}%
  \vskip3pt%
}%

\setbeamertemplate{title page}
{
  \vbox{}
  \vskip1em
  %{\huge Kapitel \insertshortlecture\par}
  {\usebeamercolor[fg]{title}\usebeamerfont{title}\inserttitle\par}%
  \ifx\insertsubtitle\@empty%
  \else%
    \vskip0.25em%
    {\usebeamerfont{subtitle}\usebeamercolor[fg]{subtitle}\insertsubtitle\par}%
  \fi%     
  \vskip1em\par
  %Vorlesung \emph{\lecturename}\ vom 
  \insertdate\par
  \vskip0pt plus1filll
  \leftskip=0pt plus1fill\insertauthor\par
  \insertinstitute\vskip1em
}

%\logo{\includegraphics[width=2cm]{beamerexample-lecture-logo.pdf}}



% Article version layout settings

\mode<article>

\makeatletter
\def\@listI{\leftmargin\leftmargini
  \parsep 0pt
  \topsep 5\p@   \@plus3\p@ \@minus5\p@
  \itemsep0pt}
\let\@listi=\@listI


\setbeamertemplate{frametitle}{\paragraph*{\insertframetitle\
    \ \small\insertframesubtitle}\ \par
}
\setbeamertemplate{frame end}{%
  \marginpar{\scriptsize\hbox to 1cm{\sffamily%
      \hfill\strut\insertshortlecture.\insertframenumber}\hrule height .2pt}}
\setlength{\marginparwidth}{1cm}
\setlength{\marginparsep}{1.5cm}

\def\@maketitle{\makechapter}

\def\makechapter{
  \newpage
  \null
  \vskip 2em%
  {%
    \parindent=0pt
    \raggedright
    \sffamily
    \vskip8pt
    %{\fontsize{36pt}{36pt}\selectfont Kapitel \insertshortlecture \par\vskip2pt}
    {\fontsize{24pt}{28pt}\selectfont \color{blue!50!black} \insertlecture\par\vskip4pt}
    {\Large\selectfont \color{blue!50!black} \insertsubtitle, \@date\par}
    \vskip10pt

    \normalsize\selectfont \@author\par\vskip1.5em
    %\hfill BLABLA
  }
  \par
  \vskip 1.5em%
}

\let\origstartsection=\@startsection
\def\@startsection#1#2#3#4#5#6{%
  \origstartsection{#1}{#2}{#3}{#4}{#5}{#6\normalfont\sffamily\color{blue!50!black}\selectfont}}

\makeatother

\mode
<all>



% Typesetting Listings

\usepackage{listings}
\lstset{language=Java}

\alt<presentation>
{\lstset{%
  basicstyle=\footnotesize\ttfamily,
  commentstyle=\slshape\color{green!50!black},
  keywordstyle=\bfseries\color{blue!50!black},
  identifierstyle=\color{blue},
  stringstyle=\color{orange},
  escapechar=\#,
  emphstyle=\color{red}}
}
{
  \lstset{%
    basicstyle=\ttfamily,
    keywordstyle=\bfseries,
    commentstyle=\itshape,
    escapechar=\#,
    emphstyle=\bfseries\color{red}
  }
}



% Common theorem-like environments

\theoremstyle{definition}
\newtheorem{exercise}[theorem]{\translate{Exercise}}




% New useful definitions:

\newbox\mytempbox
\newdimen\mytempdimen

\newcommand\includegraphicscopyright[3][]{%
  \leavevmode\vbox{\vskip3pt\raggedright\setbox\mytempbox=\hbox{\includegraphics[#1]{#2}}%
    \mytempdimen=\wd\mytempbox\box\mytempbox\par\vskip1pt%
    \fontsize{3}{3.5}\selectfont{\color{black!25}{\vbox{\hsize=\mytempdimen#3}}}\vskip3pt%
}}

\newenvironment{colortabular}[1]{\medskip\rowcolors[]{1}{blue!20}{blue!10}\tabular{#1}\rowcolor{blue!40}}{\endtabular\medskip}

\def\equad{\leavevmode\hbox{}\quad}

\newenvironment{greencolortabular}[1]
{\medskip\rowcolors[]{1}{green!50!black!20}{green!50!black!10}%
  \tabular{#1}\rowcolor{green!50!black!40}}%
{\endtabular\medskip}





\newcommand{\C}{{\mathbb  C}}
\newcommand{\Z}{{\mathbb Z}}
\newcommand{\R}{{\mathbb  R}}
\newcommand{\N}{{\mathbb  N}}
\newcommand{\Q}{{\mathbb Q}}
\begin{document}


%\subsection{Yfirlit}
\section{Upphafs- og jaðargildisverkefni fyrir breiðgerar (e.~hyperbolic)  hlutafleiðujöfnur}
%\begin{center}
%\begin{tabular}{|l|l|l|l|}\hline
%Kafli &Heiti á viðfangsefni &Bls. & Glærur\\
%\hline
%10.0 &Almenn atriði og nokkur dæmi& 883-886\\
%10.1 &Kennilínur & 886-887\\
%10.2 &Flutningsjafna --  Uppstreymismismunaaðf.& 888-899\\
%10.3 &Flutningsjafna --  MacCormack-aðferð& 902-911\\
%10.4 &Flutnings- og dreifnijafna & 914-921\\ \hline
%\end{tabular}
%\end{center}




\subsection{Flutningur, dreifni og bylgjur} 
Við ætlum nú að fjalla um jöfnur úr flokki {\it breiðgerra}
(e.~hyperbolic) hlutafleiðujafna.  
Við fylgjum greinum 11.1-3 í kennslubókinni í umfjöllun okkar um þetta
efni. 

%\pause
\smallskip
Við ætlum ekki að skilgreina flokk breiðgerra hlutafleiðujafna, 
en í honum eru jöfnur sem lýsa flutningi efna í lofti og vökva
og bylgjujöfnur.   

%\pause
\smallskip
Þessar jöfnur eru almennt settar fram 
 fyrir allar rúmvíddir, en við munum einungis
líta á jöfnur í einni rúmvídd.  

%\pause
\smallskip
Nokkur mikilvæg dæmi um fyrirbæri sem lýst er með jöfnum með einni rúmvídd :
\begin{enumerate}
\item[(i)]   Efni rennur út í á og berst með straumnum.
\item[(ii)]  Efni smitast í grunnvatn og  flyst með grunnvatnsstraumi.
\item[(iii)] Reykjarstrók slær niður og hann berst undan vindi.
\item[(iv)]  Úthafsalda fellur að langri strönd.
\item[(v)]   Sveiflur í fiðlustreng
\item[(vi)]  Hljóðbylgjur í orgelpípu
\end{enumerate}



\subsubsection{Fróðleikur á Wikipedia} 
Það er til fullt af útleiðslum á Wikipedia á einstökum jöfnum og þar
eru einnig  niðustöður úteikninga settar fram með hreyfimyndum.

  \begin{enumerate}
  \item[(i)] Bygjujafnan fyrir streng (e.~vibrating string)
  \item[(ii)] Flutnings-og-dreifnijafna (e.~Convection–diffusion
    equation) 
  \item[(iii)] Bylgjujafna á grunnu vatni
(e.~shallow water equations).
  \item[(iv)]  Flóðbylgja við strönd (e.~tsunami).  Skoðið vel
    hreyfimyndirnar neðst í greininni. 
  \end{enumerate}



\subsubsection{10.0 Nokkur dæmi} 
{\it Flutningsjafna:}
$$
\dfrac{\partial u}{\partial t}
+a(x,t,u)\dfrac{\partial u}{\partial x}=g(x,t,u).
$$
%\pause
{\it Flutnings-og-dreifnijafna:}
$$
\dfrac{\partial u}{\partial t}
+v(x,t)\dfrac{\partial u}{\partial x}=D
\dfrac{\partial^2 u}{\partial x^2}.
$$
%\pause
{\it Bylgjujafna:}
$$
\dfrac{\partial^2 u}{\partial t^2}
-c^2\dfrac{\partial^2 u}{\partial x^2}=0.
$$

Í öllum þessum jöfnum er $u=u(x,t)$ fall af tveimur breytistærðum 
þar sem við hugsum okkur að $t$ standi fyrir tíma.



\subsection{Kennilínur (e.~characteristics)} 
Lítum á einföldu flutningsjöfnuna 
$$
\dfrac{\partial u}{\partial t}
+a(x,t)\dfrac{\partial u}{\partial x}=g(x,t).
$$
sem hefur lausnina $u=u(x,t)$.  

\smallskip 
Takið eftir að við gerum ráð fyrir að föllin $a$ og $g$ séu 
einungis háð  $(x,t)$ en ekki $u$. 

%\pause
\smallskip
Nú festum við einn punkt $(x,t)$ og 
lítum nú á lausn $\xi(\tau)$ á upphafsgildisverkefninu
$$
  \dfrac{d\xi}{d\tau}=a(\xi,\tau), \qquad \xi(t)=x.
$$
Þessi jafna hefur ótvírætt ákvarðaða lausn og hún er skilgreind 
á opnu bili sem inniheldur $t$.  Við höfum þar
$$
\dfrac{d\xi}{d\tau}(\tau)= a(\xi(\tau),\tau), \qquad \tau \in I,
\quad \text{ og } \quad \xi(t)=x.
$$



%\subsection{10.1 Kennilínur} 
Ferillinn  $\tau\mapsto (\xi(\tau), \tau)$ með $\xi(t)=x$ nefnist
{\it kennilína} hlutafleiðujöfnunnar
$$
\dfrac{\partial u}{\partial t}
+a(x,t)\dfrac{\partial u}{\partial x}=g(x,t).
$$
Ef við  notfærum við okkur að fallið $u(x,t)$ uppfyllir
þessa jöfnu, þá gefur keðjuregla 
\begin{align*}
\dfrac {d}{d\tau} u(\xi(\tau),\tau)&= 
\dfrac{\partial u}{\partial x}(\xi(\tau),\tau) 
\dfrac{d\xi}{d\tau}(\tau)+
\dfrac{\partial u}{\partial t}(\xi(\tau),\tau)\\
&=\dfrac{\partial u}{\partial t}(\xi(\tau),\tau)
+a(\xi(\tau),\tau)
\dfrac{\partial u}{\partial x}(\xi(\tau),\tau) \\
&=g(\xi(\tau),\tau)  
\end{align*}



%\subsection{10.1 Kennilínur} 
Ef bilið $I$ inniheldur punktinn $\tau=0$, þá getum við heildað
síðustu jöfnu yfir bilið $[0,t]$ og þá fáum við 
$$
u(x,t)=u(\xi(0),0)+\int_0^tg(\xi(\tau),\tau)\, d\tau.
$$
Þessi jafna segir okkur að gildi lausnarinnar $u$ í punktinum $(x,t)$
ræðst af gildi hennar í skurðpunkti kennilínunnar við $x$-ásinn, sem
er sama og línan $t=0$, og af heildi fallsins  $g$ yfir kennilínuna
frá punktinum $(\xi(0),0)$ til punktsins $(x,t)$.



\subsubsection{Almenna tilfellið} 
Lítum aftur á einföldu flutningsjöfnuna 
$$
\dfrac{\partial u}{\partial t}
+a(x,t,u)\dfrac{\partial u}{\partial x}=g(x,t,u).
$$
sem hefur lausnina $u=u(x,t)$.  Nú gerum við ráð fyrir 
að föllin $a$ og $g$ séu háð öllum þremur breytunum 
$(x,t,u)$. 

%\pause
\smallskip
Nú festum við einn punkt $(x,t,u(x,t))$ á grafi fallsins 
$u$ og lítum á lausnina á upphafsgildisverkefninu
$$
  \dfrac{d\xi}{d\tau}=a(\xi,\tau,u(\xi(\tau),\tau)), \qquad \xi(t)=x.
$$
Þessi jafna hefur ótvírætt ákvarðaða lausn og hún er skilgreind 
á opnu bili sem inniheldur $t$.  



%\subsection{10.1 Kennilínur} 
Setjum nú $v(\tau)=u(\xi(\tau),\tau)$ og munum að
fallið $u$ er lausn hlutafleiðujöfnunnar
$$
\dfrac{\partial u}{\partial t}
+a(x,t,u)\dfrac{\partial u}{\partial x}=g(x,t,u).
$$
Nú fáum við með sama hætti og áður að
\begin{align*}
\dfrac{dv}{d\tau}&=\dfrac {d}{d\tau} u(\xi(\tau),\tau)= 
\dfrac{\partial u}{\partial x}(\xi(\tau),\tau) 
\dfrac{d\xi}{d\tau}(\tau)+
\dfrac{\partial u}{\partial t}(\xi(\tau),\tau)\\
&=\dfrac{\partial u}{\partial t}(\xi(\tau),\tau)
+a(\xi(\tau),\tau,v(\tau))
\dfrac{\partial u}{\partial x}(\xi(\tau),\tau) \\
&=g(\xi(\tau),\tau,v(\tau))  
\end{align*}



%\subsection{10.1 Kennilínur} 
Ef bilið $I$ inniheldur punktinn $\tau=0$, þá getum við heildað
síðustu jöfnu yfir bilið $[0,t]$ og þá fáum við 
$$
u(x,t)=u(\xi(0),0)+\int_0^tg(\xi(\tau),\tau,v(\tau))\, d\tau.
$$
Ferillinn $\tau\mapsto (\xi(\tau),\tau,v(\tau))$ þar sem 
$$
\begin{cases}
  \dfrac{d\xi}{d\tau}=a(\xi(\tau),\tau,v(\tau)),&\qquad \xi(t)=x,\\
\dfrac{dv}{d\tau}=g(\xi(\tau),\tau,v(\tau)),& \qquad v(t)=u(x,t),
\end{cases}
$$
nefnist kennlína afleiðujöfnunnar gegnum punktinn $(x,t,u(x,t))$ 

\smallskip
Kennilínan liggur í grafi lausnarinnar $u$ og gildi hennar í punktinum 
$(x,t)$ ræðst af gildi hennar í skurðpunkti kennilínunnar við planið
$t=0$ í $(x,t,u)$-hnitarúminu, en það er punkturinn
$(\xi(0),0,u(\xi(0),0))$, og heildi $g$ yfir kennilínuna
frá þessum skurðpunkti til $(x,t,u(x,t))$.



\subsection{Flutningsjafna -- Skipting á $xt$-planinu} 
Við innleiðum jafna skiptingu á $xt$-planinu með því að taka
viðmiðunarpunkt  $a\in \R$, látum $\Delta x$ tákna billengdina í
$x$-skiptingunni og $\Delta t$ billengdina í  $t$-skiptingunni.
Skiptinpunktarnir eru $x_j=a+j\, \Delta x$ og $t_n=n\, \Delta t$
Tölurnar $j$ og $n$ mega vera hvaða heiltölur sem er.

%\pause
\smallskip
Við ætlum að nálga lausnir $u(x,t)$ á flutnings-og-dreifnijöfnum 
og bylgjujöfnum í punktunum $(x_j,t_n)$.   Við innleiðum því 
nálgunargildin
\begin{multline*}
w_j^{(n)}\approx u(x_j,t_n)=u_j^{(n)}, \quad
a_j^{(n)}=a(x_j,t_n,w_j^{(n)}), \quad \\
g_j^{(n)}=g(x_j,t_n,w_j^{(n)}), o.s.frv. 
\end{multline*}



\subsubsection{Mismunakvótar fyrir flutningsjöfnu} 
Lítum nú aftur á flutningsjöfnuna
$$
\dfrac{\partial u}{\partial t}
+a(x,t,u)\dfrac{\partial u}{\partial x}=g(x,t,u).
$$

%\pause
\smallskip
Hún jafngildir því að í sérhverjum punkti $(x,t)$ gildir
\begin{multline*}
\dfrac{u(x,t+\Delta t)-u(x,t)}{\Delta t} +O(\Delta  t)\\
+a(x,t,u)\dfrac{u(x,t)-u(x-\Delta x,t)}{\Delta x}+O(\Delta x)
=g(x,t,u) 
\end{multline*}
ef við notum bakmismunakvóta til þess að reikna út $\partial
u/\partial x$ og ...



%\subsection{10.2 Mismunakvótar fyrir flutningsjöfnu} 
\begin{multline*}
\dfrac{u(x,t+\Delta t)-u(x,t)}{\Delta t} +O(\Delta  t)\\
+a(x,t,u)\dfrac{u(x+\Delta x,t)-u(x,t)}{\Delta x}+O(\Delta x)
=g(x,t,u) 
\end{multline*}
ef við notum frammismunakvóta til þess að reikna út $\partial
u/\partial x$.

%\pause
\smallskip
Ef við setjum $(x,t)=(x_j,t_n)$, þá jafngildir fyrri jafnan 
$$
u_j^{(n+1)}-u_j^{(n)}+O((\Delta t)^2)
+a_j^{(n)}\dfrac{\Delta t}{\Delta
  x}\big(u_j^{(n)}-u_{j-1}^{(n)}\big)+\Delta tO(\Delta x)=\Delta
t\, g_j^{(n)}
$$
%\pause
Nú skilgreinum við $\lambda=a_j^{(n)}{\Delta t}/{\Delta x}$ og
innleiðum nálgunarjöfnu með því að setja $w_j^{(n)}$ í stað
$u_j^{(n)}$ og fella niður leifarliðinn.



\subsubsection{Nálgunarjafna fyrir flutningsjöfnu} 
$$
w_j^{(n+1)}-w_j^{(n)}
+\lambda \big(w_j^{(n)}-w_{j-1}^{(n)}\big)=\Delta t\, g_j^{(n)}
$$
%\pause
Það er eðlilegra að setja þetta mismunasamnband fram sem 
rakningu fram í tíma, frá $t_n$ til $t_{n+1}=t_n+\Delta t$,
$$
w_j^{(n+1)}=(1-\lambda) w_j^{(n)}+\lambda w_{j-1}^{(n)}+\Delta
t\, g_j^{(n)}. 
$$
Ef við notum frammismunakvóta til þess að reikna út 
$\partial u/\partial x$ í stað bakmismunakvóta, þá fáum við í 
staðinn rakningarformúluna
$$
w_j^{(n+1)}=(1+\lambda) w_j^{(n)}-\lambda w_{j+1}^{(n)}+\Delta
t\, g_j^{(n)}. 
$$



\subsubsection{Uppstreymismismunaaðferð (e.~upwind finite difference)} 
Reikniritið sem út úr þessu kemur er
$$
w_j^{(n+1)}=
\begin{cases} 
(1-\lambda) w_j^{(n)}+\lambda w_{j-1}^{(n)}+\Delta
t\, g_j^{(n)},&a_j^{(n)}>0,\\
(1+\lambda) w_j^{(n)}-\lambda w_{j+1}^{(n)}+\Delta
t\, g_j^{(n)},&a_j^{(n)}<0.
\end{cases}
$$
%\pause
\smallskip
Valið milli $a_j^{(n)}>0$ og  $a_j^{(n)}<0$ er útskýrt út frá
kennilínunum.  

\smallskip
Ef $a_j^{(n)}>0$, þá fer kennilínan gegnum $(x,t+\Delta t)$ 
frá vinstri til hægri í $xt$-planinu.  Ef $\xi$ er stikun á henni, þá
fáum við
$$
u(x,t+\Delta t)=u(\xi(t),t)+\int_t^{t+\Delta t}
g(\xi(\tau),\tau,v(\tau))\, d\tau.
$$
Punkturinn $\xi(t)$ er vinstra megin við $x$ og því er eðlilegt að
taka bakmismun í þessu tilfelli.  Ef $a_j^{(n)}<0$ þá væri $\xi(t)$
hægra megin við $x$ og þá væri eðlilegra að taka frammismun. 



\paragraph{1.~tilbrigði} 
Lítum nú aftur á flutningsjöfnuna
$$
\dfrac{\partial u}{\partial t}
+a(x,t,u)\dfrac{\partial u}{\partial x}=g(x,t,u).
$$

%\pause
Afbrigði af þessari jöfnu er þegar $a=a(x,t)$ 
er óháð $u$ og við höfum 
$$
\dfrac{\partial u}{\partial t}
+\dfrac{\partial (au)}{\partial x}=g(x,t,u).
$$

%\pause
Reikniritið sem við fáum í þessu tilfelli er
$$
w_j^{(n+1)}=
\begin{cases} 
w_j^{(n)}+\dfrac{\Delta t}{\Delta x} 
\big(a_{j-1}^{(n)}w_{j-1}^{(n)}-a_j^{(n)}w_j^{(n)}\big)
+\Delta t\, g_j^{(n)},&a_j^{(n)}>0,\\
w_j^{(n)}+\dfrac{\Delta t}{\Delta x} 
\big(a_{j}^{(n)}w_{j}^{(n)}-a_{j+1}^{(n)}w_{j+1}^{(n)}\big)
+\Delta t\, g_j^{(n)}, &a_j^{(n)}<0.
\end{cases}
$$



\paragraph{2.~afbrigði } 
Sum varðveislulögmál eru sett fram með jöfnu
$$
\dfrac{\partial u}{\partial t}
+\dfrac{\partial f(u)}{\partial x}=g(x,t,u).
$$

%\pause
Reikniritið sem við fáum í þessu tilfelli er
$$
w_j^{(n+1)}=
\begin{cases} 
w_j^{(n)}+\dfrac{\Delta t}{\Delta x} 
\big(f(w_{j-1}^{(n)})-f(w_j^{(n)})\big)
+\Delta t\, g_j^{(n)},& \dfrac{\partial f}{\partial u}(w_j^{n})>0,\\
w_j^{(n)}+\dfrac{\Delta t}{\Delta x} 
\big(f(w_{j}^{(n)})-f(w_{j+1}^{(n)})\big)
+\Delta t\, g_j^{(n)}, &\dfrac{\partial f}{\partial u}(w_j^{n})<0.
\end{cases}
$$



\subsection{Flutningsjafna -- MacCormack-aðferð} 
Kostur við uppstreymiaðferðir er að þær byggja á kennilínuaðferðinni,
en hún er notuð  til þess að sýna fræðilega fram á að afleiðujafnan
hafi lausn.

\smallskip
Ókostur er að hún er ekki nákvæm, aðeins með fyrsta stigs
staðarskekkju, og að erfitt er að fá hana til þess að virka fyrir
hneppi.

%\pause
Við lítum aftur á fyrstu flutningsjöfnuna okkar
$$
\dfrac{\partial u}{\partial t}
+a(x,t,u)\dfrac{\partial u}{\partial x}=g(x,t,u).
$$

%\pause
Við ætlum nú að reikna út nálgunargildið $w_j^{n+1}$ í tveimur skrefum
{\it forsagnarskrefi} og {\it leiðréttingarskrefi}.  Við táknum gildið
sem út kemur eftir forsagnarskrefið með $w_j^*$.



\subsubsection{Forsagnarskref} 
$$
\dfrac{w_j^*-w_j^{(n)}}{\Delta t} 
+a_j^{(n)}\dfrac{w_{j+1}^{(n)}-w_j^{(n)}}{\Delta x}=g_j^{(n)}
$$
sem gefur okkur
$$
w_j^*=(1+\lambda) w_j^{(n)}-\lambda w_{j+1}^{(n)}+\Delta t\, g_j^{(n)},
$$
þar sem $\lambda=a_j^{(n)}\Delta t/\Delta x$.



\subsubsection{Leiðréttingarskref} 
$$
\dfrac{w_j^{(n+1)}-w_j^{(n)}}{\Delta t}
+\dfrac 12\bigg(a_j^{(n)}\dfrac{w_{j+1}^{(n)}-w_j^{(n)}}{\Delta x}
+a_j^*\dfrac{w_j^*-w_{j-1}^*}{\Delta x}\bigg)
=\dfrac  12\big(g_j^{(n)}+g_j^*\big)
$$
þar sem  $a_j^*$ er gildi fallsins $a(x,t,u)$ í punktinum
$(x_j,t_n,w_j^*)$ og $g_j^*$ er skilgreint með hliðstæðum hætti.

%\pause
\smallskip
Fyrri mismunakvótinn innan stóru sviganna kom fyrir í
forsagnarskrefinu.  Með því að nýta okkur það fáum við með smá
reikningum að 
$$
w_j^{(n+1)}=\tfrac 12
\big(w_j^{(n)}+w_j^*-\lambda^*(w_j^*-w_{j-1}^*)\big)
+\tfrac 12 \Delta t\, g_j^*,
$$
þar sem $\lambda^*=a_j^*\Delta t/\Delta x$.



\subsubsection{1.~tilbrigði} 
Við getum sett upp sams konar mismunakvóta fyrir 
$$
\dfrac{\partial u}{\partial t}
+\dfrac{\partial (au)}{\partial x}=g(x,t,u).
$$
og leyfum nú fallinu $a=a(x,t,u)$

%\pause
\smallskip
{\it Forsagnarskref:} 
$$
w_j^*=w_j^{(n)}-\dfrac{\Delta t}{\Delta x}\big( 
a_{j+1}^{(n)}w_{j+1}^{(n)}-a_j^{(n)}w_j^{(n)}\big)+\Delta t\, g_j^{(n)}.
$$

%\pause
\smallskip
{\it Leiðréttingarskref:}
$$
w_j^{(n+1)}=\tfrac 12\big(w_j^{(n)} +w_j^*-\dfrac{\Delta t}{\Delta x}
\big(a_j^*w_j^*-a_{j-1}^*w_{j-1}^*\big) \big)+\tfrac 12 \Delta t\, g_j^*
$$



\subsubsection{2.~tilbrigði} 
Við getum sett upp sams konar mismunakvóta fyrir 
$$
\dfrac{\partial u}{\partial t}
+\dfrac{\partial f(x,t,u)}{\partial x}=g(x,t,u).
$$

%\pause
\smallskip
{\it Forsagnarskref:} 
$$
w_j^*=w_j^{(n)}-\dfrac{\Delta t}{\Delta x}\big( 
f_{j+1}^{(n)}-f_j^{(n)}\big)+\Delta t\, g_j^{(n)}.
$$

%\pause
\smallskip
{\it Leiðréttingarskref:}
$$
w_j^{(n+1)}=\tfrac 12\big(w_j^{(n)} +w_j^*-\dfrac{\Delta t}{\Delta x}
\big(f_j^*-f_{j-1}^*\big) \big)+\tfrac 12 \Delta t\, g_j^*
$$



\subsubsection{MacCormack-aðferð fyrir hneppi} 
Ef hneppið er sett fram með formúlunni
$$
\dfrac{\partial \Uv}{\partial t}
+\dfrac{\partial \Fv(x,t,\Uv)}{\partial x}=\Gv(x,t,\Uv).
$$
þar sem $\Uv=[u_1,\dots,u_m]$, $\Fv=[f_1,\dots,f_m]$,
$\Gv=[g_1,\dots,g_m]$,  með $u_k=u_k(x,t)$
$f_k=f_k(x,t,\Uv)$ og $g_k=g_k(x,t,\Uv)$,
og deildað er hnit fyrir hnit, þá fæst:

%\pause
\smallskip
{\it Forsagnarskref:} 
$$
\wv_j^*=\wv_j^{(n)}-\dfrac{\Delta t}{\Delta x}\big( 
\Fv_{j+1}^{(n)}-\Fv_j^{(n)}\big)+\Delta t\, \Gv_j^{(n)}.
$$

%\pause
\smallskip
{\it Leiðréttingarskref:}
$$
\wv_j^{(n+1)}=\tfrac 12\big(\wv_j^{(n)} +\wv_j^*-\dfrac{\Delta t}{\Delta x}
\big(\Fv_j^*-\Fv_{j-1}^*\big) \big)+\tfrac 12 \Delta t\, \Gv_j^*
$$



\subsection{Kafli 10: Fræðilegar spurningar}
\begin{enumerate}
  \item Hvernig eru {\it flutningsjafna}, {\it flutnings- og
      dreifnijafna} og {\it bylgjujafna} í einni rúmvídd?
  \item Hvernig er kennilína fyrir flutningsjöfnu gegnum punktinn
    $(x,t)$ fundin? 
  \item Hvernig er lausnarformúla fyrir einfalda flutningsjöfnu fundin
    út frá gefninni kennilínu?
  \item Hvernig er mismunajafna fyrir flutningsjöfnu leidd út  ef
    rúmafleiðan er nálguð með bakmismunakvóta?
  \item Hvernig er uppstreymismismunaaðferð fyrir einfalda
    flutningsjöfnu? 
  \item Hvernig er MacCormack-aðferð fyrir einfalda flutningsjöfnu
$\dfrac{\partial u}{\partial t}+a(x,t,u)\dfrac{\partial u}{\partial x}=g(x,t,u)$?
\end{enumerate}

\end{document}
