\lecture[2]{Vika 2: Núllstöðvar; helmingunaraðferð,
fastapunktsaðferð, sniðilsaðferð og aðferð Newtons}{lecture-text}
\date{14.~og 16.~janúar 2015}

\begin{document}

\begin{frame}
	\maketitle
\end{frame}

\section*{}
\begin{frame}{Yfirlit}
\begin{block}{Vika 2: Núllstöðvar}
\begin{center}
\begin{tabular}{|l|l|l|l|}\hline
Nr. &Heiti á viðfangsefni & Bls. & Glærur\\
\hline
2.1 & Helmingunaraðferð& 58-68 & 3-7\\
2.3 & Fastapunktsaðferð& 81-93 & 8-15\\
2.5 & Sniðilsaðferð& 107-111 & 16-23\\
2.4 & Newton-aðferð& 95-104 & 24-29\\
2.4 & Samanburður á aðferðunum & & 30\\
2.4 & Matlab-forrit fyrir aðferð Newtons & & 31-35 \\
\hline
\end{tabular}
\end{center}

\end{block}
\end{frame}

\begin{frame}{2.1 Nálgun á núllstöð  $f(x)=0$:} 

\begin{block}{Upprifjun}
Munum að talan  $p\in I$ sögð vera {\it núllstöð} fallsins $f:I\to \R$ ef
\begin{equation*}
	f(p)=0.
\end{equation*}\pause
Milligildissetningin úr stærðfræðigreiningu segir: 
\begin{quote}
	Ef $f$ er samfellt á $[a,b]$ og $y$ er einhver tala á milli $f(a)$ og $f(b)$, þá er til $c$ þannig að $a < c < b$ og $f(c) = y$.
\end{quote}
\end{block} 
\pause

\begin{block}{Afleiðing}
Svo ef við höfum $a$ og $b$ þannig að $a < b$ og þannig að
$f(a)$ og $f(b)$ hafi ólík formerki, þá hefur $f$ núllstöð $p$ á bilinu
$[a,b]$.
\end{block}
\end{frame}

\begin{frame}{2.1 Helmingunaraðferð (e. bisection method):} 
Notum okkur þetta til þess að finna rætur.\pause

\begin{enumerate}
	\item[(1)] Látum $x = \frac12 (a+b)$ vera miðpunkt $[a,b]$.\pause
	\item [(2)]Reiknum $f(x)$, þá geta þrjú tilvik komið upp:
	\vspace{0.55\baselineskip}
	\begin{enumerate}
		\item[(i)] $f(x) = 0$ og leitinni að rót er lokið.\pause
		\item[(ii)] $f(a)$ og $f(x)$ hafa sama formerki, þannig að við
leitum að rót á bilinu $[x,b]$.\pause
		\item[(iii)] $f(x)$ og $f(b)$ hafa sama formerki, þannig að 
við leitum að rót á bilinu $[a,x]$.
	\end{enumerate}
	\vspace{0.85\baselineskip}
\end{enumerate}  \pause

Í tilviki (ii) segir milligildissetningin að $f$ hafi rót á bilinu
$[x,b]$, og í tilviki (iii) er rótin á bilinu $[a,x]$. Þá getum við
farið aftur í skref 1, nema með helmingi minna bil en áður. 
\pause

\smallskip
Með því að ítreka þetta ferli $n$ sinnum fáum við minnkandi runu af
bilum 
$$
[a,b]=[a_1,b_1]\supset [a_2,b_2]\supset \cdots\supset [a_n,b_n].
$$
Billengdin helmingast í hverju skrefi og milligildissetningin segir
okkur að það sé núllstöð á öllum bilunum.
\end{frame}
%
\frame{\frametitle{2.1 Helmingunaraðferð, nánar:} 


%\pause
\medskip
Rununa af bilunum 
$$
[a,b]= [a_1,b_1]\supset \cdots\supset [a_n,b_n]\supset \cdots
$$
skilgreinum við með ítrun og notum til þess rununa $x_n=\tfrac 12(a_n+b_n)$.

\pause
  
\medskip
{\bf Upphafsskref:} Setjum $a_0=a$, $b_0=b$, og $x_0=\tfrac 12(a+b)$.  

\pause
\smallskip
{\bf Ítrekunarskref:}  Gefið er $x_0,\dots,x_n$.  Reiknum $f(x_n)$.
\begin{enumerate}
  \item[(i)] Ef $f(x_n) = 0$, þá er núllstöð fundin og við hættum.
  \item[(ii)] Ef $f(x_n)$ og $f(a_n)$ hafa sama formerki,
þá setjum við \\$a_{n+1}=x_n$,\\ $b_{n+1}=b_n$, og\\
$x_{n+1}=\tfrac12(a_{n+1}+b_{n+1})$ 
\item[(iii)] annars setjum við \\
$a_{n+1}=a_n$,\\ $b_{n+1}=x_n$, og \\
$x_{n+1}=\tfrac12(a_{n+1}+b_{n+1})$. 
	\end{enumerate}
}
%
%
\begin{frame}{2.1 Skekkjumat í helmingunaraðferð:} 

Ef við látum miðpunktinn $p_n=\tfrac 12(a_n+b_n)$ vera nálgunargildi
okkar fyrir núllstöð fallsins $f$ í bilinu $[a_n,b_n]$, þá er 
skekkjan í nálguninni
$$
e_n=p-p_n
$$ 
\pause
og við höfum skekkjumatið
$$
|e_n|\leq  \dfrac{b_n - a_n}{2}\ \pause
= \frac{b_{n-1}-a_{n-1}}{2^2} = \pause \ldots = \dfrac{b_1-a_1}{2^{n}},
$$
\pause
það er
$$
|e_n| < \dfrac{b-a}{2^{n}}.
$$
\end{frame}
%
\begin{frame}{2.1. Fyrirframmat á skekkju} 

Nú er auðvelt að meta hversu margar ítrekanir þarf að framkvæma til
þess að nálgunin lendi innan gefinna skekkjumarka.  

\pause
Ef $\varepsilon>0$
er gefið og við viljum að $|e_n|<\varepsilon$, þá dugir að  
\begin{equation*}
	|e_n|\leq \dfrac{b-a}{2^{n}} <\varepsilon.
\end{equation*}\pause
Seinni ójafnan jafngildir því að 
\begin{equation*}
	n>\dfrac{\ln\big((b-a)/\varepsilon\big)}{\ln 2}.
\end{equation*}
\end{frame}


\frame{\frametitle{2.3  Fastapunktsaðferð (e.~fixed point method)} 

\begin{block}{Skilgreining}
Látum $f : [a,b] \to \mathbb R$ vera samfellt fall. Punktur $r \in
[a,b]$ þannig að 
\begin{equation*}
  f(r) = r
\end{equation*}
kallast \em fastapunktur \em fallsins $f$. \pause
\end{block}

\begin{block}{Athugasemd}
Athugum að í fastapunktum skerast graf fallsins 
$y=f(x)$ og línan $y=x$.  \pause 
Verkefnið að ákvarða fastapunkta fallsins
$r$ er því jafngilt því að athuga hvar graf $f$ sker línuna $y=x$.
\end{block}

\pause
\begin{block}{Tengin við núllstöðvar}
Verkefnið að finna fastapunkta fallsins $f(x)$ er jafngilt því að finna
núllstöðvar fallsins $g(x)=f(x)-x$.
\end{block}

}
%
%
\frame{\frametitle{2.3   Fastapunktsaðferð} 


{\bf Upphafsskref:} Valin er tala $x_0\in [a,b]$.

\pause

\medskip
{\bf Ítrekunarskref:}  Ef $x_0,\dots,x_n$ hafa verið valin, þá setjum við
$$
x_{n+1}=f(x_n)
$$


\pause

\begin{block}{Athugasemd}
Til þess að þetta sé vel skilgreind runa, þá verðum við að 
gera ráð fyrir að $f(x)\in [a,b]$ fyrir öll $x\in [a,b]$.
Þetta skilyrði er einnig skrifað
$$
f([a,b])\subset [a,b].
$$
\end{block}

\pause

\begin{block}{Athugasemd}
Ef $f$ er samfellt og runan er samleitin með markgildið $r$, þá er
$$
r=\lim_{n\to \infty}x_{n+1}=\lim_{n\to \infty}f(x_{n})
=f(\lim_{n\to \infty}x_{n})=f(r).
$$ 
Þetta segir okkur að \textbf{ef} við getum séð til þess að runan verði
samleitin, þá er markgildið fastapunktur.
\end{block}
}
%

% TEKIÐ Á TÖFLU
% \frame{\frametitle{2.2   Dæmi um ósamleitna runu:} 
% 
% Skilgreinum $f:[0,1]\to \R$ með  
% $$
% f(x)=3.4 \, x(1-x), \qquad x\in [0,1].
% $$
% Fallið $f$ tekur gildi á bilinu $[0,1]$, graf þess er partur af
% fleygboga
% (parabólu) með topppunkt $(\tfrac 12,\tfrac 14\cdot 3.4)$, 
% $f(0)=f(1)=0$ og því hefur $f$ tvo fastapunkta $r_0=0$ og $r_1\approx
% 0.?$.  
% 
% Ef við veljum $x_0=0.3$ og reiknum út $x_{n+1}=f(x_n)$ þá sést að
% runan hoppar að eilífu milli tveggja gilda sem eru nálægt $0.45$ og 
% $0.84$.
% 
% \smallskip
% Þetta segir okkur að skilyrðið $f([a,b])\subset [a,b]$ dugir ekki eitt
% til þess að runan $x_n$ verði samleitin.
% }
%

%
\frame{\frametitle{2.3 Herping} 

\begin{block}{Skilgreining}
Fall $f:[a,b]\to \R$ er sagt vera {\it herping} ef til er fasti
$\lambda\in [0,1[$ þannig að
$$
|f(x)-f(y)|\leq \lambda|x-y| \qquad \text{ fyrir öll } x,y\in [a,b].
$$
\end{block}

\pause


\begin{block}{Athugasemd}

 Sérhver herping er samfellt fall.
\end{block}

\pause

\begin{block}{Athugasemd}
 Ef $f$ er deildanlegt fall á $]a,b[$, þá gefur
  meðal\-gildis\-setningin okkur til er $\xi$ milli $x$ og $y$ þannig
  að
$$
f(x)-f(y)=f'(\xi)(x-y).
$$
\pause
Ef til er $\lambda\in[0,1[$ þannig að $|f'(x)|\leq \lambda$ fyrir öll
$x\in [a,b]$, þá er greinilegt að $f$ er herping.
\end{block}
}
%

%
\frame{\frametitle{2.3   Fastapunktssetning} 

\begin{block}{Setning}
Látum $f : [a,b] \to [a,b]$ vera herpingu. Þá hefur $f$ nákvæmlega
einn fastapunkt $r$ á bilinu $[a,b]$ og runan $(x_n)$ þar sem 
\begin{align*}
  x_0 &\in [a,b] \quad \text{ getur verið hvaða tala sem er  og } \\
  x_{n+1} &= f(x_n), \quad n \geq 0,
\end{align*}
stefnir á fastapunktinn.
\end{block}

\pause

\begin{block}{}
Sönnunina brjótum við upp í nokkur skref.
\end{block}
}

%
\frame{\frametitle{2.3   Sönnun: 1.~skref, 
herping hefur í mesta lagi einn fastapunkt} 

Sönnum þetta með mótsögn.

\pause
\smallskip

Gerum ráð fyrir að  $r$ og $s$ séu tveir ólíkir fastapunktar á $[a,b]$. \pause
 Þá er
\begin{equation*}
|r - s| = |f(r) - f(s)|
  \leq \lambda |r - s| < |r - s|
\end{equation*}
því $\lambda < 1$.   Þetta fær ekki staðist, þannig að fjöldi fastapunkta
er í mesta lagi einn
}
%

%
\frame{\frametitle{2.3 Sönnun: 2.~skref,
fallið $f$ hefur fastapunkt:} 

Látum $g(x) = f(x) - x$, þá eru núllstöðvar $g$ nákvæmlega fastapunktar $f$. 

\pause 

Þar sem $a \leq f(x) \leq b$ fyrir öll $x \in [a,b]$ er 
\begin{equation*}
  \left\{ \begin{array}{c}
      g(a) = f(a) - a \geq 0 \\
      g(b) = f(b) - b \leq 0
  \end{array} \right.
\end{equation*}
Ef annað hvort $g(a) = 0$ eða $g(b) = 0$ höfum við fundið fastapunkt
fallsins $f$ og við getum hætt. 

\pause

Ef hins vegar $g(a) > 0$ og $g(b) < 0$ þá hefur $g$ ólík formerki
í endapunktum bilsins $[a,b]$ og hefur því núllstöð $r$ á bilinu 
skv.~milligildissetninguninni. \pause
Þá er $r$ jafnframt fastapunktur $f$. 

\pause
\smallskip

Skref 1 og 2 sýna því að fallið $f$ hefur nákvæmlega einn 
fastapunkt á bilinu.
}
%


%
\frame{\frametitle{2.3 Sönnun:  3.~skref, runan $(x_n)$ er samleitin} 


Látum $r$ vera ótvírætt ákvarðaða fastapunktinn á $[a,b]$.

\pause
\smallskip

Við notfærum okkur að $f$ er herping og að $r$ er fastapunktur $f$, þá fæst 
að fyrir sérhvert $k\in \N$ þá er 
$$
|r - x_k| = |f(r) - f(x_{k-1})|  \leq \lambda |r - x_{k-1}| 
$$
\pause 
það er $|r - x_k| \leq \lambda |r - x_{k-1}|$.
\pause

Með því að nota þetta $n$-sinnum þá fæst að
\begin{align*}
  |r - x_n|   &\leq \lambda |r - x_{n-1}| & (k=n)\\
  &\leq \lambda^2 |r - x_{n-2}| & (k=n-1)\\
  &\vdots & \vdots\\
  &\leq \lambda^n |r - x_0| & (k=1).
\end{align*}
\pause
Þar sem $\lambda < 1$ er því
\begin{equation*}
  \lim\limits_{n \to +\infty} |r - x_n|
  \leq \lim\limits_{n \to +\infty} \lambda^n |r - x_0|
  = 0,
\end{equation*}
það er runan $x_n$ stefnir á $r$.
}
%

%
\frame{\frametitle{2.3 Fastapunktsaðferð er að minnsta kosti línulega 
samleitin}

Af skilgreiningunni á rununni $x_n$ leiðir beint að
$$
|e_{n+1}|=|r-x_{n+1}|=|f(r)-f(x_n)|\leq \lambda|r-x_n|=\lambda|e_n|
$$ 
sem segir okkur að fastapunktsaðferð sé að minnsta kosti línulega
samleitin ef $f$ er herping.   
}
%

% \frame{\frametitle{2.2 Samleitni af stigi $k>1$ er möguleg}
% 
% Ef til er $k>1$ þannig að $f\in C^k([a,b])$,
% $$ 
% f'(r)=\cdots=f^{(k-1)}(r)=0 \quad \text{ og } \quad 
% f^{(k)}(r)\neq 0, 
% $$
% þá gefur setning Taylors okkur að til er tala
% $\xi_n$ á milli $r$ og $x_n$ þannig að 
% $$
% f(x_n)=f(r)+\dfrac{f^{(k)}(\xi_n)}{k!}(x_n-r)^k
% $$
% $$
% e_{n+1}=r-x_{n+1}=-(f(x_n)-f(r))=(-1)^k\dfrac{f^{(k)}(\xi_n)}{k!}e_n^k
% $$ 
% Við fáum því að 
% $$
% \lim_{n\to \infty}\dfrac{|e_{n+1}|}{|e_n|^k}= \dfrac{|f^{(k)}(r)|}{k!}
% $$
% }
%

%


\frame{\frametitle{2.5 Sniðilsaðferð} 

Gefið er fallið $f:[a,b]\to \R$.  
Við ætlum að ákvarða núllstöð $f$, þ.e.a.s.
$p\in [a,b]$ þannig að 
$$
f(p)=0.
$$\pause
Rifjum upp að {\it sniðill} við graf $f$ gegnum punktana
$(\alpha,f(\alpha))$ og $(\beta,f(\beta))$ er gefinn með jöfnunni
$$
y=f(\alpha)+f[\alpha,\beta](x-\alpha)
$$
þar sem hallatalan er 
$$
f[\alpha,\beta]=\dfrac{f(\beta)-f(\alpha)}{\beta-\alpha}
=\dfrac{f(\alpha)-f(\beta)}{\alpha -\beta}.
$$


\pause
Sniðillinn sker $x$-ásinn í punkti $s$ þar sem 
$$
0=f(\alpha)+f[\alpha,\beta](s-\alpha) \quad  \text{sem jafngildir því að } \quad
s=\alpha-\dfrac{f(\alpha)}{f[\alpha,\beta]}.
$$
}

%
\frame{\frametitle{2.5  Sniðilsaðferð} 

{\bf Byrjunarskref:}  Giskað er á tvö gildi $x_0$ og $x_1$.   

\pause

\medskip
{\bf Ítrekunarskref:}  Gefin eru  $x_0,\dots,x_n$.  Punkturinn 
$x_{n+1}$ er skurðpunktur sniðilsins gegnum $(x_{n-1},f(x_{n-1}))$ og
$(x_n,f(x_n))$ við $x$-ás,
$$
x_{n+1}=x_n-\dfrac{f(x_n)}{f[x_n,x_{n-1}]}.
$$
}
%

%
\frame{\frametitle{2.5  Samleitin runa stefnir á núllstöð $f$} 


Gefum okkur að runan $(x_n)$ sé samleitin að markgildinu $r$. \pause
Meðalgildissetningin segir okkur þá að til sé punktur $\eta_n$ á milli 
$x_{n-1}$ og $x_n$ þannig að 
$$
f[x_n,x_{n-1}]=f'(\eta_n),
$$ 
\pause
og greinilegt er að $\eta_n\to r$. 

\pause
Við fáum því 
$$
r=\lim_{n\to \infty}x_{n+1}=\lim_{n\to \infty}
\bigg(x_n-\dfrac{f(x_n)}{f'(\eta_n)}\bigg) =r-\dfrac{f(r)}{f'(r)}
$$

\pause
Þessi jafna jafngildir því að $f(r)=0$.
}

%
\frame{\frametitle{2.5  Skekkjumat í  nálgun á $f(x)$ með $p_n(x)$} 

Sniðilinn sem við notum er graf 1.~stigs margliðunnar
\begin{equation*}
	p_n(x) = f(x_n) + 
		\dfrac{f(x_{n-1})-f(x_n)}{x_{n-1}-x_n}(x-x_n)
		= f(x_n) + f[x_n,x_{n-1}](x-x_n)
\end{equation*}\pause
Samkvæmt skilgreiningu er $p_n(x_{n+1}) = 0$ svo $x_{n+1}$ uppfyllir jöfnuna
\begin{equation*}
	x_{n+1} = x_n - \frac{f(x_n)}{f[x_n,x_{n-1}]}.
\end{equation*}\pause
Við þurfum að vita hver skekkjan er á því að nálga $f(x)$ með $p_n(x)$.

\smallskip

\pause
Við munum sýna fram á:  Fyrir sérhvert $x \in [a,b]$ er til $\xi_n$ sem liggur í minnsta bilinu
sem inniheldur  $x$, $x_n$ og $x_{n-1}$ þannig að 
\begin{equation*}
	f(x) - p_n(x) = \frac{1}{2}f''(\xi_n)(x-x_n)(x-x_{n-1})
\end{equation*}
}



%
\frame{\frametitle{2.5  Skekkjumat í sniðilsaðferð}
 
Gefum okkur að þessi staðhæfing sé rétt og skoðum hvað af henni
leiðir:

\pause
\smallskip
Nú er $f(r) = 0$ og því
\begin{equation*}
	-p_n(r) = \frac{1}{2}f''(\xi_n)e_n\cdot e_{n-1}.
\end{equation*}\pause
Eins er 
\begin{equation*}
	-p_n(r) = -f[x_n,x_{n-1}]e_{n+1}=-f'(\eta_n)e_{n+1},
\end{equation*}
þar sem $\eta_n$ fæst úr meðalgildissetningunni og liggur á milli $x_n$ og $x_{n+1}$.
\pause
Niðurstaðan verður því
\begin{equation*}
	e_{n+1} = \frac{-\frac{1}{2}f''(\xi_n)}
		{f[x_n, x_{n+1}]}	
	e_ne_{n-1} = \frac{-\frac{1}{2}f''(\xi_n)}
		{f'(\eta_n)}e_ne_{n-1}
\end{equation*}
}
%


%
\frame{\frametitle{2.5   Sniðilsaðferð er ofurlínuleg} 

það er 
$$
\lim_{n\to \infty}\dfrac{e_{n+1}}{e_ne_{n-1}}=
\lim_{n \to \infty} \frac{-\frac{1}{2}f''(\xi_n)}
		{f'(\eta_n)}
=
\frac{-\frac{1}{2}f''(r)}
		{f'(r)}.
$$
\pause

\begin{block}{Setning}
Ef sniðilsaðferð er samleitin, $f\in C^2([a,b])$ (tvisvar diffranlegt) 
og $f'(r)\neq 0$, þá er sniðilsaðferðin ofurlínuleg.
\end{block}
\pause
\begin{block}{Sönnun}
$$
\lim_{n\to \infty}\dfrac{|e_{n+1}|}{|e_n|} \pause=
\lim_{n\to \infty}\dfrac{|e_{n+1}e_{n-1}|}{|e_ne_{n-1}|}=
\lim_{n \to \infty} \frac{|e_{n-1}\frac{1}{2}f''(r)|}
		{|f'(r)|} = 0
$$ 
\end{block}
\pause
\begin{block}{Athugasemd}
 Nánar tiltekið þá er sniðilsaðferðin samleitin af stigi
  $\alpha = (1+\sqrt 5)/2 \approx 1,618$ og með
  $\lambda = \left(\frac{f''(r)}{2f'(r)}\right)^{\alpha -1}$, sjá 
kennslubók bls.~110.
\end{block}
}

%
\frame{\frametitle{2.5  Skekkjumat í  nálgun á $f(x)$ með $p_n(x)$} 

Við megum ekki gleyma að sanna skekkjumatið.

\pause

\begin{block}{Hjálparsetning}\pause
Til er  $\xi_n$ sem liggur í minnsta bilinu
sem inniheldur  $x$, $x_n$ og $x_{n-1}$ þannig að 
\begin{equation*}
	f(x) - p_n(x) = \frac{1}{2}f''(\xi_n)(x-x_n)(x-x_{n-1})
\end{equation*}
\end{block}

\pause
\begin{block}{Sönnun}
 Ljóst er að matið gildir ef $x=x_{n-1}$ eða $x=x_n$.  

\pause
Festum því punktinn $x$  og gerum ráð fyrir að $x\neq x_1$ og $x\neq x_n$. 

\pause
Skilgreinum fallið
$$
g(t)=f(t)-p_n(t)-\lambda(t-x_n)(t-x_{n-1})
$$
þar sem $\lambda$ er valið þannig að $g(x)=0$.  
\end{block}

}

\frame{\frametitle{ }
%2.3  Skekkjumat í  nálgun á $f(x)$ með $p_n(x)$} 

Látum nú
$\alpha<\beta<\gamma$ vera uppröðun á punktunum $x_{n-1}$, $x_n$ og
$x$. 

\pause   
Fallið 
$$
g(t)=f(t)-p_n(t)-\lambda(t-x_n)(t-x_{n-1})
$$
hefur núllstöð í öllum punktunum þremur.

\pause
Meðalgildissetningin gefur þá að $g'(t)$ hefur eina núllstöð í punkti
á bilinu $]\alpha,\beta[$ og aðra í $]\beta,\gamma[$. 

\pause
Af því leiðir
aftur að $g''(t)$ hefur núllstöð, $\xi_n$, í $[\alpha,\gamma]$, sem er
minnsta bilið sem inniheldur alla punktana $x_{n-1}$, $x_n$ og
$x$.  

\pause
Af þessu leiðir 
$$
0=g''(\xi_n)=f''(\xi_n)-2\lambda \quad \text{þþaa} \quad
\lambda=\tfrac 12 f''(\xi_n).
$$

\pause
Nú var $\lambda$ upprunalega valið þannig að $g(x)=0$. Þar með er
\begin{equation*}
	f(x) - p_n(x) = \frac{1}{2}f''(\xi_n)(x-x_n)(x-x_{n-1}).
\end{equation*}
}



\frame{\frametitle{2.4 Aðferð Newtons} 


Í sniðilsaðferðinni létum við $x_{n+1}$ vera skurðpunkt 
sniðils gegnum $(x_{n-1},f(x_{n-1}))$ og $(x_n,f(x_n))$ við
$x$-ás og fengum við rakningarformúluna
\begin{equation*}
  x_{n+1} = x_n - \frac{f(x_n)}{f[x_n,x_{n-1}]}.
\end{equation*}

\pause
Aðferð Newtons er nánast eins, nema í stað sniðils tökum við 
snertil í punktinum $(x_n,f(x_n))$.

\pause

\smallskip
Rakningarformúlan er eins, nema hallatalan verður
$f'(x_n)$ í stað $f[x_n,x_{n-1}]$
}
%

%
\frame{\frametitle{2.4  Aðferð Newtons} 


{\bf Byrjunarskref:}  Giskað er á eitt gildi $x_0$.   \pause

\medskip
{\bf Ítrekunarskref:}  Gefin eru  $x_0,\dots,x_n$.  Punkturinn 
$x_{n+1}$ er skurðpunktur snertils gegnum 
$(x_n,f(x_n))$ við $x$-ás,
$$
x_{n+1}=x_n-\dfrac{f(x_n)}{f'(x_n)}.
$$

%
\pause

\begin{block}{Upprifjun}
 Munum að snertill við graf $f$ í punktinum $x_n$ er
$$
y=f(x_n) + f'(x_n)(x-x_n),
$$
\pause
þessi lína sker $x$-ásinn ($y=0$) þegar 
$x=x_n - \frac{f(x_n)}{f'(x_n)}$.
\end{block}
}

%
\frame{\frametitle{2.4  Samleitin runa stefnir á núllstöð $f$} 

Gefum okkur að runan $(x_n)$ sé samleitin með markgildið $r$. \pause
Við fáum því 
$$
r=\lim_{n\to \infty}x_{n+1}=\lim_{n\to \infty}
\bigg(x_n-\dfrac{f(x_n)}{f'(x_n)}\bigg) =r-\dfrac{f(r)}{f'(r)}
$$

\pause
\smallskip

Þessi jafna jafngildir því að $f(r)=0$.

\pause
\smallskip

Þannig að ef runan er samleitin þá fáum við núllstöð.
}

%
\frame{\frametitle{2.4  Skekkjumat í  nálgun á $f(x)$ með $p_n(x)$} 

Snertillinn við $f$ í punktinum $x_n$ er 1.~stigs margliðan
\begin{equation*}
	p_n(x) = f(x_n) + f'(x_n)(x-x_n)
\end{equation*}
\pause
Samkvæmt skilgreiningu er $p_n(x_{n+1}) = 0$ svo $x_{n+1}$ uppfyllir jöfnuna
\begin{equation*}
	x_{n+1} = x_n - \frac{f(x_n)}{f'(x_n)}.
\end{equation*}
\pause
Athugum að $p_n$ er fyrsta Taylor nálgunin við fallið 
$f$ kringum $x_n$. \pause Setning Taylors gefur að til er  
$\xi_n$ sem liggur á milli
$r$ og $x_n$  þannig að 
\begin{equation*}
	f(r) - p_n(r) = \frac{1}{2}f''(\xi_n)(r-x_n)^2.
\end{equation*}
}
 
\frame{\frametitle{2.4  Skekkjumat í aðferð Newtons}
 
Nú er $f(r) = 0$ og því
\begin{equation*}
	-p_n(r) = \frac{1}{2}f''(\xi_n)e_n^2.
\end{equation*}
\pause
Eins er fæst af skilgreiningunni á $p_n$ að
\begin{equation*}
	-p_n(r) = -f'(x_n)e_{n+1}
\end{equation*}
\pause
Niðurstaðan verður því
\begin{equation*}
	e_{n+1} = \frac{-\frac{1}{2}f''(\xi_n)}
		{f'(x_n)}e_n^2
\end{equation*}
}

\frame{\frametitle{2.4   Aðferð  Newtons er að minnsta kosti 
ferningssamleitin} 

\begin{block}{Setning}
Ef aðferð Newtons fyrir fallið $f$  er samleitin, 
$f\in C^2([a,b])$ og $f'(r)\neq 0$, þá fáum við:
$$
\lim_{n\to \infty}\dfrac{e_{n+1}}{e_n^2}=\frac{-\frac{1}{2}f''(r)}
		{f'(r)}
$$\pause
Það er, aðferð Newtons er ferningssamleitin.
\end{block}

\pause

\begin{block}{Sönnun}
 $$
\lim_{n\to \infty}\dfrac{e_{n+1}}{e_n^2}=
\lim_{n\to \infty}\frac{-\frac{1}{2}f''(\xi_n)}{f'(x_n)} =
\frac{-\frac{1}{2}f''(r)}{f'(r)}
$$
\end{block}

\pause

\begin{block}{Athugasemd}
Athugið að það er ekki sjálfgefið að aðferð Newtons sé samleitin.

\pause
Auðvelt er að finna dæmi þar sem vond upphafságiskun $x_0$ skilar runu sem er
ekki samleitin. 
\end{block}
}


\begin{frame}{2.4 Samanburður á aðferðum}

{\small
\begin{block}{}
 \begin{table}[h]
    \begin{tabular}{|l|l|l|l|}
        \hline
        Bók            & Aðferð         & Samleitin           & Stig samleitni                    \\ \hline
        2.1 & Helmingunaraðferð      & Já, ef $f(a)f(b)<0$ & 1, línuleg                        \\ 
	& (bisection method) & & \\\hline 
        2.2 & Rangstöðuaðferð   & Já, ef $f(a)f(b)<0$ & 1, línuleg                        \\ 
        & (false position m.) && \\\hline 
	2.3 & Fastapunktsaðferð  & Ekki alltaf. En saml.         & amk 1              \\ 
        & (fixed point iteration) & ef $f$ er herping& \\\hline 
	2.4 & Aðferð Newtons  & Ekki alltaf         & 2, ef $f'(r)\neq 0$               \\ 
        & (Newtons method) & & \\\hline 
	2.5 & Sniðilsaðferð      & Ekki alltaf         & $\approx 1,618$, ef $f'(r)\neq 0$ \\
	& (secant method) & & \\
        \hline
    \end{tabular}
\end{table}
\end{block}
}

 \pause

\begin{block}{Athugasemd}
 Þó að aðferð Newtons sé samleitin af stigi 2, en sniðilsaðferðin
af stigi u.þ.b.~1,618, þá er í vissum tilfellum hagkvæmara að nota
sniðilsaðferðina ef það er erfitt að reikna gildin á afleiðunni
$f'$. 
\end{block}

\end{frame}
 


%
% \frame{\frametitle{2.4  Aðferð Newtons:  Úrlausn á ólínulegum
%     hneppum}  
% 
% Aðferð Newton er fastapunktsaðferð með ítrunarfallinu
% \begin{equation*}
%   F(x) = x - \frac{f(x)}{f'(x)}.
% \end{equation*}
% Nú er ekki sjálfgefið að aðferð Newtons uppfylli skilyrðin 
% í fastapunktssetningu og því er ekki víst að hún sé samleitin að $r$.
% 
% \smallskip
% Auðvelt er að finna dæmi þar sem vond upphafságiskun $x_0$ skilar runu sem er
% ekki samleitin. Ef aðferðin er samleitin, og $f'(r) \not= 0$, þá sjáum
% við að 
% \begin{equation*}
%   F'(r) = 1 - 
%   \frac{f'(r)\cdot f'(r) - f(r) \cdot f''(r)}{(f'(r))^2}
%   = \frac{f(r) \cdot f''(r)}{(f'(r))^2} = 0
% \end{equation*}
% því $f(r) = 0$, 
% 
% \smallskip
% Samkvæmt skekkjumatinu sem við gerður fyrir fastapunktsaðferð gefur
% þetta að Newton-aðferð er að minnsta kosti ferningssamleitin. 
% }
%
%
\frame{\frametitle{2.4  Matlab-forrit fyrir Aðferð Newtons} 
Þegar við forritum Newton aðferðina gerum við ráð fyrir að 
$f'(r) \not= 0$.  Þá er aðferðin a.m.k.\ ferningssamleitin, og við
notum  matið
\begin{equation*}
  |r-x_n| = |e_n| \approx |x_{n+1} - x_n|
\end{equation*} 
sem stöðvunarskilyrði. Við athugum þó að
\begin{equation*}
  |x_{n+1}-x_n| = 
  \left| \left( 
      x_n - \frac{f(x_n)}{f'(x_n) }
  \right) - x_n \right|
  = \left| \frac{f(x_n)}{f'(x_n)} \right|
\end{equation*}
og notum hægri hliðina sem villumat til að forðast reikniskekkjur.

}

\begin{frame}[fragile]{2.4  Matlab-forrit fyrir Aðferð Newtons} 

\hrule

\begin{verbatim}
function x = newtonNull(f,df,x0,epsilon)
%   newtonNull(f,df,x0,epsilon)
%
% Nálgar núllstöð fallsins f : R --> R með aðferð Newtons.
% Fallið df er afleiða f, x0 er upphafságiskun á núllstöð
% og epsilon er tilætluð nákvæmni.

x = x0; 
mis = f(x)/df(x);

% Ítrum meðan tilefni er til
while (abs(mis) >= epsilon)
   x = x - mis;
   mis = f(x)/df(x);
end
\end{verbatim}
\hrule
\end{frame}
%

\frame{\frametitle{2.4  Matlab-forrit fyrir aðferð Newtons} 

\begin{block}{Athugasemd}
Athugið að við þurfum ekki að skoða sérstaklega hvort {\tt x} sé
núllstöð {\tt f}, því ef svo er er {\tt abs(mis) = 0} sem er vissulega
minna en öll skynsamlega valin {\tt epsilon} og því hættir forritið
sjálfkrafa. 
\end{block}

\pause

\begin{block}{Athugasemd}
 Athugið að forritið geymir ekki $x_n$, heldur uppfærir bara ágiskunina $x$
í hvert skipti sem ítrunin er keyrð.
\end{block}

\pause

\begin{block}{Athugasemd}
 Forritið athugar ekki hversu oft það er búið að ítra, þannig að 
ef aðferðin er ekki samleitin þá hættir forritið aldrei. 
Þetta er ekki skynsamlegt.
\end{block}

}

%
\frame{\frametitle{2.4  Sýnidæmi} 

Við skulum nálga 9. rót tölunnar 1381 með nákvæmni upp á $\varepsilon = 10^{-8}$ með aðferð Newtons. Köllum rótina $r$, þá uppfyllir $r$ jöfnuna
\begin{equation*}
  r^9 - 1381 = 0
\end{equation*}
Verkefnið snýst því um að nálga núllstöð fallsins $f(x) = x^9 -
1381$. Athugið að $f$ er margliða af oddatölustigi og hefur því
virkilega núllstöð. Nú er $2^9 = 512$, svo $x_0 = 2$ er ágætis
upphafságiskun á $r$. 
}

%
\frame{\frametitle{2.4  Sýnidæmi:} 

Þegar við ítrum með forritinu okkar fæst

\begin{equation*}
  \begin{array}{c|c|c}
    n & x_n & |e_{n-1}| \approx |x_n - x_{n-1}| \\
    \hline
    0 & 2 & \\
    1 & 2.377170138888889 & 0.377170138888889 \\
    2 & 2.263516747674327 & 0.113653391214562 \\
    3 & 2.234695019689070 & 0.028821727985257 \\
    4 & 2.233115984281294 & 0.001579035407775 \\
    5 & 2.233111503379273 & 0.000004480902021 \\
    6 & 2.233111503343308 & 0.000000000035965
  \end{array}
\end{equation*}
Eftir sex ítranir er skekkjan orðin minni en $\varepsilon$, og við
nálgum því $r$ með $2.233111503$. 

\pause
\smallskip

Áhrif upphafságiskana sjást ágætlega með að prófa til dæmis $x_0 =
0.5$, þá skilar aðferðin alveg jafn góðri nálgun en þarf um 90 ítranir
til þess. 

}
%



%
\frame{\frametitle{Kafli 2: Fræðilegar spurningar:}

  \begin{enumerate}
  \item  Hvernig er ítrekunarskrefið í helmingunaraðferð?
  \item  Hvernig er  skekkjumatið í helmingunaraðferð?
  \item  Hvað þýðir að punkturinn $p$ sé fastapunktur fallsins $f$?
  \item  Hvernig er ítrekunarskrefið í fastapunktsaðferð?
  \item  Hvað þýðir að fall  $f:[a,b]\to \R$ sé {\it herping}?
  \item  Setjið fram fastapunktssetninguna.
  \item  Rökstyðjið að fastapunktsaðferð sé a.m.k.~línulega
    samleitin. 
  \item  Hvernig er ítrekunarskrefið í sniðilsaðferð?
  \item  Hvernig er skekkjuformúlan í sniðilsaðferð?
  \item  Rökstyðjið að hægt sé að nota $|x_{n+1}-x_n|$ fyrir mat á
    skekkju í sniðilsaðferð.
\item Hvernig er ítrekunarskrefið í aðferð Newtons?
\item Hvernig er skekkjumatið í aðferð Newtons?
\item Rökstyðjið að aðferð Newtons sé a.m.k.~ferningssamleitin. 
   \end{enumerate}
}



\end{document}