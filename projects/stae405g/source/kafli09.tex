\documentclass[icelandic,a4paper,12pt]{article}
\usepackage{beamerarticle}



\mode<presentation>
{
  \usetheme{boxes}
  %\useoutertheme{infolines}
  % með efnisyfirliti: Szeged, Frankfurt 
  % án efnisyfirlits: Pittsburgh
  % áhugavert: CambridgeUS, Boadilla
  %\setbeamercovered{transparent} %gegnsætt
  \setbeamercovered{invisible}

\defbeamertemplate*{footline}{infolines theme}
{
  \leavevmode%
  \hbox{%
  \begin{beamercolorbox}[wd=.333333\paperwidth,ht=2.25ex,dp=1ex,center]{author in head/foot}%
  %  \usebeamerfont{author in head/foot}\insertshortauthor~~\beamer@ifempty{\insertshortinstitute}{}{(\insertshortinstitute)}
  \end{beamercolorbox}%
  \begin{beamercolorbox}[wd=.333333\paperwidth,ht=2.25ex,dp=1ex,center]{title in head/foot}%
   % \usebeamerfont{title in head/foot}\insertshorttitle
  \end{beamercolorbox}%
  \begin{beamercolorbox}[wd=.333333\paperwidth,ht=2.25ex,dp=1ex,right]{date in head/foot}%
    %\usebeamerfont{date in head/foot}\insertshortdate{}\hspace*{2em}
    \insertshortlecture.\insertframenumber{} / \insertshortlecture.\inserttotalframenumber\hspace*{2ex} 
  \end{beamercolorbox}}%
  \vskip0pt%
}
\resetcounteronoverlays{rtaskno} %Does not increase counter rtaskno on \pause in beamer


}


\usepackage[english,icelandic]{babel}
\usepackage[utf8]{inputenc}
\usepackage{t1enc}
\usepackage{graphicx}
\usepackage{amsmath}
\usepackage{amssymb}
\usepackage{mathrsfs}
\usepackage{verbatim}
\usepackage{esint}


% RAGNAR SIGURÐSSON
%\usepackage[T1]{fontenc} 
%\usepackage[icelandic]{babel}
\usepackage{latexsym,amssymb,amsmath}
%\usepackage[utf8]{inputenc}
%\usepackage{graphicx}
\usepackage{epstopdf}
\usepackage{verbatim}
\usepackage{array,tabularx,arydshln}
\setbeamertemplate{theorems}[numbered]


\newtheorem{setning}{Setning}
\newtheorem{hjalpar}{Hjálparsetning}
\theoremstyle{definition}
\newtheorem{rithattur}{Ritháttur}
\newtheorem{skilgreining}{Skilgreining}
\newtheorem{daemi}{Dæmi}
\newtheorem{ath}{Athugasemd}

\newcommand\Wider[2][3em]{%
\makebox[\linewidth][c]{%
  \begin{minipage}{\dimexpr\textwidth+#1\relax}
  \raggedright#2
  \end{minipage}%
  }%
}

%counter used for blocks
\newcounter{rtaskno}
\DeclareRobustCommand{\rtask}[1]{%
   \refstepcounter{rtaskno}%
   \kaflanr.\thertaskno\label{#1}}

\newcommand{\C}{{\mathbb  C}}
\newcommand{\Z}{{\mathbb Z}}
\newcommand{\R}{{\mathbb  R}}
\newcommand{\N}{{\mathbb  N}}
\newcommand{\Q}{{\mathbb Q}}
\renewcommand{\phi}{\varphi}
\renewcommand{\epsilon}{\varepsilon}
\newcommand{\p}{{\partial}}
\renewcommand{\d}{{\partial}}

% RAGNAR SIGURÐSSON
\newcommand{\nin}{\mbox{$\;\not\in\;$}}
\newcommand{\dive}{\mbox{${\rm\bf div\,}$}}
\newcommand{\curl}{\mbox{${\rm\bf curl\,}$}}
\newcommand{\grad}{\mbox{${\rm\bf grad\,}$}}
\newcommand{\spann}{\mbox{${\rm Span}$}}
\newcommand{\tr}{\mbox{${\rm tr}$}}
\newcommand{\rank}{\mbox{${\rm rank}$}}
\newcommand{\image}{\mbox{${\rm image}$}}
\newcommand{\nullity}{\mbox{${\rm null}$}}
\newcommand{\proj}{\mbox{${\rm proj}$}}
\newcommand{\id}{\mbox{${\rm id}$}}
%\newcommand{\R}{\mbox{${\bf R}$}}
%\newcommand{\C}{\mbox{${\bf C}$}}
\newcommand{\Rn}{\mbox{${\bf R}^n$}}
\newcommand{\Rm}{\mbox{${\bf R}^m$}}
\newcommand{\Rk}{\mbox{${\bf R}^k$}}
\newcommand{\Av}{\mbox{${\bf A}$}}
\newcommand{\av}{\mbox{${\bf a}$}}
\newcommand{\uv}{\mbox{${\bf u}$}}
\newcommand{\vv}{\mbox{${\bf v}$}}
\newcommand{\wv}{\mbox{${\bf w}$}}
\newcommand{\xv}{\mbox{${\bf x}$}}
\newcommand{\zv}{\mbox{${\bf z}$}}
\newcommand{\yv}{\mbox{${\bf y}$}}
\newcommand{\bv}{\mbox{${\bf b}$}}
\newcommand{\cv}{\mbox{${\bf c}$}}
\newcommand{\dv}{\mbox{${\bf d}$}}
\newcommand{\ev}{\mbox{${\bf e}$}}
\newcommand{\fv}{\mbox{${\bf f}$}}
\newcommand{\gv}{\mbox{${\bf g}$}}
\newcommand{\hv}{\mbox{${\bf h}$}}
\newcommand{\iv}{\mbox{${\bf i}$}}
\newcommand{\jv}{\mbox{${\bf j}$}}
\newcommand{\kv}{\mbox{${\bf k}$}}
\newcommand{\pv}{\mbox{${\bf p}$}}
\newcommand{\nv}{\mbox{${\bf n}$}}
\newcommand{\qv}{\mbox{${\bf q}$}}
\newcommand{\rv}{\mbox{${\bf r}$}}
\newcommand{\sv}{\mbox{${\bf s}$}}
\newcommand{\tv}{\mbox{${\bf t}$}}
\newcommand{\ov}{\mbox{${\bf 0}$}}
\newcommand{\Fv}{\mbox{${\bf F}$}}
\newcommand{\Gv}{\mbox{${\bf G}$}}
\newcommand{\Uv}{\mbox{${\bf U}$}}
\newcommand{\Nv}{\mbox{${\bf N}$}}
\newcommand{\Hv}{\mbox{${\bf H}$}}
\newcommand{\Ev}{\mbox{${\bf E}$}}
\newcommand{\Sv}{\mbox{${\bf S}$}}
\newcommand{\Tv}{\mbox{${\bf T}$}}
\newcommand{\Bv}{\mbox{${\bf B}$}}
\newcommand{\Oa}{\mbox{$(0,0)$}}
\newcommand{\Ob}{\mbox{$(0,0,0)$}}
\newcommand{\Onv}{\mbox{$[0,0,\ldots,0]$}}
\newcommand{\an}{\mbox{$(a_1,a_2, \ldots,a_n)$}}
\newcommand{\xn}{\mbox{$(x_1,x_2, \ldots,x_n)$}}
\newcommand{\xnv}{\mbox{$[x_1,x_2, \ldots,x_n]$}}
\newcommand{\vnv}{\mbox{$[v_1,v_2, \ldots,v_n]$}}
\newcommand{\wnv}{\mbox{$[w_1,w_2, \ldots,w_n]$}}
\newcommand{\tvint}{\int\!\!\!\int}
\newcommand{\thrint}{\int\!\!\!\int\!\!\!\int}
\renewcommand{\ast}{{\operatorname{\text{astand}}}}


\usepackage{caption}
%\usepackage{pgfpages}
% \pgfpagesuselayout{2 on 1}[a4paper,border shrink=5mm]

\def\lecturename{Stærðfræðigreining IIB}
\title{\insertlecture}
\author{Sigurður Örn Stefánsson, \href{mailto:sigurdur@hi.is}{sigurdur@hi.is}}
\institute
{
  Verkfræði- og náttúruvísindasvið\\
  Háskóli Íslands
}
\subtitle{Stærðfræðigreining IIB, STÆ205G}
%\subject{\lecturename}

\mode<article>
{
	\usepackage[colorlinks=false,
	pdfauthor={Sigurður Örn Stefánson},
	%pdftitle={Töluleg greining}
	]{hyperref}
  %\usepackage{times}
  %\usepackage{mathptmx}
  \usepackage[left=1.5cm,right=4cm,top=1.5cm,bottom=3cm]{geometry}
}

% Beamer version theme settings

%\useoutertheme[height=0pt,width=2cm,right]{sidebar}
%\usecolortheme{rose,sidebartab}
%\useinnertheme{circles}
%\usefonttheme[only large]{structurebold}


\setbeamercolor{sidebar right}{bg=black!15}
\setbeamercolor{structure}{fg=blue}
\setbeamercolor{author}{parent=structure}

\setbeamerfont{title}{series=\normalfont,size=\LARGE}
\setbeamerfont{title in sidebar}{series=\bfseries}
\setbeamerfont{author in sidebar}{series=\bfseries}
\setbeamerfont*{item}{series=}
\setbeamerfont{frametitle}{size=}
\setbeamerfont{block title}{size=\small}
\setbeamerfont{subtitle}{size=\normalsize,series=\normalfont}


\setbeamertemplate{sidebar right}
{
  {\usebeamerfont{title in sidebar}%
    \vskip1.5em%
    \hskip3pt%
    \usebeamercolor[fg]{title in sidebar}%
    \insertshorttitle[width=2cm-6pt,center,respectlinebreaks]\par%
    \vskip1.25em%
  }%
  {%
    \hskip3pt%
    \usebeamercolor[fg]{author in sidebar}%
    \usebeamerfont{author in sidebar}%
    \insertshortauthor[width=2cm-2pt,center,respectlinebreaks]\par%
    \vskip1.25em%
  }%
  \hbox to2cm{\hss\insertlogo\hss}
  \vskip1.25em%
  \insertverticalnavigation{2cm}%
  \vfill
  \hbox to 2cm{\hfill\usebeamerfont{subsection in
      sidebar}\strut\usebeamercolor[fg]{subsection in
      sidebar}\insertshortlecture.\insertframenumber\hskip5pt}%
  \vskip3pt%
}%

\setbeamertemplate{title page}
{
  \vbox{}
  \vskip1em
  %{\huge Kapitel \insertshortlecture\par}
  {\usebeamercolor[fg]{title}\usebeamerfont{title}\inserttitle\par}%
  \ifx\insertsubtitle\@empty%
  \else%
    \vskip0.25em%
    {\usebeamerfont{subtitle}\usebeamercolor[fg]{subtitle}\insertsubtitle\par}%
  \fi%     
  \vskip1em\par
  %Vorlesung \emph{\lecturename}\ vom 
  \insertdate\par
  \vskip0pt plus1filll
  \leftskip=0pt plus1fill\insertauthor\par
  \insertinstitute\vskip1em
}

%\logo{\includegraphics[width=2cm]{beamerexample-lecture-logo.pdf}}



% Article version layout settings

\mode<article>

\makeatletter
\def\@listI{\leftmargin\leftmargini
  \parsep 0pt
  \topsep 5\p@   \@plus3\p@ \@minus5\p@
  \itemsep0pt}
\let\@listi=\@listI


\setbeamertemplate{frametitle}{\paragraph*{\insertframetitle\
    \ \small\insertframesubtitle}\ \par
}
\setbeamertemplate{frame end}{%
  \marginpar{\scriptsize\hbox to 1cm{\sffamily%
      \hfill\strut\insertshortlecture.\insertframenumber}\hrule height .2pt}}
\setlength{\marginparwidth}{1cm}
\setlength{\marginparsep}{1.5cm}

\def\@maketitle{\makechapter}

\def\makechapter{
  \newpage
  \null
  \vskip 2em%
  {%
    \parindent=0pt
    \raggedright
    \sffamily
    \vskip8pt
    %{\fontsize{36pt}{36pt}\selectfont Kapitel \insertshortlecture \par\vskip2pt}
    {\fontsize{24pt}{28pt}\selectfont \color{blue!50!black} \insertlecture\par\vskip4pt}
    {\Large\selectfont \color{blue!50!black} \insertsubtitle, \@date\par}
    \vskip10pt

    \normalsize\selectfont \@author\par\vskip1.5em
    %\hfill BLABLA
  }
  \par
  \vskip 1.5em%
}

\let\origstartsection=\@startsection
\def\@startsection#1#2#3#4#5#6{%
  \origstartsection{#1}{#2}{#3}{#4}{#5}{#6\normalfont\sffamily\color{blue!50!black}\selectfont}}

\makeatother

\mode
<all>



% Typesetting Listings

\usepackage{listings}
\lstset{language=Java}

\alt<presentation>
{\lstset{%
  basicstyle=\footnotesize\ttfamily,
  commentstyle=\slshape\color{green!50!black},
  keywordstyle=\bfseries\color{blue!50!black},
  identifierstyle=\color{blue},
  stringstyle=\color{orange},
  escapechar=\#,
  emphstyle=\color{red}}
}
{
  \lstset{%
    basicstyle=\ttfamily,
    keywordstyle=\bfseries,
    commentstyle=\itshape,
    escapechar=\#,
    emphstyle=\bfseries\color{red}
  }
}



% Common theorem-like environments

\theoremstyle{definition}
\newtheorem{exercise}[theorem]{\translate{Exercise}}




% New useful definitions:

\newbox\mytempbox
\newdimen\mytempdimen

\newcommand\includegraphicscopyright[3][]{%
  \leavevmode\vbox{\vskip3pt\raggedright\setbox\mytempbox=\hbox{\includegraphics[#1]{#2}}%
    \mytempdimen=\wd\mytempbox\box\mytempbox\par\vskip1pt%
    \fontsize{3}{3.5}\selectfont{\color{black!25}{\vbox{\hsize=\mytempdimen#3}}}\vskip3pt%
}}

\newenvironment{colortabular}[1]{\medskip\rowcolors[]{1}{blue!20}{blue!10}\tabular{#1}\rowcolor{blue!40}}{\endtabular\medskip}

\def\equad{\leavevmode\hbox{}\quad}

\newenvironment{greencolortabular}[1]
{\medskip\rowcolors[]{1}{green!50!black!20}{green!50!black!10}%
  \tabular{#1}\rowcolor{green!50!black!40}}%
{\endtabular\medskip}



\lecture[9]

\newcommand{\C}{{\mathbb  C}}
\newcommand{\Z}{{\mathbb Z}}
\newcommand{\R}{{\mathbb  R}}
\newcommand{\N}{{\mathbb  N}}
\newcommand{\Q}{{\mathbb Q}}
\begin{document}

%\subsection
%	\maketitle


%\subsection{Yfirlit}
\section{Jaðargildisverkefni fyrir sporgerar (e.~elliptic) hlutafleiðujöfnur}
%\begin{center}
%\begin{tabular}{|l|l|l|l|}\hline
%Kafli &Heiti á viðfangsefni & Bls. & Glærur\\
%\hline
%9.0 &Almenn atriði um jaðargildisverkefni & 725-730 & 3-6\\
%9.1 &Poisson jafnan í rétthyrndu svæði & & \\
%   &--  Dirichlet-jaðarskilyrði & 730-739 & 7-16\\
%9.2 &Poisson jafnan í rétthyrndu svæði & & \\
%    &-- Neumann- og Robin-jaðarskilyrði  & 742-752 & 17-24\\ \hline
%\end{tabular}
%\end{center}



\subsection{Almenn atriði um jaðargildisverkefni}

%\subsection{9.0 Inngangur}
Við ætlum að skoða annars stigs línulegar hlutafleiðujöfnur á svæði
$R$ í $\R^2$, \pause þetta eru jöfnur á forminu
\begin{multline}
A(x,y)\frac{\p^2 u}{\p x^2} + B(x,y)\frac{\p^2 u}{\p x\p y} + C(x,y)\frac{\p^2 u}{\p y^2} +\\
D(x,y)\frac{\p u}{\p x} + E(x,y)\frac{\p u}{\p y} + F(x,y)u = G(x,y).
\end{multline}
\subsubsection{Skilgreining}
 Annars stigs línuleg hlutafleiðujafna eins og að ofan kallast \emph{sporger}
 (e.~elliptic) ef 
 $$
  A(x,y)C(x,y) - B(x,y)^2 > 0,
 $$
 fyrir öll $(x,y) \in R$.





\subsubsection{Sporgerar jöfnur}
 Sporgerar jöfnur eru tímaóháðar og lýsa oft verkefnum þar sem verið er að
 leita að jafnvægisástandi \pause (lágmarksflötum, hitajafnvægi og þess háttar).
 \pause
 
 \subsubsection{Laplace/Poisson}
 Við ætlum að einskorða okkur við mikilvægasta tilvikið, \emph{Laplace jöfnuna}
 $$
  \frac{\p^2 u}{\p x^2} + \frac{\p^2 u}{\p y^2} = 0,
 $$\pause
 og hliðraða útgáfu hennar, \emph{Poisson jöfnuna}
 $$
  \frac{\partial^2 u}{\partial x^2} + \frac{\partial^2 u}{\partial y^2} = f(x,y).
 $$
 


\subsubsection{Jaðarskilyrði}
\textbf{Svæðið} $R$
  Til einföldunar þá munum við eingöngu skoða rétthyrninga á forminu
  $$
    R = \{ (x,y) \in \R^2 ; a < x < b, c < y < d \}.
  $$
 
%\subsubsection{Jaðarskilyrði}
 Eins og fyrir afleiðujöfnur með jaðarskilyrði (kafli 8) þá höfum við þrjár gerðir
 af jaðarskilyrðum,
 \begin{center}
 \begin{tabular}{ll}
  Dirichlet: & $u(x,y) = r(x,y)$ á $\p R$.\\
  Neumann:   & $\frac{\p u}{\p n}(x,y) = r(x,y)$ á $\p R$.\\
  Robin:     & $\alpha(x,y)u(x,y) + 
  \beta(x,y)\frac{\p u}{\p n}(x,y) = r(x,y)$ á  $\p R$.\\
 \end{tabular}
 \end{center}
Hér er $\p R$ jaðar svæðisins $R$ og $n$ er útvísandi þverill fyrir $\p R$.



\subsubsection{Þýð föll}
 \textbf{Skilgreining}
  Tvisvar sinnum samfellt diffranlegt fall $u$ á svæði í $\R^2$ sem uppfyllir
  Laplace jöfnuna
  $$ 
    \frac{\p^2 u}{\p x^2} + \frac{\p^2 u}{\p y^2} = 0,
  $$
  kallast \emph{þýtt}.
 
 
 \pause
 \textbf{Athugasemd}
  Tvisvar sinnum samfellt diffranlegt fall $u$ á svæði $R \subset \R^2$ er 
  þýtt þá og því aðeins að 
  $$
  u(x) = \frac{1}{\pi r^2} \int_{B(x,r)} u(y) dy
  $$
  fyrir öll $x \in R$ og öll $r$ þannig að $B(x,r) \subset R$. Hér er
  $B(x,r)$ skífan með miðju $x$ og geisla $r$.
  
  \pause
  Með öðrum orðum, fallgildi þýðs falls $u$ í punkti $x$ er jafnt ,,meðalfallgildi''
  $u$ umhverfis $x$.
 


\subsection{Poisson jafnan í rétthyrndu svæði --  Dirichlet-jaðarskilyrði }

\subsubsection{Netið}
 Byrjum á að skipta rétthyrningnum $R$ í $M \times N$ net með því að skipta
 bilinu $[a,b]$ í $N$ bil og bilinu $[c,d]$ í $M$ bil. \pause
 Til einföldunar þá gerum við ráð fyrir að hlutfall hliðarlengdanna
 í $R$ sé ræð tala, því þá getum við fundið $M$ og $N$ þannig að
 $$
  \frac{b-a}N = \frac{d-c}M = h.
 $$ \pause
 Þar með er hver reitur í netinu er af stærðinni $h \times h$.
 \pause
 \medskip
 
 Ef við köllum látum $x_j$ vera skiptipunktanna fyrir $[a,b]$,
 $$
 a=x_0 < x_1 < x_2 < \ldots x_{N-1} < x_N =b,
 $$
 og $y_k$ vera skiptipunktanna fyrir $[c,d]$,
 $$
 c=y_0 < y_1 < y_2 < \ldots y_{N-1} < y_M = d,
 $$
 þá eru skiptipunktarnir í netinu $(x_j,y_k)$.



\subsubsection{Ritháttur}
Við skoðum ætlum að skoða jöfnuna
 $$ 
    \frac{\p^2 u}{\p x^2} + \frac{\p^2 u}{\p y^2} = f(x,y),
  $$
  á $R = \{ (x,y) \in \R^2 ; a < x < b, c < y < d \}$ með Dirichlet skilyrðin
  $$
    u(x,y) = g(x,y), \qquad \text{ á } \p R.
  $$
  
  \medskip\pause
Til að einfalda rithátt þá skrifum við
\begin{align*}
 u_{j,k} &= u(x_j,y_k), \qquad j=0,\ldots,N, \quad k=0,\ldots,M,\\
 f_{j,k} &= f(x_j,y_k), \qquad j=0,\ldots,N, \quad k=0,\ldots,M,\\
 g_{j,k} &= g(x_j,y_k), \qquad j=0, j=N, k=0 \text{ eða } k=M,
\end{align*} \pause
og nálgunargildin eru 
$$
  w_{j,k} \approx u_{j,k}.
$$


\subsubsection{Strjál útgáfa Poisson jöfnunar}
 Með því að nota miðsettan mismunakvóta fyrir aðra afleiðu $u$ þá getum við 
 skrifað
 \begin{align*}
  \frac{\p^2 u}{\p x^2}|_{x_j,y_k} &= \frac{u_{j-1,k} - 2u_{j,k} + u_{j+1,k}}{h^2}
  + O(h^2), \\
  \frac{\p^2 u}{\p y^2}|_{x_j,y_k} &= \frac{u_{j,k-1} - 2u_{j,k} + u_{j,k+1}}{h^2}
  + O(h^2), 
 \end{align*}
 þar sem $j = 1,\ldots,N-1$ og $k=1,\ldots,M-1$.

\medskip\pause
Með því að setja þetta inn í 
  $$ 
    \frac{\p^2 u}{\p x^2} + \frac{\p^2 u}{\p y^2} = f(x,y),
  $$
  þá fæst, \pause 
  $$
  \frac{u_{j-1,k} - 2u_{j,k} + u_{j+1,k}}{h^2} +
  \frac{u_{j,k-1} - 2u_{j,k} + u_{j,k+1}}{h^2}
  + O(h^2) = f_{j,k}.
  $$

  
%\subsection{9.1 Strjál útgáfa Poisson jöfnunar, frh.}
Höfðum 
$$
  \frac{u_{j-1,k} - 2u_{j,k} + u_{j+1,k}}{h^2} +
  \frac{u_{j,k-1} - 2u_{j,k} + u_{j,k+1}}{h^2}
  + O(h^2) = f_{j,k}.
  $$
\pause\medskip
  
  Skiptum $u$ út fyrir nálgunargildin $w_{j,k}$, hendum skekkjuliðnum og
  margföldum í gegn með $-h^2$, \pause þá fæst
  $$
  -w_{j-1,k} - w_{j+1,k} - w_{j,k-1} - w_{j,k+1} + 4w_{j,k} = -h^2 f_{j,k},
  $$
 fyrir $j = 1,\ldots,N-1$ og $k=1,\ldots,M-1$.


\subsubsection{Fjöldi jafna}
 Hér á undan fengum við eina jöfnu fyrir hvern innri punkt netsins, 
 samtals $(N-1)(M-1)$ jöfnur. Þetta er fjöldi óþekktra $w_{j,k}$, því
 $w_{j,k}$ á jaðrinum eru þekkt, þar er 
 $$
    w_{j,k} = g_{j,k}.
 $$
 
 \pause\medskip
 Við þurfum að því leysa jöfnuhneppið
 \begin{align*}
  -w_{j-1,k} - w_{j+1,k} - w_{j,k-1} - w_{j,k+1} + 4w_{j,k} &= -h^2 f_{j,k} \\ 
   \text{fyrir } j=1&,\ldots,N-1, \ \, k=1,\ldots,M-1\\
  %w_{j,k} &= g_{j,k}, \\
  %\text{fyrir } j=0&, j=N, k=0 \text{ eða } k=M.
 \end{align*}


\subsubsection{Uppröðun á innri punktum}
 Röðum í vigur $\wv$ tölunum $w_{j,k}$ fyrir innri punkta netsins. 
 Förum í gegnum netið frá vinstri til hægri, byrjum á 
 næst neðstu línunni og förum svo upp á við,
 $$
  \wv = [ \underbrace{w_{1,1} \ w_{2,1} \ldots w_{N-1,1}}_{k=1} 
  \ \underbrace{w_{1,2} \ w_{2,2} \ldots w_{N-1,2}}_{k=2} \ 
  \underbrace{w_{1,3} \ldots  \ldots}_{k=3,\ldots,M-1} ]^T.
 $$
 
 Fyrri vísirinn ($j$) vísar til $x$-hnitsins $x_j$ og seinni vísirinn ($k$)
 til $y$-hnitsins $y_k$.


\subsubsection{Jöfnurnar aftur}
Jöfnurnar sem við leiddum út eru
 $$
  -w_{j-1,k} - w_{j+1,k} - w_{j,k-1} - w_{j,k+1} + 4w_{j,k} = -h^2 f_{j,k}.
 $$
 \pause\medskip
 
 Ef $j-1=0$, $k-1 =0$, $j+1=N$ eða $k+1=M$ þá erum við á jaðrinum og þar
 þekkjum við gildin, þau eru gefin með $g$.
 \pause\medskip
 
 T.d.~ef $j=1$ og $k=1$ þá fæst 
 \begin{align*}
  -h^2 f_{1,1} &= -w_{0,1} - w_{2,1} - w_{1,0} - w_{1,2} + 4w_{1,1} \\
  &= -g_{0,1} - w_{2,1} - g_{1,0} - w_{1,2} + 4w_{1,1}\\
 \end{align*}\pause
 Það er
 $$
  - w_{2,1} - w_{1,2} + 4w_{1,1} = 
  -h^2 f_{1,1}  + g_{0,1} + g_{1,0}.
 $$
 \pause\medskip
 
Jöfnurnar fyrir $k=1$, $j=1$, $k=M-1$ og $j=N-1$ þurfa því að taka
mið af $g$.



\subsubsection{Jöfnuhneppið}
Jöfnuhneppið sem fæst er eftirfarandi
$$
A\wv = \bv,
$$ 
þar sem $A$ er $(N-1)(M-1)\times (N-1)(M-1)$ fylkið
$$
  A = \left[\begin{array}{cccccc}
D & -I &   &   &   &  \\
-I & D & -I &   &   &  \\
  & \cdot & \cdot & \cdot &   &  \\
  &   & \cdot & \cdot & \cdot &  \\
  &   &  & -I & D & -I\\
  &   &   &   & -I & D
      \end{array}\right].
$$
Hér er $I$ er $(N-1)\times (N-1)$ einingafylkið og 
$$
  D = \left[\begin{array}{cccccc}
4 & -1 &   &   &   &  \\
-1 & 4 & -1 &   &   &  \\
  & \cdot & \cdot & \cdot &   &  \\
  &   & \cdot & \cdot & \cdot &  \\
  &   &  & -1 & 4 & -1\\
  &   &   &   & -1 & 4
      \end{array}\right] \qquad ((N-1)\times (N-1) \text{ fylki}).
$$


%\subsection{9.1 Jöfnuhneppið -- hægri hliðin}
Hægri hliðin er gefin með summunni af $-h^2 f_{j,k}$ og gildunum í þeim
jaðarpunktum sem við rekumst þegar við förum gegnum innri punktanna
(í sömu röð og tilgreind var fyrir $\wv$). \pause Það er


%\subsection{9.1 Jöfnuhneppið -- hægri hliðin, frh.}
{}\vspace{-0.2in}
{\small \begin{align*}
  \bv = [ 
  &-h^2 f_{1,1} + g_{1,0} + g_{0,1} & (\text{lína } k=1)\\
  &-h^2 f_{2,1} + g_{2,0} \\
  &\ldots \\
  &-h^2 f_{N-2,1} + g_{N-2,0} \\
  &-h^2 f_{N-1,1} + g_{N-1,0} + g_{N,1} \\
  &-h^2 f_{1,2} + g_{0,2} & (\text{lína } k=2)\\
  &-h^2 f_{2,2} \\
  &\ldots \\
  &-h^2 f_{N-2,2}\\
  &-h^2 f_{N-1,2} + g_{N,2}\\
  & \ldots \ldots \ldots  & (\text{línur } k=3,\ldots,M-2)\\
  &-h^2 f_{1,M-1} + g_{1,M-1} + g_{0,M} & (\text{lína } k=M-1)\\
  &-h^2 f_{2,M-1} + g_{2,M} \\
  &\ldots \\
  &-h^2 f_{N-1,M-1} + g_{N,M-1} + g_{N-1,M} ]\\
  \end{align*}}


\subsection{Poisson jafnan í rétthyrndu svæði -- Neumann- og Robin-jaðarskilyrði}

\subsubsection{Blönduð jaðarskilyrði}
%\subsubsection{Jaðarskilyrði}
 Rifjum upp þær þrjár gerðir jaðarskilyrða sem við höfum áhuga
 \begin{center}
 \begin{tabular}{ll}
  Dirichlet: & $u(x,y) = r(x,y)$ á $\p R$.\\
  Neumann:   & $\frac{\p u}{\p n}(x,y) = r(x,y)$ á $\p R$.\\
  Robin:     & $\alpha(x,y)u(x,y) + 
  \beta(x,y)\frac{\p u}{\p n}(x,y) = r(x,y)$ á  $\p R$.\\
 \end{tabular}
 \end{center}
Hér er $\p R$ jaðar svæðisins $R$ og $n$ er útvísandi þverill fyrir $\p R$.


%\subsubsection{}
 Dirichlet-skilyrði skoðuðum við í kafla 9.1. \pause
 
 Neumann og Robin skilyrði eru efni kafla 9.2. \pause
 Aðferðin hér er sú sama og í kafla 8.2,, við bætum við felugildum
 til þess að geta nálgað afleiðu $u$ á jaðrinum. \pause
 Þar sem jaðarinn er ferill í $\R^2$, en ekki bil eins og í kafla 8, 
 þá þarf að setja felupunkta alls staðar þar sem afleiðan $\p u/\p n$ er gefin. \pause
 Þetta veldur því að útfærslan getur orðið nokkuð flókin, 
 bókarhöfundur skoðar því eingöngu ákveðin sértilvik. 





\subsubsection{Neumann skilyrði eftir botninum}

 Skoðum svæðið 
  $$
    R = \{ (x,y) \in \R^2 ; a < x < b, c < y < d \}.
  $$ \pause
 og tilvikið þegar jaðargildin eru gefin með falli $g(x,y)$
 á hliðunum ($x=a$ og $x=b$) og toppnum ($y=d$), og 
 afleiðan eftir botninum er gefin með 
 $$
  \frac{\p u}{\p n} = \alpha(x)
 $$
 (ath. staðsetningin þar er bara háð $x$).
 
 Útvísandi þverill eftir botninum er 
 $$
 \frac{\p }{\p n} = -\frac{\p }{\p y}.
 $$
 



\subsubsection{Mismunakvótar eftir botninum}
 Gerum eins og áður ráð fyrir því að við höfum jafna skiptingu á bilinu $[a,b]$,
 $$
  a = x_0 < x_1 < \ldots < x_{N-1} < x_N = b,
 $$
 með billlengd $h$, \pause og skiptingu á bilinu $[c,d]$ með sömu billengd,
 $$
  c = y_0 < y_1 < \ldots < y_{M-1} < y_M = d.
 $$
 \pause
 
 Bætum við felupunktum $(x_j,y_{-1})$ fyrir $j=1,\ldots,N-1$. \pause
 Samhverfir mismunakvótar fyrir $\frac{\p u}{\p n}$ gefa þá
 $$
  - \left. \frac{\p u}{\p y} \right|_{x=x_j} = \alpha(x_j) 
 $$
 $$
  -\frac{w_{j,1} - w_{j,-1} }{2h} \approx \alpha(x_j).
  $$
  Það er 
  $$
    w_{j,-1}  = w_{j,1} + 2h\alpha(x_j)
  $$


\subsubsection{Jaðarskilyrðin sett inn í Poisson-jöfnuna}
 Rifjum upp að lausnin  okkar $w_{j,k}$ á $\Delta u = f$ á að uppfylla
 \begin{align*}
 -w_{j-1,k} - w_{j+1,k} - w_{j,k-1} - w_{j,k+1} + 4w_{j,k} &= -h^2 f_{j,k} 
  %\\ 
  % \text{fyrir } j=1&,\ldots,N-1, \ \, k=1,\ldots,M-1\\
  %w_{j,k} &= g_{j,k}, \\
  %\text{fyrir } j=0&, j=N, k=0 \text{ eða } k=M.
 \end{align*}
 
 Eftir botninum ($k=0$, $j=1,\ldots,N-1$) þýðir þetta að 
 \begin{align*}
  - w_{2,0} - w_{1,-1} - w_{1,1} + 4w_{1,0} &= -h^2 f_{1,0} + g(0,0),\\
  -w_{j-1,0} - w_{j+1,0} - w_{j,-1} - w_{j,1} + 4w_{j,0} &= -h^2 f_{j,0}, 
  \quad {\color{gray} j=2,\ldots,N-2},\\
  -w_{N-2,0}  - w_{N-1,-1} - w_{N-1,1} + 4w_{N-1,0} &= -h^2 f_{N-1,0} + g(N,0) 
 \end{align*} \pause
 það er
\begin{align*}
- w_{2,0}  - 2w_{1,1} + 4w_{1,0} &= -h^2 f_{1,0} + g(0,0) + 2h\alpha(x_1),\\
  -w_{j-1,0} - w_{j+1,0} - 2w_{j,1} + 4w_{j,0} &= -h^2 f_{j,0} + 2h\alpha(x_j), 
   \ \ {\color{gray} j=2,\ldots,N-2}\\
  -w_{N-2,0}  - 2w_{N-1,1} + 4w_{N-1,0} &= -h^2 f_{N-1,0} + g(N,0) + 2h\alpha(x_{N-1})
 \end{align*} 



\subsubsection{Fjöldi jafna}
\begin{enumerate}
 \item Dirichlet-jaðarskilyrði
  Fjöldi jafna í kafla 9.1 var $(N-1)(M-1)$, það er fjöldi innri punkta því
  við þekktum gildin $w_{j,k}$ á jaðrinum.
 
  \pause
  
 \item Blönduð jaðarskilyrði
  Þegar við skoðum blönduð jaðarskilyrði þá þekkjum við ekki endilega gildi 
  $w_{j,k}$ á jaðrinum og því erum við hugsanlega með fleiri jöfnur.
  
  \pause
  
 \item Blönduð jaðarskilyrði -- Dæmið okkar
  Í þessu tilviki sem við erum að skoða þá bætast við jöfnur fyrir
  $w_{j,0}$, $j=1,\ldots,N-1$, eftir botninum. Það þýðir að jöfnuhneppið 
  okkar inniheldur $(N-1)M$ jöfnur. 
  \pause
\end{enumerate}  
  Fylkið $A$ er því $(N-1)M \times (N-1)M$, og samanstendur af 
  $M \times M$ hlutfylkjum sem hvert um sig er $(N-1)\times (N-1)$.
 





\subsubsection{Jöfnuhneppið $A\wv = \bv$}
Jöfnurnar fyrir $k=0$ voru
\begin{align*}
- w_{2,0}  - {\color{red}2}w_{1,1} + 4w_{1,0} &= -h^2 f_{1,0} + g(0,0) + 
{\color{blue}2h\alpha(x_1)},\\
  -w_{j-1,0} - w_{j+1,0} - {\color{red}2}w_{j,1} + 4w_{j,0} &= -h^2 f_{j,0} + 
  {\color{blue}2h\alpha(x_j)}, 
   \ \ {\color{gray} j=2,\ldots,N-2}\\
  -w_{N-2,0}  - {\color{red}2}w_{N-1,1} + 4w_{N-1,0} &= -h^2 f_{N-1,0} + g(N,0) + 
  {\color{blue}2h\alpha(x_{N-1})}
 \end{align*} \pause
 
 Þetta breytir fyrstu hlutfylkjalínunni í jöfnuhneppinu $A\wv = \bv$, \pause
 \begin{itemize}
  \item[{\color{red} *}] $-I$ í fyrsta hlutdálki verður $-2I$.\pause
  \item[{\color{blue} *}] Við efsta hluta $\bv$ (sem tilheyrir línunni $k=0$)
  bætist við vigurinn 
  $$
    [2h\alpha(x_1),2h\alpha(x_2),\ldots,2h\alpha(x_{N-1})]^T.
   $$
 \end{itemize}




%\subsubsection{9.2 Jöfnuhneppið}
Jöfnuhneppið sem fæst er því eftirfarandi
$$
A\wv = \bv,
$$ 
þar sem $A$ er $(N-1)M\times (N-1)M$ fylkið
$$
  A = \left[\begin{array}{cccccc}
D & -2I &   &   &   &  \\
-I & D & -I &   &   &  \\
  & \cdot & \cdot & \cdot &   &  \\
  &   & \cdot & \cdot & \cdot &  \\
  &   &  & -I & D & -I\\
  &   &   &   & -I & D
      \end{array}\right].
$$
Hér er $I$ er $(N-1)\times (N-1)$ einingafylkið og 
$$
  D = \left[\begin{array}{cccccc}
4 & -1 &   &   &   &  \\
-1 & 4 & -1 &   &   &  \\
  & \cdot & \cdot & \cdot &   &  \\
  &   & \cdot & \cdot & \cdot &  \\
  &   &  & -1 & 4 & -1\\
  &   &   &   & -1 & 4
      \end{array}\right] \qquad ((N-1)\times (N-1) \text{ fylki}).
$$



%\subsection{9.2 Jöfnuhneppið -- hægri hliðin}
{\  }
\mode<presentation>{ \vspace{-1.2in} }

{\small \begin{align*}
  \bv = [ 
  &-h^2 f_{1,0} + g_{0,0} + 2h\alpha(x_1) & (\text{lína } k=0)\\
  &-h^2 f_{2,0} + 2h\alpha(x_2)  \\
  &\ldots \\
  &-h^2 f_{N-2,0} + 2h\alpha(x_{N-2}) \\
  &-h^2 f_{N-1,0} + g_{N,0} + 2h\alpha(x_{N-2}) \\
  %
  &-h^2 f_{1,1} + g_{0,1}   & (\text{lína } k=1)\\
  &-h^2 f_{2,1} \\
  &\ldots \\
  &-h^2 f_{N-2,1}  \\
  &-h^2 f_{N-1,1} +  g_{N,1} \\
  %
%   &-h^2 f_{1,2} + g_{0,2} +^2 \\
%   &-h^2 f_{2,2} \\
%   &\ldots \\
%   &-h^2 f_{N-2,2}\\
%   &-h^2 f_{N-1,2} + g_{N,2}\\
   & \ldots \ldots \ldots  & (\text{línur } k=2,\ldots,M-2)\\
  &-h^2 f_{1,M-1} + g_{1,M-1} + g_{0,M} & (\text{lína } k=M-1)\\
  &-h^2 f_{2,M-1} + g_{2,M} \\
  &\ldots \\
  %&-h^2 f_{N-2,1} + g_{N-2,0} \\
  &-h^2 f_{N-1,M-1} + g_{N,M-1} + g_{N-1,M} ]\\
  \end{align*}}




%\section*{Fræðilegar spurningar}

\subsection{Fræðilegar spurningar}
\begin{enumerate}
  \item Hvað er átt við með því að lausn hlutafleiðujöfnu á svæði  $R$
    í plani 
    uppfylli {\it Dirichlet-jaðarskilyrði}?  \\
(Samheiti er {\it fallsjaðarskilyrði}.)
\item Hvernig er {\it útvísandi þverafleiða} $\partial u/\partial n$ af 
falli $u$ á svæði $R$ í plani skilgreind?  
\item Hvað er átt við með því að lausn hlutafleiðujöfnu á svæði $R$ í plani
    uppfylli {\it Neumann-jaðarskilyrði}? \\ (Samheiti eru {\it
      afleiðujaðarskilyrði}
og {\it flæðisjaðarskilyrði}.)
  \item Hvað er átt við með því að lausn hlutafleiðujöfnu á svæði  $R$
    uppfylli {\it Robin-jaðarskilyrði}?  \\ 
(Samheiti er {\it blandað jaðarskilyrði}.)
  \item Hvernig er  nálgunarjafna fyrir 
Poisson-jöfnu $\Delta u=f$ í innri skiptipunkti
í ferningslaga neti í plani leidd út? 
  \item Hvernig eru {\it felupunktur}  og {\it felugildi} notuð til
    þess að meðhöndla blandað jaðarskilyrði $\alpha_1 u+\alpha_2
    \partial u/\partial n=\alpha_3$ í jaðarpunkti svæðis $R$ í plani
    og hvernig verður nálgunarjafnan í þeim punkti? 
  \end{enumerate}



\end{document}
