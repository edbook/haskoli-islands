\documentclass[tikz,border=10pt,multi]{standalone}

\usepackage[icelandic]{babel}
\usepackage[utf8x]{inputenc}
\usepackage[T1]{fontenc}

\usepackage{mhchem}

\begin{document}

\begin{tikzpicture}[scale=0.8]
  \draw (-0.5,-0.5) -- (-0.5,6.1) -- (4.5,6.1) -- (4.5,-0.5) -- (-0.5,-0.5);

  \node at (0,5.6) {$1s$};
  \node at (0,4.😎 {$2s$};
  \node at (0,4) {$3s$};
  \node at (0,3.2) {$4s$};
  \node at (0,2.4) {$5s$};
  \node at (0,1.6) {$6s$};
  \node at (0,0.😎 {$7s$};
  \node at (0,0) {$8s$};

  \node  at (0.8,4.😎 {$2p$};
  \node  at (0.8,4) {$3p$};
  \node  at (0.8,3.2) {$4p$};
  \node  at (0.8,2.4) {$5p$};
  \node  at (0.8,1.6) {$6p$};
  \node  at (0.8,0.😎 {$7p$};
  \node  at (0.8,0) {$8p$};

  \node  at (1.6,4) {$3d$};
  \node  at (1.6,3.2) {$4d$};
  \node  at (1.6,2.4) {$5d$};
  \node  at (1.6,1.6) {$6d$};
  \node  at (1.6,0.😎 {$7d$};
  \node  at (1.6,0) {$8d$};

  \node  at (2.4,3.2) {$4f$};
  \node  at (2.4,2.4) {$5f$};
  \node  at (2.4,1.6) {$6f$};
  \node  at (2.4,0.😎 {$7f$};
  \node  at (2.4,0) {$8f$};

  \node  at (3.2, 2.4) {$5g$};
  \node  at (3.2, 1.6) {$6g$};
  \node  at (3.2, 0.😎 {$7g$};
  \node  at (3.2, 0) {$8g$};

  \node  at (4,1.6) {$6h$};
  \node  at (4,0.😎 {$7h$};
  \node  at (4, 0 ) {$8h$};

  \draw [->, dashed] (0.4,6) -- (-0.4,5.2);
  \draw [->, dashed] (0.4,5.2) -- (-0.4,4.4);
  \draw [->, dashed] (1.2,5.2) -- (-0.4,3.6);
  \draw [->, dashed] (1.2,4.4) -- (-0.4,2.8);
  \draw [->, dashed] (2,4.4) -- (-0.4,2);
  \draw [->, dashed] (2,3.6) -- (-0.4,1.2);
  \draw [->, dashed] (2.8,3.6) -- (-0.4,0.4);
  \draw [->, dashed] (2.8,2.😎 -- (-0.4,-0.4);
  \draw [->, dashed] (3.6,2.😎 -- (0.4,-0.4);

  \draw [red] (0.8,2.4 ) circle [radius=0.3] node [anchor = north west] {\small [Xe]};
  \draw [red] (0.8,1.6) circle [radius=0.3] node [anchor = north west] {\small [Rn]};
  \draw [red] (0.8,0.😎 circle [radius=0.3] node [anchor = north west] {\small [118]};
  \draw [red] (0.8,0 ) circle [radius=0.3] node [anchor = north west] {\small [168]};


\end{tikzpicture}

\end{document}