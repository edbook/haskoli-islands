\def \kaflanr {26}
\lecture[\kaflanr]{\kaflanr. Setning Stokes}{lecture-text}
\date{8.~apríl 2015}
\newcounter{mycount}
\refstepcounter{mycount}

\begin{document}

\begin{frame}
	\maketitle
\end{frame}




\begin{frame}{} 

\begin {block}{Skilgreining \rtask{}}
Látum $\cal S$ vera áttanlegan flöt sem er
reglulegur á köflum með
jaðar $\cal C$ og einingarþver\-vigrasvið $\Nv$.  Áttun $\cal C$ út frá
$\Nv$ finnst með að hugsa sér að gengið sé eftir $\cal C$
þannig að skrokkurinn vísi í stefnu $\Nv$ og göngustefnan sé valin 
þannig að flöturinn sé á vinstri hönd. 
\end{block}

\end{frame}

\begin{frame}{} 

\begin {block}{Setning \rtask{} (Setning Stokes)}
Látum $\cal S$ vera áttanlegan flöt
sem er reglulegur á köflum og látum $\Nv$ tákna einingarþvervigrasvið
á $\cal S$.  Táknum með $\cal C$ jaðar $\cal S$ og
áttum $\cal C$ með tilliti til $\Nv$.    
Ef $\Fv$ er samfellt diffranlegt vigursvið
skilgreint á opnu mengi sem inniheldur $\cal S$  þá er  
$$\tvint_{\cal S} \curl\Fv\cdot\Nv\,dS=\oint_{\cal C}\Fv\cdot \Tv\,ds.$$

\end{block}

\end{frame}


\begin{frame}{} 

\begin {block}{Setning \rtask{}}
 Látum $\Fv$ vera samfellt
diffranlegt vigursvið skilgreint á opnu mengi $D$ í $\R^3$.    Látum
$P$ vera punkt á skilgreiningarsvæði $\Fv$ og $C_\epsilon$ vera
hring með miðju í $P$ og geisla $\epsilon$.  Látum $\Nv$ vera
einingarþvervigur á planið sem hringurinn liggur í.  Áttum hringinn
jákvætt.
Þá er
$$\Nv\cdot\curl \Fv(P)=\lim_{\epsilon\rightarrow 0^+}
\frac{1}{\pi\epsilon^2}\oint_{C_\epsilon}\Fv\cdot d\rv.$$
   
\end{block}

\end{frame}


\begin{frame}{} 

\begin {block}{Setning \rtask{}}
Látum $\cal S$ vera lokaðan flöt sem er
reglulegur á köflum.  Táknum með $D$ rúmskikann sem $\cal S$ umlykur.
Látum $\Nv$ vera einingarþvervigrasvið á $\cal S$   sem vísar út úr
$D$.  Ef $\Fv$ er samfellt diffranlegt vigursvið skilgreint á opnu
mengi sem inniheldur $D$, 
þá er 
$$\oiint_{\cal S}\curl\Fv\cdot\Nv\,dS=0.$$
\end{block}

\end{frame}

\end{document}