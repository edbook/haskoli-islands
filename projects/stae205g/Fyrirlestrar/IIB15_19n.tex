\def \kaflanr {19}
\lecture[\kaflanr]{\kaflanr. Ferilheildi og stigulsvið}{lecture-text}
\date{9.~mars 2015}


\begin{document}

\subsection{}
\maketitle





\subsection{} 

\subsubsection{Setning \rtask{}}
Látum $\Fv(x,y)$ vera samfellt stigulsvið skilgreint á svæði $D$ í $\R^2$ og látum $\phi$ vera fall skilgreint á $D$ þannig að $\Fv(x,y)=\nabla \phi(x,y)$ fyrir alla punkta $(x,y)\in D$.   Látum $\rv:[a,b]\rightarrow D$ vera stikaferill sem er samfellt diffranlegur á köflum og stikar feril $\cal C$ í $D$.  Þá er
$$\int_{\cal C} \Fv\cdot \,d\rv=\phi(\rv(b))-\phi(\rv(a)).$$
(Samsvarandi gildir fyrir vigursvið skilgreint á svæði $D\subseteq \R^3$.)





\subsection{} 

\subsubsection{Fylgisetning \rtask{} \label{thm:atob}}
Látum $\Fv$ vera samfellt stigulsvið
skilgreint á mengi $D\subseteq \R^2$.   Látum $\rv:[a,b]\rightarrow D$ vera
stikaferil sem er samfellt diffranlegur á köflum og lokaður (þ.e.a.s.\
$\rv(a)=\rv(b)$) og stikar feril $\mathcal{C}$.  Þá er $$\oint_{\cal C}  \Fv\cdot \,d\rv=0.$$

(Ath.~að rithátturinn $$\oint_{\cal C}$$ er gjarnan notaður þegar heildað er yfir lokaðan feril $\cal C$.)




\subsection{} 

\subsubsection{Fylgisetning \rtask{}}
Látum $\Fv$ vera samfellt stigulsvið
skilgreint á mengi $D\subseteq \R^2$.   Látum
$\rv_1:[a_1,b_1]\rightarrow D$ og $\rv_2:[a_2,b_2]\rightarrow D$ vera
stikaferla sem eru samfellt diffranlegir á köflum og stika ferlana $\mathcal{C}_1$ og $\mathcal{C}_2$.  Gerum ráð fyrir
að $\rv_1(a_1)=\rv_2(a_2)$ og $\rv_1(b_1)=\rv_2(b_2)$, þ.e.a.s.\ stikaferlarnir $\rv_1$ og $\rv_2$ hafa sameiginlega upphafs- og endapunkta.   Þá er  
$$\int_{{\cal C}_1} \Fv\cdot\,d\rv_1=\int_{{\cal C}_2} \Fv\cdot\,d\rv_2.$$ 
 





\subsection{} 

\subsubsection{Skilgreining \rtask{}}
 Segjum að heildi vigursviðs $\Fv$ sé {\em
  óháð stikaferli} ef fyrir sérhverja tvo samfellt diffranlega á
köflum stikaferla $\rv_1$ og $\rv_2$ með sameiginlega upphafs- og
endapunkta sem stika ferlana $\mathcal{C}_1$ og $\mathcal{C}_2$ gildir að  
$$\int_{{\cal C}_1} \Fv\cdot\,d\rv_1=
\int_{{\cal C}_2} \Fv\cdot\,d\rv_2.$$ 






\subsection{} 

\subsubsection{Setning \rtask{thm:beqc}}
  Ferilheildi samfellds vigursviðs $\Fv$ er óháð
stikaferli ef og aðeins ef $\oint_{\cal C} \Fv\cdot\,d\rv=0$ fyrir alla
lokaða ferla $\cal C$ sem eru samfellt diffranlegir á köflum. 







\subsection{} 
\subsubsection{Skilgreining \rtask{}}
   Segjum að mengi $D\subseteq \R^2$ sé {\em
  ferilsamanhangandi} (e. connected, path-connected)  ef fyrir
  sérhverja tvo punkta $P, Q\in D$ gildir 
að til er stikaferill $\rv:[0,1]\rightarrow D$ þannig að $\rv(0)=P$ og
$\rv(1)=Q$.

\bigskip
(Athugasemd:  Í bók er orðið {\em connected} notað fyrir hugtakið {\em
  ferilsamanhangandi}.  Venjulega er orðið {\em connected} notað yfir
  annað hugtak, skylt en samt ólíkt.)






\subsection{} 

\subsubsection{ Setning \rtask{thm:ctoa} }
 Látum $D$ vera opið mengi í $\R^2$ sem er ferilsamanhangandi.  Ef $\Fv$ er samfellt vigursvið skilgreint á $D$ og ferilheildi $\Fv$ eru óháð vegi þá er $\Fv$ stigulsvið.
    






\subsection{} 

\subsubsection{Setning \rtask{}}
 Fyrir samfellt vigursvið $\Fv$ skilgreint á opnu 
ferilsamanhangandi mengi $D\subseteq \R^2$ er eftirfarandi jafngilt:
\begin{itemize}
 \item [(a)] $\Fv$ er stigulsvið,
 \item [(b)]  $\oint_{\cal C} \Fv\cdot\,d\rv=0$ fyrir alla samfellt diffranlega
á köflum lokaða stikaferla $\rv$ í $D$, 
\item [(c)] ferilheildi $\Fv$ er óháð vegi.
\end{itemize}

\pause
\subsubsection{Sönnun: } 
(a) $\Rightarrow$ (b). Fylgisetning \kaflanr.\ref{thm:atob}. \\
(b) $\Leftrightarrow$ (c). Setning \kaflanr.\ref{thm:beqc}. \\
(c) $\Rightarrow$ (a). Setning \kaflanr.\ref{thm:ctoa}.
 




\end{document}