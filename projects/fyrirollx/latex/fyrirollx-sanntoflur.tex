%!TEX root = forallxcam.tex
\part{Sanntöflur}
\label{ch.TruthTables}

\chapter{Skilgreiningarsanntöflur}\label{s:CharacteristicTruthTables}

Í síðasta kafla skilgreindum við nákvæmlega hvað telst vera setning í setningrökfræði og hvað ekki. Við skilgreindum fyrst nákvæmlega hvaða tákn eru leyfileg, hástafir fyrir grunnsetningar, svigar og fimm setningatengi. Svo sögðum við að ekkert væri setning í setningarökfræði nema það væri smíðað úr grunnsetningunum eftir ákveðnum myndunarreglum, eina fyrir hvert setningatengi. Til dæmis er setningin $D \eand E$ samsett úr $D$ og $E$ samkvæmt myndunarreglunni fyrir „$\eand$“ (með það í huga að við sleppum ystu svigunum, eftir venju).

Markmið okkar er að greina rökfærslur með því að þýða þær yfir á táknmál setningarökfræði og næsta skrefið á þeirri leið er að fá nákvæmari útlistun á þessum tengjum. Við munum gera það með því að skilgreina nákvæmlega hvenær setningarnar sem þau mynda eru sannar og hvenær ósannar. Hugmyndin sem við munum útfæra er að sanngildi flókinnar setningar ákvarðist af sanngildi þeirra hluta sem hún er mynduð úr. Til dæmis, þá er setningin „Anna og Jón eru snjöll“ sönn ef og aðeins ef setningin „Anna er snjöll“ er sönn og setningin „Jón er snjall“ er sönn. 

Það sem við þurfum þá að gera er að vinna úr þessari hugmynd fyrir hvert setningatengjanna fimm og búa til reglur sem ákvarða hvenær setningar sem þau koma fyrir í eru sannar. Við gerum þetta með því sem kallað er \define{skilgreiningarsanntöflur}. Þær heita svo af því að þær \emph{skilgreina} undir hvaða kringumstæðum setning er sönn sem mynduð er úr tilteknu setningatengi. Við að fylla út skilgreiningarsanntöflurnar er gott að stytta sér leið og því munum við skrifa „S“ fyrir „Satt“ og „Ó“ fyrir „Ósatt“. 

Það er mikilvægt að hafa í huga að hingað til höfum við kynnst setningatengjunum með því að notast við þekkingu okkar á mæltu máli en að markmiðið núna er að skilgreina \emph{nákvæmlega} hver merking setningatengjanna er. Skilgreiningasanntöflurnar \emph{eru} þessar skilgreiningar.

\paragraph{Neitun.} Ef setningin „Anna er snjöll“ er sönn, hvað getum við þá sagt um neitun hennar, setninguna „Anna er ekki snjöll“? Jú, að hún er ósönn. Það sama gildir í hina áttina: ef setningin er ósönn, þá er neitun hennar sönn. Almennt gildir, fyrir hvaða setningu $\meta{A}$ sem er, að ef $\meta{A}$ er sönn, þá er $\enot \meta{A}$ ósönn, og ef $\enot \meta{A}$ sönn, þá er $\meta{A}$ ósönn. Við getum sett þetta upp í töflu, sem við getum kallað \emph{skilgreiningarsanntöflu} fyrir neitun:
\begin{center}
\begin{tabular}{c|c}
$\meta{A}$ & $\enot\meta{A}$\\
\hline
S & Ó\\
Ó & S 
\end{tabular}
\end{center}

Við lesum úr þessari töflu þannig: vinstra megin undir $\meta{A}$ eru möguleg sanngildi, satt og ósatt. Þar við hliðina á, eitt í hverri línu, eru þau sanngildi sem $\enot \meta{A}$ hefur, að því gefnu hvaða sanngildi $\meta{A}$ hefur. Við getum því lesið út úr töflunni hvaða sanngildi $\enot \meta{A}$ hefur, ef við vitum hvaða sanngildi $\meta{A}$ og hvaða sanngildi $\meta{A}$ hefur, ef við vitum hvaða sanngildi $\enot \meta{A}$ hefur. 

\paragraph{Og-tengi.} Eins og við sögðum hér að ofan, þá er setningin „Anna og Jón eru snjöll“ sönn ef og aðeins ef setningarnar „Anna er snjöll“ og „Jón er snjall“ eru báðar sannar. Það þýðir að ef önnur þeirra er ósönn, þá er öll setningin líka ósönn. Þetta gildir almennt fyrir hvaða setningapar $\meta{A}$ og $\meta{B}$ sem er: $\meta{A} \eand \meta{B}$ er sönn ef og aðeins ef $\meta{A}$ og $\meta{B}$ eru báðar sannar. Skilgreiningarsanntaflan fyrir „$\eand$“ lítur því svona út:

\begin{center}
\begin{tabular}{c c |c}
$\meta{A}$ & $\meta{B}$ & $\meta{A}\eand\meta{B}$\\
\hline
S & S & S\\
S & Ó & Ó\\
Ó & S & Ó\\
Ó & Ó & Ó
\end{tabular}
\end{center}
Við lesum úr þessari töflu á sama hátt. Fyrst höfum við öll möguleg sanngildi fyrir setningarnar tvær: $\meta{A}$ og $\meta{B}$ geta báðar verið sannar, það samsvarar efstu línunni, önnur þeirra getur verið sönn og hin ósönn, það samsvarar annarri og þriðju línu, og svo geta báðar verið ósannar. Það samsvarar neðstu línunni, línu fjögur. Næst sjáum við svo hvert sanngildi allrar setningarinnar, þ.e.\ $\meta{A} \eand \meta{B}$, er fyrir hvern þessara möguleika. Við sjáum að $\meta{A} \eand \meta{B}$ er einungis sönn ef báðir liðir eru sannir, annars ósönn.

Takið eftir því að samtenging er \emph{samhverf}. Sanngildi setningarinnar $\meta{A} \eand \meta{B}$ er alltaf það sama og sanngildi setningarinnar $\meta{B} \eand \meta{A}$.

\paragraph{Eða-tengi.} Munum að táknið „$\eor$“ stendur alltaf fyrir skarað eða. Það þýðir að fyrir hvaða setningar sem er, $\meta{A}$ og $\meta{B}$, setningin $\meta{A} \eor \meta{B}$ er sönn ef og aðeins ef að minnsta kosti \emph{önnur} setninganna $\meta{A}$ eða $\meta{B}$ er sönn. $\meta{A} \eor \meta{B}$ er því bara ósönn ef báðar setningarnar eru ósannar. Skilgreiningarsanntaflan fyrir „$\eor$“ er því svona:

\begin{center}
\begin{tabular}{c c|c}
$\meta{A}$ & $\meta{B}$ & $\meta{A}\eor\meta{B}$\\
\hline
S & S & S\\
S & Ó & S\\
Ó & S & S\\
Ó & Ó & Ó
\end{tabular}
\end{center}
Setningar sem hafa „$\eor$“ sem aðaltengi eru líka samhverfar. $\meta{A} \eor \meta{B}$ er sönn ef og aðeins ef $\meta{B} \eor \meta{A}$ er sönn. Rétt eins og við köllum setningar sem er búnar til með „$\eand$“ \emph{samtengingar}, þá köllum við setningar sem eru búnar til með „$\eor$“ stundum \emph{mistengingar}. Þær eru líka stundum kallaðar „eðanir“ á íslensku---en við munum forðast það orðalag.

\paragraph{Skilyrðistengi.} Næsta skilgreiningarsanntafla er sú fyrir skilyrðistengið. Hún er þannig að $\meta{A} \eif \meta{B}$ er \emph{ósönn} ef og aðeins ef $\meta{A}$ er sönn og $\meta{B}$ er ósönn. Það er að segja, við setjum S í hverja línu í sanntöflunni, nema þar sem $\meta{A}$ er sönn og $\meta{B}$ ósönn:

\begin{center}
\begin{tabular}{c c|c}
$\meta{A}$ & $\meta{B}$ & $\meta{A}\eif\meta{B}$\\
\hline
S & S & S\\
S & Ó & Ó\\
Ó & S & S\\
Ó & Ó & S
\end{tabular}
\end{center}

Þessi sanntafla hefur ýmsar skrýtnar afleiðingar. Við sjáum til dæmis að skilyrðissetning er \emph{alltaf} sönn ef forliðurinn er ósannur. Það þýðir að frá sjónarhóli setningarökfræðinnar er skilyrðissetningin „Ef $2+2=5$, þá er ég hundrað metra hár“ sönn. Það er virðist ósatt, því bakliðurinn hefur ekkert forliðinn að gera. Þessi setning samsvarar línu þrjú í töflunni og við gætum auðveldlega búið til álíka furðulegt dæmi fyrir línu fjögur.

En af hverju er sanntaflan svona? Það hefur að gera með \emph{takmarkanir} setningarökfræðinnar. Sú setningarökfræði sem þessi bók fjallar um er svokölluð \emph{klassísk} setningarökfræði. Það þýðir meðal annars að við gerum ráð fyrir því að hver setning hafi eitt af tveimur sanngildum, satt eða ósatt. Hver einasta setning er því annað hvort sönn eða ósönn og allar setningar verða að hafa eitt af þessum tveimur sanngildum. Við viljum líka að sanngildi setninga ráðist fullkomlega af sanngildum þeirra setninga sem hún er smíðuð úr. Það væri hægt að reyna að slaka á þessum kröfum---og hefur verið gert, en staðreyndin er sú að þær tillögur hafa líka alvarlega galla, auk þess að vera mun flóknari. Einhvers staðar verður maður líka að byrja og til þess að geta tekið afstöðu í þessum heimspekilegu deilumálum er nauðsynlegt að kunna góð skil á klassískri setningarökfræði fyrst. Í köflum \S\ref{s:IndicativeSubjunctive} og \S\ref{s:ParadoxesOfMaterialConditional} munum við ræða sum af þeim álitamálum sem styrr stendur um.

Ef við fyllum svo sanntöfluna út línu fyrir línu, þá sjáum við af hverju hún hlýtur að vera svona. Látum $\meta{A} \eif \meta{B}$ standa fyrir setninguna „Ef þessi fugl er hrafn, þá er hann svartur“. Ef bakliðurinn og forliðurinn eru báðir sannir, þ.e.\ ef fuglinn er hrafn og hann er svartur, þá er setningin ljóslega sönn---að minnsta kosti væri skrýtið að segja að hún sé þar með \emph{ósönn}. 

Lína tvö virðist líka í lagi. Ef bakliðurinn er ósannur, þ.e.\ ef þessi fugl er \emph{ekki} svartur, þá virðist eðlilegt að segja að skilyrðissetningin sé þar með afsönnuð, og því ósönn. Við getum líka litið á skilyrðissetninguna sem \emph{loforð}. Ef ég segði við þig: „Ef þú stendur þig vel í rökfræði, þá býð ég þér í bíó“ og þú stæðir þig svo vel, þá væri eðlilegt að segja að ég hafi \emph{svikið} loforðið ef ég byði þér svo \emph{ekki} í bíó. Það væri á sama hátt eðlilegt að segja að ég hafi \emph{staðið} við það, ef ég myndi bjóða þér í bíó. Það samsvarar línu eitt.

Hvað með línur þrjú og fjögur? Skoðum fyrst línu fjögur. Hér eru bæði forliðurinn og bakliðurinn ósannir. Það samsvarar því að fuglinn sé hvorki hrafn né svartur. Kannski er hann hvítur svanur. Nú höfum við tvo möguleika. Ef við segjum að setningin sé ósönn, þá hefði það í för með sér að tilvist hvítra svana \emph{afsanni} þá skilyrðissetningu að ef fuglinn sé hrafn, þá sé hann svartur. Það væri verra en sá valkostur að segja einfaldlega að setningin sé sönn, og við verðum að gera annað hvort. Þetta passar líka ágætlega við hugmyndina um skilyrðissetningar sem loforð: ef þú stendur þig illa, þá væri ég augljóslega ekki að brjóta loforðið um að bjóða þér í bíó \emph{ef þú stendur þig vel}, ef ég byði þér \emph{ekki} í bíó.

Lína þrjú er svipuð. Ef fuglinn er svartur og ekki hrafn, þá virðumst við líka nauðbeygð til að segja að setningin sé sönn. Ef við segðum að hún væri ósönn, þá myndi tilvist svartra fugla annarra en hrafna afsanna skilyrðissetninguna að ef hann sé hrafn, þá sé hann svartur. En af hverju ætti það að fuglinn sé kráka að sýna að það sé ósatt að ef hann er hrafn, þá sé hann svartur? Það virðist mun verri kostur en að segja bara að setningin sé sönn. Sama gildir ef við lítum á skilyrðissetninguna sem loforð. Ef ég lofa að bjóða þér í bíó ef þú stendur þig vel, þá væri skrýtið að segja að ég hafi \emph{svikið} loforðið ef ég býð þér samt. Það eina sem loforðið segir er að \emph{ef} þú stendur þig vel, \emph{þá} býð ég þér í bíó. Það segir ekkert um hvað gerist ef þú stendur þig illa.

Það er líklega auðveldast að leggja þessa sanntöflu á minnið ef maður lítur á skilyrðissetningar sem loforð: þær eru bara ósannar ef það sem þær „lofa“ rættist ekki. Við myndum einmitt segja að ef ég lofa því að bjóða þér í bíó ef þú stendur þig vel, og geri það svo ekki, þá hafi ég svikið loforðið, annars ekki. Það er þá eina línan þar sem skilyrðissetningin er ósönn, hinar eru allar sannar. Að hugsa um skilyrðistengið svona sýnir líka að þessi skilgreiningarsanntafla er kannski ekki alveg jafn slæm og maður myndi annars halda.

Skilyrðissetningar eru \emph{ekki samhverfar}. Það er ekki hægt að víxla forlið og baklið án þess að breyta þar með merkingu setningarinnar. $\meta{A} \eif \meta{B}$ og $\meta{B} \eif \meta{A}$ hafa ólíkar sanntöflur, enda merkir setningin „ef þú stendur þig vel, þá býð ég þér í bíó“ ekki það sama og setningin „ef ég býð þér í bíó, þá stendur þú þig vel“.

\paragraph{Jafngildistengi.} Jafngildissetningar eru í raun samtenging skilyrðissetninga sem ganga í báðar áttir: $(\meta{A} \eif \meta{B}) \eand (\meta{B} \eif \meta{A})$. Þær eru þá sannar ef og aðeins ef báðar setningarnar eru sannar, það er að segja, sannar ef báðar setningarnar eru sannar og ósannar ef báðar setningarnar eru ósannar. Jafngildistengi er því satt þegar báðar setningarnar hafa sama sanngildi, en annars ósannar. Skilgreiningarsanntaflan fyrir jafngildistengið er því svona:

\begin{center}
\begin{tabular}{c c|c}
$\meta{A}$ & $\meta{B}$ & $\meta{A}\eiff\meta{B}$\\
\hline
S & S & S\\
S & Ó & Ó\\
Ó & S & Ó\\
Ó & Ó & S
\end{tabular}
\end{center}
Eins og við sjáum, þá er jafngildistengið samhverft: $\meta{A} \eiff \meta{B}$ er það sama og $\meta{B}  \eiff \meta{A}$.

\chapter{Sannföll}\label{s:TruthFunctionality}

\section{Hvað eru sannföll?}

Eftirfarandi er mikilvæg hugmynd í rökfræði:
	\factoidbox{Setningatengi er \define{sannfall} eff sanngildi setningarinnar þar sem tengið er aðaltengi er fullkomlega ákvarðað af sanngildum setninganna sem það tengir.}
Öll setningatengin í setningarökfræði eru sannföll. Sanngildi neitunar er ákvarðað fullkomlega af sanngildi þeirrar setningar sem neitað er. Við þurfum ekki að vita neitt annað til að vita sanngildið. Það sama gildir um hin setningatengin. Sanngildi samtengingar er fullkomlega ákvarðað af sanngildi setninganna sem það tengir og sanngildi mistengis (þ.e.\ setningar sem hefur „$\eor$“ sem aðaltengi) er fullkomlega ákvarðað af sanngildi setninganna sem það tengir, o.s.frv. Til þess að vita sanngildi setningar í setningarökfræði er nóg að vita sanngildi setninganna sem hún er smíðuð úr.
	
Almennt er þetta ekki svona í mæltu máli. Til dæmis getum við búið til nýja setningu á íslensku úr öðrum setningum með því að setja „Það er nauðsynlega satt að\ldots“ fyrir framan þær. Sanngildi slíkrar setningar er \emph{ekki} fullkomlega ákvarðað af sanngildi setningarinnar sem hún var búin til úr. Skoðum tvö dæmi:
	\begin{earg}
		\item $2 + 2 = 4$
		\item Halldór Laxness skrifaði fjórtán skáldsögur.
	\end{earg}
Þessar setningar eru báðar sannar, en þó að það sé nauðsynlega satt að $2 + 2 = 4$, þá er það \emph{ekki} nauðsynlega satt að Halldór Laxness hafi skrifað fjórtán skáldsögur. Það hefði til dæmis vel getað gerst að seinni heimsstyrjöldin hefði komið í veg fyrir að hann lyki við Íslandsklukkuna, og þá hefði hann bara skrifað þrettán skáldsögur. Það er því ekki nóg að vita bara sanngildi setningarinnar sem „Það er nauðsynlega satt að\ldots“ er skeytt við til að vita sanngildi setningarninnar sem verður til við slíka skeytingu. „Það er nauðsynlega satt að\ldots“ er því ekki sannfall.

\section{Þýðingar yfir á táknmál setningarökfræði}

Öll setningatengi setningarökfræðinnar eru sannföll. En í raun eru þau heldur ekkert meira en það: þau segja okkur bara hvert er sanngildi setningar ef við vitum sanngildi annarra setninga eða setningar, nefnilega hlutasetninganna sem setningin samanstendur af.

Þegar við þýðum setningu yfir á táknmál setningarökfræði þá einblínum við á sanngildi hlutasetninganna sem mynda setninguna og \emph{hunsum} allt annað. En í mæltu máli er margt annað hluti af merkingu setningarinnar, til dæmis kaldhæðni, ljóðrænn blær, áhersla, eða að eitthvað sé gefið í skyn. Þetta eru allt mikilvægir hlutir við hversdagslega notkun tungumálsins. En setningarökfræðin er algjörlega blind á slík litbrigði málsins og allt nema sanngildið glatast við slíka þýðingu. Skoðum eftirfarandi setningar sem dæmi:%\note{Ég er auðvitað að vísa í þekktan málshátt, en þekkir einhver þetta orð í dag}
	\begin{earg}
		\item Anna er smá og kná.
		\item Þó að Anna sé smá, þá er hún kná.
		\item Þrátt fyrir að vera smá, þá er Anna kná.
		\item Anna er smá, en kná.
		\item Þrátt fyrir smæðina, þá er Anna samt kná.
	\end{earg}
Þessar setningar yrðu allar þýddar yfir á táknmál setningarökfræði með sama hætti, kannski sem $S \eand K$.

Það er því mikilvægt að taka allt tal um „þýðingar“ yfir á táknmál setningarökfræði ekki of hátíðlega. Almennt segjum við að góð þýðing sé sú sem fangar sem flest blæbrigði og hughrif þess sem þýtt er, en táknmál setningarökfræði er ekki í stakk búið til þess. Það eina sem skiptir okkur máli er sanngildið.

Þetta hefur áhrif á það hvernig best er að skilja þýðingarlykla. Tökum sem dæmi:
	\begin{ekey}
		\item[S] Anna er smá.
		\item[K] Anna er kná.
	\end{ekey}
Þegar við segjum að við notum þennan þýðingarlykil til að \emph{þýða} setningu yfir á táknmál setningarökfræði, þá ættum við ekki að skilja það sem svo að \emph{merking} setningastafanna sé sú sama og merking setninganna. Það sem við erum að gera er að segja að \emph{sanngildi} setningastafanna eigi að vera það sama og sanngildi setninganna sem þeir þýða. Með þessum þýðingarlykli erum við því að segja að grunnsetningin $S$ eigi að vera sönn ef Anna er smá, og ósönn annars, og að grunnsetningin $K$ eigi að vera sönn ef Anna er kná, og ósönn annars.
	\factoidbox{
	Þegar við \emph{þýðum} setningu á mæltu máli yfir á táknmál setningarökfræði, þá erum við að tilgreina sanngildi fyrir þá setningu.
	}

\section{Framsöguháttur og viðtengingarháttur}\label{s:IndicativeSubjunctive}
Til að hnykkja enn frekar á því að setningarökfræðin fáist aðeins við sannföll, ætla ég að segja nokkur orð í viðbót um skilyrðissetningar. Þegar ég kynnti skilgreiningarsanntöfluna fyrir skilyrðistengið til sögunnar, þá reyndi ég að sýna fram á að sanntaflan væri í raun og veru vel valin. Ég ætla að byrja á því að fara yfir eitt dæmi til viðbótar. Dæmið er tekið frá Dorothy Edgington.\footnote{Dorothy Edgington, „Conditionals, 2014, í \emph{Stanford Encyclopedia of Philosophy} (\url{http://plato.stanford.edu/entries/conditionals/}).} 

Segjum að Kristín vinkona mín hafi teiknað nokkur form á blað og litað sum þeirra. Ég hef ekki séð neitt þeirra, en segi samt:
	\begin{quote}
		Ef nokkurt form er grátt, þá er það form líka hringlaga.
	\end{quote}
Það vill svo til að Kristín hefur teiknað eftirfarandi form:	
\begin{center}
\begin{tikzpicture}
	\node[circle, grey_shape] (cat1) {A};
	\node[right=10pt of cat1, diamond, phantom_shape] (cat2)  { } ;
	\node[right=10pt of cat2, circle, white_shape] (cat3)  {C} ;
	\node[right=10pt of cat3, diamond, white_shape] (cat4)  {D};
\end{tikzpicture}
\end{center}
Núna er það sem ég sagði ljóslega satt. Form C og D eru ekki grá, og geta því varla verið \emph{mótdæmi} við þá fullyrðingu að ef eitthvað sé grátt, þá sé það hringlaga. A \emph{er} grátt, en svo vill til að það er líka hringlaga. Kristín getur því ekki bent á nein dæmi sem ganga gegn því sem ég sagði og því liggur beinast við að segja að það sem ég sagði hafi verið satt. Það þýðir líka að eftirfarandi setningar eru líka sannar:
	\begin{ebullet}
		\item Ef A er grátt, þá er það hringlaga \hfill (sannur forliður, sannur bakliður)
		\item Ef C er grátt, þá er það hringlaga \hfill (ósannur forliður, sannur bakliður)
		\item Ef D er grátt, þá er það hringlaga \hfill (ósannur forliður, ósannur bakliður)
	\end{ebullet}
Segjum svo að Kristín teikni eitt form í viðbót, svona:	

\begin{center}
\begin{tikzpicture}
	\node[circle, grey_shape] (cat1) {A};
	\node[right=10pt of cat1, diamond, grey_shape] (cat2)  {B};
	\node[right=10pt of cat2, circle, white_shape] (cat3)  {C};
	\node[right=10pt of cat3, diamond, white_shape] (cat4)  {D};
\end{tikzpicture}
\end{center}
Núna er það sem ég sagði ósatt, að ef form er grátt, þá er það hringlaga. Eftirfarandi fullyrðing er því líka ósönn:
	\begin{ebullet}
		\item Ef B er grátt, þá er það hringlaga \hfill (sannur forliður, ósannur bakliður)
	\end{ebullet}
Við munum að öll setningatengi í setningarökfræði eiga að vera sannföll. Það þýðir að ekkert nema sanngildi for- og bakliðar ákvarðar sanngildi skilyrðissetningarinnar sem þeir mynda. Við getum því séð af þessum fjórum dæmum hver skilgreiningarsanntafla skilyrðistengisins hlýtur að vera, enda eru þetta allir möguleikarnir, einn fyrir hverja línu í sanntöflunni. 
	
Þetta dæmi sýnir, með öðrum orðum, að setningatengið „$\eif$“ sem við skilgreindum hér að ofan með skilgreiningarsanntöflu hefði ekki getað verið öðruvísi. Þetta setningatengi er \emph{besta skilyrðistengið sem setningarökfræðin hefur upp á að bjóða}. En hversu vel virkar það sem þýðing á skilyrðissetningum sem við notum í mæltu máli? Skoðum tvö dæmi:
	\begin{earg}
		\item[\ex{brownwins1}] Ef Halla Tómasdóttir hefði farið með sigur af hólmi í forsetakosningunum árið 2016, þá hefði hún orðið önnur konan til að gegna embætti forseta.
		\item[\ex{brownwins2}] Ef Halla Tómasdóttir hefði farið með sigur af hólmi í forsetakosningunum árið 2016, þá hefði hún breyst í snjókarl.
	\end{earg}
Setning \ref{brownwins1} er sönn; setning \ref{brownwins2} er ósönn. En báðar setninganna hafa ósanna forliði og ósanna bakliði (Halla Tómasdóttir vann ekki; hún varð ekki önnur konan til að gegna embætti forseta; og við ættum að geta slegið því föstu að hún hefði ekki breyst í snjókarl ef svo hefði verið). Það sýnir að sanngildi \ref{brownwins2} í heild er ekki fullkomlega ákvarðað af sanngildum hlutasetninganna.

Það sem skiptir mestu máli hér er að setningar \ref{brownwins1} og \ref{brownwins2} eru í \emph{viðtengingarhætti}, fremur en \emph{framsöguhætti}. Þegar við setjum fram þessa hugsun, um hvað hefði gerst ef Halla Tómasdóttir hefði fengið flest atkvæði í kosningunum, þá erum við að ímynda okkur eitthvað sem gerðist ekki og svo ímynda okkur eitthvað annað sem \emph{hefði} þá líka gerst. Slíkt ræður „$\eif$“ einfaldlega ekki við.

Við munum segja meira um vandkvæðin sem eru bundin skilyrðissetningum í \S\ref{s:ParadoxesOfMaterialConditional}. Þangað til, þá verðum við að sætta okkur við að „$\eif$“ er eina mögulega sannfallið sem gegnt getur hlutverki skilyrðistengis í setningarökfræði, en á sama tíma að til séu setningar í mæltu máli sem ekki er hægt að þýða með því að nota það. Setningarökfræðin er að þessu leyti takmörkuð og við getum því ekki hugsunarlaust gert ráð fyrir því að allar setningar sem verða á vegi okkar sé hægt að þýða yfir á táknmál hennar, svo vel sé.

\chapter{Fullar sanntöflur}\label{s:CompleteTruthTables}

Hingað til höfum við notað þýðingarlykla til að tiltaka sanngildi setninga í setningarökfræði \emph{óbeint}. Með því að láta grunnsetninguna „$A$“ standa til dæmis fyrir setninguna „Almannagjá er á Þingvöllum“ þá höfum við þar með sagt að grunnsetningin „$A$“ eigi að vera sönn ef og aðeins ef Almannagjá er á Þingvöllum. Almannagjá \emph{er} á Þingvöllum, svo grunnsetningin „$A$“ er sönn samkvæmt þessum þýðingarlykli. En við getum líka tiltekið sanngildi grunnsetninga \emph{beint}. Við getum ákveðið, ef við viljum, að grunnsetningin „$A$“ sé sönn án þess að blanda þýðingarlyklum í málið, ef þess er ekki þörf (nú eða að hún sé ósönn, ef það hentar okkur betur). Við getum \emph{úthlutað} grunnsetningum sanngildum að vild.

Ef við ákveðum tiltekin sanngildi fyrir \emph{allar} grunnsetningar í setningu, þá köllum við slíka úthlutun \emph{sanngildadrefingu}:
	\factoidbox{Tiltekin úthlutun sanngilda (satt eða ósatt) á allar grunnsetningar í setningu eða setningum kallast \define{sanngildadreifing}.
	}
	
Í þessu liggur styrkur sanntafla. Hver einasta lína í fullri sanntöflu stendur fyrir mögulega sanngildadreifingu og því stendur sanntaflan sjálf fyrir allar mögulegar sanngildadreifingar. Við getum því notað sanntöflur til að reikna út sanngildi samsettra setninga fyrir allar mögulegar sanngildadreifingar. En þetta má kannski best sjá með dæmi.	
	
\section{Dæmi um sanntöflu}

Tökum setninguna $(H\eand I)\eif H$ sem dæmi. Hægt er að úthluta sanngildunum „satt“ og „ósatt“ á þessar grunnsetningar á fjóra vegu: „$H$“ og „$I$“ geta báðar verið sannar, önnur þeirra getur verið ósönn eða þær geta báðar verið ósannar. Það eru því fjórar mögulegar sanngildadreifingar fyrir þessar tvær grunnsetningar. Við getum táknað þær svona:

\begin{center}
\begin{tabular}{c c|d e e e f}
$H$&$I$&$(H$&\eand&$I)$&\eif&$H$\\
\hline
 S & S\\
 S & Ó\\
 Ó & S\\
 Ó & Ó
\end{tabular}
\end{center}
Til þess að reikna út sanngildi samsettu setningarinnar $(H \eand I) \eif H$, þá byrjum við á því að afrita sanngildin úr dálkinum vinstra megin fyrir hvern setningarstaf og skrifum þau beint fyrir neðan þann staf í samsettu setningunni hægra megin:

\begin{center}
\begin{tabular}{c c|d e e e f}
$H$&$I$&$(H$&\eand&$I)$&\eif&$H$\\
\hline
 S & S & {S} & & {S} & & {S}\\
 S & Ó & {S} & & {Ó} & & {S}\\
 Ó & S & {Ó} & & {S} & & {Ó}\\
 Ó & Ó & {Ó} & & {Ó} & & {Ó}
\end{tabular}
\end{center}

Skoðum núna hlutasetninguna $(H\eand I)$. Þetta er samtenging á forminu $\meta{A} \eand \meta{B}$ þar sem $H$ hefur hlutverk $\meta{A}$ og $I$ hefur hlutverk $\meta{B}$. Skilgreiningasanntaflan fyrir „$\eand$“ segir okkur nákvæmlega hvenær \emph{hvaða} setning sem er á þessu formi er sönn og hvenær ósönn, sama hvað $\meta{A}$ og $\meta{B}$ eru. Samkvæmt henni er samtenging sönn ef og aðeins ef báðir liðir hennar eru sannir. Í þessu tilfelli eru liðirnir grunnsetningarnar $H$ og $I$. Við sjáum að þær eru bara sannar á fyrstu línu sanntöflunnar. Þá getum við skrifað niður sanngildi samtengingar þeirra, $(H\eand I)$, á öllum fjórum línum sanntöflunnar.

\begin{center}
\begin{tabular}{c c|d e e e f}
 & & $\meta{A}$ & \eand & $\meta{B}$ & & \\
$H$&$I$&$(H$&\eand&$I)$&\eif&$H$\\
\hline
 S & S & S & {S} & S & & S\\
 S & Ó & S & {Ó} & Ó & & S\\
 Ó & S & Ó & {Ó} & S & & Ó\\
 Ó & Ó & Ó & {Ó} & Ó & & Ó
\end{tabular}
\end{center}

Nú erum við búin að fylla út hluta sanntöflunnar. Munum að setningin sem við erum að skoða er setning sem hefur „$\eif$“ sem aðaltengi, $\meta{A} \eif \meta{B}$, þar sem $(H \eand I)$ hefur hlutverk $\meta{A}$ og $H$ hefur hlutverk $\meta{B}$. Við vitum með því að skoða skilgreiningartöfluna fyrir skilyrðistengið að skilyrðissetning er sönn þegar forliðurinn er ósannur. Því getum við skrifað „S“ í línu tvö, þrjú og fjögur undir tákninu fyrir skilyrðistengið, „$\eif$“, ( en forliðurinn er ósannur í öllum þessum línum). Þá er eftir lína eitt, og ef við kíkjum á skilgreiningarsanntöfluna fyrir skilyrðistengið, þá sjáum við að þar ættum við líka að setja „S“, enda eru báðar setningarnar sannar þar og þá er heildin líka sönn, samkvæmt töflunni.

Þá fáum við:

\begin{center}
\begin{tabular}{c c| d e e e f}
 & &  & $\meta{A}$ &  &\eif &$\meta{B}$ \\
$H$&$I$&$(H$&\eand&$I)$&\eif&$H$\\
\hline
 S & S &  & {S} &  &{S} & S\\
 S & Ó &  & {Ó} &  &{S} & S\\
 Ó & S &  & {Ó} &  &{S} & Ó\\
 Ó & Ó &  & {Ó} &  &{S} & Ó
\end{tabular}
\end{center}
Skilyrðistengið ($„\eif“$) er aðaltengi setningarinnar. Dálkurinn undir skilyrðistenginu sýnir okkur því að setningin $(H \eand I)\eif H$ er alltaf sönn, sama hvaða sanngildi $H$ og $I$ hafa. Grunnsetningarnar geta verið sannar eða ósannar, í hvaða samsetningu sem er, og samsetta setningin verður alltaf sönn. Þetta þýðir að það skiptir ekki máli hvernig við úhlutum sanngildum á grunnsetningarnar, setningin $(H \eand I)\eif H$ er alltaf sönn. Við segjum þá að hún sé sönn fyrir allar mögulegar sanngildadreifingar.

Í dæminu hér að ofan skrifaði ég ekki hvert einasta sanngildi undir hverju einasta setningatengi. Það var til þess að auðveldara væri að lesa töfluna og sjá hvert aðaltengið er og hvaða sanngildi liggja undir því. En þegar maður skrifar út sanntöflur með blaði og penna, þá er ekki mjög praktískt að stroka út sanngildi sem maður hefur áður skrifað eða að skrifa út nýja töflu fyrir hvert skref. Það er því líka hægt að skrifa töfluna svona:

\begin{center}
\begin{tabular}{c c| d e e e f}
$H$&$I$&$(H$&\eand&$I)$&\eif&$H$\\
\hline
 S & S & S & {S} & S & \TTbf{S} & S\\
 S & Ó & S & {Ó} & Ó & \TTbf{S} & S\\
 Ó & S & Ó & {Ó} & S & \TTbf{S} & Ó\\
 Ó & Ó & Ó & {Ó} & Ó & \TTbf{S} & Ó
\end{tabular}
\end{center}

Sá dálkur sem mestu máli skiptir---og hinir eru bara notaðir til að reikna út---er sá sem er undir \emph{aðaltengi} setningarinnar. Hann segir okkur hvert sanngildi setningarinnar í heild er, og þess vegna er hann feitletraður hér. Þegar maður handskrifar svona töflu er oft gott að gera eitthvað svipað, t.d.\ með að strika undir dálkinn, eða eitthvað slíkt.

\section{Að fylla út sanntöflur}

\define{Full sanntafla} hefur eina línu fyrir hverja mögulega sanngildadreifingu grunnsetninganna, þ.e.\ eina línu fyrir hverja úthlutun á sanngildunum „satt“ og „ósatt“ á grunnsetningarnar. Hver lína stendur því fyrir eina mögulega sanngildadreifingu og full tafla hefur eina línu fyrir hverja mögulega dreifingu.

Stærð sanntöflunnar ræðst því af fjölda grunnsetninga í setningunni sem verið er að skoða. Ef setning inniheldur einungis eina grunnsetningu, þá þarf tvær línur, rétt eins og í skilgreiningarsanntöflunni fyrir neitun. Það skiptir engu þó að sami stafurinn sé endurtekinn oft, til dæmis í setningunni $[(C\eiff C) \eif C] \eand \enot(C \eif C)$. Full tafla fyrir þessa setningu er bara tvær línur, því það eru bara tvær mögulegar sanngildadreifingar: $C$ getur verið sönn eða ósönn. Full sanntafla fyrir þessa setningu lítur svona út:
\begin{center}
\begin{tabular}{c| d e e e e e e e e e e e e e e f}
$C$&$[($&$C$&\eiff&$C$&$)$&\eif&$C$&$]$&\eand&\enot&$($&$C$&\eif&$C$&$)$\\
\hline
 S &    & S &  S  & S &   & S  & S & &\TTbf{Ó}&  Ó& &   S &  S  & S &   \\
 Ó &    & Ó &  S  & Ó &   & Ó  & Ó & &\TTbf{Ó}&  Ó& &   Ó &  S  & Ó &   \\
\end{tabular}
\end{center}
Ef við skoðum dálkinn undir aðaltengi setningarinnar (sá feitletraði), þá sjáum við að setningin er ósönn í báðum línum. Setningin er því ósönn, sama hvort $C$ er sönn eða ekki. Hún er ósönn fyrir allar sanngildadreifingar.

Full sanntafla fyrir setningu sem samanstendur af tveimur grunnsetningum er fjórar línur, rétt eins og skilgreiningasanntöflurnar fyrir öll setningatengin nema neitun, eða setninguna $(H \eand I)\eif H$.

Full sanntafla fyrir setningu sem inniheldur þrjár grunnsetningar er átta línur, t.d.\ þessi:
\begin{center}
\begin{tabular}{c c c|d e e e f}
$M$&$N$&$P$&$M$&\eand&$(N$&\eor&$P)$\\
\hline
%           M        &     N   v   P
S & S & S & S & \TTbf{S} & S & S & S\\
S & S & Ó & S & \TTbf{S} & S & S & Ó\\
S & Ó & S & S & \TTbf{S} & Ó & S & S\\
S & Ó & Ó & S & \TTbf{Ó} & Ó & Ó & Ó\\
Ó & S & S & Ó & \TTbf{Ó} & S & S & S\\
Ó & S & Ó & Ó & \TTbf{Ó} & S & S & Ó\\
Ó & Ó & S & Ó & \TTbf{Ó} & Ó & S & S\\
Ó & Ó & Ó & Ó & \TTbf{Ó} & Ó & Ó & Ó
\end{tabular}
\end{center}
Þessi tafla sýnir að setningin $M\eand(N\eor P)$ getur verið hvort tveggja, sönn og ósönn, allt eftir því hvaða sanngildi grunnsetningarnar $M$, $N$ og $P$ hafa.

Full sanntafla fyrir setningu sem er sett saman úr fjórum grunnsetningum þarf svo 16 línur, sanntafla með fimm grunnsetningum 32 línur og sanntafla með sex grunnsetningum þarf 64 línur. Almennt gildir að full sanntafla með $n$ grunnsetningum er $2^n$ línur.

Til þess að fylla út sanntöflu er best að byrja á að fylla út sanngildin fyrir grunnsetninguna sem er lengst til hægri. Það er dálkurinn undir „$P$“ hér að ofan. Þar er best að skrifa „S“ efst og svo „Ó“ og „S“ á víxl í línurnar fyrir neðan. Fyrir næstu grunnsetningu til vinstri skrifar maður svo tvö „S“ efst, og svo tvö „Ó“ fyrir neðan, o.s.frv. Almennt gildir að fyrir næstu grunnsetningu til vinstri við þá sem maður var að fylla út, þá fyllir maður út tvöfalt fleiri „S“ í einu og tvöfalt fleiri „Ó“. Ef þetta er gert rétt, þá mun sanntaflan hafa allar mögulegar sanngildadreifingar.

\emph{Síðasti dálkurinn sem við fyllum út er dálkurinn undir aðaltengi setningarinnar. Því þurfum við að finna aðaltengið og setningarnar sem það tengir saman og þar gildir það sama: síðasti dálkurinn sem við fyllum út í þeim er dálkurinn undir aðaltenginu sem tengir þær saman, o.s.frv. Við vinnum okkur því svona niður þangað til við komum að grunnsetningunum og þannig getum við fyllt út töfluna}.

Við getum líka, ef við viljum, fyllt út sanntöflur þar sem metabreytur hafa sama hlutverk og grunnsetningarnar. Til dæmis gætum við fyllt út eftirfarandi sanntöflu, þar sem $\meta{A}$ stendur í stað $C$: 

\begin{center}
\begin{tabular}{c| d e e e e e e e e e e e e e e f}
$\meta{A}$&$[($&$\meta{A}$&\eiff&$\meta{A}$&$)$&\eif&$\meta{A}$&$]$&\eand&\enot&$($&$\meta{A}$&\eif&$\meta{A}$&$)$\\
\hline
 S &    & S &  S  & S &   & S  & S & &\TTbf{Ó}&  Ó& &   S &  S  & S &   \\
 Ó &    & Ó &  S  & Ó &   & Ó  & Ó & &\TTbf{Ó}&  Ó& &   Ó &  S  & Ó &   \\
\end{tabular}
\end{center}Hún sýnir að allar setningar á þessu \emph{formi} hljóta að vera ósannar.

\section{Fleiri svigavenjur}\label{s:MoreBracketingConventions}
Skoðum eftirfarandi tvær setningar:
	\begin{align*}
		((A \eand B) \eand C)\\
		(A \eand (B \eand C))
	\end{align*}
Þessar setningar hafa ekki sömu myndunarsögu. Sú fyrri er mynduð úr $(A \eand B)$ og $C$, en sú seinni úr $A$ og $(B \eand C)$. Þær hafa þrátt fyrir það báðar sömu sanntöflu. Það skiptir því engu máli---frá sjónarhóli setningarökfræðinnar---hvora setninguna við notum, því setningarökfræðin hefur bara áhuga á sanngildum (sjá \S\ref{s:TruthFunctionality}). Við getum því sleppt því að skrifa svigana, því þeir skipta ekki máli. Við getum því sparað okkur örlítið ómak með því að skrifa:
	\begin{align*}
		A \eand B \eand C
	\end{align*}
Þetta gildir almennt um setningar af þessu tagi: ef við höfum margar samtengingar hver á eftir annarri, þá getum við sleppt innri svigunum (en við vorum þegar búin að leyfa að sleppa þeim ytri hér að ofan í \S\ref{s:TFLSentences}).

Það sama má segja um mistengingar, þ.e.\ margar setningar í röð tengdar saman með „$\eor$“. Þar sem eftirfarandi tvær setningar hafa sömu sanntöflu:
	\begin{align*}
		((A \eor B) \eor C)\\
		(A \eor (B \eor C))
	\end{align*}
getum við einfaldlega skrifað:
	\begin{align*}
		A \eor B \eor C
	\end{align*}
Við getum því líka sleppt innri svigum ef við höfum margar mistengingar hver á eftir annarri. En við þurfum að fara \emph{varlega}! Þessar tvær setningar hafa \emph{ólíkar} sanntöflur:
	\begin{align*}
		((A \eif B) \eif C)\\
		(A \eif (B \eif C))
	\end{align*}
Það þýðir að setningin:
	\begin{align*}
		A \eif B \eif C
	\end{align*}
er tvíræð. Það er ekki ljóst hvort hún standi fyrir setninguna $((A \eif B) \eif C)$ eða setninguna $(A \eif (B \eif C))$. Við getum því \emph{ekki} sleppt svigum þegar um er að ræða skilyrðissetningar. Sama gildir um setningar af þessu tagi:
	\begin{align*}
		((A \eor B) \eand C)\\
		(A \eor (B \eand C))
	\end{align*}
Þær hafa ólíkar sanntöflur. Ef við myndum þá skrifa:
	\begin{align*}
		A \eor B \eand C
	\end{align*}
þá væri setningin líka tvíræð. \emph{Við sleppum því aldrei innri svigum þegar um er að ræða blandaðar setningar}. Við getum \emph{bara} sleppt þessum svigum þegar um er að ræða margar setningar í röð tengdar saman með \emph{og-tengjum} eða \emph{eða-tengjum}. Aldrei annars.

\practiceproblems
\problempart
Fyllið út sanntöflur fyrir eftirfarandi setningar:
\begin{earg}
\item $A \eif A$ %taut
\item $C \eif\enot C$ %contingent
\item $(A \eiff B) \eiff \enot(A\eiff \enot B)$ %tautology
\item $(A \eif B) \eor (B \eif A)$ % taut
\item $(A \eand B) \eif (B \eor A)$  %taut
\item $\enot(A \eor B) \eiff (\enot A \eand \enot B)$ %taut
\item $\bigl[(A\eand B) \eand\enot(A\eand B)\bigr] \eand C$ %contradiction
\item $[(A \eand B) \eand C] \eif B$ %taut
\item $\enot\bigl[(C\eor A) \eor B\bigr]$ %contingent
\end{earg}
\problempart
Sýnið að nýju svigavenjurnar sem voru kynntar til sögunnar í \S\ref{s:MoreBracketingConventions} séu í lagi, þ.e.\ sýnið:
\begin{earg}
	\item að $((A \eand B) \eand C)$ og $(A \eand (B \eand C))$ hafi sömu sanntöflu,
	\item að $((A \eor B) \eor C)$ og $(A \eor (B \eor C))$ hafi sömu sanntöflu,
	\item að $((A \eor B) \eand C)$ og $(A \eor (B \eand C))$ hafi \emph{ekki} sömu sanntöflu,
	\item að $((A \eif B) \eif C)$“ og $(A \eif (B \eif C))$ hafi \emph{ekki} sömu sanntöflu.
\end{earg}
Sýnið líka að :
\begin{earg}
	\item[5.] $((A \eiff B) \eiff C)$ og $(A \eiff (B \eiff C))$ hafi sömu sanntöflu.
\end{earg}

Ef þið viljið æfa ykkur frekar, þá er hægt að fylla út sanntöflur fyrir setningarnar og rökfærslurnar sem komu fyrir í æfingum síðasta kafla.

\chapter{Merkingarfræðileg hugtök}\label{s:semanticconcepts}

Í þessum hluta höfum við talað um sanngildadreifingar og hvernig er hægt að ákvarða sanngildi hvaða setningar sem er í setningarökfræði, sama hvaða sanngildadreifingu við veljum fyrir grunnsetningarnar með sanntöflum. Núna ætlum við að fjalla um skyld hugtök og sýna hvernig hægt er að nota sanntöflur til hjálpa okkur við beitingu þeirra.

Við köllum þessi hugtök \emph{merkingarfræðileg hugtök}. Það er að vissu leyti óheppileg nafngift, því þessi hugtök hafa að gera með \emph{sannleika} og \emph{gildi}. En þetta er það sem þau eru kölluð og því vissara að halda sig við hefðina.

\section{Klifanir og mótsagnir}

Í \S\ref{s:BasicNotions}, fjölluðum við um \emph{nauðsynlega sannar} og \emph{nauðsynlega ósannar} setningar. Þessi hugtök eiga sér hliðstæðu í setningarökfræði. Hliðstæðu nauðsynlegra sannra setninga í setningarökfræði köllum við \emph{klifanir}:
	\factoidbox{
	Setning er \define{klifun} ef og aðeins ef hún er sönn fyrir allar mögulegar sanngildadreifingar.
	}Við getum notað sanntöflur til að ákvarða hvort setning sé klifun. Ef setningin er sönn á hverri línu sanntöflu sinnar, þá er hún sönn fyrir allar sanngildadreifingar, og þá er hún klifun. Ein af setningunum hér að ofan í \S\ref{s:CompleteTruthTables}, $(H \eand I) \eif H$, er klifun. Hún er sönn, sama hvað. Annað dæmi um einfalda klifun er $P \eor \enot P$.
	
\emph{Klifun} er þó ekki nema hliðstæða nauðynlegs sannleika. Sumar fullyrðingar eru nauðsynlega sannar án þess þó að vera þýðanlegar yfir á táknmál setningarökfræði. Dæmi um slíka setningu er $2+2=4$. Hún er nauðsynlega sönn, en þó getum við ekki þýtt hana yfir á táknmál setningarökfræði. Það besta sem við getum gert er að þýða hana sem grunnsetningu---og engar grunnsetningar eru klifanir. Ef hins vegar er hægt að þýða setningu á mæltu máli yfir á táknmál setningarökfræði svo vel sé, og sú setning er klifun, þá er setningin nauðynlega sönn.

Svipuð hliðstæða er til fyrir nauðsynlega ósannar setningar. Þær köllum við \emph{mótsagnir}:
	
	\factoidbox{
		Setning er \define{mótsögn} ef og aðeins ef hún er ósönn fyrir allar mögulegar sanngildadreifingar.
	}
Við getum líka notað sanntöflur til að ákvarða hvort setning sé mótsögn. Ef setningin er ósönn á hverri línu sanntöflu sinnar, þá er hún ósönn fyrir allar sanngildadreifingar, og þá er hún mótsögn. Ein af setningunum hér að ofan í \S\ref{s:CompleteTruthTables}, $[(C\eiff C) \eif C] \eand \enot(C \eif C)$, er mótsögn. Hún er ósönn, sama hvað. Dæmi um einfalda mótsögn er $P \eand \enot P$.

\section{Rökfræðilegt jafngildi}

Hér er annað gagnlegt hugtak: 

	\factoidbox{
		Setningar (tvær eða fleiri) eru \define{rökfræðilega jafngildar} eff þær hafa sama sanngildi í öllum mögulegum sanngildadreifingum.
	}
Við höfum nú þegar nýtt okkur þetta hugtak, í \S\ref{s:MoreBracketingConventions}: Við getum sleppt svigum í $(A \eand B) \eand C$ og $A \eand (B \eand C)$ af því að þessar setningar eru rökfræðilega jafngildar. Það er auðvelt að nota sanntöflur til að ganga úr skugga um hvort setningar séu rökfræðilega jafngildar. Skoðum dæmi um tvær setningar, $\enot(P \eor Q)$ og $\enot P \eand \enot Q$. Við fyllum út sanntöflu fyrir þessar tvær setningar samtímis, svona:
	
\begin{center}
\begin{tabular}{c c|d e e f |d e e e f}
$P$&$Q$&\enot&$(P$&\eor&$Q)$&\enot&$P$&\eand&\enot&$Q$\\
\hline
 S & S & \TTbf{Ó} & S & S & S & Ó & S & \TTbf{Ó} & Ó & S\\
 S & Ó & \TTbf{Ó} & S & S & Ó & Ó & S & \TTbf{Ó} & S & Ó\\
 Ó & S & \TTbf{Ó} & Ó & S & S & S & Ó & \TTbf{Ó} & Ó & S\\
 Ó & Ó & \TTbf{S} & Ó & Ó & Ó & S & Ó & \TTbf{S} & S & Ó
\end{tabular}
\end{center}

Við skoðum dálkana sem liggja undir aðaltengjum setninganna. Í fyrri setningunni er aðaltengið neitun, en og-tengi í þeirri seinni. Við sjáum að á fyrstu þremur línunum eru báðar setningarnar ósannar, en í þeirri síðustu eru báðar sannar. Setningarnar eru því sannar fyrir sömu sanngildadreifingar og eru þar af leiðandi rökfræðilega jafngildar.

\section{Rökfræðileg samkvæmni}

Í \S\ref{s:BasicNotions} hér að ofan sögðum við að setningar væru \emph{samrýmanlegar} ef og aðeins ef það er mögulegt fyrir þær að vera allar sannar samtímis. Við höfum líka hliðstæðu við það í setningarökfræði: 
	\factoidbox{
	Setningar (tvær eða fleiri) eru \define{rökfræðilega samkvæmar} eff til er sanngildadreifing þar sem þær eru báðar/allar sannar.
	}
Á sama hátt segjum við að setningar séu \define{rökfræðilega ósamkvæmar} eff ekki er til sanngildadreifing þar sem þær eru allar sannar. Við getum auðveldlega notað sanntöflur til að athuga hvort setningar séu rökfræðilega samkvæmar. Við förum alveg eins að við það og hér að ofan, nema í þetta skiptið er nóg að athuga hvort til sé \emph{ein} lína þar setningarnar eru báðar sannar. Ef við viljum kanna rökfræðilega \emph{ósamkvæmni}, þá þurfum við að fullvissa okkur um að \emph{engin} slík lína sé til.

\section{Rökfræðileg afleiðing og gildi}

Nú komum við að setningarfræðilegri hliðstæðu gildis:
	\factoidbox{ $\meta{B}$ \define{leiðir rökfræðilega af} $\meta{A}_1, \meta{A}_2, \ldots, \meta{A}_n$ ef og aðeins ef ekki er til sanngildadreifing þar sem $\meta{A}_1, \meta{A}_2, \ldots, \meta{A}_n$ eru allar sannar, en $\meta{B}$ ósönn.} Þessi skilgreining er almenn, og þarf að gilda fyrir hvaða fjölda setninga sem er og hvaða setningar sem er. Þess vegna skrifum við feitletraða stafi með lágvísum á þennan hátt. 
	
En ef við skoðum einfaldara dæmi með einungis tveimur grunnsetningum, þá verður hugmyndin ef til vill skýrari. Við segjum  að $Q$ leiði rökfræðilega af $P \eif Q$ og $P$ ef og aðeins ef \emph{ekki} er til sanngildadreifing þar sem $P \eif Q$ og $P$ eru báðar sannar, en $Q$ ósönn. Við sjáum að svo er ekki: ef $P \eif Q$ og $P$ eru sannar, þá vitum við forliður $P \eif Q$ er sannur og að bakliðurinn er ekki ósannur (skoðið skilgreiningasanntöfluna fyrir „$\eif$“ til að fullvissa ykkur um það!) og þá hlýtur $Q$ að vera sönn líka.
	
Þetta má kanna með sanntöflum. Tökum annað, flóknara dæmi. Til að vita hvort $J$ leiði rökfræðilega af $\enot L \eif (J \eor L)$ og $\enot L$ þá þurfum við að athuga hvort til sé sanngildadreifing þar sem $\enot L \eif (J \eor L)$ og $\enot L$ eru sannar, en $J$ ósönn. Ef svo er \emph{ekki}, þá $J$ leiðir rökfræðilega af $\enot L \eif (J \eor L)$ og $\enot L$. Svona liti sanntaflan út:

\begin{center}
\begin{tabular}{c c|d e e e e f|d f| c}
$J$&$L$&\enot&$L$&\eif&$(J$&\eor&$L)$&\enot&$L$&$J$\\
\hline
%J   L   -   L      ->     (J   v   L)
 S & S & Ó & S & \TTbf{S} & S & S & S & \TTbf{Ó} & S & \TTbf{S}\\
 S & Ó & S & Ó & \TTbf{S} & S & S & Ó & \TTbf{S} & Ó & \TTbf{S}\\
 Ó & S & Ó & S & \TTbf{S} & Ó & S & S & \TTbf{Ó} & S & \TTbf{Ó}\\
 Ó & Ó & S & Ó & \TTbf{Ó} & Ó & Ó & Ó & \TTbf{S} & Ó & \TTbf{Ó}
\end{tabular}
\end{center}
Eina línan þar sem $\enot L \eif (J \eor L)$ og $\enot L$ eru báðar sannar er lína tvö, og þar er $J$ líka sönn. Það er því engin sanngildadreifing þar sem forsendurnar eru allar sannar, $\enot L \eif (J \eor L)$ og $\enot L$, en $J$ ósönn. $J$ leiðir því rökfræðilega af $\enot L \eif (J \eor L)$ og $\enot L$. 

Rökfræðileg samkvæmni er nátengd gildi:\footnote{Munið að í \S\ref{s:UseMention} notuðum við „$\therefore$“ til að gefa til kynna hvaða setningar eru forsendur og hvaða setning er niðurstaða. Án þessa tákns, þá hefðum við þurft að skrifa í boxinu hér að ofan: Ef $\meta{B}$ leiðir rökfræðilega af $\meta{A}_1, \meta{A}_2, \ldots, \meta{A}_n$, þá er rökfærslan sem hefur $\meta{A}_1, \meta{A}_2, \ldots, \meta{A}_n$ sem forsendur og $\meta{B}$ sem niðurstöðu gild.}

	\factoidbox{
		Ef $\meta{B}$ leiðir rökfræðilega af $\meta{A}_1, \meta{A}_2, \ldots, \meta{A}_n$, þá er $\meta{A}_1, \meta{A}_2, \ldots, \meta{A}_n \therefore \meta{B}$ gild rökfærsla.
	}

Ástæðan er þessi: Ef $\meta{B}$ leiðir rökfræðilega af $\meta{A}_1, \meta{A}_2, \ldots, \meta{A}_n$, þá er ekki til nein sanngildadreifing þar sem $\meta{A}_1, \meta{A}_2, \ldots, \meta{A}_n$ eru sannar en $\meta{B}$ ósönn. Það er því \emph{rökfræðilega ómögulegt} að $\meta{A}_1, \meta{A}_2, \ldots, \meta{A}_n$ séu allar sannar en $\meta{B}$ ósönn. En það er einmitt það sem skilgreiningin okkar á gildi sagði, að rökfærsla sé gild ef og aðeins ef það er ómögulegt að forsendurnar séu allar sannar en niðurstaðan ósönn.

Við höfum því loks fundið \emph{aðferð} til að kanna gildi rökfærsla á mæltu máli: fyrst þýðum við hana yfir á táknmál setningarökfræði og svo könnum við rökfræðilega afleiðingu með því að nota sanntöflur. Það er þó vert að nefna að hér notum við hugtakið „rökfræðileg afleiðing“ sem tæknilegt hugtak samkvæmt skilgreiningunni hér að ofan. Í mæltu máli er það hins vegar oft notað sem samheiti yfir gildi, eða eitthvað svipað.

\section{Takmarkanir þessarar aðferðar}\label{s:ParadoxesOfMaterialConditional}

Þetta er þó merkur áfangi: aðferð til að meta gildi rökfærsla! En þessi aðferð hefur því miður sínar takmarkanir. Skoðum þrjú dæmi:

Skoðum fyrst eftirfarandi rökfærslu: 
	\begin{earg}
		\item Búkolla er með fjóra fætur. Þar af leiðandi er Búkolla með fleiri en tvo fætur.
	\end{earg}
	Til að þýða þessa rökfærslu yfir á táknmál setningarökfræði erum við nauðbeygð til að nota mismunandi grunnsetningar---kannski $F$ og $T$---fyrir forsenduna og niðurstöðuna. En það er augljóst að $T$ leiðir ekki rökfræðilega af $F$. En þó er rökfærslan þó greinilega gild!
	
Skoðum nú eftirfarandi setningu:
	\begin{earg}
\setcounter{eargnum}{1}
		\item\label{n:JanBald} Jón er hvorki sköllóttur né ekki-sköllóttur.
	\end{earg}
Það væri eðlilegt að þýða þessa setningu yfir á táknmál setningarökfæði með $\enot J \eand \enot \enot J$. En þetta er mótsögn (eins og þið ættuð að kanna með sanntöflu). En setningin sjálf virðist ekki vera mótsögn. Til dæmis gætum við sagt að Jón sé á mörkum þess að vera sköllóttur.

Hér er eitt dæmi í viðbót:
	\begin{earg}
\setcounter{eargnum}{2}	
		\item\label{n:GodParadox} Það er ekki satt að ef Guð er til, þá svari hún bölbænum.	
	\end{earg}
Ef við þýddum þessa setningu yfir á táknmál setningarökfræði, þá væri $\enot (G \eif M)$ eðlileg þýðing. En $G$ leiðir rökfræðilega af $\enot (G \eif M)$ (og þetta ættuð þið að athuga með sanntöflu). En þetta þýðir að ef við þýðum \ref{n:GodParadox} yfir á táknmál setningarökfræði, þá virðumst við hafa sannað tilvist Guðs! En það er varla svona auðvelt, eða hvað? Ættu ekki jafnvel örgustu trúleysingar að geta tekið undir að \ref{n:GodParadox} sé sönn, án þess að lenda þar með í mótsögn við sjálfa sig (og ef þið takið ekki undir það, athugið þá að við gætum smíðað sambærilega setningu um Óðinn eða Þór)?

Þessi dæmi sýna, hvert á sinn hátt, takmarkanir þess að notast við mál sem reiðir sig \emph{eingöngu} á sannföll eins og setningarökfræðin gerir. Þessar takmarkanir vekja upp ýmsar áhugaverðar spurningar í heimspekilegri rökfræði (þ.e.\ þeim hluta heimspekinnar sem fjallar um rökfræðileg málefni). Dæmið um hárið á Jóni (eða skortinn þar á) vekur til dæmis upp spurningar hvernig best sé að eiga við setningar sem tjá það sem er óljóst eða loðið: hvenær er maður sköllóttur og hvenær verða sandkorn að hrúgu? Dæmið um Guð er svo tengt hinum svokölluðu \emph{skilyrðisþversögnum} (en við höfum séð fleiri slíkar, til dæmis að skilyrðissetning með ósönnum forlið er alltaf sönn). 

Af hverju þá að læra setningarökfræði? Hluti af svarinu er að það er ekkert eitt kerfi sem er augljóslega betra og það er umdeilt hvernig bregðast eigi við þessum spurningum. Til þess að geta tekist á við slíkar spurningar þarf maður því að byrja einhvers staðar og setningarökfræðin er þar langbesti kosturinn. Hún er einföld, vel þekkt og í raun og veru furðulega sterk.

\section{Sérstakt tákn fyrir rökfræðilega afleiðingu}

Við munum tala mikið um rökfræðilega afleiðingu í því sem eftir er af bókinni. Það er þess vegna þægilegt að innleiða nýtt tákn til að tala um hana. Við munum því sjaldan segja beint út að setninguna $\meta{B}$ leiði rökfræðilega af $\meta{A}_1, \meta{A}_2, \ldots, \meta{A}_n$, heldur frekar skammstafa það með því að skrifa: $$\meta{A}_1, \meta{A}_2, \ldots, \meta{A}_n \entails \meta{B}$$
Við notum táknið „$\entails$“ því til að tákna rökfræðilega afleiðingu.

Tökum samt \emph{vel} eftir því að $\entails$“ er ekki tákn á táknmáli setningarökfræði. Það er tákn sem við skilgreinum í framsetningarmálinu, í okkar tilfelli íslensku, til að eiga hægara með að tala \emph{um} setningar í setningarökfræði (sjá \S\ref{s:UseMention} fyrir frekari umfjöllun um þennan greinarmun).

Eftirfarandi setning á \emph{framsetningarmálinu}:
	\begin{ebullet}
		\item $P, P \eif Q \entails Q$
	\end{ebullet}
er því bara stytting eða skammstöfun á þessari setningu, sem líka er hluti af viðfangsmálinu:	
	\begin{ebullet}
		\item Setninguna $Q$ leiðir rökfræðilega af setningunum $P$ og $P \eif Q$.
	\end{ebullet}
Það eru engin takmörk fyrir því hversu margar setningar í setningarökfræði við getum talað um í einu með þessu tákni. En við getum líka sleppt því að setja nokkra setningu vinstra megin og skrifað: 
	$$\phantom{\meta{A}}\entails \meta{B}$$
Þetta segir að það sé engin sanngildadreifing sem er þannig að allar setningarnar vinstra megin við „$\entails$“ séu sannar og að $\meta{B}$ sé ósönn. Þar sem það eru \emph{engar} setningar vinstra megin, þá látum við þetta merkja að $\meta{B}$ sé sönn fyrir \emph{allar} sanngildadreifingar. Það er að segja: $\meta{B}$ er klifun. Við höfum nú þegar séð dæmi um slíkar setningar. Til dæmis gildir að $$ \entails P \eor \enot P$$

Við notum svipaðan rithátt við að segja að $\meta{B}$ sé mótsögn:
	$$\meta{B} \entails\phantom{\meta{C}}$$
Þetta segir að $\meta{B}$ sé ósatt fyrir allar sanngildadreifingar.

Stundum viljum við neita því að setningu leiði rökfræðilega af annarri. Það er að segja:
\begin{center}
	það er ekki satt að $\meta{A}_1, \meta{A}_2, \ldots, \meta{A}_n \entails \meta{B}$
\end{center}
Hér styttum við okkur aftur leið og strikum einfaldlega yfir táknið:

$$\meta{A}_1, \meta{A}_2, \ldots, \meta{A}_n \nentails \meta{B}$$

Þetta merkir að það er til \emph{einhver} sanngildadreifing sem er þannig að $\meta{A}_1, \meta{A}_2, \ldots, \meta{A}_n$ eru allar sannar en $\meta{B}$ ósönn. Athugið \emph{alveg sérstaklega vel} að þetta er ekki það sama og $\meta{A}_1, \meta{A}_2, \ldots, \meta{A}_n \entails \enot \meta{B}$! Það myndi merkja að $\enot \meta{B}$ leiði rökfræðilega af $\meta{A}_1, \meta{A}_2, \ldots, \meta{A}_n$.

\section{„$\entails$“ og „$\eif$“}

Hér að ofan sagði ég að „$\entails$“ væri hluti af framsetningarmálinu en ekki hluti af táknmáli setningarökfræðinnar. Á gagnstæðan hátt er „$\eif$“ hluti af táknmáli setningarökfræðinnar, en ekki hluti af framsetningarmálinu. Það eru þó tengsl þarna á milli.

Af ofansögðu vitum við að $\meta{A} \entails \meta{B}$ ef og aðeins ef ekki er til sanngildadreifing þar sem $\meta{A}$ er sönn og $\meta{B}$ ósönn.

Við vitum líka að: $\meta{A} \eif \meta{B}$ er klifun ef og aðeins ef ekki er til sanngildadreifing þar sem $\meta{A} \eif \meta{B}$ er ósönn. Við vitum líka að skilyrðissetning er alltaf sönn, nema þegar forliðurinn er sannur en bakliðurinn ósannur, og því er $\meta{A} \eif \meta{B}$ kilfun ef og aðeins ef ekki er til sanngildadrefing þar sem $\meta{A}$ er sönn en $\meta{B}$ ósönn. Með því að setja þetta tvennt saman, þá sjáum við að $\meta{A} \eif \meta{B}$ er klifun ef og aðeins ef $\meta{A} \entails \meta{B}$. 

Þrátt fyrir það er mikilvægt halda þessum tveimur táknum aðskildum: 
	\factoidbox{„$\eif$“ er setningatengi í táknmáli setningarökfræði.\\ „$\entails$“ er tákn sem við bættum við framsetningarmálið, íslensku. 
	}
	
Þegar við setjum tvær setningar á máli setningarökfræði sitthvoru megin við „$\eif$“, þá er útkoman lengri setning á máli setningarökfræðinnar. Á hinn bóginn, þegar við notum „$\entails$“, þá er um að ræða setningu á framsetningarmálinu sem \emph{talar um} setningar á máli setningarökfræði.

\practiceproblems
\problempart
Skoðið setningarnar í æfingu \S\ref{s:CompleteTruthTables}\textbf{A} hér að ofan. Athugið með sanntöflum hvaða setningar eru klifanir, hverjar mótsagnir og hverjar eru hvorki klifanir né mótsagnir.

\problempart

Notið sanntöflur til að ákvörða hverjar af eftirfarandi setningum eru rökfræðilega samkvæmar og hverjar eru rökfræðilega ósamkvæmar:
\begin{earg}
\item $A\eif A$, $\enot A \eif \enot A$, $A\eand A$, $A\eor A$ %consistent
\item $A\eor B$, $A\eif C$, $B\eif C$ %consistent
\item $B\eand(C\eor A)$, $A\eif B$, $\enot(B\eor C)$  %inconsistent
\item $A\eiff(B\eor C)$, $C\eif \enot A$, $A\eif \enot B$ %consistent
\end{earg}

\problempart
Notið sanntöflur til að meta eftirfarandi rökfærslur:
\begin{earg}
\item $A\eif A \therefore A$ %invalid
\item $A\eif(A\eand\enot A) \therefore \enot A$ %valid
\item $A\eor(B\eif A) \therefore\enot A \eif \enot B$ %valid
\item $A\eor B, B\eor C, \enot A \therefore B \eand C$ %invalid
\item $(B\eand A)\eif C, (C\eand A)\eif B \therefore (C\eand B)\eif A$ %invalid
\end{earg}

\problempart
Svarið eftirfarandi spurningum og rökstyðjið svarið.
\begin{earg}

\item Gerum ráð fyrir að $\meta{A}$ og $\meta{B}$ séu rökfræðilega jafngildar. Hvað getum við sagt um $\meta{A} \eiff \meta{B}$?
%\meta{A} and \meta{B} have the same truth value on every line of a complete truth table, so $\meta{A}\eiff\meta{B}$ is true on every line. It is a tautology.

\item Gerum ráð fyrir að $(\meta{A} \eand \meta{B}) \eif \meta{C}$ sé hvorki klifun né mótsögn. Hvað getum við sagt um $\meta{A}, \meta{B} \entails \meta{C}$?
%The sentence is false on some line of a complete truth table. On that line, \meta{A} and \meta{B} are true and \meta{C} is false. So the argument is invalid.

\item Gerum ráð fyrir að $\meta{A}$, $\meta{B}$ og $\meta{C}$ séu rökfræðilega ósamkvæmar. Hvað getum við sagt um $\meta{A} \eand \meta{B} \eand\meta{C}$?

\item Gerum ráð fyrir að $\meta{A}$ sé mótsögn. Hvað getum við sagt um eftirfarandi rökfærslu: $\meta{A}, \meta{B} \therefore~\meta{C}$?
%Since \meta{A} is false on every line of a complete truth table, there is no line on which \meta{A} and \meta{B} are true and \meta{C} is false. So the argument is valid.

\item Gerum ráð fyrir því að $\meta{C}$ sé klifun. Hvað getum við sagt um eftirfarandi rökfærslu: $\meta{A}, \meta{B} \therefore~\meta{C}$?
%Since \meta{C} is true on every line of a complete truth table, there is no line on which \meta{A} and \meta{B} are true and \meta{C} is false. So the argument is valid.

\item Gerum ráð fyrir að $\meta{A}$ og $\meta{B}$ séu rökfræðilega jafngildar. Hvað getum við sagt um $\meta{A} \eor \meta{B}$?
%Not much. $(\meta{A}\eor\meta{B})$ is a tautology if \meta{A} and \meta{B} are tautologies; it is a contradiction if they are contradictions; it is contingent if they are contingent.

\item Gerum ráð fyrir að $\meta{A}$ og $\meta{B}$ séu \emph{ekki} rökfræðilega jafngildar. Hvað getum við sagt um $\meta{A} \eor \meta{B}$?
%\meta{A} and \meta{B} have different truth values on at least one line of a complete truth table, and $(\meta{A}\eor\meta{B})$ will be true on that line. On other lines, it might be true or false. So $(\meta{A}\eor\meta{B})$ is either a tautology or it is contingent; it is \emph{not} a contradiction.
\end{earg}
\problempart 
Skoðum eftirfarandi reglu:
	\begin{ebullet}
		\item Gerum ráð fyrir að $\meta{A}$ og $\meta{B}$ séu rökfræðilega jafngildar. Ef rökfærsla inniheldur $\meta{A}$, annað hvort sem forsendu eða niðurstöðu, þá væri gildi rökfærslunnar óbreytt, ef við skiptum $\meta{A}$ út fyrir $\meta{B}$.
	\end{ebullet}
Er þessi regla rétt? Rökstyðjið svarið.

\chapter{Að stytta sér leið}

Með æfingu er fljótlega hægt að verða ansi lunkinn við að fylla út og nota sanntöflur. Það er þó hægt að stytta sér leið með ýmsum hætti og í þessum hluta ætla ég að nefna nokkra leiðir til þess.

\section{Styttri leiðir við að fylla út sanntöflur}
Það fyrsta sem ég vil nefna er að strangt til tekið þarf ekki að afrita sanngildin undir hverri grunnsetningu vinstra megin yfir undir hverja grunnsetningu hægra megin. Það er hægt að skrifa einfaldlega:

\begin{center}
\begin{tabular}{c c|d e e e e f}
$P$&$Q$&$(P$&\eor&$Q)$&\eiff&\enot&$P$\\
\hline
 S & S &  & S &  & \TTbf{Ó} & Ó\\
 S & Ó &  & S &  & \TTbf{Ó} & Ó\\
 Ó & S &  & S & & \TTbf{S} & S\\
 Ó & Ó &  & Ó &  & \TTbf{Ó} & S
\end{tabular}
\end{center}
En þetta er þó tvíeggjað sverð. Þegar maður sleppir afrituninni, þá aukast líkurnar á klaufavillum verulega og því ekki alltaf ljóst að þetta spari manni mikla vinnu til lengri tíma litið. En fyrir þá sem hafa skarpa sjón og örugga rithönd er gott að vita af þessum möguleika.

En hér er traustari möguleiki: Við vitum að setning sem tengd er saman með eða-tengi er sönn þegar önnur setninganna sem hún er sett saman úr er sönn. Svo ef við sjáum að ein þeirra er sönn, þá er engin ástæða til að leita að hinni og athuga sanngildi þeirra. Því er hægt að skrifa:

\begin{center}
\begin{tabular}{c c|d e e e e e e f}
$P$&$Q$& $(\enot$ & $P$&\eor&\enot&$Q)$&\eor&\enot&$P$\\
\hline
 S & S & Ó & & Ó & Ó& & \TTbf{Ó} & Ó\\
 S & Ó &  Ó & & S& S& &  \TTbf{S} & Ó\\
 Ó & S & & &  & & & \TTbf{S} & S\\
 Ó & Ó & & & & & &\TTbf{S} & S
\end{tabular}
\end{center}
Við vitum líka að samtenging er ósönn ef og aðeins ef önnur setninganna sem hún er sett saman úr er ósönn. Ef við sjáum að önnur þeirra er ósönn, er því engin ástæða til að athuga hvort hin sé sönn eða ósönn. Við vitum strax að samtengingin er ósönn. Þess vegna getum við skrifað:

\begin{center}
\begin{tabular}{c c|d e e e e e e f}
$P$&$Q$&\enot &$(P$&\eand&\enot&$Q)$&\eand&\enot&$P$\\
\hline
 S & S &  &  & &  & & \TTbf{Ó} & Ó\\
 S & Ó &   &  &&  & & \TTbf{Ó} & Ó\\
 Ó & S & S &  & Ó &  & & \TTbf{S} & S\\
 Ó & Ó & S &  & Ó & & & \TTbf{S} & S
\end{tabular}
\end{center}

Svipuðu máli gegnir um skilyrðissetningar. Við vitum að skilyrðissetningar eru sannar ef forliðurinn er ósannur eða bakliðurinn sannur (skilyrðissetning er jú bara \emph{ósönn} ef forliðurinn er sannur og bakliðurinn ósannur). Þá getum við stytt okkur leið svona:

\begin{center}
\begin{tabular}{c c|d e e e e e f}
$P$&$Q$& $((P$&\eif&$Q$)&\eif&$P)$&\eif&$P$\\
\hline
 S & S & &  & & & & \TTbf{S} & \\
 S & Ó &  &  & && & \TTbf{S} & \\
 Ó & S & & S & & Ó & & \TTbf{S} & \\
 Ó & Ó & & S & & Ó & &\TTbf{S} & 
\end{tabular}
\end{center}
Setningin „$((P \eif Q) \eif P) \eif P$“ er því klifun---hún er sönn fyrir hvaða sanngildadreifingu sem er. Þessi setning er raunar dæmi um hið svokallaða \emph{Pierce-lögmál}, en það er kennt við rökfræðinginn Charles Sanders Peirce.

\section{Styttri leiðir við kanna gildi og rökfræðilega afleiðingu}

Rökfærsla er gild, eins og við munum, þegar það er engin leið fyrir forsendurnar að vera allar sannar en að niðurstöðuna sé ósanna. Við vitum líka að ef niðurstöðu leiðir rökfræðilega af forsendum rökfærslu, þá er hún gild. Hér að ofan í \S\ref{s:semanticconcepts} lærðum við að nota sanntöflur til að kanna hvort eina setningu leiðir rökfræðilega af annarri. 

Við gerðum það með því að leita að línu í sanntöflunni þar sem forsendurnar eru allar sannar en niðurstaðan er ósönn. Köllum slíka línu \emph{slæma}. Ef við finnum slæma línu, þá vitum við að rökfærslan er \emph{ekki} gild og ef við finnum \emph{ekki} slíka línu, þá vitum við að hún \emph{er} gild.

Við vitum líka að:
\begin{earg}
	\item[\textbullet] Ef niðurstaðan er sönn í einhverri tiltekinni línu, þá er sú lína ekki slæm (og við þurfum ekki að skoða neitt \emph{frekar} á þeirri línu til að fullvissa okkur um það). Allar slæmar línur hafa ósanna niðurstöðu.
	\item[\textbullet] Ef einhver af forsendunum er ósönn í tiltekinni línu, þá er sú lína ekki slæm (og við þurfum heldur ekki að skoða neitt \emph{frekar} á þeirri línu til að fullvissa okkur um það). Allar forsendur eru sannar í slæmum línum.
\end{earg}Ef engin lína er slæm, þá vitum við að rökfærslan er gild. Með þetta í huga, þá getum við flýtt fyrir okkur ansi mikið. Skoðum til dæmis eftirfarandi rökfærslu: 
$$\enot L \eif (J \eor L), \enot L \therefore J$$
Það fyrsta sem við ættum að gera er að skoða niðurstöðuna. Ef við sjáum að niðurstaðan er \emph{sönn} á einhverri tiltekinni línu, þá er sú lína ekki slæm. Þá getum við hunsað restina af henni. Eftir að hafa gert það, þá höfum við:

\begin{center}
	\begin{tabular}{c c|d e e e e f |d f|c}
		$J$&$L$&\enot&$L$&\eif&$(J$&\eor&$L)$&\enot&$L$&$J$\\
		\hline
		%J   L   -   L      ->     (J   v   L)
		S & S & &&&&&&&& {S}\\
		S & Ó & &&&&&&&& {S}\\
		Ó & S & &&?&&&&?&& {Ó}\\
		Ó & Ó & &&?&&&&?&& {Ó}
	\end{tabular}
\end{center}
Hér tákna auðu línurnar línur sem við munum láta eiga sig hér eftir (þar sem við vitum að þær eru ekki slæmar) og spurningamerkin tákna línur sem við þurfum að skoða betur.

Það er auðvelt að kanna dálkinn þar sem „$\enot L$“ kemur fyrir og því gerum við það næst:
\begin{center}
	\begin{tabular}{c c|d e e e e f |d f|c}
		$J$&$L$&\enot&$L$&\eif&$(J$&\eor&$L)$&\enot&$L$&$J$\\
		\hline
		%J   L   -   L      ->     (J   v   L)
		S & S & &&&&&&&& {S}\\
		S & Ó & &&&&&&&& {S}\\
		Ó & S & &&&&&&{Ó}&& {Ó}\\
		Ó & Ó & &&?&&&&{S}&& {Ó}
	\end{tabular}
\end{center}
Við sjáum að lína þrjú er ekki slæm, því einhver forsenda er ósönn í henni og því þurfum við ekki að skoða hana frekar. Loks klárum við sanntöfluna með að fylla út línu fjögur:
\begin{center}
	\begin{tabular}{c c|d e e e e f |d f|c}
		$J$&$L$&\enot&$L$&\eif&$(J$&\eor&$L)$&\enot&$L$&$J$\\
		\hline
		%J   L   -   L      ->     (J   v   L)
		S & S & &&&&&&&& {S}\\
		S & Ó & &&&&&&&& {S}\\
		Ó & S & &&&&&&{Ó}& & {Ó}\\
		Ó & Ó & S &  & \TTbf{Ó} &  & Ó & & {S} & & {Ó}
	\end{tabular}
\end{center}
Það eru engar slæmar línur í þessari sanntöflu, engar línur þar sem forsendurnar eru sannar en niðurstaðan ósönn. Rökfærslan er því gild.

Gagnsemi þessarar aðferðar sést kannski jafnvel enn betur ef við skoðum rökfærslu með fleiri grunnsetningum. Til dæmis:

$$A\eor B, \enot (B\eand C) \therefore (A \eor \enot C)$$
Við byrjum á því að skoða niðurstöðuna. Aðaltengið í henni er eða-tengi, svo við getum flýtt fyrir með reglunum sem við kynntumst hér að ofan:
\begin{center}
\begin{tabular}[t]{c c c| c|c|d e e f }
$A$ & $B$ & $C$ & $A\eor B$ & $\enot (B \eand C)$ & $(A$ &$\eor $& $\enot $ & $C)$\\
\hline
S & S & S &  &  & & \TTbf{S} & & \\
S & S & Ó &  &  & & \TTbf{S} & & \\
S & Ó & S &  &  & & \TTbf{S} & & \\
S & Ó & Ó &  &  & & \TTbf{S} & & \\
Ó & S & S & ? & ? & & \TTbf{Ó} &Ó & \\
Ó & S & Ó &  &  && \TTbf{S} & S& \\
Ó & Ó & S & ? & ? && \TTbf{Ó} & Ó& \\
Ó & Ó & Ó &  &  & & \TTbf{S} & S& \\
\end{tabular}
\end{center}
Við getum núna sleppt því að skoða allar nema þær tvær línur þar sem niðurstaðan er ósönn. Með því að halda áfram, í samræmi við reglurnar sem við höfum þegar séð, þá fáum við:
 \begin{center}
 	\begin{tabular}[t]{c c c| c|d e e f |d e e f }
 		$A$ & $B$ & $C$ & $A\eor B$ & $\enot ($&$B$&$ \eand$&$ C)$ & $(A$ &$\eor $& $\enot $ & $C)$\\
 		\hline
 		S & S & S &  & &&& & & \TTbf{S} & & \\
 		S & S & Ó &  & &&& & & \TTbf{S} & & \\
 		S & Ó & S &  & &&& & & \TTbf{S} & & \\
 		S & Ó & Ó &  & &&& & & \TTbf{S} & & \\
 		Ó & S & S & \textbf{S} & \textbf{Ó}&&S& & & \TTbf{Ó} &Ó & \\
 		Ó & S & Ó & &&& & && \TTbf{S} & S& \\
 		Ó & Ó & S & \textbf{Ó} & &&& & & \TTbf{Ó} & Ó& \\
 		Ó & Ó & Ó & &&&& && \TTbf{S} & S& \\
 	\end{tabular}
 \end{center}
Hér eru engar línur þar sem niðurstöðurnar eru báðar sannar og niðurstaðan ósönn. Niðurstöðuna leiðir því rökfræðilega af forsendunum. Þessi sanntafla er mjög stór, en með því að nota reglurnar hér að ofan, þá tókst okkur að komast hjá því að fylla út megnið af henni. Það hlýtur að mega teljast vel af sér vikið.
 
\practiceproblems
\problempart
Athugið hvort eftirfarandi setningar sú klifanir, mótsagnir eða hvorugt. Styttið ykkur leið eins og lýst var hér að ofan.

\begin{earg}
	\item $\enot B \eand B$ %contra
	\item $\enot D \eor D$ %taut
	\item $(A\eand B) \eor (B\eand A)$ %contingent
	\item $\enot[A \eif (B \eif A)]$ %contra
	\item $A \eiff [A \eif (B \eand \enot B)]$ %contra
	\item $\enot(A\eand B) \eiff A$ %contingent
	\item $A\eif(B\eor C)$ %contingent
	\item $(A \eand\enot A) \eif (B \eor C)$ %tautology
	\item $(B\eand D) \eiff [A \eiff(A \eor C)]$%contingent
\end{earg}


\chapter{Ókláraðar sanntöflur}\label{s:PartialTruthTable}

Stundum er óþarfi að skoða hverja einustu línu í sanntöflu. Stundum er nóg að vita hvað gerist á einni eða tveimur línum, til dæmis ef við viljum vita hvort ákveðin sanngildadreifing er möguleg eða ekki. Við getum sparað okkur mikla vinnu með því að reyna einfaldlega að „smíða“ slíkar sanngildadreifingar frá grunni. Í þessum hluta ætla ég að taka nokkur dæmi um slíkt.

\paragraph{Klifanir.}

Setning er klifun ef og aðeins ef hún er sönn fyrir allar sanngildadreifingar. Það þýðir að við þurfum bara eina línu í sanntöflu til að sýna að setning sé \emph{ekki} klifun: við þurfum bara eina sanngildadreifingu þar sem setningin er ósönn til að sýna að hún sé ekki klifun. Það er því nóg að sýna eina línu í sanntöflunni þar sem setningin er ósönn. Í stað þess að fylla út heila sanntöflu getum við einfaldlega reynt að búa slíka sanngildadreifingu til. Ef það er hægt, þá er setningin ekki klifun. Við köllum slíkt \define{ókláraða sanntöflu}.

Segjum að við viljum sýna að setningin $(U \eand T) \eif (S \eand W)$ sé \emph{ekki} klifun. Við byrjum svona:

\begin{center}
\begin{tabular}{c c c c |d e e e e e f}
$S$&$T$&$U$&$W$&$(U$&\eand&$T)$&\eif    &$(S$&\eand&$W)$\\
\hline
   &   &   &   &    &   &    &\TTbf{Ó}&    &   &   
\end{tabular}
\end{center}
Við höfum bara eina línu hér, þar sem við erum einungis að leita að einni sanngildadreifingu þar sem setningin er ósönn. Við skrifum því niður undir aðaltengið að setningin sé ósönn og reynum svo að fylla út línuna. Ef við getum gert það, þá er setningin ekki klifun, en ef við getum ekki gert það, þá er hún klifun.

Aðaltengi setningarinnar er skilyrðistengið. Skilyrðistengið er bara ósatt ef forliðurinn er sannur og bakliðurinn er sannur. Við getum því fyllt línuna út svona:
\begin{center}
\begin{tabular}{c c c c |d e e e e e f}
$S$&$T$&$U$&$W$&$(U$&\eand&$T)$&\eif    &$(S$&\eand&$W)$\\
\hline
   &   &   &   &    &  S  &    &\TTbf{Ó}&    &   Ó &   
\end{tabular}
\end{center}
Hlutasetningin $(U\eand T)$  er ekki sönn nema $U$ og $T$ séu báðar sannar. Því höfum við: 

\begin{center}
\begin{tabular}{c c c c|d e e e e e f}
$S$&$T$&$U$&$W$&$(U$&\eand&$T)$&\eif    &$(S$&\eand&$W)$\\
\hline
   & S & S &   &  S &  S  & S  &\TTbf{Ó}&    &   Ó &   
\end{tabular}
\end{center}
Til þess að klára línuna þurfum við bara að $(S\eand W)$ sé ósönn. Til þess er nóg að annað hvort $S$ eða $W$ séu ósannar, en þær geta líka verið báðar ósannar. Það eina sem skiptir máli er að öll setningin sé ósönn á þessari línu og hvaða leið við förum hér er okkar eigin \emph{ákvörðun}. Það skiptir ekki öllu máli hvað við veljum, svo við tökum bara af skarið og klárum töfluna. Til dæmis svona:

\begin{center}
\begin{tabular}{c c c c|d e e e e e f}
$S$&$T$&$U$&$W$&$(U$&\eand&$T)$&\eif    &$(S$&\eand&$W)$\\
\hline
 Ó & S & S & Ó &  S &  S  & S  &\TTbf{Ó}&  Ó &   Ó & Ó  
\end{tabular}
\end{center}

Þetta er möguleg sanngildadreifing. Við höfum því sýnt að það sé til sanngildadreifing þar sem $(U \eand T) \eif (S \eand W)$ er ósönn, nefnilega sanngildadreifingin þar sem $S$ er ósönn, $T$ sönn, $U$ sönn og $W$ ósönn. Við höfum því ókláraða sanntöflu sem sýnir að $(U \eand T) \eif (S \eand W)$ sé ekki klifun.
 
Þetta dæmi er auðvitað vel valið til að ganga upp. Ef setningin \emph{hefði} verið klifun, þá hefði ekki verið hægt að finna \emph{neina} sanngildadreifingu sem gerði setninguna \emph{ósanna}. En hvernig lýsir það sér við beitingu aðferðarinnar? Jú, við hefðum lent í því að þurfa að setja bæði sanngildin á sömu grunnsetningu í töflunni, þ.e.\ við hefðum verið nauðbeygð til að segja að einhver grunnsetning sé bæði sönn og ósönn. En af því að grunnsetning getur bara haft eitt sanngildi, þá sýnir það að slík sanngildadreifing er ekki til. Það sýnir þá að setningin \emph{gæti ekki} verið ósönn og þar með að hún væri klifun.

Hér er dæmi um einfalda klifun og hvernig aðferðin virkar ef ekki er til nein sanngildadreifing sem gerir setninguna sanna: 
 \begin{center}
 \begin{tabular}{c|d c e e f}
$P$&$P$&$\;\eor$&$\enot$&$P$\\
 \hline
 &  & \;Ó &  &  
 \end{tabular}
 \end{center} 
Við ætlum að reyna að sýna fram á að þessi setning sé \emph{ekki} klifun og byrjum þess vegna á að gefa okkur að öll setningin sé ósönn og skrifum því „Ó“ undir aðaltengið. Þar sem við vitum að setning sem tengd er saman með eða-tengi er aldrei ósönn nema báðar setningarnar sem mynda hana séu ósannar, þá höfum við:
 
  \begin{center}
  \begin{tabular}{c|d c e e f}
 $P$&$P$&$\;\eor$&$\enot$&$P$\\
  \hline
  Ó&  Ó& \;Ó &Ó  &  
  \end{tabular}
  \end{center}
En fyrst $\enot P$ er aldrei ósönn nema „$P$“ sé sönn, þá þurfum við að setja „S“ undir grunnsetninguna $P$---en þar erum við nú þegar búin að setja „Ó“. Það er ekki leyfilegt og því ekki til nein sanngildadreifing þar sem þessi setning er ósönn. Ef það er ekki til nein sanngildadreifing þar sem þessi setning er ósönn, þá hlýtur hún að vera sönn fyrir allar sanngildadreifingar---og þá er hún klifun.

Hér þurfum við þó að passa okkur: Ef við viljum sýna að engin sanngildadreifing af ákveðnu tagi sé til og upp koma tveir möguleikar við leitina (t.d.\ getur $P \eand Q$ verið ósönn ef $P$ er ósönn \emph{eða} $Q$ er ósönn), þá getum við ekki bara valið annan hvorn möguleikann eins og að ofan. Það er vegna þess að ef við veldum annan hvorn möguleikann og sýndum að hann gangi ekki upp, þá er alltaf mögulegt að hinn geri það. Þess vegna þurfum við að bæta við línu, ef þetta gerist: eina fyrir hvern möguleika.

Segjum sem dæmi að við viljum sýna að $\enot(P \eand \enot P)$ sé klifun. Til að gera það, þurfum við að sýna að \emph{enginn} sanngildadreifing geri þessa setningu ósanna. Við byrjum á því að gera setninguna ósanna, eins og áður: 

\begin{center}
\begin{tabular}{c|d e e e  f}
$P$&\enot&$(P$&\;\eand&\enot&$P)$\\
\hline
  & Ó&  &  &  & \\
\end{tabular}
\end{center}En við sjáum að þessi setning er ósönn ef $P$ er ósönn, eða ef $\enot P$ er ósönn, þ.e.\ ef $P$ er sönn. Við þurfum því að kanna tvo möguleika. Við byrjum á fyrstu línunni og gerum $P$ ósanna þar:

\begin{center}
\begin{tabular}{c|d e e e f}
$P$&\enot &$(P$&\;\eand&\enot&$P)$\\
\hline
  Ó&  Ó&  &  &  & 
\end{tabular}
\end{center}
En nú sjáum við að ef $P$ er ósönn, þá er $P \eand \enot P$ ósönn, og þar með er $\enot(P \eand \enot P)$ sönn. En við höfðum gefið okkur að $\enot(P \eand \enot P)$ væri ósönn og því þurfum við að skrifa tvö sanngildi á sama stað í töflunni, og það er ekki hægt. Þessi sanngildadreifing er því ekki möguleg. Við skrifum „X“ að sýna að við höfum reynt að setja tvö sanngildi á sama stað:
\begin{center}
\begin{tabular}{c|d e e e f}
$P$&\enot &$(P$&\;\eand&\enot&$P)$\\
\hline
  Ó&  X&  Ó&  Ó&  S&Ó 
\end{tabular}
\end{center}
Ef við reynum að gera $P$ satt, þá lendum við í sama vanda, því þá er $\enot P$ ósönn:
\begin{center}
\begin{tabular}{c|d e e e f}
$P$&\enot &$(P$&\;\eand&\enot&$P)$\\
\hline
  Ó&  X&  Ó&  Ó&  S&Ó \\
  S&  X&  S&  Ó&  Ó&S
\end{tabular}
\end{center}Það er því ekki til nein sanngildadreifing sem gerir þessa setningu ósanna og þess vegna hlýtur hún að vera klifun. Þetta er þó ekki nema einfalt dæmi um hvernig við förum að ef tveir möguleikar koma upp, því eins og glöggir lesendur hafa kannski tekið eftir, þá höfum við fyllt út alla sanntöfluna við að beita þessari aðferð, og því enginn tími sem sparaðist. En í flóknari dæmum, eins og við sjáum að neðan, getur oft gegnt öðru máli.

Það er þó gott að hafa í huga að ef manni hugnast ekki að beita þessari aðferð, þá er alltaf hægt að búa til fulla sanntöflu. 
 
\paragraph{Mótsagnir.}

Til að athuga hvort setning sé mótsögn eða ekki þarf heldur ekki heila sanntöflu. Til þess að sýna að setning sé mótsögn þurfum við að sýna að það sé engin sanngildadreifing til þar sem setningin er sönn. Til að sýna að setning sé \emph{ekki} mótsögn þurfum við að sýna að til sé að minnsta kosti ein sanngildadreifing þar sem hún er sönn.

Byrjum á að skoða dæmi um setningu sem er ekki mótsögn. Við þurfum að sýna að til sé sanngildadreifing þar sem setningin er sönn. Við getum notað sama dæmi og að ofan, nema núna byrjum við á að skrifa að setningin sé sönn undir aðaltenginu:
\begin{center}
\begin{tabular}{c c c c|d e e e e e f}
$S$&$T$&$U$&$W$&$(U$&\eand&$T)$&\eif    &$(S$&\eand&$W)$\\
\hline
  &  &  &  &   &   &   &\TTbf{S}&  &  &
\end{tabular}
\end{center}
Til þess að setningin sé sönn, er nóg að forliðurinn sé ósannur, samkvæmt skilgreiningarsanntöflunni fyrir skilyrðistengið. Forliðurinn er svo samtenging og því nóg að einungis ein af setningunum sem mynda hana sé ósönn. Við getum ráðið því sjálf hverja þeirra við veljum og sett svo hvaða sanngildi sem er á hinar setningarnar. Segjum að „$U$“ sé ósönn. Þá fáum við:

\begin{center}
\begin{tabular}{c c c c|d e e e e e f}
$S$&$T$&$U$&$W$&$(U$&\eand&$T)$&\eif    &$(S$&\eand&$W)$\\
\hline
  &  & Ó &  &  Ó &  Ó  &  &\TTbf{S}& & &
\end{tabular}
\end{center}

Nú getum við svo sett hvaða sanngildi sem við viljum á hinar setningarnar og endum til dæmis með: \begin{center}
\begin{tabular}{c c c c|d e e e e e f}
$S$&$T$&$U$&$W$&$(U$&\eand&$T)$&\eif    &$(S$&\eand&$W)$\\
\hline
 Ó & S & Ó & Ó &  Ó &  Ó  & S  &\TTbf{S}&  Ó &   Ó & Ó
\end{tabular}
\end{center}
Þessi sanngildadreifing er þannig að setningin er sönn, og því getur hún ekki verið mótsögn. En hvað ef setningin \emph{hefði} verið mótsögn? Þá hefði ekki verið til neinn sanngildadreifing þar sem hún er sönn. Þá hefðum við ekki getað fyllt út línuna án þess að þurfa að setja bæði sanngildin á einu og sömu grunnsetninguna. Þetta er alveg hliðstætt við dæmið hér að ofan um klifanir.

\paragraph{Rökfræðilegt jafngildi.}

Til að sýna að tvær setningar séu ekki rökfræðilega jafngildar er nóg að sýna að til sé a.m.k.\ ein sanngildadreifing þar sem þær hafa ólík sanngildi. Til þess þurfum við bara eina línu: við látum eina vera sanna og hina ósanna. Ef við getum klárað sanngildadreifinguna, þá eru setningarnar rökfræðilega jafngildar.

En hvað ef setningar \emph{eru} rökfræðilega jafngildar? Við gætum prófað að gefa setningunum ólík sanngildi og fyllt út ókláraða sanntöflu: ef við getum búið til slíka sanngildadreifingu, þá eru setningarnar ekki rökfræðilega jafngildar. En rétt eins og að ofan er málið ekki alveg svona einfalt. 

Ef um er að ræða tvær setningar, $\meta{A}$ og $\meta{B}$, og við prófuðum að gefa $\meta{A}$ sanngildið „S“ og $\meta{B}$ sanngildið „Ó“ og tækist svo ekki að fylla út línuna í sanntöflunni, þá gætum við ekki dregið þá ályktun að $\meta{A}$ og $\meta{B}$ hljóti að vera rökfræðilega jafngildar. Af hverju? Jú, af því að þá hefðum við bara sýnt að ekki sé til sanngildadreifing þar sem $\meta{A}$ er sönn en $\meta{B}$ ósönn. Hugsanlega væri til sanngildadreifing þar sem $\meta{B}$ er sönn en $\meta{A}$ ósönn. Hér þyrftum við því aftur tvær línur, eina fyrir hvern möguleika.

Hér er dæmi: 

\begin{center}
	\begin{tabular}{c c|d e e f | d e f}
		$P$&$Q$&$\enot$&$P\;$&\eor&$Q$&$P$&\eif&$Q$\\
		\hline
		%J   L   -   L      ->     (J   v   L)
		 &  & & & S & & & Ó & \\
		 &  & & & Ó & & & S &
	\end{tabular}
\end{center}
Við byrjum á því að gefa setningunum ólík sanngildi, fyrst þannig að $\enot P \eor Q$ sé sönn en $P \eif Q$ ósönn, og svo öfugt.
 
Við byrjum á að fylla út efstu línuna: skilyrðissetning er bara ósönn ef forliðurinn er sannur og bakliðurinn ósannur. Þá vitum við að $P$ er sönn á þeirri línu og $Q$ ósönn. Þá höfum við:

\begin{center}
	\begin{tabular}{c c|d e e f | d e f}
		$P$&$Q$&$\enot$&$P\;$&\eor&$Q$&$P$&\eif&$Q$\\
		\hline
		%J   L   -   L      ->     (J   v   L)
		 S & Ó & & S & S & Ó & S & Ó & Ó\\
		 &  & & & Ó & & & S &
	\end{tabular}
\end{center}
En $\enot P$ er ósönn ef $P$ er sönn og því þyrftum við að setja „Ó“ í dálkinn undir „$\enot$“. En þar verður að vera „S“ svo að $\enot P \eor Q$ sé sönn. Þessi sanngildadreifing er því ekki möguleg. Við setjum aftur „X“ til að gefa það til kynna:

\begin{center}
	\begin{tabular}{c c|d e e f | d e f}
		$P$&$Q$&$\enot$&$P\;$&\eor&$Q$&$P$&\eif&$Q$\\
		\hline
		%J   L   -   L      ->     (J   v   L)
		 S & Ó & X & S & S & Ó & S & Ó & Ó\\
		 &  & & & Ó & & & S &
	\end{tabular}
\end{center}
Við höldum svo áfram með hina línuna. Við vitum að $\enot P \eor Q$ er bara ósönn ef báðar hlutasetningarnar eru ósannar. Þá vitum við að $Q$ og $\enot P$ hljóta báðar að vera ósannar. $P$ er því sönn. Þá höfum við:

\begin{center}
	\begin{tabular}{c c|d e e f | d e f}
		$P$&$Q$&$\enot$&$P\;$&\eor&$Q$&$P$&\eif&$Q$\\
		\hline
		%J   L   -   L      ->     (J   v   L)
		 S & Ó & X & S & S & Ó & S & Ó & Ó\\
		 S & Ó & Ó & S & Ó & Ó & & S &
	\end{tabular}
\end{center}
Núna vitum við hvaða sanngildi $P$ og $Q$ hljóta að hafa ef $\enot P \eor Q$ er ósönn. 

Ef við myndum svo halda áfram að fylla út línu tvö, þá myndum við sjá að forliður skilyrðissetningarinnar hlýtur að vera sannur en bakliðurinn ósannur. Því er skilyrðissetningin í heild ósönn, en af því að við gáfum okkur að hún væri sönn, þá yrðum við að skrifa bæði sanngildin, satt og ósatt, í dálkinn fyrir neðan skilyrðistengið. Það er ekki hægt, og því er engin sanngildadreifing þar sem $\enot P \eor Q$ er ósönn en $P \eif Q$ sönn. Við höfum því prófað báða möguleikana og sýnt að ekki er til sanngildadreifing þar sem þessar tvær setningar hafa ólík sanngildi, og því eru þær rökfræðilega jafngildar.

Það er þó gott að hafa í huga að manni ber engin skylda til að nota þessa aðferð við að meta hvort tvær setningar séu rökfræðilega jafngildar. Það er alltaf hægt að fylla út alla sanntöfluna og athuga hvort þær hafi sama sanngildi í öllum línum.

\paragraph{Rökfræðileg samkvæmni.}

Til að sýna að setningar, tvær eða fleiri, séu rökfræðilega samkvæmar hverri annarri þá þurfum við að sýna að til sé sanngildadreifing þar sem þær eru allar sannar. Við gerum það með sama hætti og að ofan: við setjum upp ókláraða sanntöflu þar sem allar setningarnar eru sannar og fyllum hana út: ef okkur tekst það, þá eru þær samkvæmar, annars ekki.

\paragraph{Gildi/rökfræðileg afleiðing.}
 
Til að sýna að rökfærsla sé ógild er nóg að sýna að til sé sanngildadreifing þar sem forsendurnar eru sannar en niðurstaðan ósönn. Við getum því reynt að smíða slíka sanngildadreifingu með því að láta forsendurnar vera allar sannar, en niðurstöðuna ósanna. Athugum hvort $Q, P \eif Q \therefore P$ sé gild rökfærsla:

\begin{center}
	\begin{tabular}{c c|c| d e f | c}
		$P$&$Q$&$Q$&$P$&\eif&$Q$&$P$\\
		\hline
		%J   L   -   L      ->     (J   v   L)
		 & & S & &S & & Ó
	\end{tabular}
\end{center}
Hér höfum við látið báðar forsendurnar vera sannar en niðurstöðuna ósanna. Svo fyllum við línuna til samræmis og fáum:

\begin{center}
	\begin{tabular}{c c|c| d e f | c}
		$P$&$Q$&$Q$&$P$&\eif&$Q$&$P$\\
		\hline
		%J   L   -   L      ->     (J   v   L)
		 Ó & S & S & Ó & S & S & Ó
	\end{tabular}
\end{center}
Hér er því komin sanngildadreifing þar sem forsendurnar eru báðar sannar en niðurstaðan ósönn. Þetta er því ógild rökfærsla.

Til að sýna að rökfærsla sé \emph{gild} þarf að sýna að \emph{engin} sanngildadreifing sé til þar sem niðurstöðurnar eru báðar sannar en niðurstaðan ósönn. Við gerum þetta á nákvæmlega sama hátt og í hinum dæmunum hér að ofan. Það er þó mikilvægt að hafa í huga að ef tveir möguleikar eru í boði þegar við smíðum sanndreifinguna, þá verðum við að hafa eina línu fyrir hvorn möguleika, rétt eins og talað var um hér að ofan.

Hér er dæmi:

\begin{center}
	\begin{tabular}{c c c | d d f | d d d d d d f | d d f}
		$P$&$Q$&$R$&$P$&$\eiff$&$Q$&($Q$&$\eand$&$R$)&$\eiff$&($P$&$\eor$&$R$)&$Q$&$\eiff$&$R$ \\
		\hline
		%J   L   -   L      ->     (J   v   L)
		 & & & & S & & & & & S & & & & & Ó &
	\end{tabular}
\end{center}
Hér höfum við látið forsendurnar vera sannar en niðurstöðuna ósanna. En engin frekari gildi í sanntöflunni eru ákvörðuð af því sem við höfum nú þegar valið. Til dæmis er $Q \eiff R$ ósönn ef og aðeins ef báðir liðir hafa mismunandi sanngildi. Við þurfum því að bæta við annarri línu, einni fyrir hvorn möguleika. Við skrifum því tvær línur, eina þar sem $Q$ er satt en $R$ ósatt, og svo öfugt:

\begin{center}
	\begin{tabular}{c c c | d d f | d d d d d d f | d d f}
		$P$&$Q$&$R$&$P$&$\eiff$&$Q$&($Q$&$\eand$&$R$)&$\eiff$(&$P$&$\eor$&$R$)&$Q$&$\eiff$&$R$ \\
		\hline
		%J   L   -   L      ->     (J   v   L)
		  & S & Ó & & S & & & & & S & & & & S & Ó & Ó \\
		  & Ó & S &  & S & & & & & S & & & & Ó& Ó & S\\
	\end{tabular}
\end{center}

Svo höldum við áfram að fylla út töfluna, sem er lítið mál, þar sem við höfum nú þegar ákvarðað sanngildi tveggja af þremur grunnsetningum og bara spurning um að afrita þau á rétta staði og halda áfram í samræmi við skilgreiningarsanntöflurnar fyrir setningatengin:

\begin{center}
	\begin{tabular}{c c c | d d f | d d d d d d f | d d f}
		$P$&$Q$&$R$&$P$&$\eiff$&$Q$&($Q$&$\eand$&$R$)&$\eiff$&($P$&$\eor$&$R$)&$Q$&$\eiff$&$R$ \\
		\hline
		%J   L   -   L      ->     (J   v   L)
		 S & S & Ó & S & S & S & S & & & S & S & & & S & Ó & Ó \\
		 Ó & Ó & S & Ó & S & Ó & Ó & & & S & Ó & & & Ó & Ó & S \\
	\end{tabular}
\end{center}
Takið eftir því að sanngildið á $P$ ræðst af sanngildi $Q$ og því að við höfum ákveðið að fremsta jafngildissetningin sé sönn.

Þegar hér er komið við sögu lendum við þó í vandræðum. Skoðum fyrst efri línuna. Við vitum að $R$ hlýtur að vera ósönn þar, því niðurstaðan er jafngildissetning og $R$ hefur því ekki sama sanngildi og $Q$, en sanngildi hennar vorum við þegar búin að ákvarða. Það þýðir að $Q \eand R$ er ósönn og þar með vitum við að $P\eor R$ hlýtur líka að vera ósönn (því jafngildissetningin er sönn). En þá ætti $P$ líka að vera ósönn, en við vorum búin að ákvarða að hún væri sönn. Þessi sanngildadreifing er því ekki möguleg.

Sömu sögu má segja um línu tvö: þar sem $R$ er satt, ætti $P \eor R$ líka að vera satt. En þar sem $Q$ er ósatt hlýtur $Q \eand R$ líka að vera ósatt og þá ætti öll jafngildissetningin að vera ósönn. En við vorum búin að ákvarða að hún væri sönn. Þessi sanngildadreifing er því heldur ekki möguleg. Við getum því ekki smíðað \emph{neina} sanngildadreifingu þar sem forsendurnar eru báðar sannar og niðurstaðan ósönn, og því er samsvarandi rökfærsla \emph{gild}.


\practiceproblems
\problempart
Notið sanntöflur (fullar eða ókláraðar eftir hentisemi) til að ákvarða hvort þessi setningapör séu rökfræðilega jafngild:

\begin{earg}
\item $A$, $\enot A$ %No
\item $A$, $A \eor A$ %Yes
\item $A\eif A$, $A \eiff A$ %Yes
\item $A \eor \enot B$, $A\eif B$ %No
\item $A \eand \enot A$, $\enot B \eiff B$ %Yes
\item $\enot(A \eand B)$, $\enot A \eor \enot B$ %Yes
\item $\enot(A \eif B)$, $\enot A \eif \enot B$ %No
\item $(A \eif B)$, $(\enot B \eif \enot A)$ %Yes
\end{earg}

\problempart
Notið sanntöflur (fullar eða ókláraðar eftir hentisemi) til að ákvarða hvort setningarnar í hverri línu séu rökfræðilega samkvæmar eða rökfræðilega ósamkvæmar:

\begin{earg}
\item $A \eand B$, $C\eif \enot B$, $C$ %inconsistent
\item $A\eif B$, $B\eif C$, $A$, $\enot C$ %inconsistent
\item $A \eor B$, $B\eor C$, $C\eif \enot A$ %consistent
\item $A$, $B$, $C$, $\enot D$, $\enot E$, $F$ %consistent
\end{earg}

\problempart
Notið sanntöflur (fullar eða ókláraðar eftir hentisemi) til að ákvarða hvort þessar rökfærslur séu gildar eða ógildar:

\begin{earg}
\item $A\eor\bigl[A\eif(A\eiff A)\bigr] \therefore A$ %invalid
\item $A\eiff\enot(B\eiff A) \therefore A$ %invalid
\item $A\eif B, B \therefore A$ %invalid
\item $A\eor B, B\eor C, \enot B \therefore A \eand C$ %valid
\item $A\eiff B, B\eiff C \therefore A\eiff C$ %valid
\end{earg}