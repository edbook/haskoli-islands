%!TEX root = fyrirollx.tex
\part{Grunnatriði}
\label{ch.intro}

\chapter{Rökfærslur}\label{argRaining}\label{s:Arguments}

Flest viljum við hafa vel ígrundaðar skoðanir. Ef við teljum að vel ígrunduð skoðun sé skoðun sem studd er góðum rökum---þ.e.\ ástæðum sem renna stoðum undir hana---þá vaknar sú spurning hvað góð rök séu eiginlega og hvernig við getum metið hvaða rök eru í raun og veru góð. Frá sjónarhóli heimspekinnar er það hlutverk rökfræðinnar, að greina góðan rökstuðning frá vondum. Hér er dæmi um rökstuðning:

	\begin{earg}
		\item[] Það hellirignir.
		\item[] Ef þú tekur ekki með þér regnhlíf, þá blotnarðu.
		\item[Þar af leiðandi:] Þú ættir að taka með þér regnhlíf.
	\end{earg}
	Hér eru fyrstu tvær setningarnar gefnar sem ástæða fyrir því að trúa þeirri síðustu. Við köllum þessar ástæður \emph{forsendur} og setninguna sem þær rökstyðja \emph{niðurstöðu}. Loks köllum við setningarnar allar saman sem heild \emph{rökfærslu}. Við segjum að rökfærsla sé eitthvert safn af setningum þar sem allar setningarnar nema ein eru forsendur en sú sem út af stendur niðurstaða.\footnote{Oft er gerður greinarmunur á \emph{setningum} og \emph{fullyrðingum}. Fullyrðingar eru þá \emph{það sem setningar segja}. Til dæmis, þá segja setningarnar „það er úrhelli“ og „það rignir eins og hellt sé úr fötu“ það sama, nefnilega \emph{að það sé mjög mikil rigning}. Við segjum þá að þær tjái báðar þá fullyrðingu að það sé mjög mikil rigning. Við munum ekki gera mikið með þennan greinarmun og tölum einfaldlega um setningar. } Viðfangsefni rökfræðinnar eru tengslin milli forsendanna og niðurstöðunnar.

Í dæminu hér að ofan notuðum við tvær aðskildar setningar til að tjá báðar forsendur rökfærslunnar og þá þriðju til að tjá niðurstöðuna. Oft eru rökfærslur með þessum hætti. En það er líka vel hægt að tjá rökfærslu í einni málsgrein:

	\begin{quote}
		 Ég var með hattinn, svo það hlýtur að hafa verið sól.
	\end{quote}
Þessi rökfærsla hefur eina forsendu og niðurstöðu. Oft byrja rökfærslur á forsendunum og enda á niðurstöðunni, enda er það um margt eðlileg röð hlutanna. En það þarf ekki að vera. Rökfærslan hér að ofan hefði allt eins getað verið sett fram með niðurstöðuna fyrst:

	\begin{quote}
		Þú ættir að taka með þér regnhlífina. Það er jú hellidemba. Og ef þú tekur hana ekki, þá verðurðu gegndrepa. 
	\end{quote}
Sama rökfærsla gæti líka hafa haft niðurstöðuna í miðjunni:
	\begin{quote}
		Það er úrhelli. Þú ættir þess vegna að taka regnhlífina, annars verðurðu alvot.
	\end{quote}
Þegar við skoðum rökfærslur, þá viljum við vita hvort niðurstöðuna \emph{leiði af} forsendunum eða ekki. Til þess að geta það, þá verðum við að greina niðurstöðuna frá forsendunum---vita hvort er hvað. Það er ekki alltaf einfalt mál, en eftirfarandi orð og orðasambönd benda oft til þess að það sem á eftir kemur sé niðurstaða:	
	\begin{center}
		þar af leiðandi, svo að, af því leiðir, þannig að, þess vegna 
	\end{center}
Á hinn bóginn gefa eftirfarandi orðasambönd til kynna að næsta málsgrein sé forsenda, frekar en niðurstaða: 
	\begin{center}
		vegna þess, út af því að, að því gefnu að, því, þar eð, í ljósi þess að, enda		
	\end{center}
	
Athugið samt að þessi listi er ekki tæmandi og oft eru engin sérstök orð sem gefa til kynna að um rökfærslu sé að ræða. Hér er ekki hægt að gefa skýrar og einfaldar leiðbeiningar sem hægt er að fylgja hugsunarlaust, enda er það ákveðin list að greina rökfærslur svo vel sé, frekar en vísindi. Þar, eins og í flestu öðru, skapar æfingin meistarann.	

\practiceproblems

Í lok kaflahluta eru oft æfingar sem fara yfir efni hvers kafla. Það kemur ekkert í stað þess að gera æfingarnar, enda er tilgangur þess að læra rökfræði ekki sá að læra staðreyndir utanbókar, heldur að skerpa rökhugsunina. Að mörgu leyti hugsar maður með pennanum við rökfræðinám, ekki bara höfðinu.

\
\\Hér er fyrsta æfingin. Finnið setninguna sem tjáir niðurstöðu hverrar rökfærslu:

\begin{earg}
	\item Sólin skín í heiði. Ég ætti að taka sólgleraugun.
	\item Það hlýtur að hafa verið sól í gær. Ég fór jú í sund.
	\item Enginn nema þú varst í eldhúsinu gær. Og þú ert allur í kökumylsnu. Þú stalst kökunni úr krúsinni í gær!
	\item Skafti og Skapti voru í lesstofunni þegar morðið var framið. Og Prófessor Vandráður var með kertastjakann í borðstofunni, og við vitum að hann var með hreinar hendur. Svo Hörður hlýtur að hafa framið verknaðinn í eldhúsinu með blýrörinu. Það hafði nefnilega ekki verið hleypt af byssunni.
\end{earg}


\chapter{Gildar rökfærslur}\label{s:Valid}

Hér að ofan í \S\ref{s:Arguments} gáfum við mjög víða skilgreiningu á því hvað rökfærsla er. Það sést ágætlega á eftirfarandi dæmi:
	\begin{earg}
		\item[] Reykjavík er syðsta höfuðborg Evrópu.
		\item[Þar af leiðandi:] Guðni Th.\ er í grænum buxum.
	\end{earg}
Hér höfum við forsendu og niðurstöðu, og þar með höfum við rökfærslu, samkvæmt skilgreiningunni að ofan.

Þetta er að vísu mjög ósannfærandi og vond rökfærsla, en það mun einfalda okkur lífið sem rökfræðingar töluvert að telja hana með sem slíka og halda okkar víðu skilgreiningu. Annars þyrftum við að reyna að draga mörkin nákvæmlega, og það er hvort tveggja, mjög erfitt og fullkomlega óþarfi, eins og við munum sjá þegar fram líða stundir.

\section{Rökfærslur geta farið úrskeiðis á tvenna vegu}

En hvers vegna er þessi rökfærsla jafn vond og raun ber vitni? Það eru tvær meginástæður fyrir því. Í fyrsta lagi er forsendan augljóslega ósönn: Reykjavík er í raun nyrsta höfuðborg Evrópu og liggur rétt sunnan við heimsskautsbaug. Í öðru lagi leiðir ekkert um klæðaburð forseta Íslands af hnattrænni stöðu Reykjavíkur, jafnvel þó að svo kynni að vera að Guðni Th.\ sé í grænum buxum þegar þessi orð eru skrifuð (eða lesin). Við getum einfaldlega ekki dregið neina ályktun þar að lútandi.

Hvað með röksemdafærsluna sem notuð var sem dæmi hér að ofan í \S\ref{s:Arguments}? Forsendur hennar gætu verið ósannar; það gæti vel verið að úti skíni sól og ekki sjáist ský á himni, nú eða að þú munir blotna þrátt fyrir að vera með regnhlíf. En það gæti líka allt eins verið að forsendurnar séu báðar sannar, og ef svo er, þá myndi það ekki endilega þýða að þú ættir að taka með þér regnhlíf. Kannski hefurðu sérstaka ánægju af gönguferðum í úrhelli, ellegar að sérvitur milljónamæringur bjóðist til láta stórfé af hendi rakna til góðgerðamála, en bara ef þú skilur regnhlífina eftir heima. Forsendurnar tvær tryggja því alls ekki að niðurstaðan sé sönn: þær gætu báðar verið sannar, en niðurstaðan samt sem áður ósönn.

Skoðum aðra rökfærslu:

	\begin{earg}
		\item[] Þú ert að lesa þessa bók.
		\item[] Þetta er rökfræðibók.
		\item[Þar af leiðandi:] Þú ert rökfræðinemi.
	\end{earg}

Þetta er ekki svo slæm rökfærsla. Báðar forsendurnar eru sannar og flestir sem lesa þessa bók eru líklega rökfræðinemar. En samt sem áður, þá er það vel mögulegt að einhver sem ekki er rökfræðinemi lesi hana. Ef vinur þinn tæki hana til dæmis upp og fletti í gegnum hana, þá yrði hann ekki þar með rökfræðinemi. 

En eins og áður segir, þá er þessi rökfærsla ekki endilega alslæm, og það er rökfærslan í \S\ref{s:Arguments} ekki heldur. Ef það rignir, þá gefur sú vitneskja að regnhlífar hlífi manni við því að vökna vissulega einhverja ástæðu fyrir því að maður ætti að taka með sér regnhlíf og það er á sama hátt vissulega líklegt að þú, lesandi þessarar bókar, sért rökfræðinemi og því gefa forsendur þeirrar rökfærslu líka einhverja ástæðu til að halda að niðurstaðan sé sönn. En í hvorugu dæminu \emph{tryggja} forsendurnar að niðurstaðan sé sönn, jafnvel þó að þær séu allar sannar.

Þessi dæmi sýna því að rökfærslur geta farið úrskeiðis á tvenna vegu:
%
	\begin{ebullet}
		\item Ein eða fleiri forsendur eru ósannar. 
		\item Niðurstöðuna leiðir ekki af forsendunum.
	\end{ebullet}
Það er auðvitað mjög mikilvægt að geta skorið úr um hvort forsendur rökfærslu séu sannar. En það er ekki viðfangsefni rökfræðinnar. Forsendur geta nefnilega fjallað um hvaða efni sem er undir (og yfir) sólinni: innihald ísskápsins heima hjá mér, efnasamsetningu kviku í jarðskorpunni, vegalengdir í geimnum, atburði fortíðar eða hvað sem er annað. Ef þetta allt væri viðfangsefni rökfræðinnar, þá væri hún víðfemasta fræðigreinin og rökfræðingar sérfræðingar í öllu. Það væri ómögulegt! Við höfum þess vegna takmarkaðan áhuga á því þegar við stundum rökfræði hvort tilteknar forsendur séu sannar eða ósannar \emph{almennt} og einbeitum okkur að seinni valkostinum. Með öðrum orðum: Viðfangsefni rökfræðinnar er hvort tiltekna niðurstöðu \emph{leiði af} ákveðnum forsendunum.

\section{Gildi}

Viðfangsefni rökfræðinnar er eins og áður sagði það að meta hvort niðurstöðu rökfærslu leiðir af forsendunum. Við viljum vita hvort niðurstaðan \emph{hljóti} að vera sönn, \emph{ef} forsendurnar eru allar sannar. Ef svo er, þá segjum við að rökfærslan sé \emph{gild} og við munum notast við eftirfarandi skilgreiningu á \emph{gildi} rökfærslu: 

	\factoidbox{
		Rökfærsla er \define{gild} ef og aðeins ef það er ómögulegt fyrir allar forsendur hennar að vera sannar en niðurstöðuna ósanna.
	}
Aðalatriðið hér er að ef rökfærsla er gild, þá hlýtur niðurstaðan (nauðsynlega) að vera sönn, ef forsendurnar eru allar sannar: Hér er dæmi:	
	
	\begin{earg}
		\item[] Appelsínur eru annað hvort ávextir eða hljóðfæri.
		\item[] Appelsínur eru ekki ávextir.
		\item[Þar af leiðandi:] Appelsínur eru hljóðfæri.
	\end{earg}
Niðurstaða þessarar rökfærslu er augljóslega út í hött, enda eru appelsínur ávextir. Hana leiðir samt sem áður af forsendunum, því \emph{ef} báðar forsendurnar eru sannar, þá \emph{hlýtur} niðurstaðan að vera sönn. Þessi rökfærsla er þess vegna gild.

Þetta dæmi sýnir að gildar rökfærslur þurfa hvorki að hafa sannar forsendur né sanna niðurstöðu. Á gagnstæðan hátt er ekki nóg að rökfærsla hafi sannar forsendur og sanna niðurstöðu til þess að teljast gild. Það sést vel af næsta dæmi:
	
	\begin{earg}
		\item[] Stokkhólmur er í Svíþjóð.
		\item[] Kaupmannahöfn er í Danmörku.
		\item[Þar af leiðandi:] París er í Frakklandi.
	\end{earg}
Forsendur og niðurstaða þessarar rökfærslu eru allar sannar. En rökfærslan er engu að síður ekki gild. Ef París hefði til dæmis lýst yfir sjálfstæði frá Frakklandi árið 1871, þá væri niðurstaðan ósönn, jafnvel þó að hinar forsendurnar væru enn báðar sannar. Þetta sýnir að það er \emph{mögulegt} fyrir forsendur þessarar rökfærslu að vera allar sannar en að niðurstaðan sé ósönn. Rökfærslan er því ekki gild, eða einfaldlega: ógild.

Það sem skiptir mestu máli að muna í sambandi við gildi er að það hefur ekkert að gera með sannleika setninganna í rökfærslunni. Það snýst um hvort niðurstaðan \emph{hljóti} að vera sönn, \emph{ef} forsendurnar eru allar sannar---að það sé engin leið fyrir forsendurnar að vera allar sannar en að niðurstaðan sé ósönn. Gildi snýst þess vegna um \emph{form} rökfærslunnar, þ.e.\ hvernig forsendurnar og niðurstaðan tengjast, en ekki innihald þeirra. Við munum samt segja að rökfærsla sé \define{rétt} ef og aðeins ef hún er bæði gild og allar forsendur hennar eru sannar.

Gildi er, eins og við munum heyra aftur og aftur, eitt mikilvægasta hugtak rökfræðinnar og mun koma mikið við sögu í þessari bók.

\section{Tilleiðslur}
Margar góðar rökfærslur eru ógildar. Skoðum til dæmis þessa hér:
	\begin{earg}
		\item[] Það rigndi í Reykjavík í nóvember árið 1997.
		\item[] Það rigndi í Reykjavík í nóvember árið 1998.
		\item[] Það rigndi í Reykjavík í nóvember árið 1999.
		\item[] Það rigndi í Reykjavík í nóvember árið 2000.
		\item[] Það rigndi í Reykjavík í nóvember árið 2001.
		\item[]Það rigndi í Reykjavík í nóvember árið 2002.
	\item[Þar af leiðandi:] Það rignir alltaf í nóvember í Reykjavík.
\end{earg}

Þessi rökfærsla alhæfir um allar kringumstæður af einhverju tagi út frá athugunum um einstakar kringumstæður af því tagi, nefnilega að það rigni alltaf í nóvember í Reykjavík, af því að það hefur rignt í nóvember í Reykjavík í þeim mánuðum sem athugaðir voru. Slíkar rökfærslur eru kallaðar \define{tilleiðslur}. Við hefðum getað styrkt þessa tilleiðslu enn frekar með að bæta við fleiri forsendum: Í nóvember 2003, rigndi í Reykjavík, Í nóvember 2004, rigndi í Reykjavík, Í nóvember 2005, rigndi í Reykjavík, og svo framvegis. En það skiptir engu máli hversu mörgum forsendum af þessu tagi við bætum við rökfærsluna, það er alltaf mögulegt að hann hangi þurr í Reykjavík allan næsta nóvember, jafnvel þó að það hafi rignt þar í þeim mánuði á hverju ári síðan landið reis úr sæ.

Þetta sýnir að tilleiðslur, jafnvel þó að þær séu góðar, eru ekki gildar. Þær eru ekki fullkomlega \emph{öruggar}. Það skiptir ekki máli hversu ólíklegt það er, það er alltaf \emph{mögulegt} að niðurstaða slíkrar rökfærslu sé ósönn, jafnvel þó að allar forsendurnar séu sannar. 

Það er samt mikilvægt að hafa í huga að margar ógildar rökfærslur eru engu síður traustsins verðar---og mögulega mikilvægar fyrir okkur. Tökum sem dæmi eftirfarandi tvær rökfærslur: 

\begin{earg}
	\item Sólin hefur risið alla daga á minni ævi. 
	\item[Þar af leiðandi:] Sólin mun rísa á morgun.
\end{earg}

\begin{earg}
	\item Lásinn á útidyrunum er brotinn. 
	\item Öll helstu verðmæti eru horfin úr íbúðinni minni.
	\item[Þar af leiðandi:] Brotist hefur verið inn hjá mér.
\end{earg}

Hvorug þessara rökfærsla er gild, en það er í sjálfu sér ekkert að því að trúa að sólin rísi á morgun vegna þess að hún hefur alltaf risið hingað til eða að innbrot hafi átt sér stað ef verðmæti vantar úr íbúðinni minni og lásinn á útidyrunum er brotinn. En þessar rökfærslur eru samt ekki gildar---það má vel ímynda sér að sólin komi ekki upp á morgun og við gætum eflaust upphugsað einhverja sögu sem skýrði af hverju allar forsendurnar í seinni rökfærslunni eru sannar en þó þannig að ekkert innbrot hafi átt sér stað. Það er með öðrum orðum mögulegt að forsendur þessara rökfærsla séu sannar, en niðurstöðurnar ósannar.

Það er því munur á rökfærslum sem ætlað er að séu gildar, það er að \emph{tryggja} að niðurstaðan sé sönn, ef forsendurnar eru það, og rökfærslum sem einungis er ætlað að renna stoðum undir niðurstöðuna, að sannfæra okkur um að hún sé \emph{líkleg}. Seinni gerðin af rökfærslu, er eins og áður segir, kölluð tilleiðsla, en sú fyrri \define{afleiðslur}. Góðar afleiðslur eru sagðar vera gildar, en góðar tilleiðslur eru sagðar vera \define{sterkar}.

Það er samt mikilvægt að hafa í huga að það er ekki alltaf hægt að vera viss um hver ætlun mælanda sem setur fram rökfærslu er, og því oft erfitt að segja til um hvort meta eigi rökfærslu eftir því hvort hún eigi að vera sterk eða gild. Það þarf því oft að sýna ákveðið örlæti við að túlka og greina rökfærslur og oft er hægt að túlka ógilda rökfærslu sem svo að hún eigi að vera sterk---og því góðra gjalda verð. Stundum eru líka faldar forsendur í rökfærslum sem settar eru fram í mæltu máli. Við sáum til dæmis að rökfærslan í \S\ref{s:Arguments} var ógild eins og hún stendur, en leiða má líkum að því að hver sá sem setti hana fram hafi í huga að forsendan „Þú vilt fyrir alla muni forðast að blotna“ sé líka sönn og ef henni er bætt við, þá verður rökfærslan gild.

Tilleiðslurökfræði, sem fæst við að meta hvort tilleiðslur séu góðar eða slæmar, er góðra gjalda verð, en í þessari bók munum við að mestu leggja tilleiðslur til hliðar og einblína á afleiðslur og verður gildi því eitt mikilvægasta hugtakið sem við tökum til skoðunar. 

\practiceproblems
\problempart

Hverjar af eftirfarandi rökfærslum eru gildar? Hverjar eru ógildar?

\begin{earg}
\item Sókrates er maður.
\item Allir menn eru rófur.
\item[Þar af leiðandi:] Sókrates er rófa.
\end{earg}

\begin{earg}
\item Vigdís Finnbogadóttir var annað hvort við nám í Frakklandi eða hún var aldrei forseti.
\item Vigdís Finnbogadóttir var aldrei forseti.
\item[Þar af leiðandi:] Vigdís Finnbogadóttir var við nám í Frakklandi. 
\end{earg}

\begin{earg}
\item Ef ég ýti á rofann, þá kviknar ljósið.
\item Ég ýti ekki á rofann.
\item[Þar af leiðandi:] Ljósið kviknar ekki.
\end{earg}

\begin{earg}
\item Jónas var annað hvort frá Hriflu eða Flugumýri.
\item Jónas var ekki frá Hriflu.
\item[Þar af leiðandi:] Jónas var frá Flugumýri.
\end{earg}

\begin{earg}
\item Ef heimurinn tæki enda í dag, þá þyrfti ég ekki að vakna á morgun.
\item Ég þarf að vakna á morgun.
\item[Þar af leiðandi:] Heimurinn tekur ekki enda á morgun.
\end{earg}

Skoðið skilgreininguna á gildi sérstaklega vel áður en þið svarið þessari:

\begin{earg}
\item Jón er núna 19 ára.
\item Jón er núna 87 ára.
\item[Þar af leiðandi:] Anna er 36 ára.
\end{earg}

\problempart

Skoðið eftirfarandi setningar. Ef fullyrðingin er sönn, sýnið dæmi, ef ekki, útskýrið hvers vegna. Er til...

	\begin{earg}
		\item Rökfærsla sem hefur eina ósanna forsendu og eina sanna?
		\item Gild rökfærsla sem hefur bara ósannar forsendur? 
		\item Gild rökfærsla með ósönnum forsendum og ósannri niðurstöðu?
		\item Gild rökfærsla með sönnum forsendum og ósannri niðurstöðu?
		\item Rétt rökfærsla með ósannri niðurstöðu?
		\item Ógild rökfærsla sem verður gild ef bætt er við nýrri forsendu?
		\item Gild rökfærsla sem verður ógild ef bætt er við nýrri forsendu? 
	\end{earg}
\chapter{Önnur mikilvæg rökfræðihugtök}\label{s:BasicNotions}

Í \S\ref{s:Valid} kynntum við til sögunnar hugtakið \emph{gildi}. Það er, eins og áður sagði, án efa eitt af mikilvægustu hugtökum rökfræðinnar. Í þessum hluta munum við kynnast öðrum hugtökum rökfræðinnar sem eru ekki síður mikilvæg.

\section{Sanngildi}

Eins og við sögðum í \S\ref{s:Arguments}, þá samanstanda rökfærslur af forsendum og niðurstöðu. En ekki geta allar setningar verið notaðar sem forsendur eða niðurstöður. Til dæmis:
	\begin{ebullet}
		\item \textbf{Spurningar}, t.d.\ „Ertu syfjuð?“
		\item \textbf{Skipanir}, t.d.\ „Vaknaðu!“
		\item \textbf{Upphrópanir}, t.d.\ „Á-i!“
	\end{ebullet}
	
Það sem þessar þrjár tegundir setninga eiga sameiginlegt er að þær staðhæfa ekkert: þær geta ekki verið sannar eða ósannar. Það hefur enga merkingu að spyrja hvort spurning sé sönn eða ósönn, einungis hvort svarið sé satt eða ósatt. Eins er það hvorki satt né ósatt að „Vaknaðu!“ eða „Á-i!“.

Þar sem við höfum áhuga á að meta gildi rökfærsla, það er að segja hvort niðurstaða rökfærslu sé sönn, ef forsendur hennar eru allar sannar, þá leyfum við einungis setningar sem geta verið sannar eða ósannar sem hráefni í rökfærslur og segjum að slíkar setningar hafi \define{sanngildi}. Í þessari bók munum við gera ráð fyrir að allar setningar hafi eitt af tveimur sanngildum, \define{satt} eða \define{ósatt}. Engin setning er bæði sönn og ósönn og engin setning er hvorugt.


\section{Samrýmanleiki}
Skoðum eftirfarandi tvær setningar:
	\begin{ebullet}
		\item[B1.] Tvíburabróðir Önnu er lágvaxnari en hún.
		\item[B2.] Tvíburabróðir Önnu er hávaxnari en hún.
	\end{ebullet}
	
Rökfræðin getur ekki sagt okkur hvor þessara setninga er sönn. En við getum sagt að \emph{ef} fyrsta setningin (B1) er sönn, \emph{þá} hljóti hin setningin (B2) að vera ósönn, og ef B2 er sönn, þá hljóti B1 að vera ósönn. Það er ómögulegt að þessar setningar séu báðar sannar (en þær geta reyndar báðar verið ósannar). Þessar setningar eru ósamrýmanlegar hverri annarri. Það er hugsunin á bak við eftirfarandi skilgreiningu:

	\factoidbox{
		Setningar eru \define{samrýmanlegar} ef og aðeins ef það er mögulegt fyrir þær að vera allar sannar samtímis.
	}
	Til samræmis við það segjum við að B1 og B2 séu \emph{ósamrýmanlegar}.

Við getum spurt um hvaða fjölda setninga sem er hvort þær séu samrýmanlegar hverri annarri. Tökum sem dæmi eftirfarandi fjórar setningar:

	\label{MartianGiraffes}
	\begin{ebullet}
		\item[G1.] Það eru að minnsta kosti fjórir selir í Húsdýragarðinum.
		\item[G2.] Það eru nákvæmlega sjö geitur í Húsdýragarðinum.
		\item[G3.] Það eru ekki fleiri en tvö dýr í Húsdýragarðinum sem eru svört að lit.
		\item[G4.] Hver einasti selur í húsdýragarðinum er svartur að lit.
	\end{ebullet}
	Það leiðir af G1 og G4 að það eru að minnsta kosti fjórir svartir selir í Húsdýragarðinum. Það er í mótsögn við G3, sem leiðir til þess að það eru ekki fleiri en tveir svartir selir þar. Setningarnar sem heild eru því ósamrýmanlegar hverri annarri. Þær geta ekki allar verið sannar. Takið samt eftir því að setningar G1, G3 og G4 eru ósamrýmanlegar án G2. En ef eitthvað safn af setningum er ósamrýmanlegt, þá skiptir ekki máli hvaða setningum við bætum við, safnið verður alltaf ósamrýmanlegt.
	
\section{Nauðsyn og hending}

Þegar við skoðum hvort rökfærsla sé gild, þá erum við að athuga hvað væri satt \emph{ef} forsendurnar eru allar sannar. En sumar setningar eru þess eðlis að þær hljóta að vera sannar, án þess að aðrar setningar komi þar við sögu. Skoðum þrjú dæmi:

	\begin{earg}
		\item[\ex{Acontingent}] Það er kveikt á ljósinu.
		\item[\ex{Atautology}] Annað hvort er kveikt á ljósinu eða ekki.
		\item[\ex{Acontradiction}] Það er bæði kveikt á ljósinu og ekki kveikt á ljósinu.
	\end{earg}
	
Til þess að vera viss um hvort \ref{Acontingent} sé sönn eða ósönn þurfum við að athuga með einhverjum hætti hvort ljósið sé kveikt eða ekki. Hún gæti verið sönn, en hún gæti líka verið ósönn. 

Öðru máli gegnir um \ref{Atautology}. Það er engin þörf á neinni athugun til þess að vita að annað hvort sé ljósið kveikt eða ekki. Ef það er ekki kveikt, þá er það slökkt, og öfugt. Þessi setning er \define{nauðsynlega sönn}.

Á sama hátt er engin ástæða til þess að athuga neitt til að meta sanngildi \ref{Acontradiction}. Hún hlýtur að vera ósönn af sömu ástæðu og \ref{Atautology} er sönn. Þessi setning er \define{nauðsynlega ósönn}.

Við segjum að setning sem \emph{getur} verið sönn eða ósönn, en er hvorki nauðsynlega sönn né nauðsynlega ósönn, sé \define{hending}. Seinna í bókinni munum við skilgreina þessi hugtök nákvæmlega.

Það er þó gott að hafa í huga að setning gæti alltaf hafa verið sönn og samt verið hending. Til dæmis gæti verið að það hafi aldrei verið færri en sjö hlutir í alheiminum og þá hefði setningin „Það eru til að minnsta kosti sjö hlutir“ alltaf verið sönn. En hún er samt hending: heimurinn hefði getað verið þannig að einungis sex hlutir væru til, og þá hefði setningin verið ósönn.


\practiceproblems
\problempart
\label{pr.EnglishTautology}
Svarið því hvort eftirfarandi setningar séu hendingar, nauðsynlega sannar eða nauðsynlega ósannar.
\begin{earg}
\item Sesar hélt yfir Rúbíkon.
\item Einhver hélt yfir Rúbíkon.
\item Enginn hefur nokkru sinni haldið yfir Rúbíkon.
\item Ef Sesar hélt yfir Rúbíkon, þá hefur einhver gert það.
\item Jafnvel þó að Sesar hafi haldið yfir Rúbíkon, þá hefur aldrei neinn haldið yfir Rúbíkon.
\item Ef einhver hefur haldið yfir Rúbíkon, þá var það Sesar.
\end{earg}

\problempart
\label{pr.MartianGiraffes}
Skoðið aftur setningar G1--G4 hér að ofan (um seli og geitur í Húsdýragarðinum) og segið til um hver af eftirfarandi setningasöfnum séu samrýmanleg og hver ósamrýmanleg.

\begin{earg}
\item G2, G3, og G4
\item G1, G3, og G4
\item G1, G2, og G4
\item G1, G2, og G3
\end{earg}

\

\problempart
Svarið eftirfarandi spurningum. Ef svarið er já, sýnið dæmi, ef svarið er nei, útskýrið af hverju:
\begin{earg}
\item Eru til gildar rökfærslur með nauðsynlega ósannri niðurstöðu?
\item Eru til ógildar rökfærslur með nauðsynlega sannri niðurstöðu?
\item Eru til samrýmanlegar setningar, þar sem ein er nauðsynlega ósönn?
\item Eru til ósamrýmanlegar setningar, þar sem ein er nauðsynlega sönn?
\end{earg}
